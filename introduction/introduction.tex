% SVN info for this file
\svnidlong
{$HeadURL$}
{$LastChangedDate$}
{$LastChangedRevision$}
{$LastChangedBy$}

\chapter*{Note per la lettura}
\labelChapter{introduzione}

\begin{introduction}
‘‘Un matematico è una macchina per trasformare caffè in teoremi.''
\begin{flushright}
	\textsc{Alfréd Rényi,} studioso del teorema di Van Moka-mpen.
\end{flushright}
\end{introduction}

\lettrine[findent=1pt, nindent=0pt]{S}{enza} troppe pretese di formalità, com'è intuibile dal termine dal termine tecnico \textit{manualozzo} e dalle citazioni a inizio capitolo, queste note sono nate come appunti a quattro mani basati sul corso di \textit{Geometria 2} tenuto dai docenti Alberto Albano, Cinzia Casagrande ed Elena Martinengo nell'Anno Accademico 2020-2021 presso il Dipartimento di Matematica dell'Università degli Studi di Torino.\\
Il corso è diviso in \textit{cinque} parti, pertanto abbiamo ritenuto opportuno dividere in altrettante parti il testo, seguendo l'ordine delle lezioni: Topologia generale, Omotopia, Classificazione delle superfici topologiche, Approfondimenti di Algebra Lineare e infine Geometria proiettiva. I prerequisiti necessari sono gli argomenti trattati nei corsi di \textit{Geometria 1, Algebra 1} e \textit{Analisi 1}.\\
In aggiunta a ciò, potete trovare a fine libro delle utili \textit{postille} con alcune digressioni interessanti, nonché tabelle ed elenchi riepilogativi dei teoremi, delle definizioni e delle proprietà affrontate.\\
Per quanto ci piacerebbe esserlo, non siamo \textit{esseri infallibili}: ci saranno sicuramente sfuggiti degli errori (o degli \textit{orrori}), per cui vi chiediamo gentilmente di segnalarceli sul sito \textcolor{redill}{\url{https://maxmaci.github.io}} per correggerli e migliorare il \textit{manualozzo}.\\
I disegni sono stati realizzati da Massimo Bertolotti, l'addetto alla grafica e ai capricci di \LaTeX\ (ed è molto capriccioso, fidatevi). Inoltre, per chi vuole dilettarsi può cercare di distinguere chi fra i due autori ha scritto cosa, non dovrebbe essere troppo difficile.