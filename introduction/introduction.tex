\begin{document}
	

Senza troppe pretese di formalità, com'è intuibile dal termine dal termine tecnico \textit{manualozzo} e dalle citazioni a inizio capitolo, queste note sono nate come appunti a quattro mani basati sul corso di \textit{Geometria 2} tenuto dai docenti Alberto Albano, Cinzia Casagrande ed Elena Martinengo nell'anno accademico 2020-2021 presso il Dipartimento di Matematica dell'Università degli Studi di Torino.\\
Il corso è diviso in cinque parti, pertanto abbiamo ritenuto opportuno dividere in altrettante parti il testo, seguendo l'ordine delle lezioni: Topologia generale, Omotopia, Classificazione delle superfici topologiche, Approfondimenti di Algebra Lineare e infine Geometria proiettiva. I prerequisiti necessari sono in generale i corsi di \textit{Geometria 1, Algebra 1} e \textit{Analisi 1}.\\
Essendoci basati su quanto spiegato a lezione e quanto scritto nei testi, non abbiamo aggiunto nulla di originale e non possiamo assumercene la responsabilità, tuttavia ci saranno sicuramente sfuggiti degli errori, per cui vi chiediamo di riportarceli
%facciamo un git pubblico o come?
per poterli correggere e migliorare il manuale.\\
I disegni sono stati realizzati da Massimo Bertolotti, l'addetto alla grafica e ai capricci di \LaTeX . Infine chi vuole dilettarsi può cercare di distinguere chi fra i due ha scritto cosa, non dovrebbe essere troppo difficile.

	
	
	
	
	
	
	
	











\end{document}