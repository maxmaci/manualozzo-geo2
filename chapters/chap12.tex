% SVN info for this file
\svnidlong
{$HeadURL$}
{$LastChangedDate$}
{$LastChangedRevision$}
{$LastChangedBy$}

\chapter{Geometria proiettiva}
\labelChapter{geoproiettiva}

\begin{introduction}
	‘‘BEEP BOOP INSERIRE CITAZIONE QUA BEEP BOOP.''
	\begin{flushright}
		\textsc{NON UN ROBOT,} UN UMANO IN CARNE ED OSSA BEEP BOOP.
	\end{flushright}
\end{introduction}
%inserire citazione
\section{Spazi proiettivi}
% mettere ref superfici
Abbiamo già approfondito lo \textbf{spazio proiettivo reale} e le sue caratteristiche sia \textit{topologicamente} nel \autoref{chap:azionidigruppo}, sia come \textit{varietà topologica} nel \autoref{chap:superfici}. In questo capitolo, ci dedicheremo a \textit{generalizzare} il concetto per un \textit{qualsiasi} spazio vettoriale su campo $\kamp$, utilizzando gli strumenti dell'algebra lineare. 
\begin{define}\textsc{Spazio proiettivo.}\\
Sia $\kamp$ un campo e $V$ uno spazio vettoriale di dimensione \textit{finita} su $\kamp$. Lo \textbf{spazio proiettivo}\index{spazio!proiettivo} associato a $V$ è l'insieme quoziente:
\begin{equation}
	\proj{V}=\frac{V\setminus\left\{0\right\}}{\sim}
\end{equation}
Dove $\sim$ è la relazione di equivalenza data su $V\setminus\left\{0\right\}$ definita dall'azione del gruppo moltiplicativo $\kamp\setminus\left\{0\right\}$:
\begin{equation}
	\forall v,\ w\in V\setminus\left\{0\right\}\ v\sim w \iff \exists \lambda\in\kamp\setminus\left\{0\right\}\ \colon v=\lambda w
\end{equation}
Lo spazio proiettivo $\proj{V}$ si dice anche il \textbf{proiettivizzato}\seeonlyindex{proiettivizzato}{spazio!proiettivo} di $V$.
\end{define}
\begin{demonstration}
Dimostriamo che è una relazione di equivalenza:
\begin{itemize}
\item \textsc{Riflessiva}: $v\sim v?$ Basta porre $\lambda=1$, in quanto $x=1\cdot x$.
\item \textsc{Simmetrica}: Per ipotesi $y=\lambda x$, allora $x=\nicefrac{1}{\lambda} y$ ($nicefrac{1}{\lambda}\in \kamp\setminus\left\{0\right\}$).
\item \textsc{Transitiva}: Poichè $y=\lambda x,\ z=\mu y$, segue $z=\mu \left(\lambda x\right)=\left(\mu\lambda\right)x$ e $\mu\lambda\in \kamp\setminus\left\{0\right\}$.
\end{itemize}
\vspace{-3mm}
\end{demonstration}
\begin{define}\textsc{Dimensione di uno spazio proiettivo.}\\
La \textbf{dimensione}\index{dimensione di uno spazio proiettivo} di $\proj{V}$ è:
\begin{equation}
	\dim\proj{V}=\dim V-1
\end{equation}
Se $V=\left\{0\right\}$, allora $\proj{V}=\emptyset$ e si pone $\dim\emptyset\coloneqq -1$.
\end{define}
\begin{define}\textsc{Proiezione al quoziente e classe.}\\
Si denota con $\funz{\pi}{V\setminus\left\{0\right\}}{\proj{V}}$ la \textbf{proiezione al quoziente}\index{proiezione!al quoziente} e con $\left[v\right]\in\proj{V}$ la \textbf{classe}\index{classe!dello spazio proiettivo} di $v\in V\setminus\left\{0\right\}$.
\end{define}
\begin{observe}
	Si ha una corrispondenza biunivoca:
	\begin{equation}
		\begin{array}{c}
			\proj{V}\leftrightarrow\left\{\text{sottospazi vettoriali }1\text{-dimensionali di }V\right\}\\
			\left[v\right]\leftrightarrow\lin{v}
		\end{array}
	\end{equation}
In altre parole, possiamo pensare a $\proj{V}$ come l'insieme delle \textbf{rette vettoriali} in $V$.
\end{observe}
\begin{define}\textsc{Altre nomenclature proiettive.}\\
	\begin{itemize}
		\item Se $\dim V=1$, allora $\proj{V}$ è un \textbf{punto}\index{punto!proiettivo} e $\dim\proj{V}=0$.
		\item Se $\dim \proj{V}=1$, si parla di \textbf{retta proiettiva}\index{retta!proiettiva}.
		\item Se $\dim \proj{V}=2$, si parla di \textbf{piano proiettivo}\index{piano!proiettivo}.
		\item Se $\kamp=\realset$ o $\kamp=\complexset$, si parla rispettivamente di \textbf{spazio proiettivo reale}\index{spazio!proiettivo!reale} o di \textbf{spazio proiettivo complesso}\index{spazio!proiettivo!complesso}.
	\end{itemize}
\vspace{-3mm}
\end{define}
Gli esempi più frequenti di spazi proiettivi si ottengono considerando $V=\kamp^{n+1}$.
\begin{define}\textsc{Spazio proiettivo numerico.}\\
	Lo \textbf{spazio proiettivo numerico}\index{spazio!proiettivo!numerico} o \textbf{spazio proiettivo standard}\seeonlyindex{spazio!proiettivo!standard}{spazio!proiettivo!numerico} è lo spazio proiettivo su $\kamp^{n+1}$:
\begin{equation}
	\proj{\ }=\proj[n]{\kamp}=\proj{\kamp^{n+1}}
\end{equation}
Essi sono spazi di dimensione $\dim\proj[n]{\ }=n$.
\end{define}
\section{Sottospazi proiettivi}
Sia $W\subseteq V$ un sottospazio vettoriale. Allora $W\setminus\left\{0\right\}\subseteq V\setminus\left\{0\right\}$ è chiuso rispetto alla relazione di equivalenza $\sim$ precedentemente definita e $\proj{W}$ è naturalmente un sottoinsieme di $\proj{V}$.
\begin{define}\textsc{Sottospazio proiettivo.}\\
	Se $W\subseteq V$ è un sottospazio vettoriale, allora $\proj{W}$ è detto \textbf{sottospazio proiettivo}\index{sottospazio!proiettivo}:
	\begin{equation*}
		\begin{array}{rl}
			\proj{W}&=\pi\left(W\setminus\left\{0\right\}\right)=\left\{\left[w\right]\in\proj{V}\mid w\in W\right\}\\
			&=\left\{\text{sottospazi vettoriale }1\text{-dimensione di }V\text{ contenuti in }W\right\}
		\end{array}
	\end{equation*}
La dimensione del sottospazio proiettivo è $\dim\proj{W}=\dim W-1$.
\end{define}
	\begin{itemize}
	\item Se $W=\left\{0\right\}$, allora $\proj{W}=\emptyset$.
	\item Se $\dim W=1$, allora $\proj{W}$ è un punto, che indichiamo con $\left[w\right]$ per un $w\in W$.
	\item Se $\dim W=2$ ($\dim\proj{W}=1$), allora $\proj{W}$ è \textbf{retta proiettiva}\index{retta!proiettiva} in $\proj{V}$.
	\item Se $\dim W=3$ ($\dim\proj{W}=2$), allora $\proj{W}$ è \textbf{piano proiettivo}\index{piano!proiettivo} in $\proj{V}$.
	\item Se $\dim\proj{W}=\dim \proj{V}-1$, allora $\proj{W}$ è \textbf{iperpiano (proiettivo)}\index{iperpiano!proiettivo} in $\proj{V}$.
\end{itemize}
\begin{define}\textsc{Codimensione.}\\
Si definisce la \textbf{codimensione}\index{codimensione} di $\proj{W}$ sottospazio proiettivo come:
\begin{equation}
	\codim\proj{W}=\dim\proj{V}-\dim\proj{W}
\end{equation}
\vspace{-6mm}
\end{define}
\begin{example}
	Gli iperpiani sono sottospazi di codimensione $1$.
\end{example}
\section{Coordinate omogenee e sistemi di riferimento proiettivo}
Consideriamo $\proj[n]{\kamp}=\proj{\kamp^{n+1}}$. Se $v=\left(x_0,\ \ldots,\ x_n\right)\in\kamp^{n+1}\setminus\left\{0\right\}$, denotiamo la corrispettiva classe in questa forma:
\begin{equation}
\left[v\right]=\left(x_0\colon \ldots\colon x_n\right)\in\proj[n]{\kamp},\ x_i\in\kamp
\end{equation}
\begin{observes}~{}
	\begin{enumerate}
		\item Le $x_i$ non possono mai essere tutte nulle, dato che $v\neq 0$.
		\item Due classi sono uguali se le componenti sono tutte in proporzione per uno scalare $\lambda\in\kamp$.\footnote{La notazione con i $\colon$ viene utilizzata per mettere in evidenza che la relazione fra classi e vettori è di proporzione.}
		\begin{equation*}
			\begin{array}{ccc}
			\left(x_0:\ldots:x_n\right)=\left(y_0:\ldots:y_n\right)&\iff&\left(x_0,\ \ldots,\ x_n\right)\sim\left(y_0,\ \ldots,\ y_n\right)\\&\iff& \exists \lambda\in\kamp\setminus\left\{0\right\}\ \colon y_0=\lambda x_0,\ \ldots,\ y_n=\lambda x_n
			\end{array}
		\end{equation*}
	\end{enumerate}
\end{observes}
\begin{examples}
	In $\proj[2]{\realset}$:
	\begin{gather*}
		\left(1\colon1\colon2\right) = \left(-2\colon-2\colon-4\right)\\
		\left(1\colon0\colon2\right) = \left(\frac{1}{3}\colon0\colon\frac{1}{3}\right)
	\end{gather*}
\vspace{-6mm}
\end{examples}
\begin{define}\textsc{Coordinate omogenee e sistema di riferimento proiettivo.}\\
	Sia $\basis=\left\{e_0,\ \ldots,\ e_n\right\}$ una base di $V$, con $\dim V=n+1$. Se $v\in V\setminus\left\{0\right\}$, si ha:
	\begin{equation*}
		v=x_0e_0+\ldots+x_ee_n,\ \text{con}\ x_i\in\kamp
	\end{equation*}
Diciamo che $\left(x_0\colon\ldots\colon x_n\right)$ sono le \textbf{coordinate omogenee}\index{coordinate omogenee} di $\left[v\right]\in\proj{V}$ definite dalla base $\basis$ e scriviamo:
\begin{equation}
	\left[v\right]=\left(x_0\colon\ldots\colon x_n\right)
\end{equation}
La base $\basis$ definisce su $\proj{V}$ un \textbf{sistema di riferimento proiettivo}, cioè ad ogni punto vengono assegnate delle coordinate omogenee. 
\end{define}
\begin{observes}~{}
		\begin{itemize}
		\item Le coordinate omogenee non possono \textit{mai} essere \textit{tutte nulle}.
		\item Le coordinate omogenee sono definite \textit{solo a meno di multipli}.
		\item $\proj[n]{\kamp}$ ha delle coordinate omogenee ‘‘naturali'' date dalla base canonica di $\kamp^{n+1}$.
		\item Basi \textit{multiple} definiscono lo stesso riferimento proiettivo di $\proj{V}$, cioè le stesse coordinate omogenee.
	\end{itemize}
\vspace{-3mm}
\end{observes}
\begin{demonstration}
	Dimostriamo l'ultimo punto. Siano:
	\begin{equation*}
		\basis=\left\{e_0,\ \ldots,\ e_n\right\}\quad\basis'=\left\{\mu e_0,\ \ldots,\ \mu e_n\right\}
	\end{equation*}
Con $\mu\in\kamp\setminus\left\{0\right\}$. Si ha:
\begin{equation*}
	v=x_0e_0+\ldots+x_ne_n=\frac{x_0}{\mu}\left(\mu e_0\right)+\ldots + \frac{x_n}{\mu}\left(\mu e_n\right)
\end{equation*}
Passando allo spazio proiettivo:
\begin{equation*}
\underbrace{\left(x_0\colon\ldots\colon x_n\right)}_{\text{coordinate omogenee rispetto a }\basis}=\underbrace{\left(\frac{x_0}{\mu}\colon\ldots\colon \frac{x_n}{\mu}\right)}_{\text{coordinate omogenee rispetto a }\basis'}
\end{equation*}
\end{demonstration}
\begin{define}\textsc{Punti fondamentali e punto unità.}\\
	Data la base $\basis$, i punti:
	\begin{equation}
		\begin{array}{l}
			P_0=\left[e_0\right]=\left(1\colon0\colon\ldots\colon0\right)\\			P_1=\left[e_1\right]=\left(0\colon1\colon\ldots\colon0\right)\\
			\ldots\\
			P_n=\left[e_n\right]=\left(0\colon0\colon\ldots\colon1\right)\\
		\end{array}
	\end{equation}
Sono detti \textbf{punti fondamentali}\index{punto!fondamentale} o \textbf{punti coordinati}\seeonlyindex{punto!coordinato}{punto!fondamentale}, mentre il punto:
\begin{equation*}
	U=\left[e_0+e_1+\ldots+e_n\right]=\left(1\colon1\colon\ldots\colon1\right)
\end{equation*}
È detto \textbf{punto unità}\index{punto!unità}.
\end{define}
\subsubsection{Descrizione dei sottospazi proiettivi in coordinate}
Siano $\left(x_0\colon\ldots\colon x_n\right)$ coordinate omogenee su $\proj{V}$, indotte da una base $\basis$, e consideriamo l'equazione lineare omogenea:
\begin{equation*}
	\textcolor{green}{\circled{\ast}}\quad a_0x_0+a_1x_1+\ldots+a_nx_n=0
\end{equation*}
Con $a_i\in\kamp$ non tutti nulli.
\begin{itemize}
	\item In $V$ l'equazione omogenea rappresenta un \textit{iperpiano vettoriale} $H$.
	\item I punti $P=\left[v\right]\in\proj{V}$, le cui coordinate soddisfano l'equazioni, sono quelli tali per cui $v\in H$, cioè sono tutti e soli i punti dell'iperpiano proiettivo $\proj{H}\subseteq\proj{V}$. L'equazione lineare $\textcolor{green}{\circled{\ast}}$ è l'\textbf{equazione (cartesiana) dell'iperpiano proiettivo} $\proj{H}$.
\end{itemize}
\begin{define}\textsc{Iperpiano coordinato.}\\
	Gli iperpiani di equazione cartesiana $x_i=0$, cioè tutti i punti la cui $i$-esima coordinata omogenea è nulla, si dicono $i$\textbf{-esimi iperpiani coordinati}\index{iperpiano!proiettivo!coordinato}.
\end{define}
\begin{example}\item
	In $\proj[1]{\kamp}$, cioè una \textit{retta proiettiva} ($\dim \proj[1]{\kamp}=1$), i sottospazi proiettivi sono:
	\begin{itemize}
		\item $\emptyset$.
		\item I punti, che in questo caso sono gli iperpiani.
		\item Tutto $\proj[1]{\kamp}$.
	\end{itemize}
Il punto $\left(a\colon b\right)$ ha equazione cartesiana:
\begin{equation}
	bx_0-ax_1=0
\end{equation}
Ovvero l'equazione della retta in $\kamp^2$ generata dal vettore $\left(a,\ b\right)$, ottenuta pertanto dal determinante $\left| \begin{array}{cc}
	a & b \\
	x_0 & x_1
\end{array} \right|=0$.
\end{example}
\begin{attention}
	In $\proj{V}$ un sottospazio proiettivo di \textit{dimensione zero} è un singolo punto $\left[v\right]=\proj{\lin{v}}$.
\end{attention}
Più in generale: fissata una base $\basis$ di $V$, ogni \textit{sottospazio vettoriale} $W$ di $V$ può essere visto, in \textit{coordinate} rispetto alla base, come l'\textit{insieme delle soluzioni} di un \textit{sistema lineare omogeneo}.
\begin{equation*}
	Ax=O
\end{equation*}
Dove $A=\left(a_{ij}\right)$ è di dimensioni $t\times \left(n+1\right)$ a elementi in $\kamp$, mentre si ha:
\begin{gather}
	x=\left(\begin{array}{c}
		x_0 \\
		\vdots \\
		x_n
	\end{array}\right)\\
\textcolor{green}{\circled{\ast}}\begin{cases}
	a_{1,0}x_0+\ldots+a_{1,n}x_n=0\\
	\vdots\\
	a_{t,0}x_0+\ldots+a_{t,n}x_n=0\\
\end{cases}
\end{gather}
Il sistema $\textcolor{green}{\circled{\ast}}$ dà delle \textit{equazioni cartesiane} per il sottospazio proiettivo $\proj{W}$ nelle coordinate omogenee $\left(x_0\colon\ldots\colon x_n\right)$.\\
Posto dunque $t$ come il numero delle \textit{equazioni}, notiamo che:
\begin{equation*}
	\begin{array}{cc}
		\dim W=n+1-\rk A \\
\begin{array}{lll}
	\codim W &=&\rk A \\
	\shortparallel &  \\
	\dim V-\dim W &=&\dim\proj{V}-\dim\proj{W}=\codim\proj{W}
\end{array}\\
	\implies t\geq \rk\left(A\right)=\codim\proj{W}
	\end{array}
\end{equation*}
\textit{Scartando} delle equazioni possiamo sempre ricondurci ad un sistema in cui:
\begin{equation}
	t=\rk A=\codim\proj{W}
\end{equation}
\vspace{-6mm}
\begin{intuit}
	Per facilitare la visualizzazione degli spazi proiettivi possiamo pensare allo spazio $\kamp^{n+1}$ come lo \textbf{spazio affine} $\aff{\kamp^{n+1}}$ in cui sia fissato un punto $O$ come origine: in questo modo, le classi di $\proj[n]{\kamp}$ corrispondono alle \textit{rette affini passanti per} $O$ (identificate con le rette vettoriali di $\kamp^{n+1}$):
	\begin{equation*}
		\left(x_0\colon\ldots\colon x_n\right)\leftrightarrow\begin{array}{c}
			\text{retta affine di }\aff{\kamp^{n+1}}\text{ formata}\\
			 \text{dai punti }\left(tx_0,\ \ldots,\ tx_n\right)\text{ al variare di t}\in\realset
		\end{array}
	\end{equation*}
	Approfondiremo formalmente la relazione tra gli spazi affini e gli spazi proiettivi più avanti, a pag. \ref{spaziaffini}.
\end{intuit}
\begin{examples}~{}
	\begin{itemize}
		\item Il piano proiettivo $\proj[2]{\kamp}$ ha, come sottospazi \textit{non banali}, i punti e le rette.
		\begin{itemize}
			\item Una \textit{retta proiettiva} viene da un \textit{piano}, che nel riferimento \textit{affine} possiamo prendere passante per l'origine: $a_0x_0+a_1x_0+a_2x_2=0$.
			\item Un \textit{punto} servono due equazioni, in sostanza vedendolo come \textit{intersezione di due rette proiettive}; ad esempio, $\left(1\colon0\colon0\right)$ ha equazioni $x_1=x_2=0$, mentre $\left(1\colon2\colon3\right)$ ha equazioni $\begin{cases}
				x_1=2x_0\\
				x_2=3x_0
			\end{cases}$
		\end{itemize}
	\item Nel piano proiettivo reale $\proj[2]{\realset}$, le \textit{rette proiettive} vengono da \textit{piani vettoriali}, mentre nel modello affine di $\aff{\realset^3}$ essi sono passanti per l'\textit{origine}; utilizzando la \textit{sfera unitaria} ai quali identifichiamo i punti antipodali in una relazione di equivalenza, la retta proiettiva si visualizza facilmente come l'\textit{intersezione} della sfera in un \textit{cerchio massimo}.\\
	In questo modo, considerando la \textit{semisfera superiore}, la \textbf{proiezione} dell'intersezione su di essa sul disco unitario $D$ è la rappresentazione della retta proiettiva sul \textit{modello piano} ben noto. Dunque, guardando le rette proiettive nel \textit{modello piano}, se ne hanno di \textit{tre tipi}:\\
	\begin{minipage}{0.77\textwidth}
			\begin{enumerate}[series=proj]
			\item La \textit{retta} con equazione $z=0$, ovvero al piano $xy$ in $\realset^3$: sul modello piano corrisponde al \textbf{bordo del disco} $D$ (cioè $S^1$).
		\end{enumerate}
		\end{minipage}
	\begin{minipage}{0.32\textwidth}
		\includegraphics[trim=0cm 0cm 0cm 0cm,clip,scale=0.50]{images/projectivelinesdisk1.pdf}
	\end{minipage}\\
	\begin{minipage}{0.77\textwidth}
	\begin{enumerate}[resume=proj]
			\item Le \textit{rette} con equazione $ax+by=0$, ovvero ai \textit{piani perpendicolari} in $\realset^3$ passanti per le rette con quell'equazione $ax+by=0$:  sul modello piano corrisponde a \textbf{diametri colleganti due punti} sul bordo.
		\end{enumerate}
	\end{minipage}
\begin{minipage}{0.32\textwidth}
\includegraphics[trim=0cm 0cm 0cm 0cm,clip,scale=0.50]{images/projectivelinesdisk2.pdf}
\end{minipage}\\
\begin{minipage}{0.77\textwidth}
	\begin{enumerate}[resume=proj]
			\item Nel caso generale $ax+by+cz=0$, proiettando l'\textit{arco di cerchio massimo} viene un \textbf{arco di ellisse} in $D$.
		\end{enumerate}
	\end{minipage}
	\begin{minipage}{0.32\textwidth}
		\includegraphics[trim=0cm 0cm 0cm 0cm,clip,scale=0.50]{images/projectivelinesdisk3.pdf}
	\end{minipage}
	\end{itemize}
\vspace{-3mm}
\end{examples}
\section{Operazioni con i sottospazi}
Se $W_1,\ W_2\subseteq V$ sono sottospazi vettoriali, allora $W_1\cap W_2$ è un sottospazio vettoriale e si ha che l'\textbf{intersezione}\index{sottospazio!intersezione} dei corrispettivi spazi proiettivi è ancora un sottospazio proiettivo.
\begin{equation*}
	\proj{W_1\cap W_2}=\proj{W_1}\cap\proj{W_2}
\end{equation*}
\vspace{-6mm}
\begin{observe}
	Si ha:
	\begin{equation*}
		\proj{W_1}\cap\proj{W_2}=\emptyset\iff W_1\cap W_2=\left\{0\right\}
	\end{equation*}
In tal caso diciamo che i due sottospazi sono \textbf{sghembi}\index{sottospazio!proiettivo!sghembo} o \textbf{disgiunti}\seeonlyindex{sottospazio!proiettivo!disgiunto}{sottospazio!proiettivo!sghembo}.
\end{observe}
Come per i sottospazi vettoriali, in generale l'\textbf{unione} di due sottospazi proiettivi \textit{non} è un sottospazio proiettivo.
\begin{define}\textsc{Sottospazio generato da un sottoinsieme.}\\
	Sia $S\subseteq \proj{V}$ un sottoinsieme non vuoto. Il \textbf{sottospazio generato}\index{sottospazio!proiettivo!generato} da $S$, denotato con $\left<S\right>$, è l'intersezione in $\proj{V}$ di tutti i sottospazi proiettivi contenenti $S$, ed è il più piccolo sottospazio contenente $S$.
\end{define}
\begin{itemize}
	\item $\left<S\right>=S\iff S$ è un sottospazio proiettivo.
	\item Se $S=\left\{P_1,\ \ldots,\ P_m\right\}$ è finito, scriviamo $\left<P_1,\ \ldots,\ P_m\right>$ per il sottospazio generato da $P_1,\ \ldots,\ P_m$.
\end{itemize}
\begin{define}\textsc{Sottospazio somma.}\\
	Dati due sottospazi proiettivi $T_1,\ T_2\subseteq \proj{V}$, cioè:
	\begin{equation*}
		T_i=\proj{W_i}\quad W_i\subseteq V,\ i=1,\ 2
	\end{equation*}
	Allora il sottospazio generato da $T_1\cup T_2$ è denotato con $T_1+T_2=\left<T_1,\ T_2\right>$ e si chiama \textbf{sottospazio somma}\index{sottospazio!somma}. In particolare, si ha:
	\begin{equation}
		\left<T_1,\ T_2\right>=\proj{W_1+W_2}
	\end{equation}
\vspace{-6mm}
\end{define}
\begin{demonstration}~{}\\
	$\includedx$ $\proj{W_1+W_2}$ è un sottospazio proiettivo che contiene, in quanto $W_1\subseteq W_1+W_2,\ W_2\subseteq W_1+W_2$ vettorialmente, sia $T_1=\proj{W_1}$ sia $T_2=\proj{W_2}$. In particolare, contiene la loro unione\footnote{Ricordiamo che non è essa un sottospazio, ma un sottoinsieme.}, dunque $\left<T_1,\ T_2\right>\left<T_1\cup T_2\right>\subseteq \proj{W_1+W_2}$.\\
	$\includesx$ Abbiamo che $T_i\subseteq \left<T_1,\ T_2\right>=\proj{U}$, con $U$ un sottospazio vettoriale di $V$. In particolare, si ha che $W_1,\ W_2\subseteq U$, da cui $W_1+W_2\subseteq U$. Passando allo spazio proiettivo:
	\begin{equation*}
		\left<T_1,\ T_2\right>=\proj{U}\subseteq \proj{W_1+W_2}
	\end{equation*}
\vspace{-6mm}
\end{demonstration}
\begin{proposition}\textsc{Formula di Grassmann proiettiva}\index{formula!di Grassmann proiettiva}\\
	Siano $T_1,\ T_2$ sottospazi proiettivi di $\proj{V}$. Si ha:
	\begin{equation}
				\dim\left<T_1,\ R_2\right>+\dim\left(T_1\cap T_2\right)=\dim T_1+\dim T_2
	\end{equation}
\vspace{-6mm}
\end{proposition}
\begin{demonstration}
	Posti $T_i=\proj{W_i}$, con $W_i\subseteq V$ sottospazi vettoriali. Dalla \textit{formula di Grassmann vettoriale}:
	\begin{equation*}
		\dim\left(W_1+W_2\right)+\dim\left(W_1\cap W_2\right)=\dim W_1+\dim W_2
	\end{equation*}
Sottraendo $1$ a tutte le dimensioni, otteniamo le dimensioni dei corrispettivi spazi proiettivi e dunque la formula proiettiva.
\end{demonstration}
\begin{corollary}
	Siano $T_1,\ T_2$ sottospazi proiettivi di $\proj{V}$ con $\dim\proj{P}=n$. Allora:
	\begin{equation}
		\dim \left(T_1\cap T_2\right)\geq \dim T_1+\dim T_2-n
	\end{equation}
In particolare $T_1\cap T_2\neq \emptyset$ se $\dim T_1+\dim T_2\geq n$.
\end{corollary}
\begin{demonstration}
	\begin{equation*}
		\dim\left(T_1\cap T_2\right)=\dim T_1+\dim T_2-\dim\left<T_1,\ T_2\right>\geq \dim T_1+\dim T_2-n
	\end{equation*}
Chiaramente, se $\dim T_1+\dim T_2\geq n$, allora $\dim \left(T_1\cap T_2\right)\geq 0$ e dunque $T_1\cap T_2\neq \emptyset$.
\end{demonstration}
\begin{example}
	Nel piano proiettivo, due rette sono \textit{sempre incidenti}. Infatti, le rette hanno dimensione $1$, mentre $\dim\proj[2]{\kamp}=2$, dunque vale $1+1\leq 2$, pertanto due rette si incontrano sempre.
\end{example}
\begin{observe}
	Se consideriamo l'insieme \textit{finito di punti}, possiamo considerare lo spazio $S$ \textit{generato} da $P_1,\ \ldots,\ P_m$, cioè $S=\left<P_1,\ \ldots,\ P_m\right>$; inoltre, si ha:
	\begin{equation*}
		\dim S\leq m-1
	\end{equation*}
	Infatti, se $P_i=\left[v_i\right]$ con $v_i\in V$, allora:
	\begin{equation*}
		S=\underbrace{\proj{\lin{v_1,\ \ldots,\ v_m}}}_{\dim \mathcal{L} \leq m}
	\end{equation*}
\vspace{-6mm}
\end{observe}
\section{Punti linearmente indipendenti e in posizione generale}
\begin{define}\textsc{Punti linearmente indipendenti.}\\
	Siano $P_1,\ \ldots, P_m\in\proj{V}$. Diciamo che i punti $P_1,\ \ldots,\ P_m$ sono \textbf{linearmente indipendenti}\index{linearmente indipendenti} se, scelti $v_1,\ \ldots,\ v_m\in V\setminus\left\{0\right\}$ tali che $P_i=\left[v_i\right]\ \forall i$, i vettori $v_1,\ \ldots,\ v_m$ sono \textit{linearmente indipendenti} in $V$.\\
	Se così non è, diciamo che $P_1,\ \ldots, P_m$ sono linearmente dipendenti.
\end{define}
\begin{observes}~{}
	\begin{itemize}
		\item La definizione è \textit{ben posta}. Dati $\lambda_1,\ \ldots,\ \lambda_m\in\kamp\setminus\left\{0\right\}$, si ha che:
		\begin{equation*}
			v_1,\ \ldots,\ v_m\text{ sono indipendenti}\iff\lambda_1 v_1,\ \ldots,\ \lambda_m v_m\text{ sono indipendenti.}
		\end{equation*}
		\item Se $\dim \proj{V}=n$, $\proj{V}$ contiene al più $n+1$ punti indipendenti.
		\item $P_1,\ \ldots,\ P_m$ sono indipendenti se e solo se $\dim\left<P_1,\ \ldots,\ P_m\right>=m-1$.
	\end{itemize}
\end{observes}
\begin{examples}~{}
	\begin{itemize}
		\item \textit{Due} punti $P,\ Q$ sono indipendenti se e solo se $P\neq Q$. Infatti, se $P=\left[v\right]$ e $Q=\left[w\right]$, allora:
		\begin{equation*}
			P\text{ e }Q\text{ sono indipendenti}\iff v\text{ e }w\text{ sono indipendenti}\iff v\nsim w\iff P\neq Q
		\end{equation*}
		In tal caso $\left<P,\ Q\right>$ è l'unico \textit{retta} contenente $P$ e $Q$, che indicheremo anche con $\overline{PQ}$.
		\item \textit{Tre} punti $P_1,\ P_2,\ P_3$ sono indipendenti se e solo se sono \textit{distinti} e \textit{non} sono \textit{allineati}, cioè appartenenti alla stessa retta. In tal caso $\left<P_1,\ P_2,\ P_3\right>$ è l'unico \textit{piano} contenente i tre punti.
	\end{itemize}
\vspace{-3mm}
\end{examples}
\begin{define}\textsc{Punti in posizione generali.}\\
	Dati dei punti $P_1,\ \ldots,\ P_m\in\proj{V}$, diciamo che sono \textbf{in posizione generale}\index{in posizione generale}\index{posizione generale} se vale una delle due condizioni seguenti:
	\begin{itemize}
		\item $m\leq n+1$ e i punti sono \textit{linearmente indipendenti}.
		\item $m>n+1$ e ogni scelta di $n+1$ punti tra loro sono linearmente indipendenti.
	\end{itemize}
\vspace{-3mm}
\end{define}
\begin{example}~{}
		\begin{itemize}
		\item Se $n=1$, cioè $\proj{V}$ è una \textit{retta proiettiva}, allora $P_1,\ \ldots,\ P_m$ sono in posizione generale se e solo se $P_1,\ \ldots,\ P_m$ sono \textit{tutti distinti}.
		\item Se $n=2$, cioè $\proj{V}$ è una \textit{piano proiettivo}, allora $P_1,\ \ldots,\ P_m$ sono in posizione generale se e solo se $P_1,\ \ldots,\ P_m$ sono a $3$ a $3$ \textit{non} allineati.
	\end{itemize}
\vspace{-3mm}
\end{example}
\subsection{Impratichiamoci! Punti linearmente indipendenti}
\begin{exercise}\textsc{F.F.P., 2.1.}\\
	Si mostri che i punti del piano proiettivo reale:
	\begin{equation*}
		\left(\frac{1}{2}\colon 1 \colon 1\right)\quad \left(1\colon \frac{1}{3} \colon \frac{4}{3}\right)\quad \left(2\colon -1 \colon 2\right)
	\end{equation*}
Sono allineati, e si determini un'equazione della retta che li contiene.
\end{exercise}
\begin{solution}
	Per verificare che i $3$ punti sono allineati, dobbiamo verificare che i corrispondenti vettori di $\realset^3$ sono dipendenti. Riscriviamo i seguenti punti per facilitarci i calcoli:
\begin{equation*}
	\left(\frac{1}{2}\colon 1 \colon 1\right)=\left(1\colon 2 \colon 2\right)\quad \left(1\colon \frac{1}{3} \colon \frac{4}{3}\right)=\left(3\colon 1 \colon 4\right)
\end{equation*}
Verifichiamolo la dipendenza con il determinante.
	\begin{equation*}
		\det\left|\begin{array}{ccc}
			1 & 2 & 2\\
			3 & 1 & 4\\
			2 & -1 & 2
		\end{array}\right|=0
	\end{equation*}
L'equazione della retta è data dall'equazione del piano vettoriale in $\realset^3$ generate da $2$ dei $3$ vettori:
\begin{equation*}
	0=\left|\begin{array}{ccc}
		x_0 & x_1 & x_2\\
		1 & 2 & 2\\
		3 & 1 & 4
	\end{array}\right|=x_0\left(8-2\right)-x_1\left(4-6\right)+x_2\left(1-6\right)=6x_0+2x_1-5x_2
\end{equation*}
Verifichiamo che contenga anche il terzo:
\begin{equation*}
	6\cdot 2 + 2\cdot \left(-1\right) -5\cdot 2=0
\end{equation*}
\vspace{-6mm}
\end{solution}
\section{Rappresentazione parametrica di un sottospazio proiettivo}
Sia $S\subseteq\proj{V}$ un sottospazio proiettivo di dimensione $m$. Allora esistono sempre $m+1$ punti $P_0,\ \ldots,\ P_m\in S$ linearmente indipendenti che generano $S$. Infatti, se $S=\proj{W}$ con $W\subseteq V$ sottospazio vettoriale di dimensione $m+1$, possiamo scegliere una base $\left\{w_0,\ \ldots,\ w_m\right\}$ di $W$ tale per cui:
\begin{equation*}
	P_i=\left[w_i\right]\in S
\end{equation*}
Sono linearmente indipendenti (perché lo sono i vettori della base) e generano $S$.\\
Allora, tutti e soli i punti di $S$ sono della forma:
\begin{equation*}
	\left[\lambda_0 w_0+\ldots+\lambda_m w_m\right]\quad \lambda_0,\ \ldots,\ \lambda_m\in\kamp
\end{equation*}
Supponiamo ora di aver fissato una base $\left\{e_0,\ \ldots,\ e_n\right\}$ di $V$ e quindi di aver considerato il corrispondente \textit{riferimento proiettivo}. In coordinate vettoriali di $V$, un punto di $W$ è $x=\left(x_0,\ \ldots,\ x_n\right)$ se e solo se:
\begin{equation*}
	x=x_0e_0+\ldots+x_ne_n=\lambda_0 w_0+\ldots+\lambda_m w_m
\end{equation*}
Il punto $P_i$ in $V$ avrà coordinate $\left(P_{0,i},\ \ldots,\ P_{n,i}\right)\ \forall i=1,\ \ldots,\ m$, dunque il generico vettore $x$ di $W$ è espresso da:
\begin{equation}
	\begin{cases}
		x_0=\lambda_0 P_{0,0}+\lambda_1P_{0,1}+\ldots+\lambda_mP_{0,m}\\
		\vdots\\
		x_n=\lambda_0 P_{n,0}+\lambda_1P_{n,1}+\ldots+\lambda_mP_{n,m}
	\end{cases}
\end{equation}
Anche i punti di $S$ sono date da queste coordinate, dunque questa viene definita la \textbf{rappresentazione parametrica}\index{rappresentazione!parametrica} del sottospazio $S$, con $\left(\lambda_0\colon\ldots\colon\lambda_m\right)$ le coordinate omogenee di $\proj{W}$ date dalla base $\left\{w_0,\ \ldots,\ w_m\right\}$.
\begin{example}
	In $\proj[3]{\realset}$ consideriamo i punti:
	\begin{equation*}
		A=\left(1\colon 0\colon -1\colon 4\right)\quad B=\left(2\colon 3\colon 0\colon 5\right)
	\end{equation*}
Allora, la rappresentazione parametrica del sottospazio $S$ con $\left(\lambda\colon \mu\right)$ è:
\begin{equation*}
	\begin{cases}
		x_0=\lambda+2\mu\\
		x_1=3\mu\\
		x_2=-\lambda\\
		x_3=4\lambda-5\mu
	\end{cases}
\end{equation*}
\vspace{-6mm}
\end{example}
\subsection{Coordinate proiettive e punti in posizione generale}
\begin{observe}
	Sia $\proj{V}$ con un riferimento proiettivo fissato. Consideriamo i punti fondamentali $P_0,\ \ldots,\ P_n$ e il punto unità $U$.
	\begin{itemize}
		\item $P_0,\ \ldots,\ P_n,\ U$ sono $n+2$ punti.
		\item $P_0,\ \ldots,\ P_n,\ U$ sono in posizione generale: essendo $P_i=\left[e_i\right]$ con $e_0,\ \ldots,\ e_n$ base di $V$, allora $P_0,\ \ldots,\ P_n$ sono indipendenti. Se sostituiamo l'$i$-esimo punto con $U=\left[e_1+\ldots+e_n\right]$, allora:
		\begin{equation*}
			P_0,\ \ldots,\ \check{P}_i,\ \ldots,\ U
		\end{equation*}
	Sono indipendenti $\forall i=0,\ \ldots,\ n$.\footnote{Indichiamo con $\check{P}_{i}$ il punto che sostituiamo.}
	\end{itemize}
\vspace{-6mm}
\end{observe}
\begin{observe}
	Sia $\basis=\left\{e_0,\ \ldots,\ e_n\right\}$ una base che induce un \textit{riferimento proiettivo} su $\proj{V}$.\\
	Per ogni $i$ sia $\lambda_i\in\kamp\setminus\left\{0\right\}$ e consideriamo $v_i=\lambda_i e_i$. Allora $\basis'=\left\{v_0,\ \ldots,\ v_n\right\}$ è ancora una base e i \textit{punti fondamentali} del riferimento indotto da $\basis'$ sono \textit{gli stessi} del riferimento indotto da $\basis$. Infatti:
	\begin{equation*}
		\left[e_i\right]=\left[v_i\right]=P_i
	\end{equation*}
	Però i due riferimenti sono \textbf{diversi}; dato $v$ espresso nella base $\basis$:
	\begin{equation*}
		v=x_0e_0+\ldots+x_ne_n
	\end{equation*}
	La sua classe in $\proj{V}$, rispetto a $\basis$, è:
	\begin{equation*}
		\left[v\right]=\left(x_0\colon\ldots\colon x_n\right)
	\end{equation*}
	Possiamo partire dall'espressione di $v$ nella base $\basis$ a quella nella base $\basis'$, moltiplicando e dividendo ogni $e_i$ per il corrispettivo $\lambda_i$:
	\begin{equation*}
		v=\frac{x_0}{\lambda_0}\left(\lambda_0e_0\right)+\ldots+\frac{x_n}{\lambda_n}\left(\lambda_ne_n\right)=\frac{x_0}{\lambda_0}v_0+\ldots+\frac{x_n}{\lambda_n}v_n
	\end{equation*}
	Passiamo dunque alla base $\basis'$ alla classe in $\proj{\kamp}$:
	\begin{equation*}
		\left[v\right]=\left(\frac{x_0}{\lambda_0}\colon\ldots\colon\right)
	\end{equation*}
	Notiamo che effettivamente il punto $\left[v\right]$ non cambia, ma i riferimenti \textit{non} sono multipli e quindi sono diversi!
	\begin{itemize}
		\item \textit{Conoscere} i punti fondamentali \textit{non basta} a determinare la base $\basis$.
		\item Riferimenti proiettivi \textit{diversi} possono avere gli \textit{stessi} punti fondamentali.
	\end{itemize}
\end{observe}
\begin{observe}\label{puntigeneraleindipendentiosserva}
	Supponiamo di avere $n+2$ punti $P_0,\ \ldots,\ P_{n+1}$ in $\proj{V}$, cioè $\forall i=0,\ \ldots,\ n+1\ \exists v_i\in V\ \colon P_i=\left[v_i\right]$. Allora:
	\begin{gather*}
		P_0,\ \ldots,\ P_{n+1}\text{ sono in posizione generale}\iff v_0,\ \ldots,\ v_n\text{ sono indipendenti e}\\
		v_{n+1}=a_0v_0+\ldots+a_nv_n\text{ con }a_i\neq 0\ \forall i=0,\ \ldots,\ n 
	\end{gather*}
Infatti, se $v_0,\ \ldots,\ v_n$ è una base (in quanto sono indipendenti), $v_0,\ \ldots,\ \check{v_i},\ \ldots,\ v_n,\ v_{n+1}$ sono indipendenti se e solo se $a_i\neq 0$.
\end{observe}
\begin{theorema}
	Sia $\proj{V}$ di dimensione $n$. Dati $n+2$ punti $P_0,\ \ldots,\ P_{n+1}$ in \textit{posizione generale}, esiste una base $\basis=\left\{e_0,\ \ldots,\ e_n\right\}$ di $V$ tale che:
	\begin{equation}
		P_0=\left[e_0\right],\ \ldots,\ P_n=\left[e_n\right],\ P_{n+1}=\left[e_0+\ldots+e_n\right]
	\end{equation}
Inoltre, se $\basis'=\left\{f_0,\ \ldots,\ f_n\right\}$ è un'altra base di $V$ che soddisfa la condizione sopra, allora $\basis$ è proporzionale a $\basis$, cioè $\exists\lambda\in\kamp\setminus\left\{0\right\}\ \colon f_i=\lambda e_i\ \forall i=0,\ \ldots,\ n$.
\end{theorema}
\begin{demonstration}
	Sia $P_i=\left[v_i\right]$ al variare di $i=0,\ \ldots,\ n+1$. I punti $P_0,\ \ldots,\ P_n$ sono indipendenti\footnote{Perchè se $N+2$ punti sono in posizione generale, presi $n+1$ punti fra di loro sono indipendenti.}, dunque per definizione $v_0,\ \ldots,\ v_n$ è una base di $V$. Definiamo:
	\begin{equation*}
		v_{n+1}=\lambda_0v_0+\ldots+\lambda_nv_n\quad \lambda_i\in\kamp
	\end{equation*}
	Ma allora, per l'osservazione precedente, $\lambda_i\neq 0\ \forall i$ perché i punti sono in posizione generale.\\
	Consideriamo $_0=\lambda_0v_0,\ e_1=\lambda_1v_1,\ \ldots,\ e_n=\lambda_nv_n$. Si ha che $\basis=\left\{e_0,\ \ldots,\ e_n\right\}$ è una base di $V$ perché $\lambda_i\neq 0$\ $\forall i$. Segue che:
	\begin{gather*}
		\left[e_i\right]=\left[v_i\right]=P_i\ \forall i=0,\ \ldots,\ n\\
		\left[e_0+\ldots+e_n\right]=\left[\lambda_0v_0+\ldots+\lambda_nv_n\right]=\left[v_{n+1}\right]=P_{n+1}
	\end{gather*}
Adesso, sia $\basis')\left\{f_0,\ \ldots,\ f_n\right\}$ come da ipotesi. Allora $\left[f_i\right]=P_i=\left[e_n\right]\ \forall i=0,\ \ldots,\ n$, cioè $\exists \mu_i\in\kamp\setminus\left\{0\right\}\ \colon f_i=\mu_i e_i\ \forall i=0,\ \ldots, n$. Inoltre, soddisfa anche $\left[f_0+\ldots+f_n\right]=P_{n+1}$, pertanto:
\begin{equation*}
	\left[f_0+\ldots+f_n\right]=\left[e_0+\ldots+e_n\right]
\end{equation*}
In altre parole, $\exists \mu\in\kamp\setminus\left\{0\right\}$ tale che:
\begin{equation*}
	\begin{array}{ccc}
		f_0+\ldots+f_n&=&\mu\left(e_0+\ldots+e_n\right)\\
		\shortparallel&&\\
		\mu_0e_0+\ldots+\mu_ne_n&&
	\end{array}
\end{equation*}
$e_0,\ \ldots,\ e_n$ è una base: per l'unicità della scrittura deve essere $\mu=\mu_0=\ldots=\mu_n$, cioè $f_i=\mu e_i\ \forall i=0,\ \ldots,\ n$.
\end{demonstration}
\section{Trasformazioni proiettive}
\begin{define}\textsc{Trasformazione proiettiva e proiettività.}\\
	Un'applicazione $\funz{f}{\proj{V}}{\proj{V'}}$ tra spazi proiettivi si dice \textbf{trasformazione proiettiva}\index{trasformazione proiettiva} o \textbf{isomorfismo proiettivo}\seeonlyindex{isomorfismo!proiettivo}{trasformazione proiettiva} se $\exists \funz{\phi}{V}{V'}$ isomorfismo che induce un altro isomorfismo lineare:
	\begin{equation}
		\begin{tikzcd}
			\widetilde{\phi}\ \colon\\
		\end{tikzcd}
			\funztot{\ }{\proj{V}}{\proj{V'}}{\left[v\right]}{\left[\phi\left(v\right)\right]}
	\end{equation}
	Tale per cui $f=\widetilde{\phi}$.\\
	Se $V=V'$, diciamo che $f$ è una \textbf{proiettività}\index{proiettività} di $\proj{V}$.
\end{define}
\begin{demonstration}~{}
	\begin{itemize}
		\item $\widetilde{\phi}$ \textbf{è ben definita}:
		\begin{enumerate}
			\item $\phi\left(v\right)\neq 0$ perché $v\neq 0$ e $\phi$ è iniettiva, pertanto $\ker\phi=\left\{0\right\}$ e dunque l'unico vettore mappato a $0$ tramite $\phi$ è solo $0$.
			\item Se $\left[v\right]=\left[w\right]$, allora $w\sim v$, cioè $w=\lambda v$ con $\lambda\in\kamp\setminus\left\{0\right\}$; segue che per linearità di $\phi$ $\phi\left(w\right)=\lambda \left(v\right)\implies \left[\phi\left(w\right)\right]=\left[\phi\left(v\right)\right]$
		\end{enumerate}
		\item $\widetilde{\phi}$ \textbf{è iniettiva}: se $\widetilde{\phi}\left(\left[v\right]\right)=\widetilde{\phi}\left(\left[w\right]\right)$, allora
		\begin{equation*}
			\left[\phi\left(v\right)\right]=\left[\phi\left(w\right)\right]\implies\exists\lambda\in\kamp\setminus\left\{0\right\}\ \colon\phi\left(w\right)=\lambda\phi\left(v\right)=\phi\left(\lambda v\right)
		\end{equation*}
	Poichè $\phi$ è iniettiva, segue che $w=\lambda v$ e dunque $\left[v\right]=\left[w\right]$.
		\item $\widetilde{\phi}$ \textbf{è suriettiva}: infatti, se $\left[w\right]\in\proj{V'}$, essendo $\phi$ suriettiva esiste un vettore $v$ tale che $w=\phi\left(v\right)$. Segue che $\left[w\right]=\left[\phi\left(v\right)\right]=\phi\left(\left[v\right]\right)$.
	\end{itemize}
\vspace{-3mm}
\end{demonstration}
Dato che spazi \textit{vettoriali} della \textit{stessa dimensione} sono sempre \textit{isomorfi}, due spazi \textit{proiettivi} della \textit{stessa dimensione} sono sempre \textit{isomorfi} e $\proj{V}$ è sempre isomorfo a $\proj[n]{\kamp}$, con $\dim V=n+1$.
\begin{lemming}
	Siano $\funz{\phi,\ \psi}{V}{V'}$ isomorfismi. Allora:
	\begin{equation}
		\widetilde{\phi}=\widetilde{\psi}\iff\exists\lambda\in\kamp\setminus\left\{0\right\}\ \colon\psi=\lambda\phi
	\end{equation}
\vspace{-6mm}
\end{lemming}
\begin{demonstration}~{}\\
$\impliessx$ Se $v\in\kamp\setminus\left\{0\right\}$, allora $\psi\left(v\right)=\lambda\psi\left(v\right)$. Segue:
\begin{equation*}
	\implies \widetilde{\psi}\left(\left[v\right]\right)=\left[\psi\left(v\right)\right]=\left[\psi\left(v\right)\right]=\widetilde{\psi}\left(\left[v\right]\right)
\end{equation*}
$\impliesdx$ Sia $h\coloneqq \funz{\phi^{-1}\circ \psi}{V}{V}$ automorfismo. Vogliamo mostrare che $h=\lambda Id_V$ con $\lambda\in\kamp\setminus\left\{0\right\}$. Se $v\in V\setminus\left\{0\right\}$, abbiamo:
\begin{equation*}
	\begin{array}{c}
\begin{array}{ccc}
	\widetilde{\phi}\left(\left[v\right]\right)&=&\psi\left(\left[v\right]\right)\\
	\shortparallel&&\shortparallel\\
	\left[\phi\left(v\right)\right]&&\left[\psi\left(v\right)\right]
\end{array}
\implies \lambda_v\in\kamp\setminus\left\{0\right\}\ \colon \phi\left(v\right)=\lambda_v\psi\left(v\right)
\implies h\left(v\right)=\psi^{-1}\left(\phi\left(v\right)\right)=\lambda_v v
\end{array}
\end{equation*}
Segue che $v$ è autovettore di $h\ \forall v\in V\setminus\left\{0\right\}$, in particolare ogni vettore non nullo è autovettore di $h$. Segue che $h$ è diagonalizzabile e ha un unico autovalore $\lambda$. Infatti, presi $\lambda_1$ e $\lambda_2$, si avrebbero i seguenti autovalori indipendenti:
\begin{equation*}
	v_1\in V_{\lambda_1}\setminus\left\{0\right\}\qquad v_2\in V_{\lambda_2}\setminus\left\{0\right\}
\end{equation*}
E considerato che:
\begin{equation*}
	\begin{array}{l}
		h\left(v_1\right)=\lambda_1 v_1\\
		h\left(v_2\right)=\lambda_2 v_2\\
		h\left(v_1+v_2\right)=\lambda\left(v_1+v_2\right)\\
		h\left(v_1+v_2\right)=h\left(v_1\right)+h\left(v_2\right)\\
	\end{array}
	\implies \lambda\left(v_1+v_2\right)=\lambda_1 v_1+\lambda_2 v_2
\end{equation*}
Da cui segue, in quanto $v_1,\ v_2,\ v_1+v_2\neq 0$, che $\lambda=\lambda_1=\lambda_2$ e quindi è unico.\\
Allora, $h=\lambda Id_{V}$ e pertanto $\phi=\lambda \psi$.
\end{demonstration}
\subsection{Gruppo lineare proiettivo}
\begin{observe}
	Consideriamo $\proj{V}$ e l'insieme delle proiettività $\funz{\ }{\proj{V}}{\proj{V}}$.
	\begin{itemize}
		\item La \textit{composizione} di proiettività è una \textit{proiettività} (banalmente \textit{indotta} dalla composizione delle applicazioni lineari).
		\item Poichè $Id_{\proj{V}}=\widetilde{Id_V}\implies$ L'identità $Id_{\proj{V}}$ è una \textit{proiettività}.
		\item Se $\widetilde{\phi}\! \funz{\! }{\proj{V}}{\proj{V}}$, allora $\widetilde{\phi^{-1}}=\funz{f^{-1}}{\proj{V}}{\proj{V}}$. In altre parole, l'\textit{inversa} di una proiettività è ancora una proiettività.
	\end{itemize}
L'insieme delle proiettività risulta un \textbf{gruppo} rispetto alla \textit{composizione}.
\end{observe}
\begin{define}\textsc{Gruppo lineare proiettivo.}\\
Il \textbf{gruppo lineare proiettivo}\index{gruppo!lineare proiettivo} $\projgl{V}$ è il gruppo delle proiettività dello spazio vettoriale $V$ con operazione la composizione di proiettività ed elemento neutro $Id_{\proj{V}}$.
\vspace{-3mm}
\end{define}
\subsubsection{Descrizione matriciale del gruppo lineare proiettivo}
Consideriamo gli isomorfismi $\funz{\ }{\kamp^{n+1}}{\kamp^{n+1}}$: sappiamo che la matrice associata agli isomorfismi è una matrice invertibile, cioè si ha una \textit{isomorfismo di gruppi} fra l'insieme degli isomorfismi in $\kamp^{n+1}$ al \textit{gruppo generale lineare} $\gl\left(n+1,\ \kamp\right)$:
\begin{equation*}
	\left\{\text{isomorfismi}\funz{\ }{\kamp^{n+1}}{\kamp^{n+1}}\right\}\leftrightarrow\gl\left(n+1,\ \kamp\right)
\end{equation*}
E con il gruppo lineare proiettivo si può fare? Consideriamo:
\begin{equation}
	\funztot{\oldphi}{\gl\left(n+1,\ \kamp\right)}{\projgl[n+1]{\kamp}}{\phi_A}{\widetilde{\phi}_A}
\end{equation}
 $\phi$ è \textit{omomorfismo} di gruppi \textit{suriettivo}, ma non iniettivo. Infatti, il nucleo non è \textit{triviale}:
 \begin{equation*}
 	\ker \oldphi=\left\{\phi_A\mid \phi_A=Id_{\proj[n]{\kamp}}=\widetilde{Id}_{\kamp^{n+1}}\right\}=\left\{\phi_A\mid \phi=\lambda I,\ \lambda\in\kamp\setminus\left\{0\right\}\right\}=\left\{\phi_A\mid A=\lambda I,\ \lambda\in\kamp\setminus\left\{0\right\}\right\}
 \end{equation*}
Tuttavia, possiamo per il Teorema dell'isomorfismo per i gruppi considerare il seguente diagramma commutativo:
\[\begin{tikzcd}
	{\gl\left(n+1,\ \kamp\right)} & {\projgl[n+1]{\kamp}} \\
	{\frac{\gl\left(n+1,\ \kamp\right)}{\left\{\lambda I\ \mid\ \lambda\in\kamp\setminus\left\{0\right\}\right\}}}
	\arrow["{\pi}"', from=1-1, to=2-1]
	\arrow["{\exists \overline f}"', from=2-1, to=1-2, dashed]
	\arrow["{\oldphi}", from=1-1, to=1-2]
\end{tikzcd}\]
E si ha pertanto l'isomorfismo:
\begin{equation*}
	\projgl[n+1]{\kamp}\cong \frac{\gl\left(n+1,\ \kamp\right)}{\left\{\lambda I\mid\lambda\in\kamp\setminus\left\{0\right\}\right\}}=\frac{\gl\left(n+1,\ \kamp\right)}{\left\{\lambda I\right\}}
\end{equation*}
Si può anche considerare l'isomorfismo tra $\left\{\lambda I\mid\lambda\in\kamp\setminus\left\{0\right\}\right\}$ e $\kamp\setminus\left\{0\right\}$, e riscrivere l'isomorfismo trovato come:
\begin{equation*}
	\projgl[n+1]{\kamp}\cong \frac{\gl\left(n+1,\ \kamp\right)}{\kamp\setminus\left\{0\right\}}
\end{equation*}

\begin{example}
	Consideriamo la seguente proiettività della \textit{retta proiettiva} $\proj[1]{\realset}$:
	\begin{equation*}
		\funztot{f}{\proj[1]{\realset}}{\proj[1]{\realset}}{\left(x_0\colon x_1\right)}{\left(ax_0+bx_1\colon cx_0+dx_1\right)}
	\end{equation*}
	Considerato il gruppo lineare proiettivo $\proj[2]{\realset}=\frac{\gl\left(2,\ \realset\right)}{\left\{\lambda I\right\}}$, per definizione di $f$ si ha $f=\widetilde{\phi}$. In particolare, la matrice associata a $\phi$ è:
	\begin{equation*}
		A=\left(\begin{array}{cc}
			a & b\\
			c & d
		\end{array}\right)
	\end{equation*}
	E dunque possiamo scrivere l'applicazione lineare $\phi$ come:
	\begin{equation*}
		\funztot{f}{\realset^2}{\realset^2}{\left(\begin{array}{c}
				x_0 \\
				x_1
			\end{array}\right)}{A\left(\begin{array}{c}
				x_0 \\
				x_1
			\end{array}\right)}
	\end{equation*}
	E dunque $f$ si può anche scrivere come:
	\begin{equation*}
		\funztot{f}{\proj[1]{\realset}}{\proj[1]{\realset}}{\left[v\right]}{\left[Av\right]}
	\end{equation*}
	Notiamo che se la matrice associata a $\phi$ fosse $2A$, per \textit{proporzionalità} si avrebbe comunque la proiettività $f$. In modo analogo, $\lambda\in\realset\setminus\left\{0\right\}$ induce la \textit{stessa proiettività} $f$ di $A$.
	\vspace{-3mm}
\end{example}
\subsection{Altri aspetti delle trasformazioni proiettive}
\begin{observe}
	Se $f$ è una proiettività di $\proj{V}$ e $S\subseteq \proj{V}$ un sottospazio proiettivo, allora $f\left(S\right)$ è ancora un sottospazio proiettivo della stessa dimensione di $S$. Se $S=\proj{W}$ e consideriamo per definizione $f=\widetilde{\phi}$ con $\funz{\phi}{V}{V}$, allora:
	\begin{equation*}
		\forall \left[v\right]\in S\ f\left(\left[v\right]\right)=\widetilde{\phi}\left(\left[v\right]\right)=\left[\phi\left(v\right)\right],\ \phi\left(v\right)\in W
	\end{equation*}
	\begin{equation}
		f\left(S\right)=\proj{\phi\left(W\right)}
	\end{equation}
\vspace{-6mm}
\end{observe}
\begin{define}\textsc{Proiettivamente equivalenti.}\\
	Due sottoinsiemi $A,\ B$ di $\proj{V}$ si dicono \textbf{proiettivamente equivalenti}\index{proiettivamente equivalenti} se $\exists f$ proiettività di $\proj{V}$ tale che:
	\begin{equation}
		B=f\left(A\right)
	\end{equation}
\vspace{-6mm}
\end{define}
\begin{example}
	Due sottospazi proiettivi di $\proj{V}$ della \textit{stessa} dimensione sono sempre \textit{proiettivamente equivalenti}.
\end{example}
\begin{theorema}
	Siano $\proj{V}$ e $\proj{V'}$ di dimensione $n$. Siano:
	\begin{itemize}
		\item $P_0,\ \ldots,\ P_{n+1}\in\proj{V}$ in posizione generale.
		\item $Q_0,\ \ldots,\ Q_{n+1}\in\proj{V'}$ in posizione generale.
	\end{itemize}
Allora $\exists!\funz{f}{\proj{V}}{\proj{V'}}$ trasformazione proiettiva tale che $f\left(P_i\right)=Q_i\ \forall i=0,\ \ldots,\ n+1$.\\
In particolare: se una proiettività fissa $n+2$ punti in posizione generale, allora è l'identità.
\end{theorema}
\begin{demonstration}~{}
	\begin{itemize}
		\item \textbf{Esistenza}: Siano, $\forall i$:
		\begin{itemize}
			\item $P_i=\left[v_i\right]\ v_i\in V$.
			\item $Q_i=\left[w_i\right]\ w_i\in V'$.
		\end{itemize}
	Sappiamo, dall'osservazione a pag. \ref{puntigeneraleindipendentiosserva}, che:
	\begin{itemize}
		\item $v_0,\ \ldots,\ v_n$ è base di $V$, con $v_{n+1}=\lambda_0v_0+\ldots+\lambda_n v_n$ con $\lambda_i\neq 0\ \forall i$.
		\item $w_0,\ \ldots,\ w_n$ è base di $V'$, con $w_{n+1}=\mu_0w_0+\ldots+\mu_n w_n$ con $\mu_i\neq 0\ \forall i$.
	\end{itemize}
		A meno di cambiare i rappresentanti dei punti, possiamo supporre senza perdita di generalità che $\lambda_i=\mu_i=1$. Si ha dunque:
		\begin{gather*}
			v_{n+1}=v_0+\ldots+v_n\\
			w_{n+1}=w_0+\ldots+w_n
		\end{gather*}
	Sia $\funz{\phi}{V}{V'}$ l'applicazione lineare tale per cui $\phi\left(v_i\right)=w_i\ \forall i=0,\ \ldots,\ n$. Per linearità:
	\begin{equation*}
		\phi\left(v_{n+1}\right)=\phi\left(v_0+\ldots+v_n\right)=\phi\left(v_0\right)+\ldots+\phi\left(v_n\right)=w_0+\ldots+w_n=w_{n+1}
	\end{equation*}
Poiché $\im \phi$ contiene una base per costruzione, $\phi$ è suriettiva. In particolare, essendo endomorfismo ($\dim V=\dim V'$), $\phi$ è anche \textit{isomorfismo}.\\
Allora $f\coloneqq\widetilde{\phi}\funz{\ }{\proj{V}}{\proj{V'}}$ è una \textit{trasformazione proiettiva} e:
\begin{equation*}
	f\left(P_i\right)=f\left(\left[v_i\right]\right)=\left[\phi\left(v_i\right)\right]=\left(w_i\right)=Q_i\ \forall i=0,\ \ldots,\ n+1
\end{equation*}
\item \textbf{Unicità}: sia $\funz{g}{\proj{V}}{\proj{V'}}$ un'altra trasformazione proiettiva tale che $g\left(P_i\right)=Q_i\ \forall i=0,\ \ldots,\ n+1$. Per definizione, esiste $\funz{\psi}{V}{V'}$ isomorfismo per cui $g=\widetilde{\psi}$ e:
\begin{equation*}
	\left[\psi\left(v_i\right)\right]=\left[w_i\right]\ \forall i
\end{equation*}
Si ha che $\exists a_i\in\kamp\setminus\left\{0\right\}$ tale che $\psi\left(v_i\right)=a_iw_i$. Allora:
\begin{equation*}
	\begin{array}{ccccc}
		a_{n+1}w_{n+1}&=&\psi\left(v_{n+1}\right)&=&\psi\left(v_0+\ldots+v_n\right)\\
		\shortparallel&&&&\shortparallel\\
		a_{n+1}\left(w_0+\ldots+w_n\right)&&&&\psi\left(v_0\right)+\ldots+\psi\left(v_n\right)\\
		\shortparallel&&&&\shortparallel\\
		a_{n+1}w_0+\ldots+a_{n+1}w_n&&&&a_0w_0+\ldots+a_nw_n
	\end{array}
\end{equation*}
Poiché $w_0,\ \ldots,\ w_n$ è base, la scrittura è unica. Segue che $a_0=a_1=\ldots=a_{n+1}=a$. Allora:
\begin{equation*}
	\begin{array}{l}
		\psi\left(v_i\right)=aw_i=a\phi\left(v_i\right)\\
		\implies \psi=a\phi\\
		\implies g=\widetilde{\psi}=\widetilde{\phi}=f
	\end{array}
\end{equation*}
	\end{itemize}
\vspace{-6mm}
\end{demonstration}
\begin{examples}
	\begin{itemize}
		\item In una \textit{retta proiettiva} ($\dim 1$), una proiettività è determinata dalle immagini di $3$ \textit{punti distinti}, dato che è equivalente alla condizione di ‘‘\textit{punti in posizione generale}''.
		\item In un \textit{piano proiettivo} ($\dim 2$), una proiettività è determinata dalle \textit{immagini} di $4$ punti, a $3$ a $3$ \textit{non allineati}.
		\item Se $A,\ B\subseteq \proj{V}$ sono insiemi finiti, ciascuno contenente $k$ punti in posizione generale, con $k\leq n+2$, allora $A$ e $B$ sono sempre proiettivamente equivalenti.
	\end{itemize}
\vspace{-3mm}
\end{examples}
\begin{example}
	Approfondiamo l'ultimo esempio. In $\proj[2]{\kamp}$ ($\dim 2$), si prenda $A=\left\{P_1,\ P_2\right\},\ B=\left\{Q_1,\ Q_2\right\}$ con $P_1\neq P_2$, $Q_1\neq Q_2$. Ho due punti distinti sia in $A$ e $B$, dunque esiste sempre una proiettività $\funz{f}{\proj[2]{\ }}{\proj[2]{\ }}$ tale che $f\left(A\right)=B$.\\
	Se invece $A$ e $B$ contengono $3$ punti, se i $3$ punti in $A$ \textit{sono allineati} mentre i $3$ punti in $B$ \textit{non} lo sono, allora $A$ e $B$ \textit{non} sono proiettivamente equivalenti.
\end{example}
\subsection{Trasformazioni proiettive in coordinate}
Supponiamo di avere fissato dei \textit{riferimenti proiettivi} su $\proj{V}$ e $\proj{V'}$, dati da delle basi $\basis$ di $V$ e $\basis'$ di $V'$, e sia $\funz{f}{\proj{V}}{\proj{V}}$ una trasformazione proiettiva. Sappiamo che $f=\widetilde{\phi}$ con $\funz{\phi}{V}{V'}$ isomorfismo lineare.\\
Sia $A\in\gl\left(n+1,\ \kamp\right)$ la matrice associata a $\phi$ rispetto alle basi $\basis$ e $\basis'$. Abbiamo visto che $\phi$ è determinata solo a meno di multipli: chiaramente, lo stesso è vero anche per $A$.\\
Siano allora:
\begin{gather*}
	P=\left(x_0\colon\ldots\colon x_n\right)\in\proj{V}\\
	f\left(P\right)=\left(y_0\colon\ldots\colon y_n\right)\in\proj{V'}
\end{gather*}
Allora $\exists\rho\in\kamp\setminus\left\{0\right\}$ tale che $\rho y=Ax$.
\begin{observe}\textsc{Cambiamenti di coordinate.}\\
Se in $\proj{V}$ abbiamo due riferimenti proiettivi, uno dalla dalla base $\basis$, e uno dalla base $\basis'$, sia $M$ la \textit{matrice del cambiamento di base} in $V$ tale che:
\begin{equation*}
	x'=Mx
\end{equation*}
Con $x$ in coordinate rispetto alla base $\basis$ e $x'$ in coordinate rispetto alla base $\basis'$. Allora, se $P\in\proj{V}$ ha coordinate:
\begin{equation*}
	\left(x_0\colon\ldots\colon x_n\right)\text{ rispetto a }\basis\\
	\left(x_0'\colon\ldots\colon x_n'\right)\text{ rispetto a }\basis'
\end{equation*}
Esiste $\rho\in\kamp\setminus\left\{0\right\}$ tale che $\rho x'=M x$.
\end{observe}
\subsection{Punti fissi di proiettività}
\begin{define}\textsc{Punto fisso.}\\
	Sia $\funz{f}{\proj{V}}{\proj{V}}$ una proiettività. Un \textbf{punto fisso}\index{punto!fisso} è un punto $P\in\proj{V}$ tale che:
	\begin{equation}
		f\left(P\right)=P
	\end{equation}
\vspace{-6mm}
\end{define}
Sia $\funz{\phi}{V}{V}$ un \textit{automorfismo} tale che $f=\widetilde{\phi}$, e sia $P=\left[v\right]$, con $v\in V\setminus\left\{0\right\}$. Allora:
\begin{equation*}
	\begin{array}{ll}
		& f\left(P\right)=\left[\phi\left(v\right)\right]=\left[v\right]\\
		\iff & \exists\lambda\in\kamp\setminus\left\{0\right\}\ \colon \phi\left(v\right)=\lambda v\\
		\iff & v\text{ è un autovettore per }\phi
	\end{array}
\end{equation*}
In particolare, $\phi$ è invertibile, dunque \textit{non} ha l'autovettore \textit{nullo}. Segue che i punti fissi di $f$ sono tutti e soli i punti $\left[v\right]$ con $v$ autovettore di $\phi$.
\begin{observes}~{}
	\begin{enumerate}
		\item Se $\kamp=\complexset$, allora ogni proiettività ha almeno un punto fisso, dato che $\phi$ ha sempre almeno un autovettore.
		\item Se $\kamp=\realset$ e $\dim\proj{V}=n$, allora $\dim V=n+1$. Il \textit{polinomio caratteristico} $C_\phi\left(t\right)\in\realset\left[t\right]$ ha grado $n+1$. Se $n$ è \textit{pari}, $\phi$ ha almeno un autovalore, dato che il polinomio caratteristico ha grado $n+1$ \textit{dispari}: infatti, o è di grado \textit{uno} (e quindi ha banalmente soluzione) oppure, in quanto si può decomporre in fattori a coefficienti reali al più di grado \textit{due}, ammetterà \textit{sempre} almeno un fattore di grado \textit{uno}.
		\item Portiamo un controesempio al caso $n$ dispari. Sia $\funztot{f}{\proj{\realset}}{\proj{\realset}}{\left(x\colon y\right)}{\left(-y\colon x\right)}$. La matrice $A$ associata a $f$ è:
		\begin{equation*}
			A=\left(\begin{array}{cc}
				0 & -1 \\ 
				1 & 0
			\end{array}\right)
		\end{equation*}
	Il polinomio caratteristico \textit{non} ha radici \textit{reali}:
	\begin{equation*}
		C_A\left(t\right)=\det\left(\begin{array}{cc}
			-t & -1 \\ 
			1 & -t
		\end{array}\right)=t^2-1
	\end{equation*}
Segue che $A$ non ha autovettori reali e pertanto $f$ \textit{non} ha punti fissi.
\item In generale, l'\textit{insieme} dei punti fissi di $\funz{f}{\proj{V}}{\proj{V}}$ è dato da:
\begin{equation*}
	\left\{\proj{V_{\lambda}}\mid\lambda\text{ autovalore di }\phi\right\}
\end{equation*}
Questo è un insieme di sottospazi proiettivi a 2 a 2 disgiunti.
 	\end{enumerate}
\end{observes}
\begin{define}\textsc{Insieme fisso.}\\
	Se $S\subseteq\proj{V}$ è un sottospazio, diciamo che $S$ è \textbf{fisso} per $f$ proiettività se:
		\begin{equation}
		f\left(S\right)=S
	\end{equation}
	\vspace{-6mm}
\end{define}
\subsection{Impratichiamoci! Trasformazioni proiettive}
\begin{exercise}
	In $\proj[1]{\realset}$ determinare la proiettività $f$ tale che:
	\begin{equation*}
		f\left(2\colon 1\right)=\left(1\colon1\right)\quad f\left(1\colon 2\right)=\left(0\colon1\right)\quad
		f\left(1\colon -1\right)=\left(1\colon0\right)
	\end{equation*}
\vspace{-6mm}
\end{exercise}
\begin{solution}
Notiamo che i punti:
\begin{equation*}
	\left(2\colon 1\right)\quad\left(1\colon 2\right)\quad\left(1\colon -1\right)\text{ e }	\left(1\colon 1\right)\quad\left(0\colon 1\right)\quad\left(1\colon 0\right)
\end{equation*}
Sono distinti, dunque sono in posizione generale e la proiettività è garantita. Prendiamo la generica matrice $A=\left(\begin{array}{cc}
	a & b\\
	c & d
\end{array}\right)$ associata a $\phi$ indotta da $f$ e consideriamo $\rho y=Ax$:
\begin{equation*}
	\begin{cases}
		\rho y_0=ax_0+bx_1\\
		\rho y_1=cx_0+dx_1
	\end{cases}
\end{equation*}
Imponiamo il passaggio per $f\left(2\colon 1\right)=\left(1\colon1\right)$:
\begin{equation*}
	\begin{cases}
		\rho=1a+b\\
		\rho=2c+d
	\end{cases}\implies 2a+b=2c+d
\end{equation*}
In sostanza, \textit{eliminiamo} il parametro $\rho$ per ottenere un'equazione lineare \textit{omogenea} tra gli elementi della matrice.\\
Facciamo lo stesso con i rimanenti punti $f\left(1\colon 2\right)=\left(0\colon1\right)$ e $	f\left(1\colon -1\right)=\left(1\colon0\right)$, utilizzando rispettivamente $\mu y=Ax$ e $\alpha y=Ax$:
\begin{gather*}
	\begin{cases}
		0=a+2b\\
		\rho=c+2d
	\end{cases}\implies a+2b=0\\
	\begin{cases}
		\alpha=a-b\\
		0=c-d
	\end{cases}\implies c-d=0
\end{gather*}
Costruiamo così un sistema lineare omogeneo di $3$ equazioni in $4$ incognite $a,\ b,\ c,\ d$, con una matrice dei coefficienti di rango $3$:
\begin{equation*}
	\begin{cases}
		2a+b=2c+d\\
		a+2b=0\\
		c-d=0
	\end{cases}\implies \begin{cases}
	a=2c\\
	b=-c\\
	c=c\\
	d=c
\end{cases}\implies
A=\left(\begin{array}{cc}
	2c & -c\\
	c & c
\end{array}\right)=c\left(\begin{array}{cc}
1 & -1\\
1 & 1
\end{array}\right)
\end{equation*}
A meno di multipli, $A=\left(\begin{array}{cc}
	1 & -1\\
	1 & 1
\end{array}\right)$ è la matrice cercata. Segue dunque che la proiettività cercata è:
\begin{equation*}
	\funztot{f}{\proj[1]{\realset}}{\proj[1]{\realset}}{\left(x_0\colon x_1\right)}{\left(2x_0-x_1\colon x_0+x_1\right)}
\end{equation*}
\vspace{-3mm}
\end{solution}
\section{Geometria affine e geometria proiettiva}\label{spaziaffini}
Abbiamo già accennato all'esistenza di una relazione che intercorre fra \textit{geometria affine} e \textit{geometria proiettiva}. Diamo innanzitutto qualche richiamo dei concetti della geometria affine.
\begin{define}\textsc{Spazio affine.}\\
	Sia $V$ uno spazio vettoriale di dimensione finita su un campo $\kamp$. Uno \textbf{spazio affine}\index{spazio!affine} di dimensione $n$ su $V$ (con spazio vettoriale associato $V$ di dimensione $n$ ) è un insieme $\aff{V}$ non vuoto di \textit{punti} (elementi) tale che sia data un'applicazione:
	\begin{equation*}
		\funztot{\ }{\aff{V}\times\aff{V} }{V}{\left(P,\ Q\right)}{\overrightarrow{PQ}}
	\end{equation*}
	Che alla coppia di punti $\left(P,\ Q\right)$ associa il vettore di $V$ con punto iniziale $P$ e punto finale $Q$ e tale che siano
	soddisfatti i seguenti assiomi:
	\begin{enumerate}
		\item $\forall P\in\aff{V},\ \forall v\in V$ esiste un unico punto $Q\in\aff{V}$ tale che $\overrightarrow{PQ}=v$.
		\item $\forall P,\ Q,\ R\in\aff{V}$ terna di punti di $\aff{V}$ si ha $\overrightarrow{PQ}+\overrightarrow{QR}=\overrightarrow{PR}$.
	\end{enumerate}
\end{define}
\begin{define}\textsc{Riferimento affine.}\\
	Un \textbf{riferimento affine}\index{riferimento!affine} $\mathcal{R} = (O,\  e_1,\  e_2,\ \ldots,\  e_n)$ sullo spazio $\aff{V}$ è assegnato fissando un punto $O\in \aff{V}$ detta \textbf{origine}\index{origine!affine} ed una base $\basis = (e_1,\  e_2,\ \ldots,\  e_n)$ di $V$. Dunque, per ogni $P\in \aff{V}$ si ha la $n$-upla $(X_1,\  X_2,\ \ldots,\  X_n)$ dette \textit{coordinate affini}\index{coordinate!affini} del punto $P\in\aff{V}$ (uniche per riferimento affine fissato) tale per cui:
	\begin{equation}
		P=\overrightarrow{OV}=x_1e_1+\ldots+x_ne_n
	\end{equation}
\vspace{-6mm}			 
\end{define}
Per i nostri scopi, parleremo spesso degli spazi affini di dimensione $n$ su $\kamp$.
\begin{define}\textsc{Affinità.}\\
	Un'\textbf{affinità}\index{affinità} o \textbf{trasformazione lineare affine}\seeonlyindex{trasformazione lineare affine}{affinità} di $\aff{\kamp^n}$ è un'applicazione:
	\begin{equation}
		\funz{\phi}{\aff{\kamp^n}}{\aff{\kamp^n}}
	\end{equation}
Della forma $\phi\left(x\right)=Ax+b$ con $A\in\gl\left(n,\ \kamp\right)$ un'applicazione lineare invertibile e $b$ una \textit{traslazione}.
\end{define}
\begin{define}\textsc{Sottospazio affine.}\\
	Un \textbf{sottospazio affine}\index{sottospazio!affine} di $\aff{\kamp^n}$ è un \textit{traslato} di un sottospazio vettoriale $W\subseteq \kamp^n$:
	\begin{equation}
		S=W+x_0=\left\{w+x_0\mid w\in W,\ x_0\in\aff{W}\right\}
	\end{equation}
\vspace{-6mm}
\end{define}
\begin{observes}~{}
	\begin{itemize}
		\item $W$ è l'unico traslato di $S$ per l'origine ($x_0=O$) e si dice \textbf{sottospazio direttore}  di $S$\index{sottospazio!direttore}, cioè ne dà appunto la \textit{direzione}. Si definisce $\dim S\coloneqq\dim W$.
		\item Un punto in $\aff{\kamp^n}$ è un sottospazio affine di dimensione $0$ ($W=\left\{0\right\}$; dopotutto non ha particolarmente senso parlare di direzione del punto).
		\item Una \textbf{retta affine}\index{retta!affine} $r$ in $\aff{\kamp^n}$ è un sottospazio affine di $\dim 1$: $W=\lin{v}$, cioè $r$ si può individuare assegnando un punto $P\in r$ e un qualsiasi vettore $v$ \textit{parallelo} alla retta $r$.
		\item Un \textbf{piano affine}\index{piano!affine} $\pi$ in $\aff{\kamp^n}$ è un sottospazio affine di $\dim 2$: $W=\lin{v,\ w}$, cioè $\pi$ si può individuare assegnando un punto $P\in r$ e una coppia di vettori l.i. \textit{paralleli} al piano $\pi$.
		\item Un \textbf{iperpiano affine}\index{iperpiano!affine} è un sottospazio di dimensione $n+1$.
		\item Due sottospazi affini della stessa dimensione si dicono \textbf{paralleli}\index{parallelo} se hanno lo \textit{stesso} sottospazio direttore.
	\end{itemize}
\vspace{-3mm}
\end{observes}
\begin{example}
Consideriamo $r=W+x_0$ retta affine, che ha dunque $\dim r=\dim W=1$. $W$ è la retta vettoriale in $\kamp^n$, mentre un qualunque $v\in W\setminus\left\{0\right\}$ è la \textit{direzione} della retta.
\end{example}
Un sottospazio affine $S\subseteq \kamp^n$ può essere descritti con equazioni cartesiane oppure in forma parametrica.\\
	\textbf{Equazioni cartesiane}. $S$ è visto come l'insieme delle \textit{soluzioni} del seguente sistema lineare:
	\begin{equation}
		Ax=b\qquad\left(\begin{array}{c}
			X_1\\
			\vdots\\
			X_n
		\end{array}\right)\in\aff{\kamp^{n}}
	\end{equation}
Con $b$ che descrive la traslazione dovuta a $x_0\in\aff{W}$.
In tal caso $W$ è il sottospazio vettoriale delle soluzioni del sistema lineare omogeneo associato:
\begin{equation}
	Ax=0
\end{equation}
\textbf{Forma parametrica}. Supponiamo $\dim S=\dim W=m$. Siano $v_1,\ \ldots,\ v_m\in\kamp^n$ i vettori di una base di $W$; rispetto ad una base di $\kamp^n$, e dunque rispetto ad un sistema di riferimento affine con origine $O$, essi sono espressi nelle componenti:
\begin{equation*}
	v_i=\left(V_{i,1},\ \ldots,\ V_{i,n}\right)\in\aff{\kamp^n}
\end{equation*}
Consideriamo $S=W+c$, con il punto $c=\left(C_1,\ \ldots,\ C_n\right)$ rispetto allo stesso sistema affine di prima.\\
I punti $x$ di $S$ in forma parametrica sono dati da:
\begin{equation}
	x=t_1v_1+\ldots+t_mv_m+c\quad t_1,\ \ldots,\ t_m\in\kamp
\end{equation}
Da cui otteniamo il sistema $n\times\left(m+1\right)$ seguente:
\begin{equation}
	\begin{cases}
		\begin{array}{l}
			X_1=t_1V_{1,1}+\ldots+t_mV_{m,1}+C_1\\
			\vdots\\
			X_n=t_1V_{1,n}+\ldots+t_mV_{m,n}+C_n\\			
		\end{array}
	\end{cases}
\end{equation}
\begin{example}
	La retta $r$ ($\dim W=1$) passante per $c$ con direzione $v$ è descritto parametricamente da:
\begin{equation*}
	\begin{cases}
		\begin{array}{l}
			X_1=tV_1+c_1\\
			\vdots\\
			X_n=tV_n+c_n			
		\end{array}
	\end{cases}
\end{equation*}	
\end{example}
Consideriamo ora lo \textit{spazio proiettivo numerico}:
\begin{equation*}
	\proj[n]{\ }=\proj[n]{\kamp}=\proj{\kamp^{n+1}}
\end{equation*}
E i punti in coordinate omogenee $\left(x_0\colon \ldots\colon x_n\right)$ rispetto ad un dato sistema di riferimento proiettivo.
Consideriamo il seguente sottoinsieme di $\proj[n]{\ }$:
\begin{equation}
	U_0\coloneqq \left\{P=\left(x_0\colon\ldots\colon x_n\right)\in \proj[n]{\ }\mid x_0\neq 0\right\}
\end{equation}
La condizione $x_0\neq 0$ è \textit{ben posta}; infatti, se $\lambda\in\kamp\setminus\left\{0\right\}$, allora $x_0\neq 0\iff\lambda x_0\neq 0$.\\
Consideriamo anche il suo complementare, che è l'iperpiano coordinato rispetto alla prima coordinata omogenea:
\begin{equation}
	\proj[n]{\ }\setminus=H_0=\left\{P=\left(x_0\colon\ldots\colon x_n\right)\in \proj[n]{\ }\mid x_0= 0\right\}=\left\{P=\left(0\colon\ldots\colon x_n\right)\in \proj[n]{\ }\right\}
\end{equation}
Sia $P\in U_0$: allora essendo $a_0\neq 0$ si ha $P=\left(a_0\colon\ldots\colon a_n\right)=\left(1\colon\frac{a_1}{a_0}\ldots\colon \frac{a_n}{a_0}\right)$. In particolare, $\frac{a_1}{a_0},\ \ldots,\ \frac{a_n}{a_0}$ sono univocamente determinate da $P$.
\begin{example}
 Sia $\proj[2]{\realset}=\left\{\text{rette vettoriali in }\realset^3\right\}$ con punti di componenti $\left(x_0\colon x_1\colon x_2\right)$. Allora $H_0$ è una retta proiettiva in $\proj[2]{\realset}$ e risulta:
 \begin{equation*}
 	\begin{array}{ll}
 		H_0 & =\left\{P=\left(x_0\colon x_1\colon x_2\right)\in \proj[n]{\ }\mid x_0=0\right\}=\left\{P=\left(0\colon x_1\colon x_2\right)\in \proj[n]{\ }\right\} \\
 		& =\left\{\text{rette vettoriali di }\realset^3\text{ contenute nel piano affine }x_0=0\right\}
 	\end{array}
 \end{equation*}
	Infatti, prendiamo $\aff{\realset^3}$ e consideriamo il piano $x_0=1$, parallelo al piano $x_0=0$. Se $r\subseteq \realset^3$ è una retta vettoriale che \textit{non} appartiene al piano affine $\left\{x_0=0\right\}$ ($r\nsubseteq \left\{x_0=0\right\}$), $r$ interseca il piano $x_0=1$ in un solo punto! In particolare, se $r$ ha direzione $\left(a_0,\ a_1,\ a_2\right)$, il punto nel piano $\left\{x_0=1\right\}$ avrà coordinate $\left(\frac{a_1}{a_0},\ \frac{a_2}{a_0}\right)$.
\end{example}
Possiamo identificare $U_0\subseteq\proj[n]{\ }$ con $\aff{\kamp^n}$. Consideriamo le due funzioni seguenti:
\begin{equation}
	\funztot{j=j_0}{\aff{\kamp^n}}{U_0\subseteq \proj[n]{\ }}{\left(X_1,\ \ldots,\ X_n\right)}{\left(1\colon\ x_1\colon \ldots\colon X_n\right)}\\
	\funztot{\oldphi}{U_0\subseteq \proj[n]{\ }}{\aff{\kamp^n}}{\left(x_0\colon\ \ldots\colon X_n\right)}{\left(\frac{x_1}{x_0},\ \ldots,\ \frac{x_n}{x_0}\right)}
\end{equation}
\begin{itemize}
	\item $\oldphi$ è ben definita, dato che $x_0\neq 0$ per definizione di $U_0$.
	\item $j$ e $\oldphi$ sono l'una l'inversa dell'altra:
	% https://q.uiver.app/?q=WzAsNixbMCwwLCJcXGFmZntcXGthbXBebn0iXSxbMSwwLCJVXzAiXSxbMiwwLCJcXGFmZntcXGthbXBebn0iXSxbMCwxLCJcXGxlZnQoWF8xLFxcIFxcbGRvdHMsXFwgWF9uXFxyaWdodCkiXSxbMSwxLCJcXGxlZnQoMVxcY29sb24gWF8xXFxjb2xvbiBcXGxkb3RzXFxjb2xvbiBYX25cXHJpZ2h0KSJdLFsyLDEsIlxcbGVmdChYXzEsXFwgXFxsZG90cyxcXCBYX25cXHJpZ2h0KSJdLFswLDEsImoiXSxbMSwyLCJcXG9sZHBoaSJdLFszLDQsIiIsMCx7InN0eWxlIjp7InRhaWwiOnsibmFtZSI6Im1hcHMgdG8ifX19XSxbNCw1LCIiLDAseyJzdHlsZSI6eyJ0YWlsIjp7Im5hbWUiOiJtYXBzIHRvIn19fV1d
	\[\begin{tikzcd}
		{\aff{\kamp^n}} & {U_0} & {\aff{\kamp^n}} \\[-25pt]
		{\left(X_1,\ \ldots,\ X_n\right)} & {\left(1\colon X_1\colon \ldots\colon X_n\right)} & {\left(X_1,\ \ldots,\ X_n\right)}
		\arrow["{j}", from=1-1, to=1-2]
		\arrow["{\oldphi}", from=1-2, to=1-3]
		\arrow[from=2-1, to=2-2, maps to]
		\arrow[from=2-2, to=2-3, maps to]
	\end{tikzcd}\]
% https://q.uiver.app/?q=WzAsNixbMCwwLCJVXzAiXSxbMSwwLCJcXGFmZntcXGthbXBebn0iXSxbMiwwLCJVXzAiXSxbMCwxLCJcXGxlZnQoeF8wXFxjb2xvbiBcXGxkb3RzXFxjb2xvbiB4X25cXHJpZ2h0KSJdLFsxLDEsIlxcbGVmdChcXGZyYWN7eF8xfXt4XzB9LFxcIFxcbGRvdHMsXFwgXFxmcmFje3hfbn17eF8wfVxccmlnaHQpIl0sWzIsMSwiXFxsZWZ0KDFcXGNvbG9uXFxmcmFje3hfMX17eF8wfVxcY29sb25cXGxkb3RzXFxjb2xvblxcZnJhY3t4X259e3hfMH1cXHJpZ2h0KT1cXGxlZnQoeF8wXFxjb2xvblxcbGRvdHNcXGNvbG9uIHhfblxccmlnaHQpIl0sWzAsMSwiXFxvbGRwaGkiXSxbMSwyLCJqIl0sWzMsNCwiIiwwLHsic3R5bGUiOnsidGFpbCI6eyJuYW1lIjoibWFwcyB0byJ9fX1dLFs0LDUsIiIsMCx7InN0eWxlIjp7InRhaWwiOnsibmFtZSI6Im1hcHMgdG8ifX19XV0=
\[\begin{tikzcd}
	{U_0} & {\aff{\kamp^n}} & {U_0} \\[-25pt]
	{\left(x_0\colon \ldots\colon x_n\right)} & {\left(\frac{x_1}{x_0},\ \ldots,\ \frac{x_n}{x_0}\right)} & {\left(1\colon\frac{x_1}{x_0}\colon\ldots\colon\frac{x_n}{x_0}\right)=\left(x_0\colon\ldots\colon x_n\right)}
	\arrow["{\oldphi}", from=1-1, to=1-2]
	\arrow["{j}", from=1-2, to=1-3]
	\arrow[from=2-1, to=2-2, maps to]
	\arrow[from=2-2, to=2-3, maps to]
\end{tikzcd}\]
\end{itemize}
Si ha dunque che $j$ e $\oldphi$ sono \textit{biunivoche}. In questo modo identifichiamo $U_0\subseteq \proj[n]{\ }$ con $\aff{\kamp^n}$, mentre l'iperpiano $H_0$ corrisponde allo spazio proiettivo di dimensione $n-1$; si ha dunque:
\begin{equation}
	\proj[n]{\ }=U_0\amalg H_0=\aff{\kamp^n}\amalg \proj[n-1]{\ }
\end{equation}
La coppia $\left(U_0,\ j\right)$ è detta \textbf{carta affine}\index{carta affine} di $\proj[n]{\ }$.\\
In altre parole, $\proj[n]{\ }$ si può vedere come un'estensione o \textit{ampliamento} dello spazio affine $\kamp^n$. Diciamo allora che:
	\begin{itemize}
		\item I punti di $H_0$ sono detti \textbf{punti impropri}\seeonlyindex{punto!improprio}{punto!all'infinito} o \textbf{punti all'infinito}\index{punto!all'infinito}.
		\item $H_0$ è detto \textbf{iperpiano improprio}\seeonlyindex{iperpiano!improprio}{iperpiano!all'infinito} o \textbf{iperpiano all'infinito}\index{iperpiano!all'infinito}.
		\item I punti di $U_0=\aff{\kamp^n}$ sono detti \textbf{punti propri}\index{punto!proprio}.
	\end{itemize}
\begin{intuit}
In molti casi, possiamo liberamente parlare di $\aff{\kamp^n}$ come lo spazio vettoriale $\kamp^n$ inteso in senso \textit{geometrico} come insieme di punti con un punto qualunque come origine.
\end{intuit}
\begin{example}
	Consideriamo la retta proiettiva $\proj[1]{\kamp}$. L'iperpiano all'infinito è:
	  \begin{equation*}
	 		H_0=\left\{P=\left(x_0\colon x_1\right)\in \proj[1]{\ }\mid x_0=0\right\}=\left\{\left(0\colon 1\right)\right\}
	 \end{equation*}
 Mentre invece l'insieme dei punti propri è:
 \begin{equation*}
 	U_0=\left\{P=\left(x_0\colon x_1\right)\in \proj[1]{\ }\mid x_0\neq 0\right\}=\left\{\left(0\colon 1\right)\right\}
 \end{equation*}
In particolare, si ha la corrispondenza biunivoca $U_0\stackrel{1\colon1}{\leftrightarrow}\kamp$:
\begin{equation*}
	\left(x_0\colon x_1\right)=\left(1\colon \frac{x_1}{x_0}\right)\mapsto \frac{x_1}{x_0}\in\kamp
\end{equation*}
\begin{minipage}{0.56\textwidth}
In altre parole, si può vedere la retta proiettiva come il campo $\kamp$ con l'aggiunto di un unico punto, l'\textit{infinito} $\infty$.
\begin{equation}
	\proj[1]{\ }=\kamp\cup\left(0\colon 1\right)=\kamp\cup\left\{\infty\right\}
\end{equation}
Se $\kamp=\realset$, essendo $S^1\setminus\left\{1\text{ punto}\right\}\cong \realset$, si ha:
\begin{equation}
	\proj[1]{\realset}=\realset\cup\left\{\infty\right\}\cong S^1
\end{equation}
\end{minipage}
\begin{minipage}{0.44\textwidth}
	\includegraphics[trim=0cm 0cm 0cm 0cm,clip,scale=0.75]{images/projlinetocirc.pdf}
\end{minipage}
\vspace{-6mm}
\end{example}
\subsection{Chiusura proiettiva di un sottospazio affine}
\begin{define}\textsc{Chiusura proiettiva.}\\
	Sia $r\subseteq\aff{\kamp^n}$ una retta affine. La \textbf{chiusura proiettiva}\index{chiusura proiettiva}\index{chiusura!proiettiva} di $r$ è il sottospazio proiettivo $\overline{r}\subseteq\proj[n]{\ }$ generato da $r\subseteq U_0\subseteq\proj[n]{\ }$.
\end{define}
\begin{proposition}
	$\overline{r}$ è una \textit{retta proiettiva} e si ha:
	\begin{equation}
		\overline{r}=r\union P_{\infty}
	\end{equation}
Dove $P_{\infty}=\overline{r}\cap H_0$ è detto \textit{punto all'infinito} o \textit{punto improprio} della retta $r$.
\end{proposition}
\begin{demonstration}
	Sia $v\in\kamp^n\setminus\left\{0\right\}$ la direzione di $r$ e $w\in r$ un punto della retta. Allora $r$ ha descrizione parametrica in $\aff{\kamp^n}$:
	\begin{equation*}
		\begin{cases}
			\begin{array}{l}
				X_1=tv_1+w_1\\
				\vdots\\
				X_n=tv_n+w_n\\			
			\end{array}
		\end{cases}\quad t\in\kamp
	\end{equation*}
Consideriamo la retta proiettiva $R\subseteq\proj[n]{\ }$ con descrizione parametrica:
\begin{equation*}
\begin{cases}
		\begin{array}{l}
			x_0=s\\
			x_1=tv_1+sw_1\\
			\vdots\\
			x_n=tv_n+sw_n\\			
		\end{array}
	\end{cases}\quad \left(s\colon t\right)\in\proj[1]{\ }
\end{equation*}
$R$ è la retta proiettiva per i punti:
\begin{equation*}
	t=0\ \colon\ \left(1\colon w_1\colon\ldots\colon w_n\right)\quad t=1\ \colon\ \left(0\colon v_1\colon\ldots\colon v_n\right)=P_\infty
\end{equation*}
Ponendo $s=1$ otteniamo:
\begin{equation*}
	\begin{cases}
		\begin{array}{l}
			x_0=1\\
			x_1=tv_1+w_1\\
			\vdots\\
			x_n=tv_n+w_n\\			
		\end{array}
	\end{cases}\quad t\in\kamp
\end{equation*}
Al variare di $t\in\kamp$, questi sono tutti e soli i punti di $j\left(r\right)\subseteq U_0\subseteq \proj[n]{\ }$. Si ha dunque che $R$ è una retta proiettiva contente $r$:
\begin{equation*}
	\begin{array}{ll}
			R&=r\union P_{\infty}\\
			P_{\infty}&=R\cap H_0=\left\{\left(0\colon v_1\colon v_n\right)\right\}
	\end{array}
\end{equation*}
$R$ è necessariamente il più piccolo sottospazio proiettivo contenente $r$, dato che è la retta più un solo punto. Pertanto, $R=\overline{r}$.
\end{demonstration}
\begin{observes}~{}
	\begin{enumerate}
		\item Il punto improprio di $r$ è:
		\begin{equation*}
			P_{\infty}=\left(0\colon v_1\colon \ldots\colon v_n\right)
		\end{equation*}
	E corrisponde esattamente alla \textit{direzione} $v=\left(v_1,\ \ldots,\ v_n\right)$ di $r$.
	Poiché $P_\infty=\left[v\right]$ con $v$ la direzione di $r$, ne segue che l'iperpiano improprio di $\proj[n]{\kamp }$ è:
	 \begin{equation*}
		\begin{array}{ll}
			H_0 & =\proj[n-1]{\kamp}=\proj{\kamp^n}=\left\{\text{rette vettoriali in }\kamp^n\right\}=\\
			&=\left\{\text{direzioni delle rette affini in }\aff{\kamp^n}\right\}
		\end{array}
	\end{equation*}
\item Due rette affini $r_1,\ r_2\subseteq \aff{\kamp^n}$ hanno lo stesso punto improprio se e solo hanno la \textit{stessa direzione}, cioè se sono \textit{parallele}.\\
Se $r_1\neq r_2$ e $r_1$ e $r_2$ sono parallele, allora $r_1\cap r_2=\emptyset$ in $\aff{\kamp^n}$, ma $\overline{r_1}\cap \overline{r_2}={P_{\infty}}$ in $\proj[n]{\ }$. Ciò ci porta a dire che due rette parallele $r_1$ e $r_2$ si incontrano sempre all'\textit{infinito}!
\item Se $n=2$, cioè operando in $\proj[2]{\ }$, due rette distinte $r_1,\ r_2\subseteq \aff{\kamp^2}$ sono o \textit{incidenti} o \textit{parallele}, ma in $\proj[2]{\ }$ si intersecano sempre.
\item Viceversa: sia $l\subseteq \proj[n]{\ }$ una retta proiettiva. Abbiamo due casi:
\begin{itemize}
	\item $l\subseteq H_0,\ l\cap U_0= \emptyset$.
	\item $l\nsubseteq H_0\implies l+H_0=\proj[n]{\ }$.
\end{itemize}
Infatti, si ha che $l+H_0$ è un sottospazio proiettivo che contiene strettamente $H_0$, dato che $l\nsubseteq H_0$, e usando la formula di Grassmann otteniamo:
\begin{equation*}
	\dim\left(l+H_0\right)=\dim l +\dim H_0-\dim \left(l\cap H_0\right)=1+n-1+0=n=\dim \proj[n]{\ }\implies l+H_0 = \proj[n]{\ }
\end{equation*}
Sempre dalla formula di Grassmann:
\begin{equation*}
	\dim\left(l\cap H_0\right)=0\implies l\cap H_0=\left\{1 \text{punto}\right\}=\left\{Q\right\}
\end{equation*}
Cioè $l\cap U_0=l\setminus\left\{Q\right\}$. In altre parole, $l$ è una retta affine in $\aff{\kamp^n}$ con un \textit{punto improprio} $Q$ e necessariamente $l$ è la chiusura proiettiva di $l\setminus\left\{Q\right\}$.
\item Sia $n=2$, cioè operiamo in $\proj[2]{\ }$. Una retta $r\subseteq \aff{\kamp^2}$ è descritta da un'equazione lineare:
\begin{equation}
	ax+by+c=0\quad \left(a,\ b\right)\neq\left(0,\ 0\right)
\end{equation}
Con la corrispondenza biunivoca fra le coordinate $\left(x,\ y\right)$ vettoriali e $\left(X_1,\ X_2\right)$ affini. Abbiamo tuttavia anche la corrispondenza con le coordinate omogenee in $\proj[2]{\ }$, rispettivamente $\left(x\colon y\colon z\right)$ e $\left(_0\colon x_2\colon x_3\right)$.\\
Chiamiamo $\left(x\colon y\colon z\right)$ le coordinate omogenee su $\proj[2]{\ }$ con:
\begin{equation*}
	H_0=\left\{P=\left(x\colon y\colon z\right)\in \proj[2]{\ }\mid z=0\right\}=\left\{P=\left(x\colon y\colon0\right)\in \proj[2]{\ }\right\}
\end{equation*}
Allora la chiusura proiettiva $\overline{r}\subseteq \proj[2]{\ }$ di $r$ ha in $\proj[2]{\ }$ l'equazione lineare omogenea seguente:
\begin{equation}
		ax+by+cz=0
\end{equation}
Infatti, per $z=1$ si ottiene l'equazione di $r$, mentre ponendo $z=0$ (cioè il passaggio per $H_0$) troviamo il punto improprio $P_{\infty}$ di $r$:
\begin{equation*}
		\begin{cases}
		\begin{array}{l}
			z=0\\
			ax+by=0\\	
		\end{array}
	\end{cases}\quad P_{\infty}=\left(-b\colon a\colon 0\right)
\end{equation*}
La direzione della retta $ax+by+c=0$ è data dal punto improprio $P_{\infty}$ e corrisponde al vettore $\left(-b, a\colon 0\right)$.
	\end{enumerate}
\end{observes}
Generalizziamo ora il concetto di chiusura proiettiva a un generico sottospazio affine.
\begin{define}\textsc{Chiusura proiettiva di un sottospazio.}\\
	Dato $S\subseteq \aff{\kamp^n}$ un sottospazio affine con $S\neq \emptyset$, la \textbf{chiusura proiettiva}\index{chiusura!proiettiva} $\overline{S}\subseteq \proj[n]{\ }$ di $S$ è il sottospazio proiettivo generato da $S$. Esso ha dimensione $\dim \overline{S}=\dim S=m$.
\end{define}
\textbf{Equazioni cartesiane}. Se $S$ come sottospazio affine è dato in forma cartesiana dal sistema lineare $h\times\left(n+1\right)$ seguente:
\begin{gather*}
			Ax+b=0\qquad\left(\begin{array}{c}
		X_1\\
		\vdots\\
		X_n
	\end{array}\right)\in\aff{\kamp^{n}}\\
	\begin{cases}
	\begin{array}{l}
		a_{1,1}X_1+\ldots+a_{1,n}X_n+b_1=0\\
		\vdots\\
		a_{h,1}X_1+\ldots+a_{h,n}X_n+b_h=0\\			
	\end{array}
\end{cases}
\end{gather*} 
Allora $\overline{S}$ è descritto dal sistema lineare omogeneo  $h\times\left(n+1\right)$ in $\left(x_0,\ \ldots,\ x_n\right)$ seguente:
\begin{gather*}
	\left(A\mid b\right)x=0\qquad\left(\begin{array}{c}
		x_1\\
		\vdots\\
		x_n\\
		x_0
	\end{array}\right)\in\proj[n]{\ }\\
	\textcolor{red}{\circled{\ast}}\begin{cases}
		\begin{array}{l}
			a_{1,1}x_1+\ldots+a_{1,n}x_n+b_1x_0=0\\
			\vdots\\
			a_{h,1}x_1+\ldots+a_{h,n}x_n+b_hx_0=0\\			
		\end{array}
	\end{cases}
\end{gather*}
\begin{demonstration}
	Studiamo le dimensioni di $S$ e $\overline{S}$ usando i sistemi cartesiani appena definiti:
	\begin{equation*}
		\begin{array}{l}
			\dim S=\dim \kamp^n-\rk\left(A\right)=n-\rk\left(A\right)\\
			\dim \overline{S}=\dim \proj[n]{\ }-\rk\left(A\mid b\right)-1=\left(n+1-\rk\left(A\mid b\right)\right)-1=n-\rk\left(A\mid b\right)\\
		\end{array}
	\end{equation*}
	Per Rouché-Capelli vale $\rk A=\rk \left(A\mid b\right)$ in quanto $S\neq \emptyset$. In questo modo abbiamo dimostrato che $\dim \overline{S}=\dim S$.
\end{demonstration}
I \textit{punti impropri} del sottospazio affine $S$ sono dati da $\overline{S}\cap H_0$, con $\overline{S}$ la chiusura proiettiva di $S$ e $H_0$ l'iperpiano improprio. Dal sistema $\textcolor{red}{\circled{\ast}}$ si ha che $\overline{S}\cap H_0$ è dato da:
\begin{equation*}
	\begin{cases}
		\begin{array}{l}
			x_0=0\\
			a_{1,1}x_1+\ldots+a_{1,n}x_n=0\\
			\vdots\\
			a_{h,1}x_1+\ldots+a_{h,n}x_n=0\\			
		\end{array}
	\end{cases}
\end{equation*}
Esso corrisponde al sistema lineare omogeneo in $\kamp^n$ $Ax=0$ associato al sistema lineare $Ax+b=0$ che definisce $S$. In altre parole, $\overline{S}\cap H_0$ corrisponde al \textit{sottospazio vettoriale direttore} $W\subseteq \kamp^n$ e vale $\overline{S}\cap H_0=\proj{W}$ direzione di $S$. La sua dimensione per definizione di direzione è:
\begin{equation*}
	\dim\left(\overline{S}\cap H_0\right)=\dim S-1=\dim\overline{S}-1
\end{equation*}
\textbf{Equazioni parametriche}. Se $S$ ($\dim S=m$) è data in \textit{forma parametrica} e il sottospazio direttore $W\subseteq \kamp^n$ ha una base $\left\{v_1,\ \ldots,\ v_m\right\}$ (tali che $v_i=\left(V_{i,1},\ \ldots,\ V_{i,n}\right)\in\aff{\kamp^n}$ per un dato sistema di riferimento affine), posto $c\in S$ ricordiamo che l'espressione parametrica di $S$ è:
\begin{gather*}
	X=t_1v_1+\ldots+t_mv_m+c\quad t_1,\ \ldots,\ t_m\in\kamp\\
		\begin{cases}
		\begin{array}{l}
			X_1=t_1V_{1,1}+\ldots+t_mV_{m,1}+C_1\\
			\vdots\\
			X_n=t_1V_{1,n}+\ldots+t_mV_{m,n}+C_n\\			
		\end{array}
	\end{cases}
\end{gather*}
Allora, $\overline{S}$ è il sottospazio generato dagli $m+1$ punti \textit{indipendenti}:
\begin{equation}
	\begin{array}{cc}
		\left(0\colon v_{i,1}\colon\ldots\colon v_{in}\right)&i=1,\ \ldots,\ m
		\left(1\colon c_1\colon\ldots\colon c_n\right)
	\end{array}
\end{equation}
Pertanto, $\overline{S}$ ha descrizione parametrica:
\begin{equation*}
			\begin{cases}
		\begin{array}{l}
			x_0=t_0\\
			x_1=t_1v_{1,1}+\ldots+t_mv_{m,1}+t_0C_1\\
			\vdots\\
			x_n=t_1v_{1,n}+\ldots+t_mv_{m,n}+t_0C_n\\			
		\end{array}
	\end{cases}
\end{equation*}
Con $\left(t_0\colon\ldots\colon t_m\right)\in\proj[m]{\ }$.
\subsection{Impratichiamoci! Geometria affine e geometria proiettiva}
\begin{exercise}
	Sia $\kamp=\realset$. Allora, preso $\realset^n$ con la topologia euclidea e $\proj[n]{\realset}$ con la topologia quoziente, mostrare che $U_0$ è un aperto di $\proj[n]{\realset}$ e che $\funz{j}{\realset^n}{U_0}$ è un omeomorfismo.
\end{exercise}
\begin{solution}
	... % aggiungere soluzione
\end{solution}
\begin{comment}
\section{LEZIONI 31, 32 - DIRETTA}


%LEZ 31 
Vediamo un esempio di proiettività di $\proj[1]{\ }$.
\begin{example}
	Si consideri $\proj[1]{\kamp} = \kamp \cup \{\infty\}$ con $\infty=(0\colon 1)$. Sia $f$ una proiettività (dunque biunivoca) definita come:
	\begin{equation*}
		\funztot f {\proj[1]{\ }} {\proj[1]{\ }} (x_0\colon x_1) (ax_0+bx_1\colon cx_0+dx_1)
	\end{equation*}
	Si ha che $f(0\colon 1)=(b\colon d)$, mentre la sua controimmagine è $f(-b\colon a)=(0\colon 1)$; infatti, siccome le coordinate sono omogenee, basta porre $ax_0+bx_1=0$.\newline
	Sia $t=\frac{x_1}{x_0}$ la coordinata affine su $\kamp$, se $x_0\neq 0$ tutti i punti $(x_0\colon x_1)$ si possono scrivere come $(x_0\colon x_1)=\left( 1\colon \frac{x_1}{x_0} \right)=(1\colon t)$, il che corrisponde al punto $t\in\kamp$ . Vediamo ora come si comporta l'immagine grazie a queste osservazioni se $ax_0+bx_1\neq 0$:
	\begin{gather*}
		f(x_0\colon x_1)=(ax_0+bx_1\colon cx_0+dx_1)= \left( 1\colon \frac{cx_0+dx_1}{ax_0+bx_1} \right)  =  \left( 1 \colon  \frac{ x_0 \left( c+ d \frac{x_1}{x_0} \right) }{ x_0 \left( a+ b\frac{ x_1 }{ x_0 } \right)} \right)=\left( 1\colon \frac{dt+c}{bt+a}\right)
	\end{gather*}
	Dunque la proiettività $f$ corrisponde alla trasformazione:
	\begin{equation*}
		\funz F {\kamp\cup \{\infty\}} {\kamp\cup \{\infty\}}$ con $F(t)=\begin{cases}
			\frac{dt+c}{bt+a}, & t\in\kamp, \ t\neq -\frac{a}{b}\\
			\infty, & t=-\frac{a}{b}\\
			\frac{d}{b}, & t=\infty
		\end{cases}
	\end{equation*}
	Dove per $t=-\frac{a}{b}$ si ottiene $f(-b\colon a)=(0\colon 1)=\infty$, mentre la prima equazione è detta \textit{trasformazione lineare fratta}, che è definita sulla retta affine tranne dove si annulla il denominatore. \newline
	Notiamo che $F$ diventa un'affinità $\funztot F \kamp \kamp t {\alpha t+\beta}$ se e solo se il denominatore diventa una costante ponendo $b=0$, ovvero se è della forma $F(t)=\alpha$, il che significa che la proiettività fissa il punto all'infinito, ovvero $f(0\colon 1)=(0\colon 1)$, mentre la parte affine viene mandata in sè stessa.n\newline 
	Questo ragionamento si può vedere anche in dimensione superiore.
\end{example}

\section{Spazi proiettivi complessi}
% aggiustare l'introduzione
Digressione di topologia sugli spazi proiettivi complessi
\begin{remember}
	Nel caso di $\realset^n$ si è già visto che lo spazio proiettivo reale è un quoziente del tipo $\displaystyle \proj[n]{\realset}=\nicefrac{\realset^{n+1}\setminus\{0\}}{\sim}$ dunque è dotato in maniera naturale di una topologia.\newline
	Si può anche vedere come quoziente della sfera $S^n$ dove si identificano i punti antipodali grazie alla suriezione $\surr \pi {S^n} {\proj[n]{\realset}}$, è anche una varietà topologica compatta di dimensione $n$. Inoltre $\proj[1]{\realset}\cong S^1$ e abbiamo analizzato il piano proiettivo reale $\proj[2]{\realset}$.
\end{remember}
Anche nel caso complesso per $\proj[n]{\complexset}=\nicefrac{\complexset^{n+1}\setminus\{0\}}{\sim}$ si ha in maniera naturale una topologia quoziente data dalla topologia euclidea su $\complexset^{n+1}\setminus\{0\} \cong \realset^{n+1}\setminus\{0\}$.\newline
Vogliamo vedere che è una varietà topologica compatta di dimensione $\mathbf{2n}$. Questo perché mentre $\proj[n]{\realset}$ è localmente euclideo di dimensione n, si ha che $\proj[n]{\complexset}$ localmente è come $\complexset^n\cong\realset^{2n}$, dunque la dimensione topologica è $2n$.
\begin{itemize}
	\item $\proj[n]{\complexset}$ è \textit{connesso} perché è quoziente di $\complexset^{n+1}\setminus\{0\}$ che è connesso.
	\item $\proj[n]{\complexset}$ è \textit{compatto}: per avere la tesi lo si vuole vedere come quoziente di un compatto o immagine tramite una funzione continua di un compatto. Per la relazione di equivalenza, due vettori $z\sim w\iff \exists\lambda\in\complexset\setminus\{0\}\colon w=\lambda z$. Vogliamo ora restringerci alla sfera (la cui dimensione è data da $\complexset^{n+1}=\realset^{2n+2}\supset S^{2n+1}$) e dimostrare che ogni punto del quoziente è equivalente ad un punto della sfera. Per fare ciò si sfrutta la corrispondenza fra numeri complessi e reali e la norma:
	\begin{gather*}
		z_j=x_j+iy_j\implies  z=(z_1,\dots,z_{n+1})\in\complexset^{n+1}\longleftrightarrow (x_1,y_1,\dots,x_{n+1},y_{n+1})\in\realset^{2n+2}\\
		\begin{array}{ccc}
			\displaystyle \|z\|^2=\sum_{j=1}^{n+1} \lvert z_j\rvert ^2=\sum_{j=1}^{n+1}(\lvert x_j \rvert ^2 +\lvert y_j \rvert ^2) & \displaystyle\wedge & \displaystyle\lambda\in\complexset,\  \ \|\lambda z \| =\sqrt{\sum_{j=1}^{n+1}\lvert \lambda z_j \rvert ^2}=\lvert\lambda\rvert \| z\|
		\end{array} \\
		z\in\complexset^{n+1}\setminus\{0\}\implies \|z\|\neq 0 \ \wedge \lambda=\frac{1}{\| z\|} \implies \underbrace{\frac{1}{\|z\|}z}_{\in S^{2n+1}}\sim z \implies \pi(S^{2n+1})=\proj[n]{\complexset}
	\end{gather*}
	Nel caso reale i punti sulla sfera sono equivalenti solo se antipodali, nel caso complesso $S^{2n+1}$ invece $z,w\in S^{2n+1}$ si ha $z\sim w\iff \exists \lambda\in\complexset\setminus\{0\} \colon w=\lambda z$ siccome $w,z\in S^{2n+1}$ hanno norma unitaria, dunque $1=\| w\|=\|\lambda z\|= \ \lambda | \|z\|=|\lambda|$. Siccome ci sono infiniti numeri di norma $1$ in $\complexset$, allora ci sono infiniti numeri nella stessa classe, infatti i punti $\lambda z\in S^{2n+1}$ sono tutti equivalenti. 		
	\item $\proj[n]{\complexset}$ è $T_2$, basta dimostrare che $\pi_0$ è un'identificazione chiusa. Sia $C\subset S^{2n+1}$ un chiuso, allora $\pi_0(C)\iff \pi_0^{-1}(\pi_0(C))$ è chiusa in $S^{2n+1}$. In effetti la relazione di equivalenza su $S^{2n+1}$ viene da un'azione del gruppo $S^1=\{\lambda\in\complexset \mid |\lambda |=1\}$ rispetto al prodotto con elemento neutro $1$ quale $\funztot F {S^1\times S^{2n+1}} {S^{2n+1}} {(\lambda, z)} {\lambda z}$. Si ha che $F$ è un'applicazione continua e siccome $S^1\times S^{2n+1}$ è compatto e $S^{2n+1}$ è $T_2$ allora $F$ è chiusa. Dato un chiuso $C\subseteq S^{2n+1}$, allora data la controimmagine dell'immagine agisco sui punti di $C$ con tutti gli elementi di $S^1$, ovvero prendo tutte le orbite che intersecano $C$, ottenendo quanto segue:	
	$\pi_0^{-1}(\pi_0(C))=F(S^1\times C)\subseteq S^{2n+1} \implies \pi_0^{-1}(\pi_0(C))$ chiuso $\implies \pi_0(C)$ chiuso in $\proj[n]{\complexset} \implies \pi_0$ applicazione chiusa $\implies \pi_0$ identificazione. Pertanto $\proj[n]{\complexset}$ è anche quoziente di $S^{2n+1}$. Siccome $S^{2n+1}$ è un compatto in un $T_2$ per 
	%AAA ESERCIZIO TUTORATO QUOZIENTE CERCASI	
	e siccome $\pi_0$ è un'identificazione allora $\proj[n]{\complexset}$ è $T_2$.
	\item $\proj[n]{\complexset}$ è localmente euclideo di dimensione $2n$: per dimostrarlo sfruttiamo la costruzione
	%AAA COSTRUZIONE CERCASI NELLA SCORSA LEZIONE
	ovvero identificando degli aperti di $\proj[n]{\complexset}$ con $\kamp^n$, quale la famiglia $U_j\coloneqq\{z_j\neq 0\}=\proj[n]{\complexset}\setminus H_j$, con $H_j$ $j$-esimo iperpiano coordinato. Per semplicità lavoreremo con $j=0$.\newline 
	Si considera la proiezione al quoziente $\funz \pi {\complexset^{n+1}\setminus\{0\}} {\proj[n]{\complexset}}$ e la controimmagine $\pi^{-1}(U_0)=\{z\in\complexset^{n+1}\setminus\{0\}\mid z_0\neq 0\}$ aperto in $\complexset^{n+1}\setminus\{0\}$ perché abbiamo tolto un iperpiano, pertanto $U_0$ è aperto in $\proj[n]{\complexset}$. Questo vale per tutti gli $U_j$. Ricordando la costruzione 
	%AAA COSTRUZIONE DI J E \PHI CERCASI
	si considerano le mappe biunivoche e una inversa dell'altra $\funztot j {\complexset^n} {U_0} {(z_1,\dots,z_n)} {(1\colon z_1\colon \dots\colon z_n)}$ e $\funztot \oldphi {U_0} {\complexset^n} {(z_0\colon\dots\colon z_n)} {\left( \frac{z_1}{z_0},\dots,\frac{z_n}{z_0} \right)}$. Mostriamo che $j$ e $\oldphi$ cono omeomorfismi, in questo caso siccome sono già biunivoche e una inversa dell'altra basta dimostrare che sono entrambe continue.
		\begin{minipage}[t]{0.51\textwidth}\vspace{1pt}
		Per mostrare che $j$ è continua sfruttiamo il diagramma a lato, ovvero la fattorizzazione di $j$ in $\complexset^{n+1}\setminus\{0\}$ tramite:
		\begin{equation*}
			\widetilde{j}((z_1,\dots,z_n))=(1,z_1\dots,z_n)
		\end{equation*}
		e la proiezione $\pi$. Siccome $\widetilde{j}$ e $\pi$ sono continue allora anche $j$ che è la loro composizione lo è.
	\end{minipage}\hspace{-15pt}
	\begin{minipage}[t]{0.49\textwidth}\vspace{5pt}
		% https://q.uiver.app/?q=WzAsNCxbMSwwLCJcXGNvbXBsZXhzZXRee24rMX1cXHNldG1pbnVzXFx7MFxcfSJdLFswLDIsIlxcY29tcGxleHNldF5uIl0sWzIsMiwiVV8wIl0sWzEsMl0sWzEsMCwiXFx3aWRldGlsZGV7an0iXSxbMSwyLCJqIiwyXSxbMCwyLCJcXHBpIl1d
		\[\begin{tikzcd}
			& {\complexset^{n+1}\setminus\{0\}} \\
			\\
			{\complexset^n} & {} & {U_0}
			\arrow["{\widetilde{j}}", from=3-1, to=1-2]
			\arrow["{j}"', from=3-1, to=3-3]
			\arrow["{\pi}", from=1-2, to=3-3]
		\end{tikzcd}\]
	\end{minipage}\\
	\begin{minipage}[t]{0.51\textwidth}\vspace{10pt}
		Per la continuità dell'inversa $\oldphi$ invece si procede con la fattorizzazione nella controimmagine di $U_0$ tramite $\pi$ tramite una restrizione dell'inversa di $\pi$ e $\hat{\oldphi}$ che sostanzialmente ha le stesse coordinate di $\oldphi$
	\end{minipage}\hspace{-15pt}
	\begin{minipage}[t]{0.49\textwidth}\hspace{-15pt}\vspace{5pt}
		% https://q.uiver.app/?q=WzAsNCxbMiwwLCJcXHBpXnstMX0oVV8wKSJdLFszLDAsIlxcc3Vic2V0IFxcY29tcGxleHNldF57fW4rMSJdLFswLDIsIlVfMCJdLFszLDIsIlxcY29tcGxleHNldF5uIl0sWzIsMywiXFxvbGRwaGkiLDJdLFswLDIsInAiLDJdLFswLDMsIlxcaGF0e1xcb2xkcGhpfSJdXQ==
		\[\begin{tikzcd}
			&[-35pt]& {\pi^{-1}(U_0)}\subset \complexset^{n+1} &[-15pt] {} \\
			\\
			{U_0} &&& {\complexset^n}
			\arrow["{\oldphi}"', from=3-1, to=3-4]
			\arrow["{p}"', from=1-3, to=3-1]
			\arrow["{\hat{\oldphi}}", from=1-3, to=3-4]
		\end{tikzcd}\]
	\end{minipage}
\end{itemize}
\end{comment}
%Invece per la continuità di \phi che è l'inversa si procede con la fattorizzazione con \pi^{-1}  \hat{\phi} ha le stesse coordinate di \phi. Notiamo che sono tutti vettori, non c'è il quoziente! Dall'altra parte invece c'è la restrizione di \pi alla controimmagindi di U_0 in U_0. Chiaramente \hat{\phi} è continua: divido per norma diversa da 0, dunque sapremo dire che \phi è continua se   induco al quoziente una mappa continua. Osserviamo che p è un'identificazione: il quoziente \pi è anch'esso un quoziente da un'azione di gruppo (sempre vero per spazi proiettivi, vedasi esercizio già dato) per molitplicazione su C^n+1 meno 0, inoltre è un'azione per omeomorfismi, infatti fissato \lambda in c-0, l'applicazione theta_\lambda è continua (definizione di azione per omeomorfismi) l'applicazione per un elemento fissato è continua. Avevamo dimostrato che la proiezione è aperta (vedasi proprietà delle proiezioni) se il gruppo è finito è anche chiusa!. Dunque anche p restringendosi ad un aperto è ancora un'applicazione aperta (verifica per esercizio) Quindi sicocme è aperta allroa è un'identificazione, allora tale diagramma ci dice che la mappa \oldphi è continua.
%Riasusmendo, abbiamo dedotto che U_0 è un aperto di pn, che è omeomorfo a cn, dunque a r2n. Allo stesso modo lo si verifica per tutti i j grazie alle mappe \phi_j che divide per la j-esima coordinata
%In più chiarmanet pn è l'unione di tali n+1 aperti: le coordinate omoegenee ce n'è sempre una diversa da0, dunque ogni punto sta almeno in uno di questi aerti, dunque pnc è loclamente euclideo di dimensione 2n
%su un campo qualsiasi si ha sempre la mappa j    con kn senza aspetto topologio è sempre biunivoca. Nel cso reale rimane comuneuq vero 
%END PROOF PNC VARIETà TOPOLOGICA
%nel caso compatto infatti essere a base numerabile segue dalle ltre
%ogni apero u_i ha una base umerabile, ed è facile vedere he facendone l'unione è comunque a base numerabile

%cosa succede per la retta proiettiva complessa? è una varietà topologica compatta di dimensione 2, dunque è una superficie topologica compatta, che abbiamo già classificato! che superficie sarà? l'aperto U_0 è comeomorfo a c e poi un solo punto, dunque la retta proiettiva complessa è un piano con in più solo un punto. Vediamo che è omeomorfo a S^2 e vediamo poi la differenza con il piano proiettivo reale. Si chiama anche sfera di riemann
%tutto il piano richiuso su se stesso ed un punto aggiunto, pensato come punto all'infinito
%Costruiamo l'oeomorfismo, definiamo la mappa $\funz F {S^2}  $ nella sfer in r3 si considera il polo nord ed il polo sud, la proiezione sterografica da s2-n a r2 è biunivoca ed un omeomorfismo. da r2 si va in c tramite la solita identificazione con i, essa è anche un omeomorfismo. si manda c in U_0 tramite la mappa j, anch'essa omeomorfismo. Composizione di 3 omeomorfismi: proeiz stero, identificazion standard e infine 
%definiamo f ristretto in questo modo, trmaite la composizione di tutte queste mappe. Voglio però la definizione nel polo nord per avere  tutto. MAndo il polo nord nel punto unito a U_0 ovvero (0\colon 1) per avere una biezione
%gratis: F biunivoca ed è continua in S^n-n. Manca la cotinuità su un aperto contenente il polo nord. S^2 cpt, t-2 implica chiusa, ed essendo F binuivoca allora è un omeomorfismo. 
%la coniuità sull'aperto è un conto
%PAUSA 
%anche la restrizione al plo sud (intorno di N!) si scrive in temrini di proiez stereografica dal polo sud. Vogliamo mettere in relazione la proeiz stereografica dal polo sud.
%Scriviamoci velocemente le due proiezioni stereografiche da N e d S. Fisso un punto della sfera, allora tramite la proiezione stereografica (retta uscente da N o da S che nterseca il pinao xy) scriviamo le due semirette con vettore che dà la direzoine P_0-N. Scirivamo anche la semiretta dal polo sud, cambia solo che il punto ha coordinate 0,0,-1. Intersecando con il piano xy (z=0), nel primo caso ottengo 1+t(z_0-1)  (torna tutto perché è defiita tranne in N, dunque il denominatore non è nulla), nel secondo -1+t stesse motivazioni. Dunque l'immagine di r2 si ottiene sosituendo i valorei di x e y nella semiretta. Nel caso di N viene (,), mentre nel cso della proiezione dal polo sud veien (,). Adesso considero i punti complessi associati a tali numeri tramite la mappa standard.
%Si ha la relazione w=\frac{}{} e viceversa. Verifichiamola. Mi restringo alla sfera meno entrambi i poli, dunque sono entrambe definite le proiezioni. duneuq ho c-0 (entrambi non nulli). La relazione è di inversione e coniugio.
%Verifico la relazione con un conticino: scrivo la frazione, scrivo la sua def complessa, raccolgo 1+z_0 e lo porto a numeratore, moltiplico sopra e sotto per z_0+i_0\neq 0 perché ho tolto l'asse z dalla sfera, dalla moltiplicazione si ottiene al denominatore la norma, ed essndo sulla sfera viene quanto voluto, bast semplificare.
%questo ci dice che f ristretto a s^2-s,n è data da una composizione di mappe: proiezione stereografica da N, identificazione standard con C, mappa j in U_0. Ora al posto di w posso scrivere 1/\overline{u}, che per il quoziente è pari a  grazie alle coordinate omogenee: funziona come un'eliminzione dell'indeterminazione: anche se u=0 non è un problema perché ottengo 0:1. Nella proz ster da S il punto che va nello 0 è il polo nord, e questo ci dice che F si estende in maniera continua su S^n-s mandando P nella proiez ster da S , poi ho la mappa j_1, con tutti omeomorfisi. Dunque F è continua perché composizione di omeomorfismi.
%FINE DIM SFERA OMEO A RETTA PROJ COMPLESSA

%differenza fra p1c e p2r
%sono entrambe suup cpt e sono entrambe compattificazioni del piano, ma lo fanno in modo diverso!
%p1c=c (pino) unito ad un solo punot, ottenendo la sfera
%mentre p2r è pari al pinao r2 unita alla retta impropria p1r. Topologicamente si unisce una copia della circonferenza: la copia di r2 è la copia del disco (la parte interna) mentre s1 è il bordo che ha l'identificazione dei punti antipodali

%---------------------------------------------------------------------

%torniamo al campo qualsiasi
%cos'è il birapporto
%si fa sulla reSotta proiettiva
%\begin{define}
%si considerano 4 punti
%voglio i primi 3 distinti. dipende dall'ordine, si può definire se i primi 3 sono distinti ed è un numero associato a tali 4 numeri ordinati   dove y_0\colon y_1 sono le coordinate di P_4 nel sistema di riferimento proiettivo in cui P_1 è il primo punto fondamentale, P_2 il secondo e P_3 è unita
%\end{define}

%OSS y_0, y_i\in\kamp, dnque il birapporto è un elemento del campo ed è infinito se y_0=0
%\beta è ben definito perché i 3 punti sono distinti, dunque sono in posizione generale (nella retta pro)j essere in pos gen è equivalente ad essere distinti) \implies \beta è unico (teo sul riferimento proiettivo e pos generale)
%per ipotesi i primi 3 sono distinti, nesune ipotesi su P_4. Vediamo cosa succede quando è uguale a uno degli altri punti: catena di equivalenze che porta a dire che il birapporto è pari a 0
%quindi ho 3 valori speciali del birapporto che ottengo solo quando il quarto punto coincide con uno specifico degli altri 3. contronominale
%viceversa se prendo uno scalare che non è fra quei valori e se prendo P_4=(1\colon a) allora \btea=a, dunque \beta assume tutti i valori possibili in \kamp.

%GOAL
%1)come si calcola il birapporto in un sistema di riferimento qualsias e non quello della def
%2)come mai il birapporto  interessante? quale proprietà lo caratteerizza?

%\begin{theorema}
%	Supponiamo di avere un riferimento fissato
%	Allora la formula del birapporto è data da (quoziente di determinanti 2x2)
%\end{theorema}
%comenti: birapporto viene proprio da quest formula come nome. al nume ho 1,4 e 23 (esterni e interni) al den intermedi
%ben deifnita: supponiamo di moltiplicarlo per uno scalare: succede sia al denomiatore sia al denominatore, dunque si semplifica, dunque non dipende dalla scelta delle coo omogenee. Al denominatore il primo det è sempre \neq 0 perché i punti sono distini. Il secondo det è 0 quando P_2=P_4, che torna con quanto definito prima
%\begin{demonstration}
%	Conto di algebra lineare.
%	scrivo \lambda e roba come combinazione lineare. porto dentro gli scalari. Per la costruzione della dimostrazione nella costruzione di sistema riferimento per punti speciali, essa è la base che dà il riferimento proiettivo con p_1 e P_2 come punti fond e P_3 unità: riscalo in modo tale che l'ultimo sia la somma dei vettori. Per ottenere P_4 scrivo P_4 come combinazione lineare di questi 2 vettori: \exists c,d\in\kap (\lambda_4,\mu_4)=c(a\lmbda_1, a\mu_1)
%	dunque P_4 ha coordinate c:d nel nuovo riferimento proiettivo. c e d sarebbero gli y_0 e y_1
%	notazione comoda per i determinanti
%	determinamo prima a e b: sono dati dalla relazione con\lambda_3 e \mu_3, esso diventa un piccolo sistema linearedi 2 equazioni in 2 incognite (\lambda e \mu come coeff e a e b come incognite) Risolvo il sistema con Cramer: sotto la matrice dei coeff del sistema (invertibile) 
%	faccio lo stesso con c e d
%	sostituisco il valore di a e b trovato
%	uso di nuovo Cramer: metto i termini noti , seconda colonna dei coeff e sotto la mat dei coeff del sistema. Ora il det è lineare in ogni colonna, dunque i delta escono da ogni colonna. posso riscriverli riordinando gli indici: scambio ordine colonne cambio segno, ma essneodo frazione e facendolo per entrambe il segno è invariato
%	faccio lo stesso per d
%	dunque il birapport d/c è proprio il prodotto dei determinanti
%\end{demonstration}

%TIPS AND TRICKS?
%OSS quando i 4 punti hanno entrmbi la prima neq 0 o la seconda, c'è una fromula più semplice che semplifica i conti
%questo perché per le coordinate omogenee posso mettere l'1, dunque scrivendo il determinante viene raccogliendo i \lambda 
%si usa la linearità nella prima colonna moltipl e dividendo per lamba 1 e nella seconda colonna per lambda 4, dunque vine
%idem per gli altri determinanti (semplificazione dei lambda)
%ESERCIZIO



%def birapporto e come si calcola
%manca la proprietà per cui è importante, si vedrà nella prossima lezione

\
\
\
\
\	
\
\
\
\
\
\
\
\

%LEZ 32
%
%Sia $\proj{\ }$ una retta proiettiva ed i punti $P_1, P_2, P_3, P_4$ con $P_1, P_2, P_3$ distinti, allora definisco $\beta(P_1, P_2, P_3)  \\\
%Quando una coppia di quattro punti è 
%Date due rette proiettive PV e PV' e P_1 in PV distinti e Q in PV' distinti, quindi esiste sempre ed è unica una trasformazione proiettiva tale che f(P_i)=Q_i
%Posso farlo anche con quattro punti? Posso usare il birapporto per vederlo!
%\begin{theorema}
%	Sia PV e PV' due rette proiettive e siano i punti di cui  i primi 3 distinti idem. Allora $\exists $ trasformazione proiettiva tale che f(P_i)=Q_i, \forall  \iff il birapporto di queste quaterne è lo stesso
%\end{theorema}
%PREMESSA PRIMA DELLA DIM DELL'EQUIVLENZA: si usa nella dimostazione
%consideriamo solo i primi 3 pt (scelgo rappresentanti) sia per PV sia per PV'
%base che dà rif proiettivo con i punti coordinati e punto unità
%scegliamo anche un rappresentante per P_4: v_4=. Allora in tale rif proiettivo è P_4=(a\colon b) e il birapporto è 
%Allo stesso modo si procede con Q_i, per cui 
%Siccome P_1, P_2, P_3 e Q_i sono in posizione generale, allora esiste ed è unica trasformazione proiettiva che porta f(P_1)=Q_i
%l'abbiamo costruita come indotta da un'applicazione lineare che porta la base v in quella di v'
%Cosa succede a v_4 tramite \phi fra V e V'?
%la trasf che porta i primi 3 esiste ed è unica, quindi manda p_4 dove deve. Scrivo dunque questa unica trasf fino a 3 e vediamo cosa susccede a P_4
%DIMOSTRAZIONE
%\impliesdx ddeve essere proprio la f nica, per cui   nel riferimento in cui, per cui il rapporto di Q_i è dato dal rapporto che è pari
%\impliessx se il birapporto dei Q_i=b/a  Distinguiamo i casi in cui il birapporto è in \kamp o è infinito     se a=0, allora Q_4 è il punto 0\colon 1 e coincide con f(P_4)
%il senso sta nel fatto che i primitre pt distinit la traasf exists !      stessa cosa che dre che i birapporti sono uguali
%
%
%In particolare per le proiettività esiste una proiettività che manda i p_i nei q_i se e solo se il birapporto dei P-i è pari al birapporto dei Q_i
%OSS  
%il birapporto misura l'equivalenza proiettiva su punti di una retta proiettiva
%
%
%è una relazione di equivalenza su \mathcal{S}  (verifica per esercizio) e tramite il birapporto possiamo dire che il quoziente di S per l'equiv proj, ovvero le classi di equiv proj di 4 pt distinti e ordinati in P1 sono in corrispondenza biunivoca con K\0,1 tramite il birapporto:  ad ogni quat associo il birapport (associazione suriettiva, per ogni el nel campo trovo quaterna pt distini che dà bir   due quat equivalenti se e solo se ha lo stesso birapporto, dunque quozientando diventa iniettiva e dunque biunivoca)
%Qundo ho 4 pt in sp proj di dimension maggiore di 1 in generlae il birapporto non è definito a meno che i punti non siano allineati su una retta proiettiva r
%
%
%esempio di come si usa il bir di come si studia l'equiv proj di 4 pt in pP2
%si considerano 2 quaterne di pt distinti: P_1, P_2, P_3, P_4 e 
%di ciascuna delle due quaterne vedo in che posizione sono: pos generale: a tre a tre non  allineati
%oppure se non sono in pos generale allora ce ne sono 3 allineati: fa differenza se il quarto punto sta nella retta o meno
%terza possibilità: tutti e quattro i pt allineati
%Nel secondo caso ci sono tutte le possibili permutazioni fra pt allineati
%Se i pt P sono proj equiv ai pt Q, tail posizioni devono essere mantenute: le proiettività mandano rette in rette, pos gen in pos gen
%se si verificano casi diversi per i pt P e Q si può dire che non sono proj equiv
%Supponiamo di essere nel caso 1, ovvero P_i e Q_i in pos generale: sono proj equiv oppure no? pos gener esiste unica proj fra le due (siamo nel numero giusto n+2=4)
%Caso 2: 3 allineati: la proj esiste ed è unica, ma non ci interessa l'unicità ora, basta che ce ne sia una (sono gli stessi 3 alineati con gli stessi indici) Sono proj equiv: mostriamolo con un conto: per costruire proj costruisco trasf kin inveritible: retta nella retta   sappiamo dal ris sulle trasf proj esiste sempre trasf proj che mappa i pt nei pt, soprattutto sappiamo scriverl. HìDà un'appl linneare in piani vettoriali: posso estenderla in trasf lin da un vettore che rappresent P_4 in Q_4.   Estendo in una proiettività di P2 che fa questo lavoro sostanzialmente
%Scegliamo P_i=[v_i]  tali che  posso farlo perché sono allineati: allora    è una base di K3 perché sono indipendetni dato che v_4\not in r  Idem per Q
%Definiziamo \phi sulla base lineare tale che  impongo le immagini dei rappresentanti, allora per la linearità di phi è rispettato anche 4. dunque la proj indotta da phi tilde è una proiettività che manda pi in qi
%Caso 3: osservazione: pt allineati allora è definito il loro birapporto se i pt sono proj equiv allora chiamata f la proiettività che manda i 4 pt nei 4 pt allora necessariamente manda r in s, dunque la restrizione alle due rette è una trsformazione proj che potta pi in qi per ogni i. COsa possiamo dire sui due birapporti? Ho la dimostrazione di prima /pt proj equiv allora stesso birapport/Vale il viceeversa: allora dal teorema esiste g trasformazione proiettiva che manda pi in qi per ogni ima tae g si estende facilmente ad un proiettività di p2, in maniera non unica /esercizio/ le proj si costruiscono dalle app lin: r piano vett in k3 mentre g corriponde app line fra i due piani: estendo l'app lin ad automorfismo di k3 sceglienod basi che hnno 3 vettori nel piano ed uno fuori ed uso ilterzo vettore per estenderla
%Dunqeu in questo caso i pt sono proj equiv sse hanno lo stesso birapporto BIrapporto utie anche in dimensione superiore se stanno su una rettaho così esteso l'app line per la trasf 
%
%
%FINITO LE BASI DELLO SPAZIO PROIETTIVO
%LUNEDì E MARTEDì VACANZA
%
%
%
%STUDIO DELLA GEOMETRIA DI RETTE O CURVE NEL PIANO PROIETTIVO
%OGGI RETTE E UN PO' DI IPERPIANI
%Una retta in P2 ha equazione z_0x_0+a_i   eq lin omo elle coo omogenee, dunque è determinata dai coefficienti a_0 a_1 e a_2 dell'eq con la proprietà che devono essere non tutti nulli e in più fissati i coeff l'eq non è univocamente determinata: è determinata a meno di costante molitplicativa non nulla. Dunque si può associare a r un punto del piano proiettivo dato esattamente dai coefficienti delle coordinate omogenee, per cui si ha una corrispondenza binivoca fra le rette in P2 e P"(che non va pensato come lo stessso piano)   funziona bene con p2: le coo omo si comportano proprio come i coeff dell'equazione
%
%ESEMPIO NEL CASO REALE
%quando pensiamo a P2 d destra cpme lo spazio che parametrizza le rette in P2, lo denotiamo conP2* e lo chiamiamo piano proiettivo duale
%in prima istanza questo significa semplicemente  interpretazione punto del duale visto come ppunto associato ad una retta
%collgamento co lo spazio vettoriale duale        forma lineare
%prima costruzione come corrispondenza
%
%un FASCIO  di rette in P2 è il set delle rette di equazione    con l_1 ed l_2 sono rette fissate
%i coeff variano in P1
%
%esempio
%
%collezione delle rette la cui eq si ottiene come combinazione lineare delle due
%è una famiglia di rette con lambda e mu come parametri
%
%
%famiglia dir ette: ad ogni retta un punto del proj duale: a cosa corrisonde questa costruzione di prendere il fascio?  Il fascio F corrisponde alla retta passante per i punti corrispondenti a l1 e a l2, perché quando guardo i coeff dell'eq ho l_1: a_0    \squigglyarrow  l_2:      il fascio F generato da queste due rette ha equazione e quindi eusto corrisponde esattamente ai pt nel duale scritti come (lambda    ), ovvero la retta per i due punti sopra
%
%Esempio
%
%
%
%nel duale mi dà la retta passante per i due punti
%
%
%tornando dalla situazione generale nel fascio
%OSS dure rette distinte nel piano si intersecano in un punto solo: deve stare in ogni retta del fascio perché lì si annullano, dunque la cl rimane 0
%Ogn retta del fascio passa per P e P è l'unico punto comunea tutte le rette del fascio
%Viceversa ogni retta per P appartiene al fascio
%Ciò significa che F è la famiglia delle rette per il punto fissato P PUZZA DI SPAZIO VETT CON VETT NULLO P 
%
%P è detto punto base del fascio
%
%ESEMPIO
%
%
%Voglio riprendere la descrizione parametrica del fascio, in questo esempio il fascio F corrisponde alla retta nel piano proiettivo duale per i punti 
%adesso la voglio scrivere in equazione cartesiana nelle coordinate a_    nel piano proj duale
%scrivere la retta vuol dire fare il determinante formale
%
%
%sorpresa sorpresa, i coeff della retta sono esattmante le coo di P di intersezione delle due rette
%infatti vale in generale che fissato un punto base, il set delle rette per p in P2 è un fascio di rette, corrispondente a una retta nel piano proiettivo duale, allora se P ha coordinate c_0   la retta corrispondente nel piano proj duale ha equazione c_0a_0 è   =0, questo perché dta una retta r qualsiasi di equazione a_0x_0   =0, il punto P appartiene a r sse vale l'equazione c_0    sostituisco le coordinate di P e trovo quell'eq:   condizione perchè appartenga alla retta   o al fascio, collezione delle rette per p     fisso p e retta che varia (duale )   retta r fissata e p che varia. Ma l'eq è la stessa
%di fatto il fascio si scrive in forma parametrica scegliendo due rette specifiche che passano per p
%
%ESEMPIO IN P2R
%scrivere in forma parametrica il fascio delle rette per il punto base p di coordinate (1\colon -1 \coln4)   scelgo 2rette a caso che passano per il punto p
%se faccio variare lambda e mu ottengo tutte le rette di p2 che passano per questo punto
%
%OSS SULL'INTERPRETAZIONE AFFINE
%se vedo r2 come   ed F fascio di rette proj con punto base P in p2, allora ho 2 possibilità: punto base proprio o all'infinito
%Se P è un punto proprio, allora P\in\kamp^2 e F corrisponde a meno di passare alla chiusura proj dalla retta proj a quella affine corrisponde alle rette afifni in k2 per il punto P
%Se p invece è un punto improprio, corrisponde ad una direzione di rette nel piano affine e F corrisponde a tutte le rette affini che hanno questa direazione fissata, ovvero è un fascio di rette parallele
%differenza solo fra punto proprio e improprio
%
%
%PIANO PROIETTIVO DUALE
%OSS
%P2 è il proeiettivizato di K3, e sappiamo che K3 duale è lo spazio delle forme lineari \alpha su K3 della forma ax_0+a_1    numero le coordinate da 0 a 2, per il resto è uguale a Geo1
%Quando considero la retta r di equazione a_0x_0   =0, allora stiamo dicendo che r /è il nucle di tale forma lineare \alpha / corrisponde a un piano vettoriale che è esattamente il nucleo di \alpha 
%r ha tale eq: r è proietivizzato da piano vett /da def sott/ tale sott è nucleo
%il punt corrispondente ad \alpha corrisponde alla retta r    \alpha è forma lineare non nulla determinata a meno di multipli, dunque alpha non identicamente nulla e determinata a meno di multipli
%
%base che induce le coordinate proiettive a_0   su p2 duale, quindi tale interpretazione astratta diventa quella numerica fissando la base naturale delle forme lineari e le coo proiettive associate  lpha come cl della base a_0x_0   i coeff sono le coordinate omogenee
%
%INCISO
%costruzione che si generalizza: dato V spazio vettoriale e PV spazio  proiettivo associato a V, abbiamo anche V^* spazio vettoriale duale, ovvero le forme lineari, allora il proiettivizzato del duale si dice spazio proiettivo duale di PV e si indica con l'asterisco fuori (!!)
%di particolare ha una corrispondenza biunivoca fra i punti dello spazio proiettivo duale e gli iperpiani di PV   data da [\alpha]  associo il proiettivizzato del nucleo di alpha in PV
%In coordinate: a un iperpiano di equazione a_0x_0+a_nx_ associamo il punto (a0\colon   a_n)
%/così come associo ad un vettore i coefficienti della base/
%
%
%CONICHE IN p^2(K)
%Abbiamo già visto a Geo1 cosa sono le coniche in R^2, classificandole a meno di rototraslazione
%generalizziamolo al caso del piano proiettivo. Il passaggi dalla retta alla conica   grado 1 (rette) a eq polinomiali di grado 2 Dobbiamo tenere conto del fatto che nel piano proiettivo le coordinate sono omogenee
%Consideriamo k[x_0,   x_2] sono i polinomi in x_0   e coefficienti in K. Se F è un polinomio qualsiasi, 'equazione non dà una condizione ben definita in P2
%quali sono le eq che ha senso studiare in p2: i polinomi omogenei
%definizione per numero qualsiasi di variabilitutti i monomi con coeff non nulla hanno lo stesso grado
%esempio
%omogeneo di grado 3
%non omogeneo  /grado 3, 2, 0/   PROPRIO PESSIMO
%def significativa con più variabili
%
%forma lineare è omogenea quando non ha termine noto
%
%i polinomi omogenei hanno la proprietà di 		 infatti è vero per ogni monomio
%
%
%tornando al pinao proiettivo duale con le coordinate omogenee x_0   e consideriamo F un polinomio omogeneo nelle coordinate omogenee
%Se ho un punto P di coordinate c_0   in P2, allora tutte le altre coordinate, possibili scelte per le coordinate di P, sono \lambda c_0   con \lambda\in\kamp\setminus    e quando a valutare F con lambda ottego     quesot mi dice che F si annulla in una scelta di ccordinate se e solo se si annulla in qualsiasi scelta di coordinate
%Quindi l'equazion F(a condizione)=0 è ben posta in P^2
%
%esempio
%è un subset ben definito del piano proiettivo
%
%le coniche sono esattamente i sottonisiemi di P2 che sono dati da polinoi omogenei di grado 2
%
%
%
%def i pol + va omogeneo
%eq polinomiali che hanno senso in P^2 per coo omogenee: pol omo
%dunque eq ben definite
%
%
%DEFINIZIONE
%curva algebrica piana proiettiva è data da un polinomio omogeneo F a coefficienti nel piano K nelle coordinate omogenee su P". Tale polinomio si considera a meno di multipli (si consdiera come un'equazione)
%
%la curva C ha un supporto che è il subset di P^2 dove F si annulla
%assieme al supporto vogliamo ricordarci dell'equazione
%F è l'eq di C
%La morale è che oltre al supporto vogliamo ricordarci dell'equazione a meno di multipli. Questo perhcé eq determina il supporto, ma in generale non vale il contrario  (vedasi ellissi immaginario)
%le curve che vogliamo studiare sono subset di P2 sono luoghi di zeri di pol omogenei, ma nella nozione di curva inseriamo anche l'eq F con multpli . Algebrica erché polinomi, piana perché siamon nel piano, proiettiva perché piano proiettivo
%
%
%ESEMPIO: retta
%
%Il grado della curva è il grado dell'equazione F. Dunque curve di grado 1 sono le rette proiettive. Le consiche che vogliamo studiare sono le curve di grado 2
%
%Terminologia: la curva C è reale o complessa se siamo in p2r o in p2c
%
%



