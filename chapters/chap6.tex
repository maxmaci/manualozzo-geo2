% SVN info for this file
\svnidlong
{$HeadURL$}
{$LastChangedDate$}
{$LastChangedRevision$}
{$LastChangedBy$}

\chapter{Successioni}
\labelChapter{successioni}

\begin{introduction}
‘‘BEEP BOOP INSERIRE CITAZIONE QUA BEEP BOOP.''
\begin{flushright}
	\textsc{NON UN ROBOT,} UN UMANO IN CARNE ED OSSA BEEP BOOP.
\end{flushright}
\end{introduction}

\section{Numerabilità}
\begin{define}
Un insieme $X$ è \textbf{numerabile}\index{numerabile} se è finito oppure esiste una biezione tra l'insieme $X$ e i naturali $\naturalset$.
\end{define}
\begin{define}
Uno spazio topologico $X$ è \textbf{a base numerabile}\seeonlyindex{a base numerabile}{assioma di numerabilità!secondo} se esiste una base $\basis$ della topologia tale che $\basis$ sia numerabile. %sia di cardinalità numerabile??
Si dice anche che $X$ soddisfa il \textbf{secondo assioma di numerabilità}\index{assioma di numerabilità!secondo}.
\end{define}
\begin{define}
Uno spazio topologico $X$ è \textit{primo-numerabile}\seeonlyindex{primo-numerabile}{assioma di numerabilità!primo} se ogni punto ammette un sistema fondamentale di intorni che sia numerabile. Si dice anche che $X$ soddisfa il \textbf{primo assioma di numerabilità}\index{assioma di numerabilità!primo}.
\end{define}
\begin{observe}~{}
\begin{enumerate}
\item Il secondo assioma di numerabilità implica il primo.
\item Se $X$ è finito, $X$ soddisfa sempre i due assiomi.
\item Se $X$ è spazio metrico, $X$ è sempre \textit{primo-numerabile}.
\item Se $X$ è \textit{a base numerabile}, ogni sottospazio $Y$ di $X$ è \textit{a base numerabile}. In particolare $Y$ è primo-numerabile.
\item Se $X$ e $Y$ sono \textit{a base numerabile}, allora $X\times Y$ è \textit{a base numerabile}. In particolare $X\times Y$ è primo-numerabile.
\item Non è vero che il quoziente di $X$ spazio \textit{a base numerabile} (o \textit{primo-numerabile}) è sempre \textit{a base numerabile} (o \textit{primo-numerabile}).
\end{enumerate}
\end{observe}
\begin{demonstration}~{}
\begin{enumerate}[label=\Roman*]
	\item Se $X$ ha base numerabile $\basis$ e $x\in X$, allora $\left\{A\in \basis\mid x\in A\right\}$ è un sistema fondamentali di intorni di $x$ ed è chiaramente numerabile.
	\item Ogni base e sistema fondamentale di intorni contiene necessariamente un numero finito di elementi.
	\item Preso $x\in X$, allora $\left\{B_{\nicefrac{1}{n}}\left(x\right)\right\}_{n\in\naturalset}$ è un sistema fondamentale di intorni ed è numerabile.
	\item Se $\basis$ è una base numerabile per $X$, $\left\{A\cap Y\mid A\in \basis\right\}$ è base numerabile per $Y$.
	\item Se $\basis_X$ è una base numerabile per $X$ e $\basis_Y$ base numerabile per $Y$, allora $\left\{A\times B\mid A\in \basis_X,\ B\in\basis_Y\right\}$ è base di $X\times Y$ numerabile: il prodotto cartesiano di due insiemi numerabili rimane numerabile.
	\item La contrazione di $\integerset$ in $\realset$ ad un punto, cioè il quoziente $\nicefrac{\realset}{\integerset}$, non è primo-numerabile nè tanto meno a base numerabile, pur essendo $\realset$ a base numerabile in quanto metrico\footnote{Nelle ‘‘Note aggiuntive'', a pag. \pageref{dimostrazionenonnumerabilità}, si può trovare la dimostrazione di ciò.}.
\end{enumerate}
\end{demonstration}
\begin{example}
	$\realset$ con la topologia Euclidea è \textit{a base numerabile}. Presa infatti:
	\begin{equation*}
		\basis=\left\{\left(a,\ b\right)\mid a,\ b\in \rationalset,\ a<b\right\}
	\end{equation*}
	\begin{itemize}
		\item È numerabile (è definita con i razionali $\rationalset$, che sono numerabili)
		\item È una base perché, dati $x,\ y\in\realset,\ x<y$:
		\begin{equation*}
			\left(x,\ y\right)=\union_{\substack{a, b\in\rationalset\\ x<a<b<y}}\left(a,\ b\right)
		\end{equation*}
	\end{itemize}
\end{example}
\begin{proposition}
Sia $X$ \textit{a base numerabile}. Allora ogni ricoprimento aperto di $X$ ammette un sottoricoprimento numerabile.
\end{proposition}
\begin{demonstration}
Sia $\mathcal{A}$ un ricoprimento aperto di $X$, $\basis$ una base numerabile per $X$ e $x\in X$. Allora $\exists U_x\in\mathcal{A}$ tale che $x\in U_x$. Essendo $\basis$ base, $\exists B_x\in\basis$ tale che $x\in B_x\subseteq U_x$.\\
Abbiamo così determinato un sottoinsieme numerabile della base $\basis$:
\begin{equation*}
\widetilde{\basis}\coloneqq\left\{B_x\mid x\in X\right\}
\end{equation*}
Allora esiste in particolare $E\subseteq X$ numerabile tale che:
\begin{equation*}
\widetilde{\basis}\coloneqq\left\{B_x\mid x\in E\right\}
\end{equation*}
Se consideriamo ora $\widetilde{\mathcal{A}}\coloneqq\left\{U_x\mid x\in E\right\}$, notiamo che:
\begin{itemize}
	\item $\widetilde{A}\subseteq A$.
	\item $\widetilde{A}$ è numerabile perché lo è $E$.
	\item $X=\union_{B_x\in\widetilde{\basis}}B_x=\union_{x\in E}B_x\subseteq \union_{x\in E}U_x $.
\end{itemize}
Segue che $\widetilde{\mathcal{A}}$ è un sottoricoprimento numerabile di $A$.
\end{demonstration}
\begin{define}
Uno spazio topologico $X$ si dice \textbf{separabile}\index{spazio topologico!separabile} se contiene un sottoinsieme $E$ \textit{denso} e \textit{numerabile}.
\end{define}
\begin{examples}~{}
	\begin{itemize}
		\item Se $X$ è numerabile, allora è separabile perché l'insieme stesso è un sottoinsieme numerabile e denso.
		\item $\realset^n$ con la topologia euclidea è separabile perché si ha $E=\rationalset^n$ denso in $\realset^n$.
	\end{itemize}
\end{examples}
\begin{lemming}\label{basenumseparabile}
Se $X$ è \textit{a base numerabile}, allora è \textit{separabile}.
\end{lemming}
\begin{demonstration}
	Sia $\basis$ una base numerabile. Per ogni $U\in\basis$ sia $x_U\in U$ un punto e sia:
	\begin{equation*}
		E=\left\{x_U\mid U\in \basis\right\}
	\end{equation*}
\begin{itemize}
	\item $E$ è numerabile perché lo è $\basis$: abbiamo preso un punto per ogni elemento della base numerabile.
	\item $E$ è denso: se $A\subseteq X$ è aperto \textit{non} vuoto, allora $\exists U\in\basis$ tale che $x_u\in U\subseteq A\implies x_U\in A\implies E\cap A\neq \emptyset$.
\end{itemize}
\end{demonstration}
\begin{proposition}
	Se $X$ è spazio metrico, $X$ è sempre primo-numerabile ed è:
	\begin{center}
		\textit{a base numerabile} $\iff$ \textit{separabile}
	\end{center}
\vspace{-6mm}
\end{proposition}
\begin{demonstration}~{}\\
$\impliesdx$ Sempre vera per ogni spazio anche non metrico (lemma \ref{basenumseparabile}).\\
$\impliessx$ Sia $E\subseteq X$ sottoinsieme numerabile e denso e consideriamo:
\begin{equation*}
	\basis=\left\{B_{\nicefrac{1}{n}}\left(e\right)\mid e\in E,\ n\in \naturalset\right\}
\end{equation*}
Questo insieme è numerabile: mostriamo che sia una base. Per far ciò fissiamo $U\subseteq X$ aperto e prendiamo $x\in U$: vogliamo trovare un aperto di $\basis$ contenuto in $U$ contenente $x$.\footnote{Gli elementi della base sono già aperti banalmente. Per l'arbitrarietà di $x$, troviamo un ricoprimento aperto di $U$ costituito da aperti di $\basis$ contenuto interamente in $U$, cioè $\displaystyle U=\union_{i\in I}B_i$.}\\
Sia $n\in\naturalset$ tale che $B_{\nicefrac{1}{n}}\left(x\right)\subseteq$. Cerchiamo opportuni $e\in E,\ m\in\naturalset$ tale che:
\begin{equation*}
x\in B_{\nicefrac{1}{m}}\left(e\right)\subseteq B_{\nicefrac{1}{n}}\left(x\right)\subseteq U
\end{equation*}
Consideriamo la palla $B_{\nicefrac{1}{2n}}\left(x\right)$. Siccome $E$ è denso in $X$, $\exists e\in E\cap B_{\nicefrac{1}{2n}}\left(x\right)$.\\
Prendiamo ora la balla $B_{\nicefrac{1}{2n}}\left(e\right)\in\basis$:
\begin{itemize}
	\item \textit{contiene} $x$ perché se $e\in B_{\nicefrac{1}{2n}}\left(x\right)\implies \mvf{d}{e}{y}<\frac{1}{2n}\implies x\in B_{\nicefrac{1}{2n}}\left(e\right)$
	\item $B_{\nicefrac{1}{2n}}\left(e\right)\subseteq B_{\nicefrac{1}{n}}\left(x\right)\subseteq U$; infatti, preso $y\in B_{\nicefrac{1}{2n}}\left(e\right)$ si ha:
	\begin{equation*}
		\mvf{d}{x}{y}\leq \mvf{d}{x}{e}+\mvf{d}{e}{y}<\frac{1}{2n}+\frac{1}{2n}=\frac{1}{n}
	\end{equation*}
$\implies y\in B_{\nicefrac{1}{n}}\left(x\right) \subseteq U$.
\end{itemize}
Segue la tesi.
\end{demonstration}
\begin{example}
	Si può vedere che $\realset^n$ è base numerabile anche perché è uno spazio metrico ed è separabile.
\end{example}
\begin{attention}
Un insieme con una certa topologia può essere a base numerabile (o primo-numerabile), ma non necessariamente rispetto ad un altra!
\end{attention}
\begin{example}
\textsc{Retta di Sorgenfrey}\index{retta di Sorgenfrey}.\\
Consideriamo $X=\realset$ con la topologia avente come base:
\begin{equation}
\basis=\left\{\left[a,\ b\right)\mid a,\ b\in\realset,\ a<b\right\}
\end{equation}
Mostriamo $\basis$ è base per una topologia, è separabile, primo-numerabile ma \textit{non} è a base numerabile.
\begin{itemize}
	\item \textit{Base per una topologia}: usiamo il teorema delle basi (Manetti, 3.7), pag. \pageref{teoremabasi}.
	\begin{enumerate}[label=\Roman*]
		\item $\displaystyle X=\union_{B\in \basis}B$ è ovvio.
		\item Prendiamo $A=\left[a,\ b\right),\ B=\left[c,\ d\right)$ e consideriamo:
		\begin{equation*}
			\forall x\in A\cap B=\left[\max\left\{a,\ b\right\},\ \min\left\{c,\ d\right\}\right)
		\end{equation*}
	Allora basta prendere $C=A\cap B\in\basis$ per soddisfare $x\in C\subseteq A\cap B$.
	\end{enumerate}
\item \textit{Separabile}: $E=\integerset$ è numerabile ed è denso perché vale sempre $\left[a,\ b\right)\cap \integerset \neq \emptyset$, dunque ogni aperto non vuoto interseca $E$; segue che $X$ è separabile.
\item \textit{Primo-numerabile}: s $a\in\realset$ allora $\left\{\left[a,\ a+\frac{1}{n}\right)\right\}_{n\in\naturalset}$ è un sistema fondamentale di intorni di $a$ numerabile. Preso $U$ intorno di $a$, $\exists b>a$ tale che $\left[a,\ b\right)\subseteq U$; inoltre, $\exists n\in\naturalset$ tale che $a+\frac{1}{n}<b$, cioè:
\begin{equation*}
\left[a,\ a+\frac{1}{n}\right)\subseteq\left[a,\ b\right)\subseteq U
\end{equation*}
\item \textit{Non a base numerabile}: presa una base $\widetilde{\basis}$ per $X$, mostriamo che non è numerabile. Sia $x\in\realset$. Allora:
\begin{equation*}
\left[x,\ \infty\right)=\union_{y>x}\left[x,\ y\right)
\end{equation*}
È aperto. In particolare, esiste un aperto dipendente dal punto $x$, cioè $U\left(x\right)=\left[x,\ b\right)\in\widetilde{\basis}$ (per un certo $b>x$) per cui $x\in U\left(x\right)\subseteq\left[x,\ \infty\right)$.\\
Notiamo che se $x\neq y$, allora $U\left(x\right)\neq U\left(y\right)$: preso $y>x$, segue che $x\notin \left[y,\ \infty\right)\supseteq U\left(y\right)\implies x\notin U\left(y\right)\implies U\left(x\right)\neq U\left(y\right)$. L'applicazione:
\begin{equation}
\funztot{\ }{\realset}{\widetilde{\basis}}{x}{U\left(x\right)}
\end{equation}
\end{itemize}
è iniettiva, dunque $\widetilde{\basis}$ non è in iniezione con i naturali e pertanto $\widetilde{\basis}$ \textit{non} è numerabile.
\end{example}
\begin{center}
	\begin{tikzcd}
		&  & \begin{array}{c}
			\text{Primo}\\\text{numerabile}\end{array} \arrow[color=red, lld, "\text{se }X\text{ è metrico}"' six, shift right, below] \\
		\begin{array}{c}
			\text{A base}\\\text{numerabile}
		\end{array} \arrow[rrd, shift left, shorten >= 1em] \arrow[rru, shift right] &  &                                                                                      \\
		&  & \text{Separabile}                                                                   
	\end{tikzcd}
\end{center}
\section{Successioni}
\begin{define}
Una \textbf{successione}\index{successione} in uno spazio topologico $X$ è una funzione:
$\funz{a}{\naturalset}{X}$ che indichiamo con $\left\{a_n\right\}_{n\in\naturalset}=\left\{a_n\right\}$.
\end{define}
\begin{define}
Sia $\left\{a_n\right\}$ una successione in $X$. Diciamo che $\left\{a_n\right\}$ \textbf{converge}\index{convergenza} a $p\in X$ se $\forall U\in I\left(p\right)\ \exists n_0\in \naturalset\ \colon a_n\in U,\ \forall n\geq n_0$.
\end{define}
\begin{observe}
	Se $X$ è di \textbf{Hausdorff}, una successione convergente ha un \textbf{unico}\index{limite!unicità} limite.
\end{observe}
\begin{demonstration}
	Supponiamo che $\left\{a_n\right\}$ converga a $p$ e $q$. Mostriamo che $p=q$.\\
	Siano $U\in I\left(p\right)$ e $V\in I\left(q\right)$.
	\begin{itemize}
		\item Siccome $\left\{a_n\right\}$ converge a $p$, $\exists n_0$ tale che $a_n\in U \forall n\geq n_0$.
		\item Siccome $\left\{a_n\right\}$ converge a $1$, $\exists n_1$ tale che $a_n\in V \forall n\geq n_1$.
	\end{itemize}
\begin{equation*}
\implies a_n\in U\cap V\ \forall n\geq \max_{n_0,\ n_1}\implies U\cap V\neq \emptyset\implies p=q
\end{equation*}
L'ultima implicazione deriva dal fatto che $X$ è di \textbf{Hausdorff}. Infatti, se in \textbf{Hausdorff} $p\neq q\implies U\cap V = \emptyset$, vale anche la sua negazione: $U\cap V\neq \emptyset\implies p = q$.
\end{demonstration}
\begin{define}
Se $X$ è di \textbf{Hausdorff} e $\left\{a_n\right\}$ è convergente, ha senso parlare del \textbf{limite} della successione:
\begin{equation}
p=\lim_{n \to +\infty}a_n
\end{equation}
Se $X$ \textit{non} è di \textbf{Hausdorff}, la stessa successione può convergere a più punti, dunque non esiste il limite della successione.
\end{define}
\begin{examples}~{}
	\begin{itemize}
		\item Se $X$ ha la topologia banale $\topo=\left\{X,\ \emptyset\right\}$, l'unico intorno di qualunque punto è $X$. Allora ogni successione $\left\{a_n\right\}$ in $X$ converge sempre ad un qualunque punto $p$.
		\item Se $X$ ha la topologia discreta, $\left\{a_n\right\}$ successione in $X$ converge a $p\iff \exists n_0\ \colon a_n=p,\ \forall n\geq n_0$, cioè se la successione è finitamente costante. Infatti, nella topologia discreta anche il singoletto $\left\{p\right\}$ è intorno di $p$, dunque eventualmente la successione avrà solo termini nel singoletto.
	\end{itemize}
\end{examples}
\begin{observe}
Se $X$ spazio metrico:
\begin{equation}
	a_n\textsc{ converge a } p\iff \forall \epsilon>0\ \exists n_0\ \colon \mvf{d}{x}{y}<\epsilon,\ \forall n\geq n_0
\end{equation}
\vspace{-6mm}
\end{observe}
\begin{demonstration}~{}\\
	$\impliesdx$ $U=B_{\epsilon}\left(p\right)$ è l'intorno di convergenza che soddisfa l'implicazione.\\
	$\impliessx$ Sia $U\in I\left(p\right)$. Allora $\exists \epsilon\ \colon B_{\epsilon}\left(p\right)\subseteq U$. Ma allora, dato che per le ipotesi $\exists n_0\ \colon \mvf{d}{p}{a_n}<\epsilon,\ \forall n\geq n_0$, cioè $a_n\in B_{\epsilon}\left(p\right)\subseteq U\implies a_n\in U\forall n\geq n_0$.
\end{demonstration}
\subsection{Punti di accumulazione}
\begin{define}
	Un punto $p\in X$ è \textbf{punto di accumulazione per la successione}\index{punto!di accumulazione!per una successione} $\left\{a_n\right\}$ se:
	\begin{equation}
		\forall U\in I\left(p\right),\ \forall N\in\naturalset\ \exists n\geq N\ \colon a_n\in U
	\end{equation}
\vspace{-6mm}
\end{define}
\begin{exercise}
	Se $X$ è spazio metrico, allora:
	\begin{equation}
		p\textsc{ punto di accumulazione per } a_n\iff \forall \epsilon>0\ \exists n_0\ \colon \mvf{d}{x}{y}<\epsilon,\ \forall n\geq n_0
	\end{equation}
\end{exercise}
\vspace{-6mm}
\begin{define}
	Un punto $p\in X$ è \textbf{punto di accumulazione per il sottoinsieme}\index{punto!di accumulazione!per un sottoinsieme} $B\subseteq X$ se:
	\begin{equation}
		\forall U\in I\left(p\right), \exists b\in B \colon b\in U\setminus \left\{ p\right\}
	\end{equation}
L'insieme dei punti di accumulazione per il sottoinsieme $B$ è chiamato \textbf{derivato}\index{derivato} di $B$.
\end{define}
\begin{exercise}
Data la successione $\left\{a_n\right\}$ in $X$ e definito $A\coloneqq \{a_n\mid n\in\naturalset\}$:
\begin{itemize}
\item $p\in X$ punto di accumulazione per la successione non è mai punto di accumulazione per l'insieme $A$.
\item $p\in X$ punto di accumulazione per l'insieme $A$ in generale non è punto di accumulazione per la successione; se $X$ è metrico %
allora vale l'implicazione
\end{itemize}
\end{exercise}
\subsection{Sottosuccessioni}
\begin{define}
Una \textbf{sottosuccessione}\index{successione!sottosuccessione} di $\left\{a_n\right\}$ è la composizione di $\funz{a}{\naturalset}{X}$ con un'applicazione \textit{strettamente crescente} $\funztot{k}{\naturalset}{\naturalset}{n}{k\left(n\right)}$. Si indica con $\left\{a_{k_n}\right\}$.
\end{define}
\begin{lemming}\label{lemmatriangolino} $\textcolor{blue}{\circled{{\large \blacktriangle}}}$ 	Sia $\left\{a_n\right\}$ una successione su $X$ e $p\in X$. Valgono le seguenti implicazioni:
	\begin{equation}
		\circled{1}\ \left\{a_n\right\}\text{ converge a }p
	\end{equation}
\begin{equation*}
	\begin{array}{ll}
	\Downarrow 
\end{array}
\end{equation*}
\begin{equation}
		\circled{2}\ \left\{a_n\right\}\text{ ha una sottosuccessione convergente a }p
\end{equation}
\begin{equation*}
\begin{array}{ll}
	\Downarrow &  \textcolor{red}{\circled{\ast}}
\end{array}
\end{equation*}
\begin{equation}
	\circled{3}\ p\text{ è un punto di accumulazione per }\left\{a_n\right\}
\end{equation}
	\begin{equation*}
	\begin{array}{ll}
		\Downarrow & \textcolor{red}{\circled{\ast\ast}}
	\end{array}
\end{equation*}
\begin{equation}
	\circled{4}\ p\in\overline{A}\text{ dove }A=\left\{a_n\mid n\in \naturalset\right\}\subseteq X
\end{equation}
\end{lemming}~{}\\
\begin{demonstration}
$\circled{1}\implies\circled{2}$ La sottosuccessione convergente è la successione stessa.\\
$\circled{2}\implies\circled{3}$ Sia $\left\{a_{k\left(n\right)}\right\}$ una sottosuccessione convergente a $p$ e sia $U\in I\left(p\right)$. \\
Se $a_{k\left(n\right)}$ converge a $p$ si ha che $\exists n_0\ \colon a_{k\left(n\right)}\in U,\ \forall n\geq n_0$. Poichè $k\left(n\right)$ è strettamente crescente, $\exists n_1\ \colon k\left(n\right)\geq N,\ \forall n\geq n_1$. Allora preso:
\begin{equation*}
	n=\max\left\{n_0,\ n_1\right\}
\end{equation*}
Abbiamo che $a_{k\left(n\right)}\in U,\ k\left(n\right)\geq N$.\footnote{L'intorno $U$ è arbitrario.} Segue che $p$ è punto di accumulazione per $\left\{a_n\right\}$.\\
$\circled{3}\implies\circled{4}$ $p\in \overline{A}\iff \forall U\in I\left(p\right)\ A\cap U\neq \emptyset$. Allora sia $U$ intorno di $p$: voglia che $U\cap A\neq \emptyset$. Essendo $p$ punto di accumulazione per $\left\{a_n\right\}$, $\exists n\ a_n\in U\implies U\cap A\neq \emptyset$.
\end{demonstration}
\begin{lemming}\label{primonumesucc}
Sia $X$ \textit{primo-numerabile}, $\left\{a_n\right\}$ successione in $X$ e $p\in X$. Allora vale anche il viceversa di $\textcolor{red}{\circled{\ast}}$, cioè:
\begin{equation}
	\left\{a_n\right\}\text{ ha una sottosuccessione convergente a }p\iff p\text{ è di accumulazione per }\left\{a_n\right\}
\end{equation}
\end{lemming}
\begin{demonstration}
	$\impliesdx$ Vale per $\textcolor{red}{\circled{\ast}}$.\\
	$\impliessx$ Sia $\left\{U_m\right\}_{m\in\naturalset}$ sistema fondamentale di intorni di $p$ numerabile per ipotesi ($X$ primo-numerabile). Consideriamo i seguenti insiemi:
	\begin{equation*}
	\widetilde{U}_m\coloneqq U_1\cap\ldots U_m\quad \forall m\in \naturalset
	\end{equation*}
\begin{itemize}
	\item $\widetilde{U}_m$ è intorno di $p$, in quanto intersezione \textit{finita} di intorni di $p$.
	\item $\widetilde{U}_m=U_1\cap\ldots U_m\supseteq U_1\cap\ldots U_m\cap U_{m+1}=\widetilde{U}_{m+1}$.
\end{itemize}
Segue che $\left\{\widetilde{U}_m\right\}$ è ancora un sistema fondamentale di intorni (numerabile) di $p$, infatti, se $V$ è intorno di $p$, $\exists m\ \colon V\supseteq U_m\supseteq \widetilde{U}_m$.\\
A meno di sostituire $U_m$ con $\widetilde{U}_m$, possiamo supporre che $U_1\supseteq U_2\supseteq U_3\supseteq \ldots$.\\
Costruiamo una sottosuccessione di $\left\{a_n\right\}$ convergente a $p$. Sicuramente:
\begin{itemize}
	\item $\exists k\left(1\right)\in \naturalset\ \colon a_{k\left(1\right)}\in U_1$.
	\item $\exists k\left(2\right)\geq k\left(1\right)+1\ \colon A_{k\left(2\right)}\in U_2$.
\end{itemize}
E così via: $\forall m\ \exists k\left(m\right)\geq k\left(m-1\right)+1$ tale che $a_{k\left(m\right)}\in U_m$, ottenendo una sottosuccessione $\left\{a_{k\left(m\right)}\right\}$. Notiamo in particolare che:
\begin{center}
\label{notasorridente} $\textcolor{blue}{{\large \smiley{}}}$ Se $m_2\geq m_1$, allora $a_{k\left(m_2\right)}\in U_{m_2}\subseteq U_{m_1}$.
\end{center}
Mostriamo che $\left\{A_{k\left(n\right)}\right\}$ converge a $p$.\\
Sia $V$ intorno di $p$. Dal sistema fondamentale di intorni $\exists m_0$ tale che $U_{m_0}\subseteq V$. Da $\textcolor{blue}{{\large \smiley{}}}$ si ha che $\forall m\geq m_0\ a_{k\left(m\right)}\in U_{m_0}\subseteq V$.
\end{demonstration}
\begin{proposition}
	\textsc{Caratterizzazione della chiusura in termini di successioni.}\\
	Sia $X$ uno spazio topologico \textit{primo-numerabile}. Sia $Y\subseteq X$ e $p\in X$. Sono equivalenti
	\begin{itemize}
		\item Esiste una successione in $Y$ convergente a $p$.
		\item $p$ è di accumulazione per una successione in $Y$.
		\item $p\in \overline{Y}$
	\end{itemize}
\end{proposition}
\begin{demonstration}~{}\\
$\circled{1}\implies\circled{2}$ Non è necessario che $X$ sia primo-numerabile, è immediato dal lemma \ref{lemmatriangolino} $\textcolor{blue}{\circled{{\large \blacktriangle}}}$ (pag. \pageref{lemmatriangolino}).\\
$\circled{2}\implies\circled{3}$ Non è necessario che $X$ sia primo-numerabile. Se $p$ di accumulazione per $\left\{a_n\right\}$ con $a_n\in y\ \forall n\implies A=\left\{a_n\mid n\in \naturalset\right\}\subseteq Y$. Allora segue dal lemma \ref{lemmatriangolino} $\textcolor{blue}{\circled{{\large \blacktriangle}}}$ (pag. \pageref{lemmatriangolino}) che $p\in A=\overline{\left\{a_n\mid n\in \naturalset\right\}}\subseteq \overline{Y}$ \\
$\circled{3}\implies\circled{1}$ Sia $\left\{U_n\right\}$ un sistmea fondamentale di intorni di $p$ tale che $U_n\supseteq U_{n+1}\ \forall n$. Allora:
\begin{equation*}
	p\in \overline{Y}\implies \forall n\ Y\cap U_n\neq\emptyset\implies\forall n\ \exists y_n\in Y\cap U_n
\end{equation*}
In modo analogo a $\textcolor{blue}{{\large \smiley{}}}$ (pag. \pageref{notasorridente}), se $n_2\geq n_1$, allora $y_{n_2}\in U_{n_2}\subseteq U_{n_1}$. Allora $\left\{y_n\right\}$ è una successione in $Y$, mostriamo che converge a $p$.\\
Sia $V$ intorno di $p$. Dal sistema fondamentale di intorni $\exists n_0$ tale che $U_{n_0}\subseteq V$. Dal ragionamento analogo a $\textcolor{blue}{{\large \smiley{}}}$ si ha che $\forall n\geq n_0\ y_n\in U_{n_0}\subseteq V$.
\end{demonstration}
\section{Successioni e compatti}
\begin{proposition}\label{compattocontenuto}\textsc{(Manetti, 4.46)}\\
Sia $X$ spazio topologico e sia $K_n\subseteq X\ \forall n\in\naturalset$ un sottospazio chiuso, \textit{compatto} e non vuoto. Supponiamo inoltre che:
\begin{equation*}
K_n\supseteq K_n+1\ \forall n\implies K_1\supseteq K_2\supseteq K_3\supseteq \ldots
\end{equation*}
Allora: $\inter_{n\geq 1}K_n\neq\emptyset$.
\end{proposition}
\begin{demonstration}
Iniziamo in $K_1$. Consideriamo $A_n\coloneqq K_1\setminus K_n$:
\begin{itemize}
	\item $K_n$ chiuso in $X\implies K_n$ chiuso in $K_1$. Allora $A_n$ completementare di un chiuso, dunque aperto in $K_1\ \forall m\geq 1$.
	\item $K_n\supseteq K_{n+1}\implies A_n\subseteq A_{n+1}\ \forall n\geq 1$.
\end{itemize}
Sia allora $N\in\naturalset$.
\begin{equation*}
	\union_{n=1}^{N}A_n=A_N=K_1\setminus \underbrace{K_N}_{\neq\emptyset}\subsetneqq K_1
\end{equation*}
Allora nessuna unione \textit{finita} degli $A_n$ ricopre $K_1$, cioè $\displaystyle \union_{n\in\naturalset}A_n\subsetneqq K_1$, altrimenti $\left\{A_n\right\}$ sarebbe un ricoprimento aperto di $K_1$ che \textit{non} ammette sottoricoprimento finito (assurdo, in quanto $K_1$ è \textit{compatto}!).
\begin{equation*}
\union_{n\geq 1}A_n=\union_{n\geq 1}K_1\setminus K_n=K_1\setminus \left(\inter_{n\geq 1}K_n\right)\subsetneqq K_1\implies \inter_{n\geq 1}K_n\neq\emptyset
\end{equation*}
\end{demonstration}
\begin{lemming}
In uno spazio topologico \textit{compatto} $X$ ogni successione in $X$ ha punti di accumulazione.
\end{lemming}
\begin{demonstration}
Sia	$\left\{a_n\right\}$ successione in $X$; per definizione:
\begin{center}
$p\in X$ punto di accumulazione per $\left\{a_n\right\}\iff \forall U\in I\left(p\right),\ \forall N\in\naturalset\ \exists n\geq N\ \colon a_n\in U$
\end{center}
Per $N$ fissato sia $A_N\coloneqq \left\{a_n\mid n\geq N\right\}\subseteq X$. Allora:
\begin{center}
	$p\in X$ punto di accumulazione per $\displaystyle\left\{a_n\right\}\iff \forall U\in I\left(p\right),\ \forall N\in\naturalset\ U\cap A_N\neq\emptyset\iff \forall N\in\naturalset,\ p\in\overline{A}_N\coloneqq C_N$
\end{center}
Dunque $\displaystyle\left\{\text{punti di accumulazione di} \left\{a_n\right\}\right\}=\inter_{N\in\naturalset}C_N$ e:
 \begin{center}
 	$\left\{a_n\right\}$ ha punti di accumulazione $\iff \inter_{N\in\naturalset}C_N\neq\emptyset$
 \end{center}
\begin{itemize}
	\item $A_B\neq \emptyset$ per definizione, dunque $C_N$ è un chiuso non vuoto.
	\item $X$ è compatto, $C_N$ chiuso in $X$ compatto $\implies C_N$ \textit{compatto}.
\end{itemize}
Poiché $A_N=\left\{a_n\mid n\geq N\right\}\supseteq A_ {N+1}=\left\{a_n\mid n\geq N+1\right\}$, si ha:
\begin{equation*}
C_N=\overline{A_N}\subseteq \overline{A_{N+1}}=C_{N+1}
\end{equation*}
Abbiamo trovato una successione di compatti contenuto l'uno nel successivo. Allora per la proposizione \ref{compattocontenuto} (pag. \pageref{compattocontenuto}, \textsc{(Manetti, 4.46)}). si ha che $\displaystyle\inter_{n\geq 1}C_N\neq \emptyset$.
Segue che esiste un punto di accumulazione per la successione.
\end{demonstration}
\subsection{Compattezza per successioni}
\begin{define}
Sia $X$ spazio topologico. $X$ si dice \textbf{compatto per successioni}\index{compatto!per successioni} se ogni successione ammette una sottosuccessione convergente.
\end{define}
\begin{observe}
Per il lemma \ref{lemmatriangolino} $\textcolor{blue}{\circled{{\large \blacktriangle}}}$ (pag. \pageref{lemmatriangolino}), se $X$ è compatto per successioni allora ogni successione in $X$ ha un punto di accumulazione.
\end{observe}
\begin{lemming}
	Sia $X$ \textit{primo-numerabile.} Allora:
\begin{enumerate}
	\item $X$ compatto per successioni $\iff $ Ogni successione in $X$ ha un punto di accumulazione.
	\item $X$ compatto $\implies$ $X$ compatto per successioni.
\end{enumerate}
\end{lemming}
\begin{demonstration}~{}
\begin{enumerate}[label=\Roman*]
\item $\impliesdx$ Vale per l'osservazione precedente.\\
$\impliessx$ Vale per il lemma \ref{primonumesucc}, pag. \pageref{primonumesucc}: se ogni successione ha un punto di accumulazione in $X$ primo numerabile, allora ogni sottosuccessione ammette una sottosuccessione convergente a $p$, cioè $X$ è compatto per successioni.
\item Se $X$ è compatto, allora ogni successione in $X$ ha dei punti di accumulazione e per il punto $1)$ segue che $X$ è compatto per successioni.
\end{enumerate}
\end{demonstration}
\begin{proposition}\textsc{Caratterizzazione della compattezza in termini di successioni}\\
	Sia $X$ uno spazio topologico a base numerabile. Allora sono equivalenti:
	\begin{enumerate}
		\item $X$ compatto.
		\item $X$ compatto per successioni
		\item Ogni successione in $X$ ammette un punto di accumulazione.
	\end{enumerate}
\end{proposition}
\begin{demonstration}
Sappiamo già che $2)\iff 3)$ e $1)\implies 2)$ dal lemma precedente. Dobbiamo dimostrare $2)\implies 1)$. Dimostriamo per contronominale ($\neg 1)\implies \neg 2)$): se $X$ \textit{non} è compatto, allora $X$ non è compatto per successioni, cioè esiste una sottosuccessione in $X$ che \textit{non} ha alcuna sottosuccessione convergente.\\
\begin{itemize}
	\item $X$ \textit{non} compatto$\implies \exists \widetilde{\mathcal{A}}$ ricoprimento aperto di $X$ che \textit{non} ha sottoricoprimenti finiti.
	\item $X$ a \textit{base numerabile} $\implies\exists\mathcal{A}$ sottoricoprimento di $\widetilde{\mathcal{A}}$ che sia numerabile.
\end{itemize}
Poiché ogni sottoricoprimento di $\mathcal{A}$ è anche un sottoricoprimento di $\widetilde{\mathcal{A}}$, significa che $\mathcal{A}$ \textit{non} ha sottoricoprimenti finiti. Definiamo:
\begin{equation*}
\mathcal{A}\coloneqq\left\{A_n\right\}_{n\in\naturalset}
\end{equation*}
Allora:
\begin{equation*}
\forall n\in\naturalset\quad \union_{j=1}^n A_j\subsetneqq X\implies \exists x_n\in X\setminus \union_{j=1}^n A_j
\end{equation*}
Costruisco così una successione $\left\{x_n\right\}$ successione in $X$ tale per cui:
\begin{center}
	\label{notatriste} $\textcolor{red}{{\large \frownie{}}}$ $x_n\notin A_j\ \forall j\leq n$.
\end{center}
Mostriamo che $\left\{x_n\right\}$ non ha sottosuccessioni convergenti. Sia $\left\{x_{k\left(n\right)}\right\}$ una sottosuccessione arbitraria di $\left\{x_n\right\}$ e sia $p\in X$, mostriamo che essa non converga ad un qualunque $p$.\\
\begin{itemize}
	\item $\mathcal{A}$ è un (sotto)ricoprimento di $X\implies \exists N\ \colon p\in A_N$
	\item Da $\textcolor{red}{{\large \frownie{}}}$ (pag. \pageref{notatriste}) abbiamo che $x_n \notin A_N \ \forall n\geq N$  (dato che $x_n \notin A_j \ \forall j\leq n$, in particolare in $A_N$ per ogni $n\geq N$); si ha allora $x_{k\left(n\right)}\notin A_n\ \forall n\ \colon k\left(n\right)\geq N$.
\end{itemize}
Essendo $k\left(n\right)$ crescente, $\exists n_0\ \colon k\left(n\right)\geq N\ \forall n\geq n_0$. Segue che se $n\geq n_0$ allora $x_k\left(n\right)\notin A_N$ 
Poiché $A_N$ è intorno di $p$, segue che $\left\{x_{k\left(n\right)}\right\}$ non converge a $p$.
\end{demonstration}
\begin{theorema}
Sia $X$ spazio metrico. Allora:
\begin{equation}
	X\text{ compatto}\iff X\text{ compatto per successioni}
\end{equation}
\end{theorema}
\begin{center}
\begin{tikzcd}
	&  & \begin{array}{c}
		\text{Compatto per}\\ \text{successioni}
	\end{array}  \arrow[color=red, lld, "X\text{ a base numerabile}"' sixslant2, shift right, shorten <= 0.5em] \arrow[dd] \arrow[color=red, lld, "X\text{ metrico}" sixslant1, shift right, below, shorten <= 0.5em] \\
	\text{Compatto} \arrow[rrd, shift left, ] &  &                                                                                                                                                 \\
	&  & \begin{array}{c}
		\text{Ogni successione ha}\\ \text{un punto di accumulazione}
	\end{array} \arrow[color=red, uu, "\substack{\text{primo}\\\text{numerabile}}"' nine, shift right=4]                                     
\end{tikzcd}
\end{center}
\section{Spazi metrici completi}
\begin{define}
	Sia $\left(X,\ d\right)$ uno spazio metrico. Una successione $\left\{a_n\right\}$ si dice \textbf{di Cauchy}\index{successione!di Cauchy} se:
	\begin{equation}
		\forall \epsilon>0\ \exists n_0\in\naturalset\ \colon \mvf{d}{a_n}{a_m}<\epsilon,\ \forall n,\ m\geq n_0
	\end{equation}
\end{define}
\begin{define}
	Uno spazio metrico $\left(X,\ d\right)$ si dice \textbf{completo}\index{spazio topologico!metrico!completo} se ogni successione di Cauchy è convergente.
\end{define}
\begin{observe}~{}
	\begin{enumerate}
		\item Ogni successione \textit{convergente} è di \textit{Cauchy}.
		\item Una successione di Cauchy è \textit{convergente} se e solo se ha punti di accumulazione.
		\item Una successione di Cauchy è \textit{convergente} se ha una \textit{sottosuccessione convergente}.
		\item Se $X$ è \textit{compatto}, allora ogni successione di Cauchy è \textit{convergente}.
		\item Se $X$ è spazio metrico \textit{compatto}, allora $X$ è spazio metrico \textit{completo}; non è vero il viceversa.
	\end{enumerate}
\end{observe}
\begin{demonstration}~{}
	\begin{enumerate}[label=\Roman*]
		\item Se $a_n\to p$ per $n\to +\infty$ significa che:
		\begin{equation*}
			\forall \epsilon>0\ \exists n_0\in\naturalset\ \colon \mvf{d}{a_n}{p}<\epsilon,\ \forall n\geq n_0
		\end{equation*}
	Considerati $n,\ m\geq n_0$ si ha:
	\begin{equation}
		\mvf{d}{a_n}{a_m}\leq \mvf{d}{a_n}{p}+\mvf{d}{p}{a_n}< 2\epsilon
	\end{equation}
Per l'arbitrarietà di $\epsilon$ vale la convergenza.
\item $\impliesdx$ Sempre vera per  \ref{lemmatriangolino} $\textcolor{blue}{\circled{{\large \blacktriangle}}}$ (pag. \pageref{lemmatriangolino}).\\
$\impliessx$ Sia $\left\{a_n\right\}$ una successione di Cauchy e sia $p$ un punto di accumulazione. Sia $\epsilon>0$: dalla definizione di successione di Cauchy $\exists n_0$ tale per cui $\mvf{d}{a_n}{a_m}<\epsilon\ \forall n,\ m\geq n_0$.\\
Essendo $p$ di accumulazione, $\exists n_1\geq n_0$ tale per cui $\mvf{d}{p}{a_{n_1}}<\epsilon$. Allora, se $n\geq n_0$ si ha:
	\begin{equation*}
	\mvf{d}{a_n}{p}\leq \mvf{d}{a_n}{a_{n_1}}+\mvf{d}{a_{n_1}}{p}< 2\epsilon
\end{equation*}
Dunque $\left\{a_n\right\}$ converge a $p$.
\item Poiché $X$ è metrico, $X$ è primo-numerabile, dunque avere un punto di accumulazione è equivalente ad avere una sottosuccessione convergente.
\item Se $X$ è compatto, ogni successione ha punti di accumulazione, in particolare quelle di Cauchy: per il punto $2)$ tutte le successioni di Cauchy risultano allora convergenti.
\item Segue dal punto $4)$. Un controesempio del viceversa è $R^n$, dato che è completo ma non è compatto (si veda il teorema seguente). 
\end{enumerate}
\end{demonstration}
\begin{theorema}
	$\realset^n$ in metrica euclidea è uno spazio metrico completo.
\end{theorema}
\begin{demonstration}
	Sia $\left\{a_n\right\}$ di Cauchy in $\realset^n$. Mostriamo che $\left\{a_n\right\}$ è eventualmente limitata\footnote{Supponendo chiaramente che la successione sia ben definita, ci interessa solamente che la successione sia limitata dopo un $n_0$: prima di ciò ho un numero finito di termini $a_0,\ \ldots,\ a_{n_0}<\infty$ e posso chiaramente prendere una palla (chiusa) che li contenga, ad esempio di raggio $M+1$ con $M$ definito come nella dimostrazione.}.\\
	Poiché la successione di Cauchy è definita per ogni $\epsilon$, fissiamo $\epsilon=1$. Allora:
	\begin{equation*}
	\exists n_0\ \colon \labs a_n-a_m\rabs\leq 1\ \forall n,\ m\geq n_0
	\end{equation*}
Sia $M\coloneqq \max_{n_0,\ \ldots,\ n_0}\labs a_n\rabs$. Se $n\geq n_0$ si ha:
\begin{equation*}
\labs a_n\rabs=\labs a_n - a_{n_0}+ a_{n_0}\rabs=\leq \labs a_n - a_{n_0}\rabs+\labs a_{n_0}\rabs\leq 1+M
\end{equation*}
Questo significa che $\left\{a_n\right\}\subseteq\overline{B_{1+M}\left(0\right)}$. Questa palla chiusa è uno spazio metrico \textit{indotto} in $\realset^n$ e compatto, cioè è uno \textit{spazio metrico completo}. Allora la successione di Cauchy, trovandosi in uno spazio metrico completo, converge in esso, e dunque converge anche in $\realset^n$.
\end{demonstration}
\begin{attention}
La \textbf{completezza} \textit{non} è una proprietà topologica! Per esempio, $\realset$ e $\left(0,\ 1\right)$ con metrica euclidea sono omeomorfi rispetto alla topologia indotta dalla metrica, ma $\realset$ abbiamo appena dimostrato che è completo, mentre $\left(0,\ 1\right)$ si può vedere che non lo è!
\end{attention}