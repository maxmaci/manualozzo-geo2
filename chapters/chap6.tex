% SVN info for this file
\svnidlong
{$HeadURL$}
{$LastChangedDate$}
{$LastChangedRevision$}
{$LastChangedBy$}

\chapter{Assiomi di numerabilità e successioni}
\labelChapter{successioni}

\begin{introduction}
‘‘Non ti ho tradito! Dico sul serio... Ero... rimasto senza benzina. Avevo una gomma a terra. Non avevo i soldi per prendere il taxi. La tintoria non mi aveva portato il tight. C'era il funerale di mia madre! Era crollata la casa! C'è stato un... terremoto! Una tremenda inondazione! Le cavallette! Non è stata colpa mia! Lo giuro su Dio!''
\begin{flushright}
	\textsc{Jake "Joliet" Blues,} The Blues Brothers.
\end{flushright}
\end{introduction}
\lettrine[findent=1pt, nindent=0pt]{D}{allo}  studio dell'\textit{Analisi} sono ben note le \textit{successioni} a valori reali e le loro proprietà. In questo capitolo porremo la nostra attenzione sulle \textbf{successioni} con codominio uno \textit{spazio topologico}. Inoltre, riprenderemo dal \autoref{chap:Connessocompatto} la trattazione dei \textit{compatti} per studiare la relazione che hanno con le successioni.\\
Prima fare tutto ciò, tuttavia, abbiamo bisogno di introdurre degli \textbf{assiomi di numerabilità}, in modo da garantire l'esistenza di insiemi \textit{numerabili} di oggetti topologici; la maggior parte degli spazi topologici più comuni soddisfano almeno uno di questi assiomi.\\
Il capitolo si conclude con un approfondimento delle successioni negli spazi metrici.
\section{Numerabilità}
\begin{definition}{}[Insieme numerabile]
Un insieme $X$ è \textbf{numerabile}\index{numerabile} o di \textit{cardinalità numerabile} se è finito oppure esiste una biezione tra l'insieme $X$ e i naturali $\N$.
\end{definition}
\begin{definition}{}[Secondo assioma di numerabilità {(A base numerabile)}]
Uno spazio topologico $X$ è \textbf{a base numerabile}\seeonlyindex{a base numerabile}{assioma di numerabilità!secondo}, o soddisfa il \textbf{{\small secondo assioma di numerabilità}}\index{assioma di numerabilità!secondo}, se esiste una base $\basis$ della topologia tale che $\basis$ sia di \textit{cardinalità numerabile}.
\end{definition}
\begin{definition}{}[Primo assioma di numerabilità]
Uno spazio topologico $X$ è \textit{primo numerabile}\seeonlyindex{primo numerabile}{assioma di numerabilità!primo}, o soddisfa il \textbf{primo assioma di numerabilità}\index{assioma di numerabilità!primo}, se ogni punto ammette un sistema fondamentale di intorni che sia numerabile.
\end{definition}
\begin{remark}{pn}~{}\label{metrico implica primo num}
\begin{enumerate}
\item Il secondo assioma di numerabilità implica il primo.
\item Se $X$ è finito, $X$ soddisfa sempre i due assiomi.
\item Se $X$ è spazio metrico, $X$ è sempre \textit{primo numerabile}.
\item Se $X$ è \textit{a base numerabile}, ogni sottospazio $Y$ di $X$ è \textit{a base numerabile}. In particolare $Y$ è primo numerabile.
\item Se $X$ e $Y$ sono \textit{a base numerabile}, allora $X\times Y$ è \textit{a base numerabile}. In particolare $X\times Y$ è primo numerabile.
\item Non è vero che il quoziente di $X$ spazio \textit{a base numerabile} (o \textit{primo numerabile}) è sempre \textit{a base numerabile} (o \textit{primo numerabile}).
\end{enumerate}
\end{remark}
\begin{proof}{n}~{}
\begin{enumerate}[label=\Roman*]
	\item Se $X$ ha base numerabile $\basis$ e $x\in X$, allora $\left\{A\in \basis\mid x\in A\right\}$ è un sistema fondamentali di intorni di $x$ ed è chiaramente numerabile.
	\item Ogni base e sistema fondamentale di intorni contiene necessariamente un numero finito di elementi.
	\item Preso $x\in X$, $\left\{B_{\nicefrac{1}{n}}\left(x\right)\right\}_{n\in\N}$ è un sistema fondamentale di intorni ed è numerabile.
	\item Se $\basis$ è una base numerabile per $X$, $\left\{A\cap Y\mid A\in \basis\right\}$ è base numerabile per $Y$.
	\item Se $\basis_X$ è una base numerabile per $X$ e $\basis_Y$ base numerabile per $Y$, allora \begin{equation*}
		\left\{A\times B\mid A\in \basis_X,\ B\in\basis_Y\right\}
	\end{equation*}
	è base di $X\times Y$ numerabile.
	\item La contrazione di $\Z$ in $\R$ ad un punto, cioè il quoziente $\nicefrac{\R}{\Z}$, non è primo numerabile né tanto meno a base numerabile, pur essendo $\R$ a base numerabile in quanto metrico\footnote{Nelle ‘‘Note aggiuntive'', a pag. \pageref{dimostrazionenonnumerabilità}, si può trovare la dimostrazione di ciò.}.\qedhere
\end{enumerate}
\end{proof}
\begin{example}{n}
	$\R$ con la topologia Euclidea è \textit{a base numerabile}. Infatti, presa
	\begin{equation*}
		\basis=\left\{\left(a,\ b\right)\mid a,\ b\in \Q,\ a<b\right\},
	\end{equation*}
	è una base perché, dati $x,\ y\in\R,\ x<y$,
	\begin{equation*}
		\left(x,\ y\right)=\bigcup_{\substack{a, b\in\Q\\x<a<b<y}}\left(a,\ b\right).
	\end{equation*}
	ed è numerabile, essendo definita in base ai razionali $\Q$, che sono numerabili.
\end{example}
\begin{proposition}{}[Ogni ricoprimento aperto ammette un sottoricoprimento numerabile in spazio a base numerabile]
Sia $X$ \textit{a base numerabile}. Ogni ricoprimento aperto di $X$ ammette un sottoricoprimento numerabile.
\end{proposition}
\begin{proof}{n}
Sia $\mathcal{A}$ un ricoprimento aperto di $X$ e $\basis$ una base numerabile per $X$. Preso $x\in X$ esiste un $U_x\in\mathcal{A}$ che lo contiene in quanto $\mathcal{A}$ ricoprimento e un $B_x\in\basis$ tale che $x\in B_x\subseteq U_x$ in quanto $\mathcal{B}$ base. Abbiamo così determinato un sottoinsieme numerabile della base $\basis$:
\begin{equation*}
\widetilde{\basis}\coloneqq\left\{B_x\mid x\in X\right\}
\end{equation*}
In particolare, esiste $E\subseteq X$ numerabile tale che
\begin{equation*}
\widetilde{\basis}\coloneqq\left\{B_x\mid x\in E\right\}
\end{equation*}
Se consideriamo ora $\widetilde{\mathcal{A}}\coloneqq\left\{U_x\mid x\in E\right\}$, notiamo che:
\begin{itemize}
	\item $\widetilde{\mathcal{A}}\subseteq \mathcal{A}$;
	\item $\widetilde{\mathcal{A}}$ è numerabile perché lo è $E$;
	\item Si ha
	\begin{equation*}
		X=\bigcup_{B_x\in\widetilde{\basis}}B_x=\bigcup_{x\in E}B_x\subseteq \bigcup_{x\in E}U_x.
	\end{equation*}
\end{itemize}
Segue che $\widetilde{\mathcal{A}}$ è un sottoricoprimento numerabile di $\mathcal{A}$.\qedhere
\end{proof}
\begin{definition}{}[Spazio separabile]
Uno spazio topologico $X$ si dice \textbf{separabile}\index{spazio!separabile} se contiene un sottoinsieme $E$ \textit{denso} e \textit{numerabile}.
\end{definition}
\begin{example}{pn}~{}
	\begin{itemize}
		\item Se $X$ è numerabile, allora è separabile perché l'insieme stesso è un sottoinsieme numerabile e denso.
		\item $\R^n$ con la topologia Euclidea è separabile perché si ha $E=\Q^n$ denso in $\R^n$.
	\end{itemize}
\end{example}
\begin{lemma}{}[Base numerabile implica separabile]\label{basenumseparabile}
Se $X$ è a base numerabile, allora è separabile.
\end{lemma}
\begin{proof}{n}
	Sia $\basis$ una base numerabile. Prendiamo
	\begin{equation*}
		E\coloneqq\Set{x_U\in U| U\in \basis}.
	\end{equation*}
\begin{itemize}
	\item $E$ è numerabile perché lo è $\basis$, avendo preso un punto per ogni elemento della base numerabile.
	\item $E$ è denso: se $A\subseteq X$ è aperto \textit{non} vuoto, allora esiste $U\in\basis$ tale che $x_u\in U\subseteq A$, dunque $x_U\in A$ e allora $E\cap A\neq \emptyset$.\qedhere
\end{itemize}
\end{proof}
\begin{proposition}{}[$X$ metrico a base numerabile se e solo se separabile]
	Sia $X$ spazio metrico. $X$ è sempre primo numerabile, mentre è a base numerabile se e solo se separabile.
\end{proposition}
\begin{proof}{n}
Per la prima affermazione, si veda l'osservazione a pag. \pageref{metrico implica primo num}; mostriamo ora la seconda.\\
$\rightimplies$Sempre vera per ogni spazio anche \textit{non} metrico (lemma \ref{basenumseparabile}).\\
$\leftimplies$Sia $E\subseteq X$ sottoinsieme numerabile e denso. Definiamo
\begin{equation*}
	\basis\coloneqq\Set{B_{\nicefrac{1}{n}}\left(e\right)| e\in E,\ n\in \N}.
\end{equation*}
Questo insieme è numerabile, mostriamo che sia una base. Per far ciò, fissiamo $U\subseteq X$ aperto e prendiamo $x\in U$: vogliamo trovare un aperto di $\basis$ contenuto in $U$ e contenente $x$.\footnote{Gli elementi della base sono già aperti banalmente. Per l'arbitrarietà di $x$, troviamo un ricoprimento aperto di $U$ costituito da aperti di $\basis$ contenuto interamente in $U$, cioè $U=\bigcup_{i\in I}B_i$.} Sia $n\in\N$ tale che $B_{\nicefrac{1}{n}}\left(x\right)\subseteq U$. Cerchiamo opportuni $e\in E,\ m\in\N$ tale che
\begin{equation*}
x\in B_{\nicefrac{1}{m}}\left(e\right)\subseteq B_{\nicefrac{1}{n}}\left(x\right)\subseteq U
\end{equation*}
Consideriamo la palla $B_{\nicefrac{1}{2n}}\left(x\right)$. Siccome $E$ è denso in $X$, esiste $e\in E\cap B_{\nicefrac{1}{2n}}\left(x\right)$. Prendiamo ora la balla \textit{[sic.]} $B_{\nicefrac{1}{2n}}\left(e\right)\in\basis$:
\begin{itemize}
	\item \textit{contiene} $x$ perché se $e\in B_{\nicefrac{1}{2n}}\left(x\right)$, allora $d\left(e,x\right)<\frac{1}{2n}$, dunque $x\in B_{\nicefrac{1}{2n}}\left(e\right)$
	\item $B_{\nicefrac{1}{2n}}\left(e\right)\subseteq B_{\nicefrac{1}{n}}\left(x\right)\subseteq U$; infatti, preso $y\in B_{\nicefrac{1}{2n}}\left(e\right)$ si ha
	\begin{equation*}
		d\left(x,y\right)\leq d\left(x,e\right)+d\left(e,y\right)<\frac{1}{2n}+\frac{1}{2n}=\frac{1}{n},
	\end{equation*}
da cui segue $y\in B_{\nicefrac{1}{n}}\left(x\right) \subseteq U$.\qedhere
\end{itemize}
\end{proof}
\begin{example}{n}
	Si può vedere che $\R^n$ è base numerabile in modo alternativo a quanto già vista perché è uno spazio metrico ed è separabile.
\end{example}
\begin{warning}{n}
Un insieme con una certa topologia può essere a base numerabile (o primo numerabile), ma non necessariamente rispetto ad un'altra!
\end{warning}
\begin{example}{}[Retta di Sorgenfrey]\index{retta!di Sorgenfrey}
Consideriamo $X=\R$ con la topologia avente come base
\begin{equation*}
\basis=\Set{\left[a,\ b\right)| a,\ b\in\R,\ a<b}.
\end{equation*}
Mostriamo che $\basis$ è base per una topologia, $X$ è separabile, primo numerabile ma \textit{non} è a base numerabile.
\begin{itemize}
	\item \textit{Base per una topologia}: usiamo il teorema delle basi (\ref{teoremabasi}, Manetti, 3.7).
	\begin{enumerate}[label=\Roman*]
		\item $X=\bigcup_{B\in \basis}B$ è ovvio.
		\item Prendiamo $A=\left[a,\ b\right),\ B=\left[c,\ d\right)$ e consideriamo
		\begin{equation*}
			\forall x\in A\cap B=\left[\max\left\{a,\ b\right\},\ \min\left\{c,\ d\right\}\right)
		\end{equation*}
	Basta prendere $C=A\cap B\in\basis$ per soddisfare $x\in C\subseteq A\cap B$.
	\end{enumerate}
\item \textit{Separabile}: $E=\Q$ è numerabile ed è denso perché vale sempre $\left[a,\ b\right)\cap \Q \neq \emptyset$, dunque ogni aperto non vuoto interseca $E$; segue che $X$ è separabile.
\item \textit{Primo numerabile}: se $a\in\R$, allora
\begin{equation*}
	\left\{\left[a,\ a+\frac{1}{n}\right)\right\}_{n\in\N}
\end{equation*}
è un sistema fondamentale di intorni di $a$ numerabile. Preso $U$ intorno di $a$, esiste $b>a$ tale che $\left[a,\ b\right)\subseteq U$; inoltre, esiste $n\in\N$ tale che $a+\frac{1}{n}<b$, cioè
\begin{equation*}
\left[a,\ a+\frac{1}{n}\right)\subseteq\left[a,\ b\right)\subseteq U.
\end{equation*}
\item \textit{Non a base numerabile}: presa una base $\widetilde{\basis}$ per $X$, mostriamo che non è numerabile. Sia $x\in\R$; allora
\begin{equation*}
\left[x,\ \infty\right)=\bigcup_{y>x}\left[x,\ y\right)
\end{equation*}
è aperto. In particolare, esiste un aperto $U\left(x\right)=\left[x,\ b\right)\in\widetilde{\basis}$ con $b>x$ dipendente dalla scelta del punto $x$ per cui $x\in U\left(x\right)\subseteq\left[x,\ \infty\right)$. Notiamo che se $x\neq y$, allora $U\left(x\right)\neq U\left(y\right)$: preso $y>x$, segue che se $x\notin \left[y,\ \infty\right)\supseteq U\left(y\right)$ allora $x\notin U\left(y\right)$ e quindi $U\left(x\right)\neq U\left(y\right)$. L'applicazione
\begin{equation*}
\funct{}{\R}{\widetilde{\basis}}[x][U\left(x\right)]
\end{equation*}
\end{itemize}
è iniettiva, dunque $\widetilde{\basis}$ non è in iniezione con i naturali e pertanto $\widetilde{\basis}$ \textit{non} è numerabile.
\end{example}
Riassumendo, le relazioni tra gli assiomi di numerabilità e la separabilità sono le seguenti:
\begin{center}
	\begin{tikzcd}
		&  &  \text{Separabile} \arrow[color=red, lld, "\text{se }X\text{ è metrico}"' six, shift right, below] \\
		\begin{array}{c}
			\text{A base}\\\text{numerabile}
		\end{array} \arrow[rrd, shift left, shorten >= -1em] \arrow[rru, shift right] &  &                                                                                      \\
		&  & \begin{array}{c}
			\text{Primo}\\\text{numerabile}\end{array}
		\end{tikzcd}
\end{center}
\section{Successioni}
\begin{definition}{}[Successione]
Una \textbf{successione}\index{successione} in uno spazio topologico $X$ è una funzione $\funct{}[a]{\N}{X}$. La indichiamo con $\left\{a_n\right\}_{n\in\N}=\left\{a_n\right\}\coloneqq\left\{a(n)\right\}_{n\in\N}$.
\end{definition}
\begin{definition}{}[Convergenza di una successione]
Sia $\left\{a_n\right\}$ una successione in $X$. Diciamo che $\left\{a_n\right\}$ \textbf{converge}\index{convergenza} a $p\in X$ se
\begin{equation*}
	\forall U\in I\left(p\right)\ \exists n_0\in \N\ \colon a_n\in U,\ \forall n\geq n_0.
\end{equation*}
\end{definition}
\begin{remark}{n}
	Se $X$ è di Hausdorff, una successione convergente ha un \textbf{unico}\index{limite!unicità} limite.
\end{remark}
\begin{proof}{n}
	Supponiamo che $\left\{a_n\right\}$ converga a $p$ e $q$; mostriamo che $p=q$. Siano $U\in I\left(p\right)$ e $V\in I\left(q\right)$ arbitrari.
	\begin{itemize}
		\item Siccome $\left\{a_n\right\}$ converge a $p$, esiste $n_0$ tale che $a_n\in U,$ per ogni $n\geq n_0$.
		\item Siccome $\left\{a_n\right\}$ converge a $q$, esiste $n_1$ tale che $a_n\in V,$ per ogni $n\geq n_1$.
	\end{itemize}
Allora $a_n\in U\cap V$ per ogni $n\geq \max\{n_0,\ n_1\}$, quindi $U\cap V\neq \emptyset$ e pertanto $p=q$ perché $X$ è di Hausdorff\footnote{In particolare abbiamo usato la contronominale della definizione di Hausdorff:
\begin{equation*}
	\neg\left(p\neq q\implies \exists U\in I(p), V\in I(q)\colon U\cap V=\emptyset\right)\rightarrow\forall U\in I(p), V\in I(q), U\cap V\neq \emptyset\implies p=q
	\end{equation*}}.\qedhere
\end{proof}
\begin{definition}{}[Limite]
Se $X$ è di Hausdorff e $\left\{a_n\right\}$ è convergente, ha senso parlare del \textbf{limite} della successione:
\begin{equation*}
p=\lim_{n \to +\infty}a_n.
\end{equation*}
Se $X$ \textit{non} è di Hausdorff, la stessa successione può convergere a più punti, dunque non esiste il limite della successione.
\end{definition}
\begin{example}{pn}~{}
	\begin{itemize}
		\item Se $X$ ha la topologia banale $\topo=\left\{X,\ \emptyset\right\}$, l'unico intorno di qualunque punto è $X$. Allora ogni successione $\left\{a_n\right\}$ in $X$ converge sempre ad un qualunque punto $p$.
		\item Se $X$ ha la topologia discreta, una successione $\left\{a_n\right\}$ in $X$ converge a $p$ se e solo se esiste $n_0$ tale che $a_n=p$, per ogni $n\geq n_0$, cioè se la successione è definitamente costante. Infatti, nella topologia discreta anche il singoletto $\left\{p\right\}$ è intorno di $p$, dunque alla fine la successione avrà solo termini nel singoletto.
	\end{itemize}
\end{example}
\begin{remark}{n}
Se $X$ spazio metrico, $a_n$ converge a $p$ se e solo se
\begin{equation*}
	\forall \epsilon>0\ \exists n_0\ \colon d\left(a_n,p\right)<\epsilon,\ \forall n\geq n_0
\end{equation*}
\end{remark}
\begin{proof}{n}~{}\\
	$\rightimplies$$U=B_{\epsilon}\left(p\right)$ è l'intorno di convergenza che soddisfa l'implicazione.\\
	$\leftimplies$Sia $U\in I\left(p\right)$. Allora esiste $\epsilon>0$ tale che $B_{\epsilon}\left(p\right)\subseteq U$; per le ipotesi esiste $n_0$ tale che $d\left(p,a_n\right)<\epsilon$, per ogni $n\geq n_0$, cioè $a_n\in B_{\epsilon}\left(p\right)\subseteq U$. Dunque $a_n\in U$, per ogni $n\geq n_0$.\qedhere
\end{proof}
\subsection{Punti di accumulazione}
\begin{definition}{}[Punto di accumulazione per la successione]
	Un punto $p\in X$ è \textbf{punto di accumulazione per la successione}\index{punto!di accumulazione!per una successione} $\left\{a_n\right\}$ se
	\begin{equation*}
		\forall U\in I\left(p\right),\ \forall N\in\N\ \exists n\geq N\ \colon a_n\in U.
	\end{equation*}
\end{definition}
\begin{exercise}{n}
	Se $X$ è spazio metrico, allora $p$ è un punto di accumulazione per $\{a_n\}$ se e solo se
	\begin{equation*}
		\forall \epsilon>0,\ \exists n_0\ \colon d\left(a_n,p\right)<\epsilon,\ \forall n\geq n_0.
	\end{equation*}
\end{exercise}

\begin{definition}{}[Punto di accumulazione per il sottoinsieme e derivato]
	Un punto $p\in X$ è \textbf{punto di accumulazione per il sottoinsieme}\index{punto!di accumulazione!per un sottoinsieme} $B\subseteq X$ se
	\begin{equation*}
		\forall U\in I\left(p\right),\ \exists b\in B \colon b\in U\setminus \left\{ p\right\}.
	\end{equation*}
L'insieme dei punti di accumulazione per il sottoinsieme $B$ è chiamato \textbf{derivato}\index{derivato} di $B$ e si indica con $B'$.
\end{definition}
\begin{exercise}{n}
Data la successione $\left\{a_n\right\}$ in $X$ e definito $A\coloneqq \{a_n\mid n\in\N\}$:
\begin{itemize}
\item Un punto di accumulazione $p\in X$ per la successione $\left\{a_n\right\}$ non è mai punto di accumulazione per l'insieme $A$.
\item Un punto di accumulazione $p\in X$ per l'insieme $A$ in generale non è punto di accumulazione per la successione  $\left\{a_n\right\}$; se $X$ è metrico, allora vale l'implicazione.
\end{itemize}
\end{exercise}
\subsection{Sottosuccessioni}
\begin{definition}{}[Sottosuccessione]
Una \textbf{sottosuccessione}\index{successione!sottosuccessione} di $\left\{a_n\right\}$ è la composizione di $\funct{}[a]{\N}{X}$ con un'applicazione \textit{strettamente crescente}
\begin{equation*}
	\funct{}[k]{\N}{\N}[n][k\left(n\right)].
\end{equation*}
Si indica con $\left\{a_{k_n}\right\}_{n\in\N}$=$\left\{a_{k_n}\right\}$.
\end{definition}
\begin{lemma}{}[Lemma delle successioni]\label{lemmatriangolino}
%${\textcolor{blueill}{\ovaled{{ \blacktriangle}}}}$ 
Sia $\left\{a_n\right\}$ una successione su $X$ e $p\in X$. Valgono le seguenti implicazioni:
\begin{equation*}
	\begin{array}{lc}
		1)&\left\{a_n\right\}\ \text{converge a}\ p\\
		&\Downarrow\\
		2)&\left\{a_n\right\}\ \text{ha una sottosuccessione convergente a}\ p\\
		&\Downarrow\\
		3)&p\ \text{è un punto di accumulazione per}\ \left\{a_n\right\}\\
		&\Downarrow\\
		4)&p\in\overline{A}\ \text{dove}\ A=\left\{a_n\mid n\in \N\right\}\subseteq X
	\end{array}
\end{equation*}
\end{lemma}
\begin{proof}{n}~{}\\
$1)\implies2)$ La sottosuccessione convergente è la successione stessa.\\
$2)\implies3)$ Sia $\left\{a_{k\left(n\right)}\right\}$ una sottosuccessione convergente a $p$ e sia $U\in I\left(p\right)$ arbitrario. Se $a_{k\left(n\right)}$ converge a $p$ si ha che esiste $n_0$ tale che $a_{k\left(n\right)}\in U$, per ogni $n\geq n_0$. Poiché $k\left(n\right)$ è strettamente crescente, esiste $n_1$ tale che $k\left(n\right)\geq N,$, per ogni $n\geq n_1$. Allora preso $n=\max\left\{n_0,\ n_1\right\}$, abbiamo che $a_{k\left(n\right)}\in U,\ k\left(n\right)\geq N$. Segue che $p$ è punto di accumulazione per $\left\{a_n\right\}$.\\
$3)\implies4)$ Per definizione di chiusura, $p\in \overline{A}$ se e solo se per ogni $U\in I\left(p\right)$ $A\cap U\neq \emptyset$. Prendiamo $U\in I\left(p\right)$: essendo $p$ punto di accumulazione per $\left\{a_n\right\}$, esiste $n$ tale che $a_n\in U$ e $U\cap A\neq \emptyset$.\qedhere
\end{proof}
\begin{lemma}{}[In $X$ primo numerabile $a_n$ ha una sottosuccessione convergente a $p$ se $p$ di accumulazione per $a_n$]\label{primonumesucc}
Sia $X$ \textit{primo numerabile}, $\left\{a_n\right\}$ successione in $X$ e $p\in X$. Allora vale anche $3)\implies 2)$ del teorema \ref{lemmatriangolino}. In altre parole, $\left\{a_n\right\}$ ha una sottosuccessione convergente a $p$ se e solo se $p$ è di accumulazione per $\left\{a_n\right\}$.
\end{lemma}
\begin{proof}{n}~{}\\
	$\rightimplies$Vale per $2)\implies 3)$ del teorema \ref{lemmatriangolino}.\\
	$\leftimplies$Sia $\left\{U_m\right\}_{m\in\N}$ sistema fondamentale di intorni di $p$ numerabile per ipotesi, essendo $X$ primo numerabile. Consideriamo i seguenti insiemi:
	\begin{equation*}
	\widetilde{U}_m\coloneqq U_1\cap\ldots \cap U_m,\quad \forall m\in \N
	\end{equation*}
\begin{itemize}
	\item $\widetilde{U}_m$ è intorno di $p$, in quanto intersezione \textit{finita} di intorni di $p$.
	\item $\widetilde{U}_m=U_1\cap\ldots\cap U_m\supseteq U_1\cap\ldots\cap U_m\cap U_{m+1}=\widetilde{U}_{m+1}$.
\end{itemize}
Segue che $\left\{\widetilde{U}_m\right\}$ è ancora un sistema fondamentale di intorni numerabile di $p$: infatti, se $V$ è intorno di $p$, $\exists m\ \colon V\supseteq U_m\supseteq \widetilde{U}_m$. A meno di sostituire $U_m$ con $\widetilde{U}_m$, possiamo supporre che $U_1\supseteq U_2\supseteq U_3\supseteq \ldots$. Costruiamo una sottosuccessione di $\left\{a_n\right\}$ convergente a $p$:
\begin{itemize}
	\item $\exists k\left(1\right)\in \N\ \colon a_{k\left(1\right)}\in U_1$.
	\item $\exists k\left(2\right)\geq k\left(1\right)+1\ \colon a_{k\left(2\right)}\in U_2$.
	\item $...$
\end{itemize}
E così via: per ogni $m$ esiste $k\left(m\right)\geq k\left(m-1\right)+1$ tale che $a_{k\left(m\right)}\in U_m$, ottenendo così una sottosuccessione $\left\{a_{k\left(m\right)}\right\}$. Notiamo in particolare che
\begin{equation}
	m_2\geq m_1\implies a_{k\left(m_2\right)}\in U_{m_2}\subseteq U_{m_1}\label{condizionefaccina}
\end{equation}
Mostriamo che $\left\{a_{k\left(n\right)}\right\}$ converge a $p$. Sia $V$ intorno di $p$: dal sistema fondamentale di intorni esiste $m_0$ tale che $U_{m_0}\subseteq V$. Da \eqref{condizionefaccina} si ha che per ogni $m\geq m_0\ a_{k\left(m\right)}\in U_{m_0}\subseteq V$.\qedhere
\end{proof}
\begin{proposition}{}[Caratterizzazione della chiusura con successioni]
	Sia $X$ uno spazio topologico \textit{primo numerabile}, $Y\subseteq X$ e $p\in X$. Sono equivalenti:
	\begin{itemize}
		\item esiste una successione in $Y$ convergente a $p$;
		\item $p$ è di accumulazione per una successione in $Y$;
		\item $p\in \overline{Y}$.
	\end{itemize}
\end{proposition}
\begin{proof}{n}~{}\\
$1)\implies2)$ Non è necessario che $X$ sia primo numerabile, in quanto segue dal lemma \ref{lemmatriangolino} (pag. \pageref{lemmatriangolino}).\\
$2)\implies3)$ Non è necessario che $X$ sia primo numerabile. Se $p$ di accumulazione per $\left\{a_n\right\}$ con $a_n\in y,\forall n$, allora $A=\left\{a_n\mid n\in \N\right\}\subseteq Y$. Segue dal lemma \ref{lemmatriangolino} (pag. \pageref{lemmatriangolino}) che $p\in \overline{A}=\overline{\left\{a_n\mid n\in \N\right\}}\subseteq \overline{Y}$.\\
$3)\implies1)$ Sia $\left\{U_n\right\}$ un sistema fondamentale di intorni di $p$ tale che $U_n\supseteq U_{n+1},\ \forall n$. Allora
\begin{equation*}
	p\in \overline{Y}\implies \forall n\ Y\cap U_n\neq\emptyset\implies\forall n\ \exists y_n\in Y\cap U_n.
\end{equation*}
In modo analogo a \eqref{condizionefaccina} (pag. \pageref{condizionefaccina}), se $n_2\geq n_1$, allora $y_{n_2}\in U_{n_2}\subseteq U_{n_1}$. Allora $\left\{y_n\right\}$ è una successione in $Y$ e converge a $p$. Infatti, sia $V$ intorno di $p$: dal sistema fondamentale di intorni esiste $n_0$ tale che $U_{n_0}\subseteq V$. Dal ragionamento analogo a \eqref{condizionefaccina} si ha che per ogni $n\geq n_0\ y_n\in U_{n_0}\subseteq V$.\qedhere
\end{proof}
\begin{remark}{n}\label{densitaesuccessioni}
	Preso uno spazio topologico $X$ primo numerabile, se $Y\subseteq X$ è un sottoinsieme \textbf{denso}, cioè $\overline{Y}=X$, allora per la proposizione precedente ogni elemento $p\in X$ ammette una successione in $Y$ convergente al punto $p$. In $\R$, ciò comporta che ogni \textit{reale} può essere approssimato da una successione di soli \textit{razionali}, dato che $\overline{\Q}=\R$.
\end{remark}
\section{Successioni e compatti}
\begin{proposition}{}[Successioni di compatti ; Manetti, 4.46]\label{compattocontenuto}
Sia $X$ spazio topologico e sia $K_n\subseteq X,\ \forall n\in\N$ un sottospazio chiuso, \textit{compatto} e non vuoto. Supponiamo inoltre che
\begin{equation*}
K_n\supseteq K_{n+1}\ \forall n\implies K_1\supseteq K_2\supseteq K_3\supseteq \ldots
\end{equation*}
Allora
\begin{equation*}
	\bigcap_{n\geq 1}K_n\neq\emptyset.
\end{equation*}
\end{proposition}
\begin{proof}{n}
Consideriamo $A_n\coloneqq K_1\setminus K_n$:
\begin{itemize}
	\item $K_n$ chiuso in $X$ implica che $K_n$ chiuso in $K_1$. Allora $A_n$ è complementare di un chiuso, dunque aperto in $K_1,\ \forall m\geq 1$.
	\item Dato che $K_n\supseteq K_{n+1}$, allora $A_n\subseteq A_{n+1}\ \forall n\geq 1$.
\end{itemize}
Sia $N\in\N$. Osserviamo che
\begin{equation*}
	\bigcup_{n=1}^{N}A_n=A_N=K_1\setminus K_N\subsetneqq K_1,
\end{equation*}
dunque nessuna unione \textit{finita} degli $A_n$ ricopre $K_1$, cioè
\begin{equation*}
	\bigcup_{n\in\N}A_n\subsetneqq K_1.
\end{equation*}
Allora anche l'unione arbitraria degli $A_n$ \textit{non} copre $X$, altrimenti $\left\{A_n\right\}$ sarebbe un ricoprimento aperto di $K_1$ che \textit{non} ammette sottoricoprimento finito, il che è assurdo in quanto $K_1$ è \textit{compatto}! Pertanto,
\begin{equation*}
\bigcup_{n\geq 1}A_n=\bigcup_{n\geq 1}K_1\setminus K_n=K_1\setminus \left(\bigcap_{n\geq 1}K_n\right)\subsetneqq K_1\implies \bigcap_{n\geq 1}K_n\neq\emptyset.\qedhere
\end{equation*}
\end{proof}
\begin{lemma}{}[Ogni successione in un compatto ha punti di accumulazione]
In uno spazio topologico \textit{compatto} $X$ ogni successione in $X$ ha punti di accumulazione.
\end{lemma}
\begin{proof}{n}
Sia	$\left\{a_n\right\}$ successione in $X$. Per definizione, $p\in X$ punto di accumulazione per $\left\{a_n\right\}$ se e solo se
\begin{equation*}
	\forall U\in I\left(p\right),\ \forall N\in\N\ \exists n\geq N\ \colon a_n\in U.
\end{equation*}
Per $N$ fissato sia $A_N\coloneqq \left\{a_n\mid n\geq N\right\}\subseteq X$. Allora $p\in X$ punto di accumulazione per $\displaystyle\left\{a_n\right\}$ se e solo se
\begin{equation*}
	\forall U\in I\left(p\right),\ \forall N\in\N\ U\cap A_N\neq\emptyset\iff \forall N\in\N,\ p\in\overline{A}_N\coloneqq C_N
\end{equation*}
Dunque
\begin{equation*}
	\left\{\text{punti di accumulazione di} \left\{a_n\right\}\right\}=\bigcap_{N\in\N}C_N
\end{equation*}
e $\left\{a_n\right\}$ ha punti di accumulazione se e solo se
 \begin{equation*}
 	 \bigcap_{N\in\N}C_N\neq\emptyset
 \end{equation*}
\begin{itemize}
	\item $A_B\neq \emptyset$ per definizione, dunque $C_N$ è un chiuso non vuoto.
	\item $C_N$ è chiuso in $X$ compatto, quindi $C_N$ \textit{compatto}.
\end{itemize}
Poiché $A_N=\left\{a_n\mid n\geq N\right\}\supseteq A_ {N+1}=\left\{a_n\mid n\geq N+1\right\}$, si ha
\begin{equation*}
C_N=\overline{A_N}\supseteq \overline{A_{N+1}}=C_{N+1}.
\end{equation*}
Abbiamo trovato una successione di compatti contenuto l'uno nel successivo. Allora per la proposizione \ref{compattocontenuto} (Manetti, 4.46, pag. \pageref{compattocontenuto}) si ha che
\begin{equation*}
	\bigcap_{n\geq 1}C_N\neq \emptyset.-
\end{equation*}
Segue che esiste un punto di accumulazione per la successione.\qedhere
\end{proof}
\subsection{Compattezza per successioni}
\begin{definition}{}[Compatto per successioni]
Sia $X$ spazio topologico. $X$ si dice \textbf{compatto per successioni}\index{spazio!compatto!per successioni} se ogni successione ammette una sottosuccessione convergente.
\end{definition}
\begin{remark}{n}
Per il lemma \ref{lemmatriangolino} (pag. \pageref{lemmatriangolino}), se $X$ è compatto per successioni allora ogni successione in $X$ ha un punto di accumulazione.
\end{remark}
\begin{lemma}{}[Compattezza e prima numerabilità]
	Sia $X$ \textit{primo numerabile.} Allora:
\begin{enumerate}
	\item $X$ compatto per successioni se e solo se ogni successione in $X$ ha un punto di accumulazione.
	\item $X$ compatto se e solo se $X$ compatto per successioni.
\end{enumerate}
\end{lemma}
\begin{proof}{n}~{}
\begin{enumerate}[label=\Roman*]
\item$\rightimplies$Vale per l'osservazione precedente.\\
$\leftimplies$Vale per il lemma \ref{primonumesucc} (pag. \pageref{primonumesucc}): se ogni successione ha un punto di accumulazione in $X$ primo numerabile, allora ogni sottosuccessione ammette una sottosuccessione convergente a $p$, cioè $X$ è compatto per successioni.
\item Se $X$ è compatto, allora ogni successione in $X$ ha dei punti di accumulazione e per il punto 1. segue che $X$ è compatto per successioni.\qedhere
\end{enumerate}
\end{proof}
\begin{proposition}{}[Caratterizzazione della compattezza in termini di successioni]
	Sia $X$ uno spazio topologico a base numerabile. Allora sono equivalenti:
	\begin{enumerate}
		\item $X$ compatto;
		\item $X$ compatto per successioni;
		\item ogni successione in $X$ ammette un punto di accumulazione.
	\end{enumerate}
\end{proposition}
\begin{proof}{n}
Sappiamo già che $2)\iff 3)$ e $1)\implies 2)$ dal lemma precedente. Dobbiamo dimostrare che $2)\implies 1)$. Dimostriamo per contronominale che $\neg 1)\implies \neg 2)$: se $X$ \textit{non} è compatto, allora $X$ non è compatto per successioni, cioè esiste una sottosuccessione in $X$ che \textit{non} ha alcuna sottosuccessione convergente.
\begin{itemize}
	\item Poiché $X$ \textit{non} compatto, esiste $\widetilde{\mathcal{A}}$ ricoprimento aperto di $X$ che \textit{non} ha sottoricoprimenti finiti.
	\item Poiché $X$ a \textit{base numerabile}, esiste $\mathcal{A}$ sottoricoprimento di $\widetilde{\mathcal{A}}$ che è numerabile.
\end{itemize}
Poiché ogni sottoricoprimento di $\mathcal{A}$ è anche un sottoricoprimento di $\widetilde{\mathcal{A}}$, significa che $\mathcal{A}$ \textit{non} ha sottoricoprimenti finiti. Definiamo $\mathcal{A}\coloneqq\left\{A_n\right\}_{n\in\N}$; allora
\begin{equation*}
\forall n\in\N\quad \bigcup_{j=1}^n A_j\subsetneqq X\implies \exists x_n\in X\setminus \bigcup_{j=1}^n A_j
\end{equation*}
Costruiamo così una successione $\left\{x_n\right\}$ in $X$ tale per cui $x_n\notin A_j\ \forall j\leq n$. Mostriamo che $\left\{x_n\right\}$ non ha sottosuccessioni convergenti. Sia $\left\{x_{k\left(n\right)}\right\}$ una sottosuccessione arbitraria di $\left\{x_n\right\}$ e sia $p\in X$, mostriamo che essa non converga ad un qualunque $p$.
\begin{itemize}
	\item $\mathcal{A}$ è un ricoprimento di $X\implies \exists N\ \colon p\in A_N$
	\item Per definizione della successione $\left\{x_n\right\}$, abbiamo che $x_n \notin A_N \ \forall n\geq N$, dato che $x_n \notin A_j \ \forall j\leq n$, in particolare in $A_N$ per ogni $n\geq N$; si ha allora $x_{k\left(n\right)}\notin A_n\ \forall n\ \colon k\left(n\right)\geq N$.
\end{itemize}
Essendo $k\left(n\right)$ crescente, esiste $n_0$ tale che $k\left(n\right)\geq N,\ \forall n\geq n_0$. Segue che se $n\geq n_0$ allora $x_{k\left(n\right)}\notin A_N$. Poiché $A_N$ è intorno di $p$, segue che $\left\{x_{k\left(n\right)}\right\}$ non converge a $p$.
\end{proof}
\begin{theorem}{q}[Equivalenza della compattezza per spazi metrici]
Sia $X$ spazio metrico. Allora $X$ è compatto se e solo se $X$ è compatto per successioni.\qedhere
\end{theorem}
\begin{center}
\begin{tikzcd}
	&  & \begin{array}{c}
		\text{Compatto per}\\\text{successioni}
	\end{array}  \arrow[color=red, lld, "X\text{ a base numerabile}"' sixslant2, shift right, shorten <= 0.5em] \arrow[dd] \arrow[color=red, lld, "X\text{ metrico}" sixslant1, shift right, below, shorten <= 0.5em] \\
	\text{Compatto} \arrow[rrd, shift left, ] &  &                                                                                                                                                 \\
	&  & \begin{array}{c}
		\text{Ogni successione ha}\\\text{un punto di accumulazione}
	\end{array} \arrow[color=red, uu, "\substack{\text{primo}\\\text{numerabile}}"' nine, shift right=4]
\end{tikzcd}
\end{center}
\section{Spazi metrici completi}
\begin{definition}{}[Successione di Cauchy]
	Sia $\left(X,\ d\right)$ uno spazio metrico. Una successione $\left\{a_n\right\}$ si dice \textbf{di Cauchy}\index{successione!di Cauchy} se
	\begin{equation*}
		\forall \epsilon>0\ \exists n_0\in\N\ \colon d\left(a_n,a_m\right)<\epsilon,\ \forall n,\ m\geq n_0.
	\end{equation*}
\end{definition}
\begin{definition}{}[Spazio metrico completo]
	Uno spazio metrico $\left(X,\ d\right)$ si dice \textbf{completo}\index{spazio!metrico!completo} {\small se ogni successione di Cauchy è convergente.}
\end{definition}
\begin{remark}{pn}~{}
	\begin{enumerate}
		\item Ogni successione \textit{convergente} è di \textit{Cauchy}.
		\item Una successione di Cauchy è \textit{convergente} se e solo se ha punti di accumulazione.
		\item Una successione di Cauchy è \textit{convergente} se ha una \textit{sottosuccessione convergente}.
		\item Se $X$ è \textit{compatto}, allora ogni successione di Cauchy è \textit{convergente}.
		\item Se $X$ è spazio metrico \textit{compatto}, allora $X$ è spazio metrico \textit{completo}; non è vero il viceversa.
	\end{enumerate}
\end{remark}
\begin{proof}{n}~{}
	\begin{enumerate}[label=\Roman*]
		\item Se $a_n\to p$ per $n\to +\infty$ significa che
		\begin{equation*}
			\forall \epsilon>0\ \exists n_0\in\N\ \colon d\left(a_n,p\right)<\epsilon,\ \forall n\geq n_0.
		\end{equation*}
	Considerati $n,\ m\geq n_0$ si ha
	\begin{equation*}
		d\left(a_n,a_m\right)\leq d\left(a_n,p\right)+d\left(p,a_n\right)< 2\epsilon.
	\end{equation*}
Per l'arbitrarietà di $\epsilon$ vale la convergenza.
\item $\rightimplies$Sempre vera per \ref{lemmatriangolino} (pag. \pageref{lemmatriangolino}).\\
$\leftimplies$Sia $\left\{a_n\right\}$ una successione di Cauchy e $p$ un suo punto di accumulazione. Sia $\epsilon>0$: dalla definizione di successione di Cauchy esiste $n_0$ tale per cui $d\left(a_n,a_m\right)<\epsilon\ \forall n,\ m\geq n_0$. Essendo $p$ di accumulazione, esiste $ n_1\geq n_0$ tale per cui $d\left(p,a_{n_1}\right)<\epsilon$. Allora, se $n\geq n_0$ si ha
	\begin{equation*}
	d\left(a_n,p\right)\leq d\left(a_n,a_{n_1}\right)+d\left(a_{n_1},p\right)< 2\epsilon,
\end{equation*}
dunque $\left\{a_n\right\}$ converge a $p$.
\item Poiché $X$ è metrico, $X$ è primo numerabile, dunque avere un punto di accumulazione è equivalente ad avere una sottosuccessione convergente.
\item Se $X$ è compatto, ogni successione ha punti di accumulazione, in particolare quelle di Cauchy: per il punto 2. tutte le successioni di Cauchy risultano allora convergenti.
\item Segue dal punto 4. Un controesempio del viceversa è $\mathbb{R}^n$, dato che è completo ma non è compatto (si veda il teorema seguente).\qedhere
\end{enumerate}
\end{proof}
\begin{theorem}{}[$\R^n$ in top. Euclidea è spazio metrico completo]
	$\R^n$ con metrica Euclidea è uno spazio metrico completo.
\end{theorem}
\begin{proof}{n}
	Sia $\left\{a_n\right\}$ di Cauchy in $\R^n$. Mostriamo che $\left\{a_n\right\}$ è eventualmente limitata\footnote{Supponendo chiaramente che la successione sia ben definita, ci interessa solamente che la successione sia limitata dopo un $n_0$: prima di ciò ho un numero finito di termini $a_0,\ \ldots,\ a_{n_0}<\infty$ e posso chiaramente prendere una palla (chiusa) che li contenga, ad esempio di raggio $M+1$ con $M$ definito come nella dimostrazione.}. Poiché la successione di Cauchy è definita per ogni $\epsilon$, fissiamo $\epsilon=1$. Allora
	\begin{equation*}
	\exists n_0\ \colon \norm{a_n-a_m}\leq 1\ \forall n,\ m\geq n_0.
	\end{equation*}
Sia $M\coloneqq \max_{0,\ \ldots,\ n_0} \norm{a_n}$. Se $n\geq n_0$ si ha
\begin{equation*}
 \norm{a_n}=\norm{a_n - a_{n_0}+ a_{n_0}}\leq\norm{a_n - a_{n_0}}+ \norm{a_{n_0}}\leq 1+M.
\end{equation*}
Questo significa che $\left\{a_n\right\}\subseteq\overline{B_{1+M}\left(0\right)}$. Questa palla chiusa è uno spazio metrico \textit{indotto} in $\R^n$ e compatto, cioè è uno \textit{spazio metrico completo}. Allora la successione di Cauchy, trovandosi in uno spazio metrico completo, converge in esso, e dunque converge anche in $\R^n$.\qedhere
\end{proof}
\begin{warning}{n}
La \textbf{completezza} \textit{non} è una proprietà topologica! Per esempio, $\R$ e $\left(0,\ 1\right)$ con metrica Euclidea sono omeomorfi rispetto alla topologia indotta dalla metrica, ma $\R$ abbiamo appena dimostrato che è completo, mentre $\left(0,\ 1\right)$ si può vedere che non lo è!
\end{warning}
