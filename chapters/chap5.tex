% SVN info for this file
\svnidlong
{$HeadURL$}
{$LastChangedDate$}
{$LastChangedRevision$}
{$LastChangedBy$}

\chapter{Azioni di gruppo}
\labelChapter{azionidigruppo}

\begin{introduction}
‘‘I gruppi, come l'uomo, sono noti sulla base delle loro azioni.''
\begin{flushright}
	\textsc{Guillermo Moreno,} un gruppo algebrico che pensava di essere un essere umano.
\end{flushright}
\end{introduction}
\lettrine[findent=1pt, nindent=0pt]{S}{tudieremo} ora uno strumento importante dell'Algebra: l'\textbf{azione di un gruppo} $G$ su un insieme $X$. Grazie a ciò, possiamo dire che gli elementi del gruppo ‘‘\textit{spostano}'' gli elementi dell'insieme in altri, senza dotare l'insieme di alcuna particolare \textit{struttura} od operazione. Alcuni gruppi sono definiti proprio sulla base delle azioni, come il \textit{gruppo simmetrico}: gli elementi sono le \textit{permutazioni} sull'insieme!\\
Per gli scopi della topologia, le azioni di gruppo hanno particolare importanza: se nel \autoref{chap:topologia quoziente} abbiamo visto diversi tipi di spazi quozienti, generalmente dando una \textit{relazione di equivalenza} su un sottoinsieme relativamente semplice, qui introdurremo \textit{quozienti topologici} indotti da queste azioni di gruppo. Fra questi preannunciamo il ruolo di primo piano svolto dallo \textit{spazio proiettivo reale}.
\section{Azione di un gruppo su un insieme}
\begin{definition}{}[Gruppo simmetrico]
	Sia $X$ un insieme. Si definisce il \textbf{gruppo simmetrico} sull'insieme $X$ come:
	\begin{equation*}
		S(X)\coloneqq \Set{\funct{}[f]{X}{X} | f \text{ biunivoca}}
	\end{equation*}
\end{definition}
\begin{definition}{}[Azione di un gruppo su un insieme]
	Sia $G$ un gruppo e $X$ un insieme. Un'\textbf{azione}\index{azione} di $G$ su $X$ è, equivalentemente:
		\begin{itemize}
			\item un morfismo di gruppi $\funct{}[\Phi]{G}{S(X)}$, cioè $\Phi$ soddisfa $\Phi(gh)=\Phi(g)\Phi(h)$.
			\item un applicazione $\funct{}[\phi]{G\times X}{X}[(g,\ x)][g\ldotp x]$ tale che $e\ldotp x=x, \forall x\in X$ e $g\ldotp (h\ldotp x)=(gh)\ldotp x$.
		\end{itemize}
\end{definition}
	Le due definizioni sono effettivamente equivalenti:
	\begin{itemize}
		\item Se l'azione è definita come morfismo di gruppi $\Phi$, si pone $\phi(g,\ x)\coloneqq\Phi(g)(x)$, dove $\Phi(g)\in S(X)$.
		\item Se l'azione è definita come applicazione $\phi$, si pone 
		\begin{equation*}
			\funct{}[\Phi(g)]{X}{X}[x][\phi(g,\ x)].
		\end{equation*}
	\end{itemize}
\begin{definition}{}[Relazione di equivalenza dell'azione]
	Su $X$ si definisce una \textit{relazione di equivalenza} dettata dall'azione di un gruppo $G$:
	\begin{equation*}
		x\sim y\iff \exists g\in G \colon y=g\ldotp x
	\end{equation*}
\end{definition}
\begin{proof}{n}
	Dimostriamo che è una relazione di equivalenza:
	\begin{itemize}
		\item \textit{Riflessività}: $x\sim x$ in quanto $x=e\ldotp x$.
		\item \textit{Simmetria}: poiché $y=g\ldotp x$, allora $x=g^{-1}\ldotp y$ ($g^{-1}\in G$).
		\item \textit{Transitività}: poiché $y=g\ldotp x,\ z=h\ldotp y$, allora $z=h\ldotp \left(g\ldotp x\right)=(hg)\ldotp x$ e $hg\in G$.\qedhere
	\end{itemize}
\end{proof}
\begin{definition}{}[Orbita di un elemento]
	Le classi di equivalenza date da questa relazione sono dette \textbf{orbite}\index{orbita}:
		\begin{equation*}
			[x]=O\left(x\right)=G\ldotp x\coloneqq\Set{y\in X | \exists g\in G \colon y=g\ldotp x}=\Set{ g\ldotp x\in X\mid g\in G}
		\end{equation*}
	L'insieme quoziente è detto \textbf{spazio delle orbite}\index{spazio!delle orbite} e si denota $X/G$.
\end{definition}
\begin{example}{n}
	Il gruppo $\GL(n,\R)$ agisce con l'insieme $\R^n$ con la moltiplicazione matrice per vettore $\phi(A,\mathbf{v})=A\mathbf{v}$. Analizziamo le orbite:
	\begin{itemize}
		\item Per ogni $A\in\GL(n,\R)$, $A\mathbf{0}=\mathbf{0}$; dunque $G\ldotp\mathbf{0}=\{\mathbf{0}\}$.
		\item Fissato $\mathbf{v}\neq \mathbf{0}$, per ogni vettore $\mathbf{w}\neq\mathbf{0}$ esiste sempre $A\in\GL(n,\R)$ tale che $\mathbf{w}=A\mathbf{v}$, dunque $G\ldotp\mathbf{v}=\R^n\setminus\{\mathbf{0}\}$.
	\end{itemize}
	Quindi $\R^n/\GL(n, \R)=\{ a,\ b\}$ con $a=[\mathbf{0}]$ e $b=[\mathbf{v}], \mathbf{v\neq 0}$.
\end{example}
\section{Stabilizzatore di un elemento}
\begin{definition}{}[Stabilizzatore di un elemento]
Lo \textbf{stabilizzatore di un elemento}\index{stabilizzatore di un elemento} $x$ è l'insieme degli elementi di $G$ che fissano $x$:
\begin{equation*}
H_x\coloneqq \left\{g\in G\mid g\ldotp x = x\right\}
\end{equation*}
$H_x$ è un \textit{sottogruppo di isotropia} di $x$. Se $H_x$ è banale, allora l'azione è detta \textbf{libera}\index{azione!libera}.
\end{definition}
\begin{proof}{n}
$H_x$ è chiuso rispetto all'azione:
	\begin{itemize}
		\item $1_G\in H_x$ per definizione dell'azione $g\ldotp$, in quanto $1_G\ldotp x=x, \forall x$.
		\item Per ogni $g,\ h\in H_x$, allora $\left(gh\right)\ldotp x=g\ldotp \left(h\ldotp x\right)=g\ldotp x=x$.\qedhere
	\end{itemize}
\end{proof}
\begin{remark}{n}
L'insieme $G/H_x$ dei laterali sinistri di $H_x$ in $G$ è in corrispondenza biunivoca con l'orbita $O\left(x\right)$. Se $G$ è finito, la cardinalità dell'orbita è l'indice di $H_x$ in $G$.
\end{remark}
\begin{proof}{n}
Sia data
\begin{equation*}
	\funct{}[\alpha]{G/H_x}{O\left(x\right)}[g\ldotp H_x][g\ldotp x].
\end{equation*}
Mostriamo che $\alpha$ è ben definita e biunivoca.
\begin{enumerate}
	\item \textit{Ben definizione}: se $g\ldotp H_x=\widetilde{g}\ldotp H_x$ allora $g^{-1}\widetilde{g}=h\in H_x$, dunque $\widetilde{g}=gh\in H_x$. Si ha:
	\begin{equation*}
		\alpha\left(\widetilde{g}\ldotp H_x\right)=\widetilde{g}\ldotp x = \left(gh\right)\ldotp x = g \ldotp\left( h\ldotp x\right)=g\ldotp x=\alpha\left(g\ldotp H_x\right)
	\end{equation*}
 Poiché $g\ldotp H_x=\widetilde{g}\ldotp H_x$ si ha $g\ldotp x=\widetilde{g}\ldotp x$ e la funzione è ben definita.
 \item \textit{Iniettività}: vale la seguente catena di implicazioni:
 \begin{align*}
 	\alpha\left(g_1\ldotp H_x\right)&=\alpha\left(g_2\ldotp H_x\right)&\implies g_1\ldotp x=g_2\ldotp x\implies g_2^{-1}\ldotp\left(g_1\ldotp x\right)=g_2^{-1}\ldotp\left(g_2\ldotp x\right)\\
 	&\implies \left(g_2^{-1}\ldotp g_1\right)\ldotp x=1_G\ldotp x=x
 \end{align*}
Ne segue che $\left(g_2^{-1} g_1\right)\in H_x$, dunque $g_2^{-1} g_1=h\in H_x$ e $g_1\ldotp H_x=g_2\ldotp H_x$.
\item \textit{Suriettività}: se $y\in O\left(x\right)$, per definizione $\exists g\in G\ \colon y=g\ldotp x$, cioè $y=\alpha\left(g\ldotp H_x\right)$.
\end{enumerate}
Segue, dal teorema di Lagrange, che
\begin{equation*}
	\lvert O\left(x\right)\rvert = \left[G\ \colon H_x\right]= \frac{\lvert G\rvert}{\lvert H_x\rvert}.\qedhere
\end{equation*}
\end{proof}
\begin{remark}{n}
Punti nella stessa orbita hanno stabilizzatori \textbf{coniugati}\index{coniugati}:
\begin{equation*}
x_2=g\ldotp x_1\implies H_{x_2}=g\ldotp H_{x_1}\ldotp g^{-1}
\end{equation*}
\end{remark}
\begin{proof}{n}~{}\\
$\rightinclude$Sia $h\in H_{x_2}$. Poiché  per ipotesi $x_2=g\ldotp x_1$, si ha
\begin{equation*}
h\ldotp x_2 = x_2\implies h\ldotp\left(g\ldotp x_1\right) = g\ldotp x_1 \implies \left(g^{-1} h g\right)\ldotp x_1 = x_1.
\end{equation*}
Segue che per ogni $h\in H_{x_2}$ si ha $g^{-1} h g\in H_{x_1}$, ma allora $h = g \left(g^{-1} h^{-1} g\right) g^{-1}\in g\ldotp H_{x_1}\ldotp g^{-1}$. Pertanto, per l'arbitrarietà di $h$ si ha $H_{x_2} \subseteq g \ldotp H_{x_1}\ldotp g^{-1}$.
$\leftinclude$Sia $h\in H_{x_1}$ e consideriamo $ghg^{-1}$. Se applico $ghg^{-1}$ all'elemento $x_2$:
\begin{equation*}
\left(ghg^{-1}\right)\ldotp x_2=\left(ghg^{-1}\right)\ldotp g\ldotp x_1=\left(gh\right)\ldotp \left(g^{-1}g\right)\ldotp x_1=\left(gh\right)\ldotp x_1=g\ldotp\left(h\right)\ldotp x_1=g\ldotp x_1=x_2
\end{equation*}
Pertanto, per ogni $h\in H_{x_1}$ si ha $\left(ghg^{-1}\right)\ldotp x_2= x_2$ e per l'arbitrarietà di $h$ segue $g\ldotp H_{x_1}\ldotp g^{-1}\subseteq H_{x_2}$
\end{proof}
\section{Azione per omeomorfismi}
\begin{definition}{}[Azione per omeomorfismi]
Sia $X$ uno spazio topologico e $G$ un gruppo che agisce su $X$. Diciamo che $G$ \textbf{agisce per omeomorfismi}\index{azione!per omeomorfismi} se per ogni $g\in G$ l'applicazione
\begin{equation*}
\funct{}[\theta_g]{X}{X}[x][g\ldotp x]
\end{equation*}
è un \textit{omeomorfismo}. Questo è equivalente a chiedere che l'azione sia data da un \textit{omomorfismo di gruppi}
\begin{equation*}
\funct{}[\Phi]{G}{\left\{\text{omeomorfismi}\ X\rightarrow X\right\}\leq S\left(x\right)}.
\end{equation*}
\end{definition}
\begin{lemma}{n}[{$G$} agisce per omeomorfismi se e solo se {$\theta_g$} è continua, {$\forall g\in G$}]
\end{lemma}
\begin{proof}{n}~{}\\
$\rightimplies$$G$ agisce per omeomorfismi se $\theta_g$ è un omeomorfismo per ogni $g$, dunque banalmente $\theta_g$ è continua.\\
$\leftimplies$Dobbiamo dimostrare che $\theta_g$ è un omeomorfismo. Notiamo che per ogni $g,\ h\in G$ vale $\theta_g\circ \theta_h = \theta_{gh}$:
\begin{equation*}
	\left(\theta_g\circ\theta_h\right)\left(x\right)=\theta_g\left(\theta_h\left(x\right)\right)=\theta_g\left(h\ldotp x\right)=g\ldotp\left(h\ldotp x\right)=\left(gh\right)\ldotp x=\theta_{gh}\left(x\right),\ \forall x\in X.
\end{equation*}
Inoltre, vale $\theta_e=Id_X$:
\begin{equation*}
	\theta_e\left(x\right)=e\ldotp x=x=Id\left(x\right),\ \forall x\in X.
\end{equation*}
Allora, si verifica che
\begin{equation*}
	\left(\theta_{g^{-1}}\circ\theta_g\right)\left(x\right)=\theta_{g^{-1}g}\left(x\right)=\theta_e\left(x\right)=Id_X\left(x\right)=x,\ \forall x\in X,\ \forall g\in G,
\end{equation*}
cioè l'inversa di $\theta_g$ è $\theta_{g^{-1}}$. Di conseguenza, anche $\left(\theta_{g}\right)^{-1}=\theta_{g^{-1}}$ è continua per ipotesi perché $g^{-1}\in G$, dunque segue la tesi.\qedhere
\end{proof}
\begin{proposition}{n}[Azione per omeomorfismi e proiezione aperta]\label{proiezione azione gruppo aperta}~{}
\begin{enumerate}
\item Sia $X$ uno spazio topologico e $G$ un gruppo che agisce su $X$ per omeomorfismo. La proiezione $\funct{}[\pi]{X}{X/G}$ dall'insieme allo spazio delle orbite $X/G$ è aperta e, se $G$ è finito, $\pi$ è anche chiusa.
\item Sia $X$ uno spazio topologico di Hausdorff e $G$ gruppo finito che agisce su $X$ per omeomorfismi. Allora $X/G$ è di Hausdorff.
\end{enumerate}
\end{proposition}
\begin{proof}{n}~{}
\begin{enumerate}[label=\Roman*]
\item Dato $A\subseteq X$ aperto, vogliamo dimostrare che $\pi\left(A\right)$ è aperto in $X/G$ e quindi, per definizione di topologia quoziente, che $\pi^{-1}\left(\pi\left(A\right)\right)$ è aperto in $X$. Ogni elemento di $A$ è contenuto in un orbita, dunque $\pi\left(A\right)$ contiene le orbite degli $x\in A$; la controimmagine $\pi^{-1}\left(\pi\left(A\right)\right)$ risulta dunque pari all'unione di \textit{tutte} le orbite in $X$ che intersecano l'insieme $A$, che indichiamo con $g\ldotp A$:
\begin{equation*}
\pi^{-1}\left(\pi\left(A\right)\right)=\bigcup_{g\in G} g\ldotp A.
\end{equation*}
Ma allora $g\ldotp A=\left\{g\ldotp x\mid x\in A\right\}$ è un aperto al variare di $g\in G$ poiché un omeomorfismo porta aperti in aperti; l'unione di aperti è aperta, dunque $\pi^{-1}\left(\pi\left(A\right)\right)$ è aperto in $X$ e, come desiderato, $\pi\left(A\right)$ è aperto in $X/G$. In modo analogo con $C\subseteq X$ chiuso otteniamo che
\begin{equation*}
	\pi^{-1}\left(\pi\left(C\right)\right)=\bigcup_{g\in G} g\ldotp C,
\end{equation*}
con $g\ldotp C$ chiuso per omeomorfismo. In particolare, essendo $G$ finito, segue che l'unione dei $g\ldotp C$ è finita e dunque anch'essa è un chiuso in $X$ e, come desiderato, $\pi\left(C\right)$ chiuso in $X/G$.
\item Siano $p,\ q\in X/G$ distinti; vogliamo dimostrare che esistono intorni di $p$ e $q$ disgiunti. Siano $x,\ y\in X$ tali che $\pi\left(x\right)=p$ e $\pi\left(y\right)=q$ e consideriamo il gruppo finito $G=\left\{g_1=1_G,\ g_2,\ \ldots,\ g_n\right\}$. Le orbite di $x$ e $y$ sono distinte: se così non fosse, si avrebbe $\pi\left(x\right)=\pi\left(y\right)$ e cioè $p=q$, il che è assurdo! Allora $g_i\ldotp x\neq g_y\ldotp y\quad, \forall i,\ j$. Definiti in $X$ due intorni aperti $U\in I\left(x\right)$ e $V\in I\left(y\right)$ disgiunti per Hausdorff, possiamo considerare altri intorni aperti disgiunti $g_i\ldotp U\in I\left(g_i\ldotp x\right),\ g_i\ldotp V\in I\left(g_i\ldotp y\right)$. Allora
\begin{equation*}
\widetilde{U}\coloneqq \bigcup_{i}g_i\ldotp U\qquad \widetilde{V}\coloneqq \bigcup_{i}g_i\ldotp V
\end{equation*}
sono entrambi aperti. Vogliamo costruire degli opportuni $U\in I\left(x\right)$ e $V\in I\left(y\right)$ in modo che siano intorni aperti disgiunti tali che, costruiti come sopra $\widetilde{U},\ \widetilde{V}$, si abbia $\widetilde{U}\cap \widetilde{V}=\emptyset$. In questo modo, passando al quoziente con $\pi$, si otterranno degli intorni $\pi\left(\widetilde{U}\right)$ di $p$ e $\pi\left(\widetilde{V}\right)$ di $q$ che soddisfano $\pi\left(\widetilde{U}\right)\cap \pi\left(\widetilde{V}\right)=\emptyset$.
\begin{itemize}
\item Costruiamo $U$ e $V$: per ogni $i$ sappiamo che $x\neq g_i\ldotp y$ in $X$ in quanto le orbite di $x$ e $y$ sono distinte. In quanto $X$ è di Hausdorff, si ha che per ogni $i$ $\exists U_i,\ V_i$ intorni aperti disgiunti tali che $x\in U_i$ e $g_i\ldotp y\in V_i$. Notiamo che $y\in g_i^{-1}\ldotp V_i$; allora definiamo
\begin{equation*}
U\coloneqq \bigcap_{i}^{n}U_i\in I\left(x\right)\qquad V\coloneqq \bigcap_{i=1}^{n}g_i^{-1}\ldotp V_i\in I\left(y\right).
\end{equation*}
\item Ricaviamo $\widetilde{U}$ e $\widetilde{V}$: per ogni elemento di $G$, ossia per per ogni $i$, abbiamo
\begin{equation*}
U\cap \left(g_i\ldotp V\right)\subseteq U_i \cap \left(g_i\ldotp g_i^{-1}\ldotp V_i\right)=U_i\cap V_i=\emptyset\implies U\cap \left(g_i\ldotp V\right)=\emptyset.
\end{equation*}
Allora per ogni $ i,\ j$ abbiamo
\begin{equation*}
	\left(g_i\ldotp U\right)\cap \left(g_j\ldotp V\right)=\left(g_i\ldotp U\right)\cap \left(g_ig_i^{-1}g_j\ldotp V\right)= g_i\ldotp \left(U\cap \left(g_i^{-1}g_j\right)\ldotp V\right),
\end{equation*}
ma $g_i^{-1}g_j \in G$, dunque $U\cap \left(g_i^{-1}g_j\right)\ldotp V=\emptyset$. Segue che
\begin{equation*}
	\left(g_i\ldotp U\right)\cap \left(g_j\ldotp V\right)=\emptyset\implies
	\left(\bigcup_{i}g_i\ldotp U\right)\cap \left(\bigcup_{i}g_i\ldotp V\right)=\emptyset\implies \widetilde{U}\cap \widetilde{V}=\emptyset.\qedhere
\end{equation*}	
\end{itemize}
\end{enumerate}
\end{proof}
\begin{example}{n}
	$\left(\Z,\ +\right)$ agisce in $\R$ per \textbf{traslazione}\index{traslazione}:
	\begin{equation*}
		m\ldotp x = x+m
	\end{equation*}
Se mettiamo su $\R$ la topologia Euclidea, allora l'azione è per omeomorfismi, dato che per ogni scelta di $m\in \Z$
\begin{equation*}
	\funct{}[\theta_m]{\R}{\R}[x][x+m]
\end{equation*}
è continua.
\begin{itemize}
	\item \textit{Orbite}: $O\left(x\right)=\left\{x+m\mid m\in \Z\right\}$ rappresenta tutti i numeri che hanno mantissa uguale. Ad esempio, l'orbita di $x=1.5$ continue $1.5,\ 2.5,\ -1.5,\ \ldots$.
	\item \textit{Stabilizzatore}: $H_x=\left\{m\in Z\mid x+m=x\right\}=\left\{0\right\}$ è banale, dunque l'azione è libera.
	\item \textit{Spazio delle orbite}: $\R/\Z$ è insiemisticamente in corrispondenza biunivoca con $\left[0,\ 1\right)$, in particolare un sistema di rappresentanti di $\R/\Z$ sono le orbite al variare di $x\in[0,1)$. In termini topologici, lo spazio delle orbite è compatto poiché
	\begin{equation*}
		\pi\left(\left[0,\ 1\right]\right)=\R/\Z
	\end{equation*}
	ed è omeomorfo a $S^1$.
\end{itemize}
\end{example}
\begin{proof}{n}
Consideriamo
\begin{equation*}
	\funct{}[f]{\R}{S^1\subseteq \R}[t][\left(\cos\left(2\pi t\right),\sin\left(2\pi t\right)\right)].
\end{equation*}
\begin{itemize}
\item $f$ è continua.
\item $f$ è suriettiva.
\item Si ha $f\left(t_1\right)=f\left(t_2\right)\iff t_1-t_2\in \Z \iff t_1,\ t_2$ nella stessa orbita $\iff \pi\left(t_1\right)=\pi\left(t_2\right)\iff t_1\sim t_2$
\end{itemize}
Allora la relazione di equivalenza indotta da $f$ è quella dell'azione di $\Z$ su $\R$.\\
\begin{minipage}[t]{0.83\textwidth}
 Inoltre, $f$ induce $\funct{}[\overline{f}]{\R/\Z}{S^1}$ continua per le proprietà della topologia quoziente e che rende commutativo il diagramma a lato.
 Infatti $\overline{f}$ è biunivoca in quanto suriettiva - essendolo è $f$ - e iniettiva - per conseguenza del sistema di rappresentanti che si ha su $\R/\Z$.
\end{minipage}
\begin{minipage}[t]{0.13\textwidth}\vspace{-10pt}
\begin{tikzcd}
	\R \arrow[r, "f"] \arrow[d, "\pi"']                                   & S^1 \\
	\R/\Z \arrow[ru, "\exists\overline{f}"', dotted] &
\end{tikzcd}
\end{minipage}\\
Inoltre, essendo $\R/\Z$ compatto ed $S^1$ di Hausdorff, $\overline{f}$ è chiusa e dunque $\overline{f}$ è l'\textit{omeomorfismo} cercato. Per questo motivo, si ha anche che $f$ è un'\textit{identificazione} aperta.\qedhere
\end{proof}
\begin{digression}{n}\label{complessir2}
	Si può sempre vedere $\R^2$ come lo spazio dei complessi $\C$. Allora
	\begin{equation*}
		 S^1=\left\{z\mid \lvert z\rvert=1\right\}\subset\C
	\end{equation*}
	e la funzione di prima si può anche riscrivere equivalentemente nelle forme
	\begin{gather}
		\left(\cos\left(2\pi t\right),\sin\left(2\pi t\right)\right)\leftrightarrow \cos\left(2\pi t\right)+i\sin\left(2\pi t\right)\leftrightarrow e^{2\pi i t}.
	\end{gather}
\end{digression}
\begin{example}{n}
Ricordiamo che il gruppo $\GL(n,\R)$ agisce con l'insieme $\R^n$ con la moltiplicazione matrice per vettore $\phi(A,\mathbf{v})=A\mathbf{v}$. Sappiamo che ci sono solo due orbite, dunque lo spazio delle orbite è $\R^n/\GL(n,\R)=\left\{\left[\mathbf{0}\right],\ \left[\mathbf{v}\right]\right\}$. L'azione è per omeomorfismi, dato che, fissato $A\in G$,
\begin{equation*}
	\funct{}[\theta_A]{\R^n}{\R^n}[\mathbf{v}][A\mathbf{v}],
\end{equation*}
è continua. Tuttavia, $\R^n/\GL(n,\R)$ non è di Hausdorff. Infatti, $\left\{\left[\mathbf{v}\right]\right\}$ è aperto ma non chiuso in $\R^n/G$ in quanto $\pi^{-1}\left(\left[\mathbf{v}\right]\right)=\R^n\setminus\left\{0\right\}$, che è un aperto ma \textit{non} è un chiuso di $\R^n$: $\R^n/G$ non è T1 e tanto meno è di Hausdorff.
\end{example}
\subsection{Spazio proiettivo reale}
In questa sezione introduciamo uno spazio topologico molto importante per i nostri studi, lo \textbf{spazio proiettivo reale}. Nel \autoref{chap:varietà} guarderemo il suo comportamento come \textit{varietà topologica}, mentre nel \autoref{chap:geoproiettiva} tratteremo una sua \textit{generalizzazione} su un qualunque campo $\K$ con gli strumenti dell'\textit{algebra lineare}, oltre a discutere topologicamente lo \textit{spazio proiettivo complesso}.
\begin{example}{n}
	Il gruppo moltiplicativo $\R^{\ast}=\R\setminus\left\{0\right\}$ agisce su $\R^{n+1}\setminus\left\{0\right\}$ con l'azione di moltiplicazione per uno scalare $\lambda\ldotp \mathbf{x} \coloneqq \lambda\mathbf{x}$. L'azione è per omeomorfismi in quanto, fissato $\lambda\in G$,
\begin{equation*}
	\funct{}[\theta_\lambda]{\R^{n+1}\setminus\left\{0\right\}}{\R^{n+1}\setminus\left\{0\right\}}[\mathbf{x}][\lambda \mathbf{x}]
\end{equation*}
è continua.
\begin{itemize}
	\item \textit{Orbite}: le orbite
	\begin{equation*}
		O\left(\mathbf{x}\right)=\left\{\lambda \mathbf{x}\mid \lambda\in G\right\}
	\end{equation*}
	rappresentano tutte le rette vettoriali passanti per l'origine in $\R^{n+1}$ private dell'origine.
	\item \textit{Spazio delle orbite}: $\Proj^n\left(\R\right)\coloneqq\left(\R^{n+1}\setminus\left\{0\right\}\right)/\left(\R^{n+1}\setminus\left\{0\right\}\right)$ è lo \textbf{spazio proiettivo reale}, spazio topologico rispetto alla topologia quoziente indotta dall'azione.
\end{itemize}
\end{example}
\begin{definition}{}[Spazio proiettivo reale]\label{def spazio proiettivo}
Lo \textbf{spazio proiettivo reale}\index{spazio!proiettivo!reale} $\Proj^n\left(\R\right)$ o $\R\Proj^n$ di dimensione $n$ è lo spazio topologico delle rette vettoriali passanti l'origine in $\R^{n+1}$, a cui son state tolte l'origine. È definito come lo spazio quoziente rispetto all'azione del gruppo moltiplicativo $\R^{\ast}=\R\setminus\left\{0\right\}$:
	\begin{equation*}
		\Proj^n\left(\R\right)= \left(\R^{n+1}\setminus \left\{0\right\}\right)/\left(\R^{\ast}=\R\setminus\left\{0\right\}\right).
	\end{equation*}
\end{definition}
\begin{proposition}{n}[${\Proj^n\left(\R\right)}$ è di Hausdorff, compatto e c.p.a.] \label{spazi proiettivi compatti connessi}
\end{proposition}
\begin{proof}{n}~{}
\begin{enumerate}[label=\Roman*]
\item Dati $p,\ q\in\Proj^n\left(\R\right)$, $p\neq q$, sono della forma $p=\left[\mathbf{x}\right]$ e $q=\left[\mathbf{y}\right]$. Allora
\begin{equation*}
\left[\mathbf{x}\right]\neq\left[\mathbf{y}\right]\implies\mathcal{L}_0\left(\mathbf{x}\right)\neq\mathcal{L}_0\left(\mathbf{y}\right),
\end{equation*}
con $\mathcal{L}_0\left(\mathbf{x}\right),\ \mathcal{L}_0\left(\mathbf{y}\right)$ le rette vettoriali descritte da $\mathbf{x}$ e $\mathbf{y}$. Prendiamo gli intorni aperti disgiunti $U\setminus\left\{0\right\}\in I\left(\mathbf{x}\right)$, $V\setminus\left\{0\right\}\in I\left(\mathbf{y}\right)$ in $\R^{n+1}\setminus\left\{0\right\}$. Allora, passando al quoziente, $\pi\left(U\setminus\left\{0\right\}\right)$ e $\pi\left(V\setminus\left\{0\right\}\right)$ formano due fasci di rette a forma di ‘‘doppio cono infinito'' con vertice nell'origine; questi due coni sono intorni aperti in quanto
\begin{equation*}
\pi^{-1}\left(\pi\left(U\setminus\left\{0\right\}\right)\right)=U\setminus\left\{0\right\}\qquad\pi^{-1}\left(\pi\left(V\setminus\left\{0\right\}\right)\right)=V\setminus\left\{0\right\}.
\end{equation*}
e inoltre sono intorni disgiunti di $p$ e $q$, dunque $\Proj^n\left(\R\right)$ è di Hausdorff.
\item Per dimostrare che $\Proj^n\left(\R\right)$ è compatto, dimostriamo che $\pi\left(S^n\right)=\Proj^n\left(\R\right)$, dato che $S^n\subseteq R^{n+1}\setminus\left\{0\right\}$ è compatto. Notiamo che, presa l'orbita di un vettore $\mathbf{v}$, si ha
\begin{equation*}	
\left[\mathbf{v}\right]=\left\{\lambda \mathbf{v}\mid \lambda\in \R^{\ast}\right\}=\Set{\lambda \norm{\mathbf{v}}\frac{\mathbf{v}}{\norm{\mathbf{v}}} | \lambda\in \R^{\ast}}=\left\{\mu \mathbf{x}\mid \mu\in \R^{\ast}\right\}=\left[\mathbf{x}\right]
\end{equation*}
dove $\mu\coloneqq\lambda \norm{\mathbf{v}}\in \R^{\ast}$ e  $\mathbf{x}\coloneqq\mathbf{v}/\norm{\mathbf{v}}\in S^1$; ogni orbita dello spazio proiettivo reale si può scrivere come l'orbita di un vettore appartenente alla sfera $S^n$. Segue che $\pi_{\mid S^n}$ è suriettiva, cioè $\pi_{\mid S^n}\left(S^n\right)=\Proj^n\left(\R\right)$ e dunque $\pi\left(S^n\right)=\Proj^n\left(\R\right)$. Dato che $S^n$ è compatto e c.p.a., segue che anche lo spazio proiettivo reale è compatto e c.p.a. in quanto immagine continua di $S^n$.\qedhere
\end{enumerate}
\end{proof}