% SVN info for this file
\svnidlong
{$HeadURL$}
{$LastChangedDate$}
{$LastChangedRevision$}
{$LastChangedBy$}

\chapter{Azioni di gruppo}
\labelChapter{azionidigruppo}

\begin{introduction}
‘‘BEEP BOOP INSERIRE CITAZIONE QUA BEEP BOOP.''
\begin{flushright}
	\textsc{NON UN ROBOT,} UN UMANO IN CARNE ED OSSA BEEP BOOP.
\end{flushright}
\end{introduction}

\section{Azione di un gruppo su un insieme}
\begin{center}
	[...]
\end{center}
\begin{define}
Lo \textbf{stabilizzatore di un elemento}\index{stabilizzatore di un elemento} è l'insieme degli elementi di $G$ che fissano $x$:
\begin{equation}
H_x\coloneqq \left\{g\in G\mid g\ldotp x = x\right\}
\end{equation}
$H_x$ è un \textit{sottogruppo di isotropia} di $x$.
\end{define}
\begin{demonstration} $H_x$ è chiuso rispetto all'azione:
	\begin{itemize}
		\item $1_G\in H_x$ per definizione dell'azione $g\ldotp$ ($1_G\ldotp x=x\ \forall x$).
		\item $\forall g,\ h\in H_x$, allora $\left(gh\right)\ldotp x=g\ldotp \left(h\ldotp x\right)=g\ldotp x=x$.
	\end{itemize}
\end{demonstration}
\begin{observe}
L'insieme $\nicefrac{G}{H_x}$ dei laterali sinistri di $H_x$ in $G$ è in corrispondenza biunivoca con l'\textit{orbita} $O\left(x\right)$. Inoltre, se $G$ è finito, la cardinalità dell'orbita è pari all'indice di $H_x$ in $G$.
\end{observe}
\begin{demonstration}
Sia data:
\begin{equation*}
	\funztot{\alpha}{\nicefrac{G}{H_x}}{O\left(x\right)}{g\ldotp H_x}{g\ldotp x}
\end{equation*}
Mostriamo che $\alpha$ è ben definita e biunivoca.
\begin{enumerate}
	\item \textit{Ben definizione}: se $g\ldotp H_x=\tilde{g}\ldotp H_x$ allora $g^{-1}\tilde{g}=h\in H_x\implies\tilde{g}=gh\in H_x$. Si ha:
	\begin{equation*}
		\alpha\left(\tilde{g}\ldotp H_x\right)=\tilde{g}\ldotp x = \left(gh\right)\ldotp x = g \ldotp\left( h\ldotp x\right)=g\ldotp x=\alpha\left(g\ldotp H_x\right)
	\end{equation*}
 Poiché $g\ldotp H_x=\tilde{g}\ldotp H_x\implies g\ldotp x=\tilde{g}\ldotp x$ la funzione è ben definita.
 \item \textit{Iniettività}:
 \begin{align*}
 	\alpha\left(g_1\ldotp H_x\right)=\alpha\left(g_2\ldotp H_x\right)&\implies g_1\ldotp x=g_2\ldotp x\implies g_2^{-1}\ldotp\left(g_1\ldotp x\right)=g_2^{-1}\ldotp\left(g_2\ldotp x\right)\\
 	&\implies \left(g_2^{-1}\ldotp g_1\right)\ldotp x=1_G\ldotp x=x
 \end{align*}
Ne segue che $\left(g_2^{-1} g_1\right)\in H_x\implies g_2^{-1} g_1=h\in H_x\implies g_1\ldotp H_x=g_2\ldotp H_x$.
\item \textit{Suriettività}: se $y\in O\left(x\right)$, per definizione $\exists g\in G\ \colon y=g\ldotp x$, cioè $y=\alpha\left(g\ldotp H_x\right)$.
Ne consegue, dal teorema di Lagrange, che $\lvert O\left(x\right)\rvert = \left[G\ \colon H_x\right]= \frac{\lvert G\rvert}{\lvert H_x\rvert}$.
\end{enumerate}
\end{demonstration}