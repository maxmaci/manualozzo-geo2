% SVN info for this file
\svnidlong
{$HeadURL$}
{$LastChangedDate$}
{$LastChangedRevision$}
{$LastChangedBy$}

\chapter{Omotopia}
\labelChapter{omotopia}

\begin{introduction}
	‘‘BEEP BOOP INSERIRE CITAZIONE QUA BEEP BOOP.''
	\begin{flushright}
		\textsc{NON UN ROBOT,} UN UMANO IN CARNE ED OSSA BEEP BOOP.
	\end{flushright}
\end{introduction}

\section{Lemma di incollamento}
\begin{lemming}\textsc{Lemma di incollamento}\label{lemmaincollamento}\\
	Siano $X,\ Y$ spazi topologici e $X=A\cup B$. Siano $\funz{f}{A}{Y}$ e $\funz{g}{B}{Y}$ continue tali che $f\left(x\right)=g\left(x\right)\ \forall x\in A\cap B$, cioè $f_{\mid A\cap B}=g_{\mid A\cap B}$.\\
	Consideriamo l'\textbf{incollamento}\index{incollamento} $\funz{h}{X}{Y}$ definito da:
	\begin{equation}
		h\left(x\right)=\begin{cases}
			f\left(x\right)\ x\in A\\
			g\left(x\right)\ x\in B
		\end{cases}
	\end{equation}
	Se $A$ e $B$ sono entrambi aperti in $X$, oppure se $A$ e $B$ sono entrambi chiusi in $X$, allora $h$ è continua.
\end{lemming}
\begin{demonstration}
	Supponiamo $A$ e $B$ aperti. Sia $U\subseteq Y$ aperto. Allora:
	\begin{equation*}
		h^{-1}\left(U\right)=\underbrace{f^{-1}\left(U\right)}_{\subseteq B}\cap \underbrace{g^{-1}\left(U\right)}_{\subseteq B}
	\end{equation*}
Essendo $f,\ g$ continue, segue che $f^{-1}\left(U\right)$ è aperto in $A$ e $g^{-1}\left(U\right)$ è aperto in $B$.\\
In quanto $A,\ B$ aperti, per definizione di aperto del sottospazio\footnote{Poichè un aperto del sottospazio è dato dall'intersezione del sottospazio con un aperto di $X$, se abbiamo che anche il sottospazio è aperto di $X$, l'intersezione è aperta: in questo caso ogni aperto del sottospazio è anche aperto di $X$.} $f^{-1}\left(U\right)$ e $g^{-1}\left(U\right)$ sono aperti su $X\implies h^{-1}\left(U\right)$ aperto.\\
Il caso di $A$ e $B$ chiuso è esattamente analogo. 
\end{demonstration}
\section{Componente connessa e componente c.p.a.}
Riprendiamo la trattazione delle componenti connesse e \textbf{c.p.a.} introdotte nel capitolo \textbf{XXX}.
\begin{define}
	Una \textbf{componente connessa}\index{componente!connessa} di $X$ spazio topologico è uno spazio $C\subseteq X$ \textit{connesso} tale per cui:
	\begin{equation}
		C\subseteq A\subseteq X\text{ con }A\text{ connesso}\implies C=A 
	\end{equation} 
\end{define}
\begin{observe}~{}
\begin{itemize}
	\item Le componenti connesse formano una \textit{partizione} di $X$.
	\item Se $x\in X$ si può definire la componente connessa che contiene $x$:
	\begin{equation}
		C\left(x\right)=\union\left\{C\subseteq X\mid x\in C,\ C\text{ connesso}\right\}
	\end{equation}
\ item Le componenti connesse possono essere viste come classi di equivalenza per la seguente relazione di equivalenza su $ X $:
	\begin{equation}
		x,\ y\in X\qquad x\sim_C y\iff \exists C\subseteq X\text{ connesso}\ \colon x,\ y\in C
	\end{equation}
\end{itemize}
\end{observe}
\begin{demonstration}
Innanzitutto mostriamo che la relazione è di equivalenza:
\begin{itemize}
\item \textsc{Riflessiva}: $x\sim_C x$ è vero, dato che $\left\{x\right\}$ è sempre un connesso.
\item \textsc{Simmetrica}: ovvia dalla definizione.
\item \textsc{Transitiva}: Supponiamo $x\sim_C y,\ y\sim_C z$. Allora $\exists C,\ D\subseteq X$ connessi tale che $x,\ y\in C$ e $y,\ z\in D$. Allora $C\cup D$ contiene sia $x$ che $z$. Inoltre, essendo $y\in C\cap D\implies C\cap D\neq \emptyset$, dunque $C\cup D$ è un connesso: vale $x\sim_C z$.\\
Mostriamo che le classi di equivalenza sono le componenti connesse per $x$.\\
$\includedx$ Se $C\subseteq X$ è una componente connessa, allora $\forall x, y\in C$ si ha $x\sim_C y$, cioè $C$ è interamente contenuta in $C_0=[x]=[y]$ classe di equivalenza per $\sim_C$: $C\subseteq C_0$.\\
$\includesx$ Sia $z\in C_0$ classe di equivalenza e sia $x\in C$ componente connessa. Allora: $x\sim_C z\implies \exists T\subseteq X \text{ connesso}\ \colon x,\ z\in T$.\\
Consideriamo $C\cup T$. $C$ e $T$ sono connessi, $x\in C\cap T\implies C\cap T\neq \emptyset$: $C\cup T$ è ancora connessa. In quanto $C$ è componente connessa, dato che $C\subseteq C\cup T$ per definizione segue che $C=C\cup T$, cioè $T\subseteq C$. Ma allora $z\in C$ e segue che $C_0\subseteq C$.\\
\end{itemize}
\end{demonstration}