% SVN info for this file
\svnidlong
{$HeadURL$}
{$LastChangedDate$}
{$LastChangedRevision$}
{$LastChangedBy$}

\chapter{Geometria proiettiva}
\labelChapter{geoproiettiva}

\begin{introduction}
	‘‘La Geometria proiettiva è tutta la Geometria.''
	\begin{flushright}
		\textsc{Arthur Cayley,} cercando di vendere i suoi appunti di Geometria proiettiva agli ignari studenti di Geometria uno.
	\end{flushright}
\end{introduction}
\lettrine[findent=1pt, nindent=0pt]{A}{bbiamo} già trattato lo \textit{spazio proiettivo reale} e le sue caratteristiche nel \autoref{chap:azionidigruppo} e \autoref{chap:varietà}. In questo, ci dedicheremo a generalizzare il concetto per un \textit{qualsiasi} spazio vettoriale su campo $\K$, utilizzando gli strumenti dell'algebra lineare.\\
Parliamo dunque di \textbf{Geometria proiettiva}: come in topologia studiavamo le proprietà degli spazi topologici invarianti per omeomorfismi, lo scopo della geometria proiettiva è studiare quelle degli \textbf{spazi proiettivi} invarianti per \textbf{proiettività}.
\section{Spazi proiettivi}
\begin{definition}{}[Spazio proiettivo]
Sia $\K$ un campo e $V$ uno spazio vettoriale di dimensione \textit{finita} su $\K$. Lo \textbf{spazio proiettivo}\index{spazio!proiettivo} associato a $V$ è l'insieme quoziente
\begin{equation*}
	\Proj\left(V\right)=\left(V\setminus\left\{0\right\}\right)/\!\sim
\end{equation*}
dove $\sim$ è la relazione di equivalenza data su $V\setminus\left\{0\right\}$ definita dall'azione del gruppo moltiplicativo $\K\setminus\left\{0\right\}$:
\begin{equation*}
	\forall v,\ w\in V\setminus\left\{0\right\}\ v\sim w \iff \exists \lambda\in\K\setminus\left\{0\right\}\ \colon v=\lambda w
\end{equation*}
Lo spazio proiettivo $\Proj\left(V\right)$ si dice anche il \textbf{proiettivizzato}\seeonlyindex{proiettivizzato}{spazio!proiettivo} di $V$.
\end{definition}
\begin{proof}{n}
Dimostriamo che è una relazione di equivalenza:
\begin{itemize}
\item \textit{Riflessività}: per $\lambda=1$ si ha $x=1\cdot x$ quindi $x\sim x$.
\item \textit{Simmetria}: se $x\sim y$ si ha $x=\lambda y$ per un $\lambda\neq 0$, quindi $y=\frac{1}{\lambda} x$, cosicché $y\sim x$.
\item \textit{Transitività}: se $x\sim y$ e $y\sim z$ si ha $x=\lambda y,\ z=\mu y$, da cui $z=\mu \left(\lambda x\right)=\left(\mu\lambda\right)x$ e $\mu\lambda\in \K\setminus\left\{0\right\}$, così $x\sim z$.\qedhere
\end{itemize}
\end{proof}
\begin{definition}{}[Dimensione di uno spazio proiettivo]
La \textbf{dimensione}\index{dimensione di uno spazio proiettivo} di $\Proj\left(V\right)$ è
\begin{equation*}
	\dim\Proj\left(V\right)=\dim V-1
\end{equation*}
Se $V=\left\{0\right\}$, allora $\Proj\left(V\right)=\emptyset$ e si pone $\dim\emptyset\coloneqq -1$.
\end{definition}
\begin{definition}{}[Proiezione al quoziente e classe]
Si denota con $\funct{}[\pi]{V\setminus\left\{0\right\}}{\Proj\left(V\right)}$ la \textbf{proiezione al quoziente}\index{proiezione!al quoziente} e con $\left[v\right]\in\Proj\left(V\right)$ la \textbf{classe}\index{classe!dello spazio proiettivo} di $v\in V\setminus\left\{0\right\}$.
\end{definition}
\begin{remark}{n}
	Si ha la corrispondenza biunivoca
	\begin{equation*}
		\begin{array}{c}
			\Proj\left(V\right)\leftrightarrow\left\{\text{sottospazi vettoriali }1\text{-dimensionali di }V\right\}\\
			\left[v\right]\leftrightarrow\lin{v}
		\end{array}
	\end{equation*}
In altre parole, possiamo pensare a $\Proj\left(V\right)$ come l'insieme delle \textbf{rette vettoriali} in $V$.
\end{remark}
\begin{definition}{n}[Altre nomenclature proiettive]~{}
	\begin{itemize}
		\item Se $\dim V=1$, allora $\Proj\left(V\right)$ è un \textbf{punto}\index{punto!proiettivo} e $\dim\Proj\left(V\right)=0$.
		\item Se $\dim \Proj\left(V\right)=1$, si parla di \textbf{retta proiettiva}\index{retta!proiettiva}.
		\item Se $\dim \Proj\left(V\right)=2$, si parla di \textbf{piano proiettivo}\index{piano!proiettivo}.
		\item Se $\K=\R$ o $\K=\C$, si parla rispettivamente di \textbf{spazio proiettivo reale}\index{spazio!proiettivo!reale} o di \textbf{spazio proiettivo complesso}\index{spazio!proiettivo!complesso}.
	\end{itemize}
\end{definition}
Gli esempi più frequenti di spazi proiettivi si ottengono considerando $V=\K^{n+1}$.
\begin{definition}{}[Spazio proiettivo numerico]
	Lo \textbf{spazio proiettivo numerico}\index{spazio!proiettivo!numerico} o \textbf{spazio proiettivo standard}\seeonlyindex{spazio!proiettivo!standard}{spazio!proiettivo!numerico} è lo spazio proiettivo su $\K^{n+1}$:
\begin{equation*}
	\Proj=\Proj^n\left(\K\right)=\Proj\left(\K^{n+1}\right)
\end{equation*}
Essi sono spazi di dimensione $\dim\Proj^n=n$.
\end{definition}
\section{Sottospazi proiettivi}
Sia $W\subseteq V$ un sottospazio vettoriale. Allora $W\setminus\left\{0\right\}\subseteq V\setminus\left\{0\right\}$ è chiuso rispetto alla relazione di equivalenza $\sim$ già definita e $\Proj\left(W\right)$ è un sottoinsieme di $\Proj\left(V\right)$.
\begin{definition}{}[Sottospazio proiettivo]
	Se $W\subseteq V$ è un sottospazio vettoriale, allora $\Proj\left(W\right)$ è detto \textbf{sottospazio proiettivo}\index{sottospazio!proiettivo}:
	\begin{align*}
			\Proj\left(W\right)&=\pi\left(W\setminus\left\{0\right\}\right)=\left\{\left[w\right]\in\Proj\left(V\right)\mid w\in W\right\}\\
			&=\left\{\text{sottospazi vettoriali }1\text{-dimensionali di }V\text{ contenuti in }W\right\}
	\end{align*}
La dimensione del sottospazio proiettivo è $\dim\Proj\left(W\right)=\dim W-1$.
\end{definition}
	\begin{itemize}
	\item Se $W=\left\{0\right\}$, allora $\Proj\left(W\right)=\emptyset$.
	\item Se $\dim W=1$, allora $\Proj\left(W\right)$ è un punto, che indichiamo con $\left[w\right]$ per un $w\in W$.
	\item Se $\dim W=2$ ($\dim\Proj\left(W\right)=1$), allora $\Proj\left(W\right)$ è \textbf{retta proiettiva} in $\Proj\left(V\right)$.
	\item Se $\dim W=3$ ($\dim\Proj\left(W\right)=2$), allora $\Proj\left(W\right)$ è \textbf{piano proiettivo} in $\Proj\left(V\right)$.
	\item Se $\dim\Proj\left(W\right)=\dim \Proj\left(V\right)-1$, allora $\Proj\left(W\right)$ è \textbf{iperpiano (proiettivo)}\index{iperpiano!proiettivo} in $\Proj\left(V\right)$.
\end{itemize}
\begin{definition}{}[Codimensione]
Si definisce la \textbf{codimensione}\index{codimensione} di $\Proj\left(W\right)$ sottospazio proiettivo come
\begin{equation*}
	\codim\Proj\left(W\right)=\dim\Proj\left(V\right)-\dim\Proj\left(W\right)
\end{equation*}
\end{definition}
\begin{example}{n}
	Gli iperpiani sono sottospazi di codimensione $1$.
\end{example}
\section{Coordinate omogenee e sistemi di riferimento proiettivo}
Consideriamo $\Proj^n\left(\K\right)=\Proj\left(\K^{n+1}\right)$. Se $v=\left(x_0,\ \ldots,\ x_n\right)\in\K^{n+1}\setminus\left\{0\right\}$, denotiamo la corrispettiva classe in questa forma:
\begin{equation*}
\left[v\right]=\left(x_0\colon \ldots\colon x_n\right)\in\Proj^n\left(\K\right),\ x_i\in\K
\end{equation*}
\begin{remark}{pn}~{}
	\begin{enumerate}
		\item Le $x_i$ non possono mai essere tutte nulle, dato che $v\neq 0$.
		\item Due classi sono uguali se e solo se le componenti sono tutte in proporzione per uno scalare $\lambda\in\K$.\footnote{La notazione con i $\colon$ viene utilizzata per mettere in evidenza che la relazione fra classi e vettori è di proporzione.}
		\begin{equation*}
			\begin{array}{ccc}
			\left(x_0:\ldots:x_n\right)=\left(y_0:\ldots:y_n\right)&\iff&\left(x_0,\ \ldots,\ x_n\right)\sim\left(y_0,\ \ldots,\ y_n\right)\\&\iff& \exists \lambda\in\K\setminus\left\{0\right\}\ \colon y_0=\lambda x_0,\ \ldots,\ y_n=\lambda x_n
			\end{array}
		\end{equation*}
	\end{enumerate}
\end{remark}
\begin{example}{pn}
	In $\Proj^2\left(\R\right)$:
	\begin{gather*}
		\left(1\colon1\colon2\right) = \left(-2\colon-2\colon-4\right)\\
		\left(1\colon0\colon2\right) = \left(\frac{1}{3}\colon0\colon\frac{2}{3}\right)
	\end{gather*}
\end{example}
\begin{definition}{}[Riferimento proiettivo e coordinate omogenee]
	Sia $\basis=\left\{e_0,\ \ldots,\ e_n\right\}$ una base di $V$, con $\dim V=n+1$. Se $v\in V\setminus\left\{0\right\}$, si ha
	\begin{equation*}
		v=x_0e_0+\ldots+x_n e_n,\ \text{con}\ x_i\in\K
	\end{equation*}
Diciamo che $\left(x_0\colon\ldots\colon x_n\right)$ sono le \textbf{coordinate omogenee}\index{coordinate omogenee} di $\left[v\right]\in\Proj\left(V\right)$ definite dalla base $\basis$ e scriviamo
\begin{equation*}
	\left[v\right]=\left(x_0\colon\ldots\colon x_n\right)
\end{equation*}
La base $\basis$ definisce su $\Proj\left(V\right)$ un \textbf{sistema di riferimento proiettivo}, cioè ad ogni punto vengono assegnate delle coordinate omogenee.
\end{definition}
\begin{remark}{pn}~{}
		\begin{itemize}
		\item Le coordinate omogenee non possono \textit{mai} essere \textit{tutte nulle}.
		\item Le coordinate omogenee sono definite \textit{solo a meno di multipli}.
		\item $\Proj^n\left(\K\right)$ ha delle coordinate omogenee ‘‘naturali'' date dalla base canonica di $\K^{n+1}$.
		\item Basi \textit{multiple} definiscono lo stesso riferimento proiettivo di $\Proj\left(V\right)$, cioè le stesse coordinate omogenee.
	\end{itemize}
\end{remark}
\begin{proof}{n}
	Dimostriamo l'ultimo punto. Siano
	\begin{equation*}
		\basis=\left\{e_0,\ \ldots,\ e_n\right\}\quad\basis'=\left\{\mu e_0,\ \ldots,\ \mu e_n\right\},
	\end{equation*}
con $\mu\in\K\setminus\left\{0\right\}$. Si ha
\begin{equation*}
	v=x_0e_0+\ldots+x_ne_n=\frac{x_0}{\mu}\left(\mu e_0\right)+\ldots + \frac{x_n}{\mu}\left(\mu e_n\right).
\end{equation*}
Passando allo spazio proiettivo,
\begin{equation*}
\underbrace{\left(x_0\colon\ldots\colon x_n\right)}_{\text{coordinate omogenee rispetto a }\basis}=\underbrace{\left(\frac{x_0}{\mu}\colon\ldots\colon \frac{x_n}{\mu}\right)}_{\text{coordinate omogenee rispetto a }\basis'}.\qedhere
\end{equation*}
\end{proof}
\begin{definition}{}[Punti fondamentali e punto unità]
	Data la base $\basis$, i punti
	\begin{align*}
			P_0&=\left[e_0\right]=\left(1\colon0\colon\ldots\colon0\right)\\			P_1&=\left[e_1\right]=\left(0\colon1\colon\ldots\colon0\right)\\
			\vdots&\\
			P_n&=\left[e_n\right]=\left(0\colon0\colon\ldots\colon1\right)\\
		\end{align*}
sono detti \textbf{punti fondamentali}\index{punto!fondamentale} o \textbf{punti coordinati}\seeonlyindex{punto!coordinato}{punto!fondamentale}, mentre il punto
\begin{equation*}
	U=\left[e_0+e_1+\ldots+e_n\right]=\left(1\colon1\colon\ldots\colon1\right)
\end{equation*}
è detto \textbf{punto unità}\index{punto!unità}.
\end{definition}
\subsubsection{Descrizione dei sottospazi proiettivi in coordinate}
Siano $\left(x_0\colon\ldots\colon x_n\right)$ coordinate omogenee su $\Proj\left(V\right)$, indotte da una base $\basis$, e consideriamo l'equazione lineare omogenea
\begin{equation}\label{eqlinomogenea}
	\textcolor{green}{\circled{\ast}}\quad a_0x_0+a_1x_1+\ldots+a_nx_n=0
\end{equation}
con $a_i\in\K$ non tutti nulli.
\begin{itemize}
	\item In $V$ l'equazione omogenea rappresenta un \textit{iperpiano vettoriale} $H$.
	\item I punti $P=\left[v\right]\in\Proj\left(V\right)$, le cui coordinate soddisfano le equazioni, sono quelli tali per cui $v\in H$, cioè sono tutti e soli i punti dell'iperpiano proiettivo $\Proj\left(H\right)\subseteq\Proj\left(V\right)$. L'equazione lineare \eqref{eqlinomogenea} è l'\textbf{equazione (cartesiana) dell'iperpiano proiettivo} $\Proj\left(H\right)$.
\end{itemize}
\begin{definition}{}[Iperpiano coordinato]
	Gli iperpiani di equazione cartesiana $x_i=0$, cioè tutti i punti la cui $i$-esima coordinata omogenea è nulla, si dicono $i$\textbf{-esimi iperpiani coordinati}\index{iperpiano!proiettivo!coordinato}.
\end{definition}
\begin{example}{n}
	In $\Proj^1\left(\K\right)$, cioè una \textit{retta proiettiva} in quanto $\dim \Proj^1\left(\K\right)=1$, i sottospazi proiettivi sono:
	\begin{itemize}
		\item $\emptyset$.
		\item I punti, che in questo caso sono gli iperpiani.
		\item Tutto $\Proj^1\left(\K\right)$.
	\end{itemize}
Il punto $\left(a\colon b\right)$ ha equazione cartesiana
\begin{equation*}
	bx_0-ax_1=0
\end{equation*}
ovvero l'equazione della retta in $\K^2$ generata dal vettore $\left(a,\ b\right)$, ottenuta pertanto dal determinante
\begin{equation*}
	\left| \begin{array}{cc}
		a & b \\
		x_0 & x_1
	\end{array} \right|=0.
\end{equation*}
\end{example}
\begin{warning}{n}
	In $\Proj\left(V\right)$ un sottospazio proiettivo di \textit{dimensione zero} è un singolo punto $\left[v\right]=\Proj\left(\lin{v}\right)$.
\end{warning}
Più in generale, fissata una base $\basis$ di $V$, ogni \textit{sottospazio vettoriale} $W$ di $V$ può essere visto, in \textit{coordinate} rispetto alla base, come l'\textit{insieme delle soluzioni} di un \textit{sistema lineare omogeneo}
\begin{equation*}
	Ax=O
\end{equation*}
dove $A=\left(a_{ij}\right)$ è di dimensioni $t\times \left(n+1\right)$ a elementi in $\K$ e $x=\left(\begin{array}{c}
	x_0 \\
	\vdots \\
	x_n
\end{array}\right)$
Il sistema
\begin{equation*}ì\begin{cases}
	a_{1,0}x_0+\ldots+a_{1,n}x_n=0\\
	\vdots\\
	a_{t,0}x_0+\ldots+a_{t,n}x_n=0
\end{cases}
\end{equation*}
dà delle \textit{equazioni cartesiane} per il sottospazio proiettivo $\Proj\left(W\right)$ nelle coordinate omogenee $\left(x_0\colon\ldots\colon x_n\right)$. Posto dunque $t$ come il numero delle \textit{equazioni}, notiamo che
\begin{equation*}
	\dim W=n+1-\rk A
\end{equation*}
quindi
\begin{equation*}
	\begin{array}{lll}
		\codim W &=&\rk A \\
		\shortparallel &  \\
		\dim V-\dim W &=&\dim\Proj\left(V\right)-\dim\Proj\left(W\right)=\codim\Proj\left(W\right)
	\end{array}
\end{equation*}
pertanto
\begin{equation*}
	t\geq \rk\left(A\right)=\codim\Proj\left(W\right)\qquad\dim\Proj\left(W\right)=\Proj\left(V\right)-\rk\!\left(A\right)
\end{equation*}
\textit{Scartando} delle equazioni possiamo sempre ricondurci ad un sistema in cui
\begin{equation*}
	t=\rk A=\codim\Proj\left(W\right).
\end{equation*}
\begin{intuitively}{n}
	Per facilitare la visualizzazione degli spazi proiettivi possiamo pensare allo spazio $\K^{n+1}$ come lo \textbf{spazio affine} $\aff{\K^{n+1}}$ in cui sia fissato un punto $O$ come origine: in questo modo, le classi di $\Proj^n\left(\K\right)$ corrispondono alle \textit{rette affini passanti per} $O$, identificate con le rette vettoriali di $\K^{n+1}$:
	\begin{equation*}
		\left(x_0\colon\ldots\colon x_n\right)\leftrightarrow\begin{array}{c}
			\text{retta affine di }\aff{\K^{n+1}}\text{ formata}\\
			 \text{dai punti }\left(tx_0,\ \ldots,\ tx_n\right)\text{ al variare di t}\in\R
		\end{array}
	\end{equation*}
	Approfondiremo formalmente la relazione tra gli spazi affini e gli spazi proiettivi più avanti, a pag. \pageref{spaziaffini}.
\end{intuitively}
\begin{example}{pn}~{}
	\begin{itemize}
		\item Il piano proiettivo $\Proj^2\left(\K\right)$ ha come sottospazi \textit{non banali} i punti e le rette.
		\begin{itemize}
			\item Una \textit{retta proiettiva} viene da un \textit{piano}, che nel riferimento \textit{affine} possiamo prendere passante per l'origine: $a_0x_0+a_1x_0+a_2x_2=0$.
			\item Per determinare un \textit{punto} servono due equazioni, in sostanza vedendolo come \textit{intersezione di due rette proiettive}; ad esempio, $\left(1\colon0\colon0\right)$ ha equazioni
			\begin{equation*}
				x_1=x_2=0,
			\end{equation*}
			mentre $\left(1\colon2\colon3\right)$ ha equazioni
			\begin{equation*}
				\begin{cases}
					x_1=2x_0\\
					x_2=3x_0
				\end{cases}
			\end{equation*}
		\end{itemize}
	\item Nel piano proiettivo reale $\Proj^2\left(\R\right)$, le \textit{rette proiettive} vengono da \textit{piani vettoriali}, dunque nel modello affine di $\aff{\R^3}$ essi sono piani passanti per l'\textit{origine}; utilizzando la \textit{sfera unitaria}, i cui punti antipodali sono identificati in una relazione di equivalenza, la retta proiettiva si visualizza facilmente come l'\textit{intersezione} della sfera in un \textit{cerchio massimo}; nel modello della \textit{calotta superiore}, prendiamo l'intersezione dall'equatore in su.	In questo modo, la \textbf{proiezione verticale} dell'intersezione del cerchio massimo con la calotta sul disco unitario $D$ è la rappresentazione della retta proiettiva sul \textit{modello piano del disco}. Dunque, abbiamo \textit{tre tipi} di rette:\\
	\begin{minipage}{0.77\textwidth}
			\begin{enumerate}[series=proj]
			\item La \textit{retta} con equazione $z=0$, ovvero al piano $xy$ in $\R^3$: sul modello piano corrisponde al \textbf{bordo del disco} $D$ (cioè $S^1$).
		\end{enumerate}
		\end{minipage}
	\begin{minipage}{0.32\textwidth}
		\includegraphics[trim=0cm 0cm 0cm 0cm,clip,scale=0.50]{images/projectivelinesdisk1.pdf}
	\end{minipage}\\
	\begin{minipage}{0.77\textwidth}
	\begin{enumerate}[resume=proj]
			\item Le \textit{rette} con equazione $ax+by=0$, ovvero ai \textit{piani perpendicolari} in $\R^3$ passanti per le rette con quell'equazione $ax+by=0$:  sul modello piano corrisponde a \textbf{diametri colleganti due punti} sul bordo.
		\end{enumerate}
	\end{minipage}
\begin{minipage}{0.32\textwidth}
\includegraphics[trim=0cm 0cm 0cm 0cm,clip,scale=0.50]{images/projectivelinesdisk2.pdf}
\end{minipage}\\
\begin{minipage}{0.77\textwidth}
	\begin{enumerate}[resume=proj]
			\item Nel caso generale $ax+by+cz=0$, proiettando l'\textit{arco di cerchio massimo} viene un \textbf{arco di ellisse} in $D$.
		\end{enumerate}
	\end{minipage}
	\begin{minipage}{0.32\textwidth}
		\includegraphics[trim=0cm 0cm 0cm 0cm,clip,scale=0.50]{images/projectivelinesdisk3.pdf}
	\end{minipage}
	\end{itemize}
\vspace{-3mm}
\end{example}
\section{Operazioni con i sottospazi}
Se $W_1,\ W_2\subseteq V$ sono sottospazi vettoriali, allora $W_1\cap W_2$ è un sottospazio vettoriale e si ha che l'\textbf{intersezione}\index{sottospazio!intersezione} dei corrispettivi spazi proiettivi è ancora un sottospazio proiettivo.
\begin{equation*}
	\Proj\left(W_1\cap W_2\right)=\Proj\left(W_1\right)\cap\Proj\left(W_2\right)
\end{equation*}
\begin{remark}{n}
	Si ha $\Proj\left(W_1\right)\cap\Proj\left(W_2\right)=\emptyset$ se e solo se $W_1\cap W_2=\left\{0\right\}$. In tal caso diciamo che i due sottospazi sono \textbf{sghembi}\index{sottospazio!proiettivo!sghembo} o \textbf{disgiunti}\seeonlyindex{sottospazio!proiettivo!disgiunto}{sottospazio!proiettivo!sghembo}.
\end{remark}
Come per i sottospazi vettoriali, in generale l'\textbf{unione} di due sottospazi proiettivi \textit{non} è un sottospazio proiettivo.
\begin{definition}{}[Sottospazio generato da un sottoinsieme]
	Sia $S\subseteq \Proj\left(V\right)$ un sottoinsieme non vuoto. Il \textbf{sottospazio generato}\index{sottospazio!proiettivo!generato} da $S$, denotato con $\left<S\right>$, è l'intersezione in $\Proj\left(V\right)$ di tutti i sottospazi proiettivi contenenti $S$, ed è il più piccolo sottospazio contenente $S$.
\end{definition}
\begin{itemize}
	\item $\left<S\right>=S\iff S$ è un sottospazio proiettivo.
	\item Se $S=\left\{P_1,\ \ldots,\ P_m\right\}$ è finito, scriviamo $\left<P_1,\ \ldots,\ P_m\right>$ per il sottospazio generato da $P_1,\ \ldots,\ P_m$; se, sono linearmente indipendenti, $\dim \left<P_1,\ \ldots,\ P_m\right>=m-1$.
\end{itemize}
\begin{definition}{}[Sottospazio somma]
	Dati due sottospazi proiettivi $T_1,\ T_2\subseteq \Proj\left(V\right)$, cioè
	\begin{equation*}
		T_i=\Proj\left(W_i\right)\quad W_i\subseteq V,\ i=1,\ 2,
	\end{equation*}
	allora il sottospazio generato da $T_1\cup T_2$ è denotato con $T_1+T_2=\left<T_1,\ T_2\right>$ e si chiama \textbf{sottospazio somma}\index{sottospazio!somma}. In particolare, si ha
	\begin{equation*}
		\left<T_1,\ T_2\right>=\Proj\left(W_1+W_2\right)
	\end{equation*}
\end{definition}
\begin{proof}{n}~{}\\
	$\rightinclude$ $\Proj\left(W_1+W_2\right)$ è un sottospazio proiettivo che contiene, in quanto $W_1\subseteq W_1+W_2,\ W_2\subseteq W_1+W_2$ vettorialmente, sia $T_1=\Proj\left(W_1\right)$ sia $T_2=\Proj\left(W_2\right)$. In particolare, contiene la loro unione\footnote{Ricordiamo che non è essa un sottospazio, ma un sottoinsieme.}, dunque $\left<T_1,\ T_2\right>=\left<T_1\cup T_2\right>\subseteq \Proj\left(W_1+W_2\right)$.\\
	$\leftinclude$Abbiamo che $T_i\subseteq \left<T_1,\ T_2\right>=\Proj\left(U\right)$, con $U$ un sottospazio vettoriale di $V$. In particolare, si ha che $W_1,\ W_2\subseteq U$, da cui $W_1+W_2\subseteq U$. Passando allo spazio proiettivo
	\begin{equation*}
		\left<T_1,\ T_2\right>=\Proj\left(U\right)\supseteq \Proj\left(W_1+W_2\right.\qedhere
	\end{equation*}
\end{proof}
\begin{proposition}{}[Formula di Grassmann proiettiva]\index{formula!di Grassmann proiettiva}
Siano $T_1,\ T_2$ sottospazi proiettivi di $\Proj\left(V\right)$. Si ha
	\begin{equation*}
				\dim\left<T_1,\ T_2\right>+\dim\left(T_1\cap T_2\right)=\dim T_1+\dim T_2.
	\end{equation*}
\end{proposition}
\begin{proof}{n}
	Posti $T_i=\Proj\left(W_i\right)$, con $W_i\subseteq V$ sottospazi vettoriali. Dalla \textit{formula di Grassmann vettoriale} si ha
	\begin{equation*}
		\dim\left(W_1+W_2\right)+\dim\left(W_1\cap W_2\right)=\dim W_1+\dim W_2.
	\end{equation*}
Sottraendo $1$ a tutte le dimensioni, otteniamo le dimensioni dei corrispettivi spazi proiettivi e dunque la formula proiettiva.\qedhere
\end{proof}
\begin{corollary}{}[Condizioni sulla dimensione dell'intersezione]
	Siano $T_1,\ T_2$ sottospazi proiettivi di $\Proj\left(V\right)$ con $\dim\Proj\left(V\right)=n$. Allora
	\begin{equation*}
		\dim \left(T_1\cap T_2\right)\geq \dim T_1+\dim T_2-n
	\end{equation*}
In particolare $T_1\cap T_2\neq \emptyset$ se $\dim T_1+\dim T_2\geq n$.
\end{corollary}
\begin{proof}{n}
	Si ha
	\begin{equation*}
		\dim\left(T_1\cap T_2\right)=\dim T_1+\dim T_2-\dim\left<T_1,\ T_2\right>\geq \dim T_1+\dim T_2-n.
	\end{equation*}
Chiaramente, se $\dim T_1+\dim T_2\geq n$, allora $\dim \left(T_1\cap T_2\right)\geq 0$ e dunque $T_1\cap T_2\neq \emptyset$.\qedhere
\end{proof}
\begin{example}{n}
	Nel piano proiettivo, due rette sono \textit{sempre incidenti}. Infatti, le rette hanno dimensione $1$, mentre $\dim\Proj^2\left(\K\right)=2$, dunque vale $1+1\leq 2$, pertanto due rette si incontrano sempre.
\end{example}
\begin{remark}{n}
	Se consideriamo un insieme \textit{finito di punti}, possiamo considerare lo spazio $S$ \textit{generato} da $P_1,\ \ldots,\ P_m$, cioè $S=\left<P_1,\ \ldots,\ P_m\right>$; inoltre, si ha
	\begin{equation*}
		\dim S\leq m-1.
	\end{equation*}
	Infatti, se $P_i=\left[v_i\right]$ con $v_i\in V$, allora
	\begin{equation*}
		S=\underbrace{\Proj\left(\lin{v_1,\ \ldots,\ v_m}\right)}_{\dim \mathcal{L} \leq m}.
	\end{equation*}
\end{remark}
\section{Punti linearmente indipendenti e in posizione generale}
\begin{definition}{}[Punti linearmente indipendenti]
	Siano $P_1,\ \ldots, P_m\in\Proj\left(V\right)$. Diciamo che i punti $P_1,\ \ldots,\ P_m$ sono \textbf{linearmente indipendenti}\index{linearmente indipendenti} se, scelti $v_1,\ \ldots,\ v_m\in V\setminus\left\{0\right\}$ tali che $P_i=\left[v_i\right]\ \forall i$, i vettori $v_1,\ \ldots,\ v_m$ sono \textit{linearmente indipendenti} in $V$. Se così non è, diciamo che $P_1,\ \ldots, P_m$ sono linearmente dipendenti.
\end{definition}
\begin{remark}{pn}~{}
	\begin{itemize}
		\item La definizione è \textit{ben posta}. Dati $\lambda_1,\ \ldots,\ \lambda_m\in\K\setminus\left\{0\right\}$, si ha che
		\begin{equation*}
			v_1,\ \ldots,\ v_m\text{ sono indipendenti}\iff\lambda_1 v_1,\ \ldots,\ \lambda_m v_m\text{ sono indipendenti.}
		\end{equation*}
		\item Se $\dim \Proj\left(V\right)=n$, $\Proj\left(V\right)$ contiene al più $n+1$ punti indipendenti.
		\item $P_1,\ \ldots,\ P_m$ sono indipendenti se e solo se $\dim\left<P_1,\ \ldots,\ P_m\right>=m-1$.
	\end{itemize}
\end{remark}
\begin{example}{pn}~{}
	\begin{itemize}
		\item \textit{Due} punti $P,\ Q$ sono indipendenti se e solo se $P\neq Q$. Infatti, se $P=\left[v\right]$ e $Q=\left[w\right]$, allora
		\begin{equation*}
			P\text{ e }Q\text{ sono indipendenti}\iff v\text{ e }w\text{ sono indipendenti}\iff v\nsim w\iff P\neq Q
		\end{equation*}
		In tal caso $\left<P,\ Q\right>$ è l'unica \textit{retta} contenente $P$ e $Q$, che indicheremo anche con $\overline{PQ}$.
		\item \textit{Tre} punti $P_1,\ P_2,\ P_3$ sono indipendenti se e solo se sono \textit{distinti} e \textit{non} sono \textit{allineati}, cioè appartenenti alla stessa retta. In tal caso $\left<P_1,\ P_2,\ P_3\right>$ è l'unico \textit{piano} contenente i tre punti.
	\end{itemize}
\end{example}
\begin{definition}{}[Punti in posizione generale]
	Dati dei punti $P_1,\ \ldots,\ P_m\in\Proj\left(V\right)$, diciamo che sono \textbf{in posizione generale}\index{in posizione generale}\index{posizione generale} se vale una delle due condizioni seguenti:
	\begin{itemize}
		\item $m\leq n+1$ e i punti sono \textit{linearmente indipendenti}.
		\item $m>n+1$ e ogni scelta di $n+1$ punti tra loro sono linearmente indipendenti.
	\end{itemize}
\end{definition}
\begin{example}{n}~{}
		\begin{itemize}
		\item Se $n=1$, cioè $\Proj\left(V\right)$ è una \textit{retta proiettiva}, allora $P_1,\ \ldots,\ P_m$ sono in posizione generale se e solo se $P_1,\ \ldots,\ P_m$ sono \textit{tutti distinti}.
		\item Se $n=2$, cioè $\Proj\left(V\right)$ è una \textit{piano proiettivo}, allora $P_1,\ \ldots,\ P_m$ sono in posizione generale se e solo se $P_1,\ \ldots,\ P_m$ sono a $3$ a $3$ \textit{non} allineati.
	\end{itemize}
\end{example}
\subsection{Impratichiamoci! Punti linearmente indipendenti}
\begin{exercise}{}[F.F.P., 2.1]
	Si mostri che i punti del piano proiettivo reale
	\begin{equation*}
		\left(\frac{1}{2}\colon 1 \colon 1\right)\quad \left(1\colon \frac{1}{3} \colon \frac{4}{3}\right)\quad \left(2\colon -1 \colon 2\right)
	\end{equation*}
sono allineati, e si determini un'equazione della retta che li contiene.
\end{exercise}
\begin{solution}{n}
	Per verificare che i $3$ punti sono allineati, dobbiamo verificare che i corrispondenti vettori di $\R^3$ sono dipendenti. Riscriviamo i seguenti punti per facilitarci i calcoli:
\begin{equation*}
	\left(\frac{1}{2}\colon 1 \colon 1\right)=\left(1\colon 2 \colon 2\right)\quad \left(1\colon \frac{1}{3} \colon \frac{4}{3}\right)=\left(3\colon 1 \colon 4\right)
\end{equation*}
Verifichiamolo la dipendenza con il determinante.
	\begin{equation*}
		\det\left|\begin{array}{ccc}
			1 & 2 & 2\\
			3 & 1 & 4\\
			2 & -1 & 2
		\end{array}\right|=0
	\end{equation*}
L'equazione della retta è data dall'equazione del piano vettoriale in $\R^3$ generato da $2$ dei $3$ vettori:
\begin{equation*}
	0=\left|\begin{array}{ccc}
		x_0 & x_1 & x_2\\
		1 & 2 & 2\\
		3 & 1 & 4
	\end{array}\right|=x_0\left(8-2\right)-x_1\left(4-6\right)+x_2\left(1-6\right)=6x_0+2x_1-5x_2
\end{equation*}
Verifichiamo che contenga anche il terzo:
\begin{equation*}
	6\cdot 2 + 2\cdot \left(-1\right) -5\cdot 2=0
\end{equation*}
\vspace{-6mm}
\end{solution}
\section{Rappresentazione parametrica di un sottospazio proiettivo}
Sia $S\subseteq\Proj\left(V\right)$ un sottospazio proiettivo di dimensione $m$. Allora esistono sempre $m+1$ punti $P_0,\ \ldots,\ P_m\in S$ linearmente indipendenti che generano $S$. Infatti, se $S=\Proj\left(W\right)$ con $W\subseteq V$ sottospazio vettoriale di dimensione $m+1$, possiamo scegliere una base $\left\{w_0,\ \ldots,\ w_m\right\}$ di $W$ tale per cui
\begin{equation*}
	P_i=\left[w_i\right]\in S.
\end{equation*}
Sono linearmente indipendenti (perché lo sono i vettori della base) e generano $S$.\\
Allora, tutti e soli i punti di $S$ sono della forma
\begin{equation*}
	\left[\lambda_0 w_0+\ldots+\lambda_m w_m\right]\quad \lambda_0,\ \ldots,\ \lambda_m\in\K
\end{equation*}
Supponiamo ora di aver fissato una base $\left\{e_0,\ \ldots,\ e_n\right\}$ di $V$ e quindi di aver considerato il corrispondente \textit{riferimento proiettivo}. In coordinate vettoriali di $V$, un punto di $W$ è $x=\left(x_0,\ \ldots,\ x_n\right)$ se e solo se
\begin{equation*}
	x=x_0e_0+\ldots+x_ne_n=\lambda_0 w_0+\ldots+\lambda_m w_m
\end{equation*}
Il punto $P_i$ in $V$ avrà coordinate $\left(P_{0,i},\ \ldots,\ P_{n,i}\right)\ \forall i=1,\ \ldots,\ m$, dunque il generico vettore $x$ di $W$ è espresso da
\begin{equation*}
	\begin{cases}
		x_0=\lambda_0 P_{0,0}+\lambda_1P_{0,1}+\ldots+\lambda_mP_{0,m}\\
		\vdots\\
		x_n=\lambda_0 P_{n,0}+\lambda_1P_{n,1}+\ldots+\lambda_mP_{n,m}
	\end{cases}
\end{equation*}
Anche i punti di $S$ sono date da queste coordinate, dunque questa viene definita la \textbf{rappresentazione parametrica}\index{rappresentazione!parametrica} del sottospazio $S$, con $\left(\lambda_0\colon\ldots\colon\lambda_m\right)$ le coordinate omogenee di $\Proj\left(W\right)$ date dalla base $\left\{w_0,\ \ldots,\ w_m\right\}$.
\begin{example}{n}
	In $\Proj^3\left(\R\right)$ consideriamo i punti
	\begin{equation*}
		A=\left(1\colon 0\colon -1\colon 4\right)\quad B=\left(2\colon 3\colon 0\colon 5\right)
	\end{equation*}
Allora, la rappresentazione parametrica del sottospazio $S=\left< A,\ B\right>$ con $\left(\lambda\colon \mu\right)$ è
\begin{equation*}
	\begin{cases}
		x_0=\lambda+2\mu\\
		x_1=3\mu\\
		x_2=-\lambda\\
		x_3=4\lambda-5\mu
	\end{cases}
\end{equation*}
\end{example}
\subsection{Coordinate proiettive e punti in posizione generale}
\begin{remark}{n}
	Sia $\Proj\left(V\right)$ con un riferimento proiettivo fissato. Consideriamo i punti fondamentali $P_0,\ \ldots,\ P_n$ e il punto unità $U$.
	\begin{itemize}
		\item $P_0,\ \ldots,\ P_n,\ U$ sono $n+2$ punti.
		\item $P_0,\ \ldots,\ P_n,\ U$ sono in posizione generale: essendo $P_i=\left[e_i\right]$ con $e_0,\ \ldots,\ e_n$ base di $V$, allora $P_0,\ \ldots,\ P_n$ sono indipendenti. Se sostituiamo l'$i$-esimo punto con $U=\left[e_1+\ldots+e_n\right]$, allora:
		\begin{equation*}
			P_0,\ \ldots,\ \check{P}_i,\ \ldots,\ U
		\end{equation*}
	Sono indipendenti per ogni $i=0,\ \ldots,\ n$.\footnote{Indichiamo con $\check{P}_{i}$ il punto che sostituiamo.}
	\end{itemize}
\end{remark}
\begin{warning}{n}
	Sia $\basis=\left\{e_0,\ \ldots,\ e_n\right\}$ una base che induce un \textit{riferimento proiettivo} su $\Proj\left(V\right)$.\\
	Per ogni $i$ sia $\lambda_i\in\K\setminus\left\{0\right\}$ e consideriamo $v_i=\lambda_i e_i$. Allora $\basis'=\left\{v_0,\ \ldots,\ v_n\right\}$ è ancora una base e i \textit{punti fondamentali} del riferimento indotto da $\basis'$ sono \textit{gli stessi} del riferimento indotto da $\basis$. Infatti
	\begin{equation*}
		\left[e_i\right]=\left[v_i\right]=P_i
	\end{equation*}
	Però i due riferimenti sono \textbf{diversi}; dato $v$ espresso nella base $\basis$
	\begin{equation*}
		v=x_0e_0+\ldots+x_ne_n,
	\end{equation*}
	la sua classe in $\Proj\left(V\right)$, rispetto a $\basis$, è
	\begin{equation*}
		\left[v\right]=\left(x_0\colon\ldots\colon x_n\right).
	\end{equation*}
	Possiamo partire dall'espressione di $v$ nella base $\basis$ a quella nella base $\basis'$, moltiplicando e dividendo ogni $e_i$ per il corrispettivo $\lambda_i$:
	\begin{equation*}
		v=\frac{x_0}{\lambda_0}\left(\lambda_0e_0\right)+\ldots+\frac{x_n}{\lambda_n}\left(\lambda_ne_n\right)=\frac{x_0}{\lambda_0}v_0+\ldots+\frac{x_n}{\lambda_n}v_n.
	\end{equation*}
	Passiamo dunque alla base $\basis'$ alla classe in $\Proj\left(\K\right)$:
	\begin{equation*}
		\left[v\right]=\left(\frac{x_0}{\lambda_0}\colon\ldots\colon\right).
	\end{equation*}
	Notiamo che effettivamente il punto $\left[v\right]$ non cambia, ma i riferimenti \textit{non} sono multipli e quindi sono diversi!
	\begin{itemize}
		\item \textit{Conoscere} i punti fondamentali \textit{non basta} a determinare la base $\basis$.
		\item Riferimenti proiettivi \textit{diversi} possono avere gli \textit{stessi} punti fondamentali.
	\end{itemize}
\end{warning}
\begin{remark}{n}\label{puntigeneraleindipendentiosserva}
	Supponiamo di avere $n+2$ punti $P_0,\ \ldots,\ P_{n+1}$ in $\Proj\left(V\right)$, cioè $\forall i=0,\ \ldots,\ n+1\ \exists v_i\in V\ \colon P_i=\left[v_i\right]$. Allora
	\begin{gather*}
		P_0,\ \ldots,\ P_{n+1}\text{ sono in posizione generale}\iff v_0,\ \ldots,\ v_n\text{ sono indipendenti e}\\
		v_{n+1}=a_0v_0+\ldots+a_nv_n\text{ con }a_i\neq 0\ \forall i=0,\ \ldots,\ n
	\end{gather*}
Infatti, se $v_0,\ \ldots,\ v_n$ è una base (in quanto sono indipendenti), $v_0,\ \ldots,\ \check{v_i},\ \ldots,\ v_n,\ v_{n+1}$ sono indipendenti se e solo se $a_i\neq 0$.
\end{remark}
\begin{theorem}{n}[Esistenza di una base dati $n+2$ punti in posizione generale]\label{puntiposizionegeneraleesistenzabase}
	Sia $\Proj\left(V\right)$ di dimensione $n$. Dati $n+2$ punti $P_0,\ \ldots,\ P_{n+1}$ in \textit{posizione generale}, esiste una base $\basis=\left\{e_0,\ \ldots,\ e_n\right\}$ di $V$ tale che
	\begin{equation*}
		P_0=\left[e_0\right],\ \ldots,\ P_n=\left[e_n\right],\ P_{n+1}=\left[e_0+\ldots+e_n\right].
	\end{equation*}
Inoltre, se $\basis'=\left\{f_0,\ \ldots,\ f_n\right\}$ è un'altra base di $V$ che soddisfa la condizione sopra, allora $\basis'$ è proporzionale a $\basis$, cioè esiste $\mu\in\K\setminus\left\{0\right\}$ tale che $f_i=\mu e_i\ \forall i=0,\ \ldots,\ n$.
\end{theorem}
\begin{proof}{n}
	Sia $P_i=\left[v_i\right]$ al variare di $i=0,\ \ldots,\ n+1$. I punti $P_0,\ \ldots,\ P_n$ sono indipendenti\footnote{Perché se $n+2$ punti sono in posizione generale, presi $n+1$ punti fra di loro sono indipendenti.}, dunque per definizione $v_0,\ \ldots,\ v_n$ è una base di $V$. Poniamo
	\begin{equation*}
		v_{n+1}=\lambda_0v_0+\ldots+\lambda_nv_n\quad \lambda_i\in\K
	\end{equation*}
	Allora, per l'osservazione precedente, $\lambda_i\neq 0\ \forall i$ perché i punti sono in posizione generale. Consideriamo $e_0=\lambda_0v_0,\ e_1=\lambda_1v_1,\ \ldots,\ e_n=\lambda_nv_n$. Si ha che $\basis=\left\{e_0,\ \ldots,\ e_n\right\}$ è una base di $V$ perché $\lambda_i\neq 0$\ $\forall i$. Segue che
	\begin{gather*}
		\left[e_i\right]=\left[v_i\right]=P_i\ \forall i=0,\ \ldots,\ n\\
		\left[e_0+\ldots+e_n\right]=\left[\lambda_0v_0+\ldots+\lambda_nv_n\right]=\left[v_{n+1}\right]=P_{n+1}
	\end{gather*}
Adesso, sia $\basis'=\left\{f_0,\ \ldots,\ f_n\right\}$ come da ipotesi. Allora $\left[f_i\right]=P_i=\left[e_i\right]\ \forall i=0,\ \ldots,\ n$, cioè $\exists \mu_i\in\K\setminus\left\{0\right\}\ \colon f_i=\mu_i e_i\ \forall i=0,\ \ldots, n$. Inoltre, soddisfa anche $\left[f_0+\ldots+f_n\right]=P_{n+1}$, pertanto
\begin{equation*}
	\left[f_0+\ldots+f_n\right]=\left[e_0+\ldots+e_n\right].
\end{equation*}
In altre parole, esiste $\mu\in\K\setminus\left\{0\right\}$ tale che
\begin{equation*}
	\begin{array}{ccc}
		f_0+\ldots+f_n&=&\mu\left(e_0+\ldots+e_n\right)\\
		\shortparallel&&\\
		\mu_0e_0+\ldots+\mu_ne_n&&
	\end{array}
\end{equation*}
$e_0,\ \ldots,\ e_n$ è una base: per l'unicità della scrittura deve essere $\mu=\mu_0=\ldots=\mu_n$, cioè $f_i=\mu e_i\ \forall i=0,\ \ldots,\ n$.\qedhere
\end{proof}
\section{Trasformazioni proiettive}
\begin{definition}{}[Trasformazione proiettiva e proiettività]
	Un'applicazione $\funct{}[f]{\Proj\left(V\right)}{\Proj\left(V'\right)}$ tra spazi proiettivi si dice \textbf{trasformazione proiettiva}\index{trasformazione!proiettiva} o \textbf{isomorfismo proiettivo}\seeonlyindex{isomorfismo!proiettivo}{trasformazione!proiettiva} se esiste $\funct{}[\phi]{V}{V'}$ isomorfismo che induce un altro isomorfismo lineare
	\begin{equation*}
		\begin{tikzcd}
			\widetilde{\phi}\ \colon\\
		\end{tikzcd}
			\funct{}{\Proj\left(V\right)}{\Proj\left(V'\right)}[\left[v\right]][\left[\phi\left(v\right)\right]]
	\end{equation*}
	tale per cui $f=\widetilde{\phi}$. Se $V=V'$, diciamo che $f$ è una \textbf{proiettività}\index{proiettività} di $\Proj\left(V\right)$.
\end{definition}
\begin{proof}{n}~{}
	\begin{itemize}
		\item $\widetilde{\phi}$ \textit{è ben definita}:
		\begin{enumerate}
			\item $\phi\left(v\right)\neq 0$ perché $v\neq 0$ e $\phi$ è iniettiva, pertanto $\ker\phi=\left\{0\right\}$ e dunque l'unico vettore mappato a $0$ tramite $\phi$ è solo $0$.
			\item Se $\left[v\right]=\left[w\right]$, allora $w\sim v$, cioè $w=\lambda v$ con $\lambda\in\K\setminus\left\{0\right\}$; segue che per linearità di $\phi$ vale $\phi\left(w\right)=\lambda \phi\left(v\right)$ e quindi $\left[\phi\left(w\right)\right]=\left[\phi\left(v\right)\right]$.
		\end{enumerate}
		\item $\widetilde{\phi}$ \textit{è iniettiva}: se $\widetilde{\phi}\left(\left[v\right]\right)=\widetilde{\phi}\left(\left[w\right]\right)$, allora
		\begin{equation*}
			\left[\phi\left(v\right)\right]=\left[\phi\left(w\right)\right]\implies\exists\lambda\in\K\setminus\left\{0\right\}\ \colon\phi\left(w\right)=\lambda\phi\left(v\right)=\phi\left(\lambda v\right)
		\end{equation*}
	Poiché $\phi$ è iniettiva, segue che $w=\lambda v$ e dunque $\left[v\right]=\left[w\right]$.
		\item $\widetilde{\phi}$ \textit{è suriettiva}: infatti, se $\left[w\right]\in\Proj\left(V'\right)$, essendo $\phi$ suriettiva esiste un vettore $v$ tale che $w=\phi\left(v\right)$. Segue che $\left[w\right]=\left[\phi\left(v\right)\right]=\phi\left(\left[v\right]\right)$.\qedhere
	\end{itemize}
\end{proof}
Dato che spazi \textit{vettoriali} della \textit{stessa dimensione} sono sempre \textit{isomorfi}, due spazi \textit{proiettivi} della \textit{stessa dimensione} sono sempre \textit{isomorfi} e $\Proj\left(V\right)$ è sempre isomorfo a $\Proj^n\left(\K\right)$, con $\dim V=n+1$.
\begin{lemma}{}[Uguaglianza di proiettività]
	Siano $\funct{}[\phi,\ \psi]{V}{V'}$ isomorfismi. Allora:
	\begin{equation*}
		\widetilde{\phi}=\widetilde{\psi}\iff\exists\lambda\in\K\setminus\left\{0\right\}\ \colon\psi=\lambda\phi
	\end{equation*}
\end{lemma}
\begin{proof}{n}~{}\\
$\leftimplies$Se $v\in\K\setminus\left\{0\right\}$, allora $\psi\left(v\right)=\lambda\phi\left(v\right)$. Segue che
\begin{equation*}
	\widetilde{\phi}\left(\left[v\right]\right)=\left[\phi\left(v\right)\right]=\left[\psi\left(v\right)\right]=\widetilde{\psi}\left(\left[v\right]\right)
\end{equation*}
$\rightimplies$Sia $h\coloneqq \funct{}[\psi^{-1}\circ \phi]{V}{V}$ automorfismo. Vogliamo mostrare che $h=\lambda Id_V$ con $\lambda\in\K\setminus\left\{0\right\}$. Se $v\in V\setminus\left\{0\right\}$, abbiamo
\begin{equation*}
	\begin{array}{c}
\begin{array}{ccc}
	\widetilde{\phi}\left(\left[v\right]\right)&=&\widetilde{\psi}\left(\left[v\right]\right)\\
	\shortparallel&&\shortparallel\\
	\left[\phi\left(v\right)\right]&&\left[\psi\left(v\right)\right]
\end{array}
\implies \lambda_v\in\K\setminus\left\{0\right\}\ \colon \phi\left(v\right)=\lambda_v\psi\left(v\right)
\implies h\left(v\right)=\psi^{-1}\left(\phi\left(v\right)\right)=\lambda_v v
\end{array}
\end{equation*}
Segue che $v$ è autovettore di $h\ \forall v\in V\setminus\left\{0\right\}$, in particolare ogni vettore non nullo è autovettore di $h$. Segue che $h$ è diagonalizzabile e ha un unico autovalore $\lambda$. Infatti, presi $\lambda_1$ e $\lambda_2$, si avrebbero i seguenti autovalori indipendenti:
\begin{equation*}
	v_1\in V_{\lambda_1}\setminus\left\{0\right\}\qquad v_2\in V_{\lambda_2}\setminus\left\{0\right\}
\end{equation*}
Considerato che
\begin{equation*}
	\begin{cases}
		h\left(v_1\right)=\lambda_1 v_1\\
		h\left(v_2\right)=\lambda_2 v_2\\
		h\left(v_1+v_2\right)=\lambda\left(v_1+v_2\right)\\
		h\left(v_1+v_2\right)=h\left(v_1\right)+h\left(v_2\right)
	\end{cases}
	\implies \lambda\left(v_1+v_2\right)=\lambda_1 v_1+\lambda_2 v_2
\end{equation*}
Da cui segue, in quanto $v_1,\ v_2,\ v_1+v_2\neq 0$, che $\lambda=\lambda_1=\lambda_2$ e quindi è unico. Allora, $h=\lambda Id_{V}$ e pertanto $\phi=\lambda \psi$.\qedhere
\end{proof}
\subsection{Gruppo lineare proiettivo}
\begin{remark}{n}
	Consideriamo $\Proj\left(V\right)$ e l'insieme delle proiettività $\funct{}{\Proj\left(V\right)}{\Proj\left(V\right)}$.
	\begin{itemize}
		\item La \textit{composizione} di proiettività è una \textit{proiettività} (banalmente \textit{indotta} dalla composizione delle applicazioni lineari).
		\item Poiché $Id_{\Proj\left(V\right)}=\widetilde{Id_V}\implies$ L'identità $Id_{\Proj\left(V\right)}$ è una \textit{proiettività}.
		\item Se $\funct{}[\widetilde{\phi}]{\Proj\left(V\right)}{\Proj\left(V\right)}$, allora $\funct{}[\widetilde{\phi}^{-1}=\widetilde{\phi^{-1}}]{\Proj\left(V\right)}{\Proj\left(V\right)}$. In altre parole, l'\textit{inversa} di una proiettività è ancora una proiettività.
	\end{itemize}
L'insieme delle proiettività risulta un \textbf{gruppo} rispetto alla \textit{composizione}.
\end{remark}
\begin{definition}{}[Gruppo lineare proiettivo]
Il \textbf{gruppo lineare proiettivo}\index{gruppo!lineare proiettivo} $\Proj\GL\left(V\right)$ è il gruppo delle proiettività dello spazio vettoriale $V$ con operazione la composizione di proiettività ed elemento neutro $Id_{\Proj\left(V\right)}$.
\end{definition}
\subsubsection{Descrizione matriciale del gruppo lineare proiettivo}
Consideriamo gli isomorfismi $\funct{}{\K^{n+1}}{\K^{n+1}}$. Sappiamo che la matrice associata agli isomorfismi è una matrice invertibile, cioè si ha una \textit{isomorfismo di gruppi} fra l'insieme degli isomorfismi in $\K^{n+1}$ al \textit{gruppo generale lineare} $\GL\left(n+1,\ \K\right)$:
\begin{equation*}
	\left\{\text{isomorfismi}\ \funct{}{\K^{n+1}}{\K^{n+1}}\right\}\leftrightarrow\GL\left(n+1,\ \K\right)
\end{equation*}
E con il gruppo lineare proiettivo si può fare? Consideriamo
\begin{equation*}
	\funct{}[\oldphi]{\GL\left(n+1,\ \K\right)}{\Proj\GL_{n+1}\left(\K\right)}[\phi_A][\widetilde{\phi}_A]
\end{equation*}
 $\oldphi$ è \textit{omomorfismo} di gruppi \textit{suriettivo}, ma non iniettivo. Infatti, il nucleo non è \textit{triviale}:
 \begin{align*}
 	\ker \oldphi &= \Set{\phi_A | \phi_A=Id_{\Proj^n\left(\K\right)}=\widetilde{Id}_{\K^{n+1}}} = \Set{\phi_A | \phi_A=\lambda Id_{\mathbb{K}^{n+1}},\ \lambda\in\K\setminus\left\{0\right\}} =\\
 	&= \Set{\phi_A | A=\lambda I,\ \lambda\in\K\setminus\left\{0\right\}}
 \end{align*}
Tuttavia, per il \textit{primo teorema di isomorfismo per i gruppi} possiamo considerare il seguente diagramma commutativo:
\[\begin{tikzcd}
	{\GL\left(n+1,\ \K\right)} & {\Proj\GL_{n+1}\left(\K\right)} \\
	{\frac{\GL\left(n+1,\ \K\right)}{\left\{\lambda I\ \mid\ \lambda\in\K\setminus\left\{0\right\}\right\}}}
	\arrow["{\pi}"', from=1-1, to=2-1]
	\arrow["{\exists \overline f}"', from=2-1, to=1-2, dashed]
	\arrow["{\oldphi}", from=1-1, to=1-2]
\end{tikzcd}\]
Si ha pertanto l'isomorfismo
\begin{equation*}
	\Proj\GL_{n+1}\left(\K\right)\cong\GL\left(n+1,\ \K\right)/\left\{\lambda I\mid\lambda\in\K\setminus\left\{0\right\}\right\}
\end{equation*}
Si può anche considerare l'isomorfismo tra $\left\{\lambda I\mid\lambda\in\K\setminus\left\{0\right\}\right\}$ e $\K\setminus\left\{0\right\}$, e riscrivere l'isomorfismo trovato come
\begin{equation*}
	\Proj\GL_{n+1}\left(\K\right)\cong \GL\left(n+1,\ \K\right)/\left(\K\setminus\left\{0\right\}\right)
\end{equation*}

\begin{example}{n}
	Consideriamo la seguente proiettività della \textit{retta proiettiva} $\Proj^1\left(\R\right)$:
	\begin{equation*}
		\funct{}[f]{\Proj^1\left(\R\right)}{\Proj^1\left(\R\right)}[\left(x_0\colon x_1\right)][\left(ax_0+bx_1\colon cx_0+dx_1\right)]
	\end{equation*}
	Considerato il gruppo lineare proiettivo $\Proj^2\left(\R\right)=\GL\left(2,\ \R\right)/\left\{\lambda I\right\}$, per definizione di $f$ si ha $f=\widetilde{\phi}$, dove a $\phi$ è associata la matrice
	\begin{equation*}
		A=\left(\begin{array}{cc}
			a & b\\
			c & d
		\end{array}\right)
	\end{equation*}
	e dunque possiamo scrivere l'applicazione lineare $\phi$ come
	\begin{equation*}
		\funct{}[f]{\R^2}{\R^2}[\left(\begin{array}{c}
				x_0 \\
				x_1
			\end{array}\right)][A\left(\begin{array}{c}
				x_0 \\
				x_1
			\end{array}\right)]
	\end{equation*}
	Pertanto, $f$ si può anche scrivere come
	\begin{equation*}
		\funct{}[f]{\Proj\GL_2\left(\K\right)}{\Proj^1\left(\R\right)}[\left[v\right]][\left[Av\right]]
	\end{equation*}
	Notiamo che se la matrice associata a $\phi$ fosse $2A$, per \textit{proporzionalità} si avrebbe comunque la proiettività $f$. In modo analogo, $\lambda A$ con $\lambda\in\R\setminus\left\{0\right\}$ induce la \textit{stessa proiettività} $f$ di $A$.
\end{example}
\subsection{Altri aspetti delle trasformazioni proiettive}
\begin{remark}{n}
	Se $f$ è una proiettività di $\Proj\left(V\right)$ e $S\subseteq \Proj\left(V\right)$ un sottospazio proiettivo, allora $f\left(S\right)$ è ancora un sottospazio proiettivo della stessa dimensione di $S$. Se $S=\Proj\left(W\right)$ e consideriamo per definizione $f=\widetilde{\phi}$ con $\funct{}[\phi]{V}{V}$, allora
	\begin{equation*}
		\forall \left[v\right]\in S\ f\left(\left[v\right]\right)=\widetilde{\phi}\left(\left[v\right]\right)=\left[\phi\left(v\right)\right],\ \phi\left(v\right)\in W
	\end{equation*}
	\begin{equation*}
		f\left(S\right)=\Proj\left(\phi\left(W\right)\right)
	\end{equation*}
\end{remark}
\begin{definition}{}[Proiettivamente equivalenti]
	Due sottoinsiemi $A,\ B$ di $\Proj\left(V\right)$ si dicono \textbf{proiettivamente equivalenti}\index{proiettivamente equivalenti} se esiste una proiettività $f$ di $\Proj\left(V\right)$ tale che:
	\begin{equation*}
		B=f\left(A\right)
	\end{equation*}
\end{definition}
\begin{example}{n}
	Due sottospazi proiettivi di $\Proj\left(V\right)$ della \textit{stessa} dimensione sono sempre \textit{proiettivamente equivalenti}.
\end{example}
\begin{theorem}{}[Esistenza e unicità di una trasformazione proiettiva dati $n+2$ punti in posizione generale]\label{puntigeneralitrasformazioni}
Siano $\Proj\left(V\right)$ e $\Proj\left(V'\right)$ di dimensione $n$. Siano:
	\begin{itemize}
		\item $P_0,\ \ldots,\ P_{n+1}\in\Proj\left(V\right)$ $n+2$ punti in posizione generale.
		\item $Q_0,\ \ldots,\ Q_{n+1}\in\Proj\left(V'\right)$ $n+2$ punti in posizione generale.
	\end{itemize}
Allora esiste ed è unica $\funct{}[f]{\Proj\left(V\right)}{\Proj\left(V'\right)}$ trasformazione proiettiva tale che $f\left(P_i\right)=Q_i\ \forall i=0,\ \ldots,\ n+1$. In particolare, se una proiettività fissa $n+2$ punti $f\left(P_i\right)=P_i\ \forall i=0,\ \ldots,\ n+1$ in posizione generale, allora è l'identità.
\end{theorem}
\begin{proof}{n}~{}
	\begin{itemize}
		\item \textbf{Esistenza}: siano, $\forall i$
		\begin{itemize}
			\item $P_i=\left[v_i\right]\ v_i\in V$.
			\item $Q_i=\left[w_i\right]\ w_i\in V'$.
		\end{itemize}
	Sappiamo, dall'osservazione a pag. \pageref{puntigeneraleindipendentiosserva}, che
	\begin{itemize}
		\item $v_0,\ \ldots,\ v_n$ è base di $V$, con $v_{n+1}=\lambda_0v_0+\ldots+\lambda_n v_n$ con $\lambda_i\neq 0\ \forall i$.
		\item $w_0,\ \ldots,\ w_n$ è base di $V'$, con $w_{n+1}=\mu_0w_0+\ldots+\mu_n w_n$ con $\mu_i\neq 0\ \forall i$.
	\end{itemize}
		A meno di cambiare i rappresentanti dei punti, possiamo supporre senza perdita di generalità che $\lambda_i=\mu_i=1$. Si ha dunque
		\begin{align*}
			v_{n+1}&=v_0+\ldots+v_n\\
			w_{n+1}&=w_0+\ldots+w_n
		\end{align*}
	Sia $\funct{}[\phi]{V}{V'}$ l'applicazione lineare tale per cui $\phi\left(v_i\right)=w_i\ \forall i=0,\ \ldots,\ n$. Per linearità
	\begin{equation*}
		\phi\left(v_{n+1}\right)=\phi\left(v_0+\ldots+v_n\right)=\phi\left(v_0\right)+\ldots+\phi\left(v_n\right)=w_0+\ldots+w_n=w_{n+1}
	\end{equation*}
Poiché $\img \phi$ contiene una base per costruzione, $\phi$ è suriettiva. In particolare, essendo endomorfismo ($\dim V=\dim V'$), $\phi$ è anche \textit{isomorfismo}. Allora $\funct{}[f\coloneqq\widetilde{\phi}]{\Proj\left(V\right)}{\Proj\left(V'\right)}$ è una \textit{trasformazione proiettiva} e
\begin{equation*}
	f\left(P_i\right)=f\left(\left[v_i\right]\right)=\left[\phi\left(v_i\right)\right]=\left[w_i\right]=Q_i\ \forall i=0,\ \ldots,\ n+1
\end{equation*}
\item \textbf{Unicità}: sia $\funct{}[g]{\Proj\left(V\right)}{\Proj\left(V'\right)}$ un'altra trasformazione proiettiva tale che $g\left(P_i\right)=Q_i\ \forall i=0,\ \ldots,\ n+1$. Per definizione, esiste $\funct{}[\psi]{V}{V'}$ isomorfismo per cui $g=\widetilde{\psi}$ e
\begin{equation*}
	\left[\psi\left(v_i\right)\right]=\left[w_i\right]\ \forall i
\end{equation*}
Si ha che $\exists a_i\in\K\setminus\left\{0\right\}$ tale che $\psi\left(v_i\right)=a_iw_i$. Allora
\begin{equation*}
	\begin{array}{ccccc}
		a_{n+1}w_{n+1}&=&\psi\left(v_{n+1}\right)&=&\psi\left(v_0+\ldots+v_n\right)\\
		\shortparallel&&&&\shortparallel\\
		a_{n+1}\left(w_0+\ldots+w_n\right)&&&&\psi\left(v_0\right)+\ldots+\psi\left(v_n\right)\\
		\shortparallel&&&&\shortparallel\\
		a_{n+1}w_0+\ldots+a_{n+1}w_n&&&&a_0w_0+\ldots+a_nw_n
	\end{array}
\end{equation*}
Poiché $w_0,\ \ldots,\ w_n$ è base, la scrittura è unica. Segue che $a_0=a_1=\ldots=a_{n+1}=a$. Allora
\begin{equation*}
		\psi\left(v_i\right)=aw_i=a\phi\left(v_i\right)\implies \psi=a\phi\implies g=\widetilde{\psi}=\widetilde{\phi}=f\qedhere
\end{equation*}
	\end{itemize}
\end{proof}
\begin{example}{pn}~{}
	\begin{itemize}
		\item In una \textit{retta proiettiva} ($\dim 1$), una proiettività è determinata dalle immagini di $3$ \textit{punti distinti}, dato che è equivalente alla condizione di ‘‘\textit{punti in posizione generale}''.
		\item In un \textit{piano proiettivo} ($\dim 2$), una proiettività è determinata dalle \textit{immagini} di $4$ punti, a $3$ a $3$ \textit{non allineati}.
		\item Se $A,\ B\subseteq \Proj\left(V\right)$ sono insiemi finiti, ciascuno contenente $k$ punti in posizione generale, con $k\leq n+2$, allora $A$ e $B$ sono sempre proiettivamente equivalenti.
	\end{itemize}
\end{example}
\begin{example}{n}
	Approfondiamo l'ultimo esempio. In $\Proj^2\left(\K\right)$, cioè $\dim 2$, si prenda $A=\left\{P_1,\ P_2\right\},\ B=\left\{Q_1,\ Q_2\right\}$ con $P_1\neq P_2$, $Q_1\neq Q_2$. Ho due punti distinti sia in $A$ e $B$, dunque esiste sempre una proiettività $\funct{}[f]{\Proj^2}{\Proj^2}$ tale che $f\left(A\right)=B$. Se invece $A$ e $B$ contengono $3$ punti, se i $3$ punti in $A$ \textit{sono allineati} mentre i $3$ punti in $B$ \textit{non} lo sono, allora $A$ e $B$ \textit{non} sono proiettivamente equivalenti.
\end{example}
\subsection{Trasformazioni proiettive in coordinate}
Supponiamo di avere fissato dei \textit{riferimenti proiettivi} su $\Proj\left(V\right)$ e $\Proj\left(V'\right)$, dati da delle basi $\basis$ di $V$ e $\basis'$ di $V'$, e sia $\funct{}[f]{\Proj\left(V\right)}{\Proj\left(V'\right)}$ una trasformazione proiettiva. Sappiamo che $f=\widetilde{\phi}$ con $\funct{}[\phi]{V}{V'}$ isomorfismo lineare. Sia $A\in\GL\left(n+1,\ \K\right)$ la matrice associata a $\phi$ rispetto alle basi $\basis$ e $\basis'$. Abbiamo visto che $\phi$ è determinata solo a meno di multipli: chiaramente, lo stesso è vero anche per $A$. Siano allora
\begin{align*}
	P&=\left(x_0\colon\ldots\colon x_n\right)\in\Proj\left(V\right)\\
	f\left(P\right)&=\left(y_0\colon\ldots\colon y_n\right)\in\Proj\left(V'\right)
\end{align*}
Allora $\exists\rho\in\K\setminus\left\{0\right\}$ tale che $\rho y=Ax$.
\begin{remark}{}[Cambiamenti di coordinate]
Se in $\Proj\left(V\right)$ abbiamo due riferimenti proiettivi, uno dalla base $\basis$, e uno dalla base $\basis'$, sia $M$ la \textit{matrice del cambiamento di base} in $V$ tale che
\begin{equation*}
	x'=Mx
\end{equation*}
con $x$ in coordinate rispetto alla base $\basis$ e $x'$ in coordinate rispetto alla base $\basis'$. Allora, se $P\in\Proj\left(V\right)$ ha coordinate $\left(x_0\colon\ldots\colon x_n\right)$ rispetto a $\basis$ e
$\left(x_0'\colon\ldots\colon x_n'\right)$ rispetto a $\basis'$. Esiste $\rho\in\K\setminus\left\{0\right\}$ tale che $\rho x'=M x$.
\end{remark}
\subsection{Punti fissi di proiettività}
\begin{definition}{}[Punto fisso]
	Un punto $P\in\Proj\left(V\right)$ è \textbf{fisso}\index{punto!fisso} per $\funct{}[f]{\Proj\left(V\right)}{\Proj\left(V\right)}$ proiettività se $f\left(P\right)=P$.
\end{definition}
Sia $\funct{}[\phi]{V}{V}$ un \textit{automorfismo} tale che $f=\widetilde{\phi}$, e sia $P=\left[v\right]$, con $v\in V\setminus\left\{0\right\}$. Allora
\begin{equation*}
	f\left(P\right)=\left[\phi\left(v\right)\right]=\left[v\right] \iff \exists\lambda\in\K\setminus\left\{0\right\}\ \colon \phi\left(v\right)=\lambda v\iff v\text{ è un autovettore per }\phi
\end{equation*}
In particolare, $\phi$ è invertibile, dunque \textit{non} ha l'autovettore \textit{nullo}. Segue che i punti fissi di $f$ sono tutti e soli i punti $\left[v\right]$ con $v$ autovettore di $\phi$.
\begin{remark}{pn}~{}
	\begin{enumerate}
		\item Se $\K=\C$, allora ogni proiettività ha almeno un punto fisso, dato che $\phi$ ha sempre almeno un autovettore.
		\item Se $\K=\R$ e $\dim\Proj\left(V\right)=n$, allora $\dim V=n+1$. Il \textit{polinomio caratteristico} $C_\phi\left(t\right)\in\R\left[t\right]$ ha grado $n+1$. Se $n$ è \textit{pari}, $\phi$ ha almeno un autovalore, dato che il polinomio caratteristico ha grado $n+1$ \textit{dispari}: infatti, o è di grado \textit{uno} (e quindi ha banalmente soluzione) oppure, in quanto si può decomporre in fattori a coefficienti reali al più di grado \textit{due}, ammetterà \textit{sempre} almeno un fattore di grado \textit{uno}.
		\item Portiamo un controesempio al caso $n$ dispari. Sia
		\begin{equation*}
			\funct{}[f]{\Proj\left(\R\right)}{\Proj\left(\R\right)}[\left(x\colon y\right)][\left(-y\colon x\right)].
		\end{equation*}
		La matrice $A$ associata a $f$ è
		\begin{equation*}
			A=\left(\begin{array}{cc}
				0 & -1 \\
				1 & 0
			\end{array}\right)
		\end{equation*}
	Il polinomio caratteristico \textit{non} ha radici \textit{reali}:
	\begin{equation*}
		C_A\left(t\right)=\det\left(\begin{array}{cc}
			-t & -1 \\
			1 & -t
		\end{array}\right)=t^2+1
	\end{equation*}
Segue che $A$ non ha autovettori reali e pertanto $f$ \textit{non} ha punti fissi.
\item In generale, l'\textit{insieme} dei punti fissi di $\funct{}[f]{\Proj\left(V\right)}{\Proj\left(V\right)}$ è dato da
\begin{equation*}
	\Set{\Proj\left(V_{\lambda}\right)|\lambda\text{ autovalore di }\phi}
\end{equation*}
Questo è un insieme di sottospazi proiettivi a 2 a 2 disgiunti.
 	\end{enumerate}
\end{remark}
\begin{definition}{}[Insieme fisso]
	Se $S\subseteq\Proj\left(V\right)$ è un sottospazio, diciamo che $S$ è \textbf{fisso} per una proiettività $f$ se $f\left(S\right)=S$
\end{definition}
\subsection{Impratichiamoci! Trasformazioni proiettive}
\begin{exercise}{n}
	In $\Proj^1\left(\R\right)$ determinare la proiettività $f$ tale che
	\begin{equation*}
		f\left(2\colon 1\right)=\left(1\colon1\right)\quad f\left(1\colon 2\right)=\left(0\colon1\right)\quad
		f\left(1\colon -1\right)=\left(1\colon0\right)
	\end{equation*}
\end{exercise}
\begin{solution}{n}
Notiamo che i punti
\begin{equation*}
	\left(2\colon 1\right)\quad\left(1\colon 2\right)\quad\left(1\colon -1\right)\qquad\text{ e }\qquad	\left(1\colon 1\right)\quad\left(0\colon 1\right)\quad\left(1\colon 0\right)
\end{equation*}
sono distinti, dunque sono in posizione generale e la proiettività è garantita. Prendiamo la generica matrice $A=\left(\begin{array}{cc}
	a & b\\
	c & d
\end{array}\right)$ associata a $\phi$ indotta da $f$ e consideriamo $\rho y=Ax$:
\begin{equation*}
	\begin{cases}
		\rho y_0=ax_0+bx_1\\
		\rho y_1=cx_0+dx_1
	\end{cases}
\end{equation*}
Imponiamo il passaggio per $f\left(2\colon 1\right)=\left(1\colon1\right)$:
\begin{equation*}
	\begin{cases}
		\rho=2a+b\\
		\rho=2c+d
	\end{cases}\implies 2a+b=2c+d
\end{equation*}
In sostanza, \textit{eliminiamo} il parametro $\rho$ per ottenere un'equazione lineare \textit{omogenea} tra gli elementi della matrice. Facciamo lo stesso con i rimanenti punti $f\left(1\colon 2\right)=\left(0\colon1\right)$ e $	f\left(1\colon -1\right)=\left(1\colon0\right)$, utilizzando rispettivamente $\mu y=Ax$ e $\eta y=Ax$:
\begin{align*}
	\begin{cases}
		0=a+2b\\
		\mu=c+2d
	\end{cases}&\implies a+2b=0\\
	\begin{cases}
		\eta=a-b\\
		0=c-d
	\end{cases}&\implies c-d=0
\end{align*}
Costruiamo così un sistema lineare omogeneo di $3$ equazioni in $4$ incognite $a,\ b,\ c,\ d$, con una matrice dei coefficienti di rango $3$:
\begin{equation*}
	\begin{cases}
		2a+b=2c+d\\
		a+2b=0\\
		c-d=0
	\end{cases}\implies \begin{cases}
	a=2c\\
	b=-c\\
	c=c\\
	d=c
\end{cases}\implies
A=\left(\begin{array}{cc}
	2c & -c\\
	c & c
\end{array}\right)=c\left(\begin{array}{cc}
2 & -1\\
1 & 1
\end{array}\right)
\end{equation*}
A meno di multipli, $A=\left(\begin{array}{cc}
	2 & -1\\
	1 & 1
\end{array}\right)$ è la matrice cercata. Segue dunque che la proiettività cercata è
\begin{equation*}
	\funct{}[f]{\Proj^1\left(\R\right)}{\Proj^1\left(\R\right)}[\left(x_0\colon x_1\right)][\left(2x_0-x_1\colon x_0+x_1\right)]
\end{equation*}
\vspace{-3mm}
\end{solution}
\section{Geometria affine e geometria proiettiva}\label{spaziaffini}
Abbiamo già accennato all'esistenza di una relazione che intercorre fra \textit{geometria affine} e \textit{geometria proiettiva}. Diamo innanzitutto qualche richiamo dei concetti della geometria affine.
\begin{definition}{}[Spazio affine]
	Sia $V$ uno spazio vettoriale di dimensione finita su un campo $\K$. Uno \textbf{spazio affine}\index{spazio!affine} di dimensione $n$ su $V$ (con spazio vettoriale associato $V$ di dimensione $n$ ) è un insieme $\aff{V}$ non vuoto di \textit{punti} (elementi) tale che sia data un'applicazione
	\begin{equation*}
		\funct{}{\aff{V}\times\aff{V}}{V}[\left(P,\ Q\right)][\overrightarrow{PQ}]
	\end{equation*}
	che alla coppia di punti $\left(P,\ Q\right)$ associa il vettore di $V$ con punto iniziale $P$ e punto finale $Q$ e tale che siano	soddisfatti i seguenti assiomi:
	\begin{enumerate}
		\item $\forall P\in\aff{V},\ \forall v\in V$ esiste un unico punto $Q\in\aff{V}$ tale che $\overrightarrow{PQ}=v$.
		\item $\forall P,\ Q,\ R\in\aff{V}$ terna di punti di $\aff{V}$ si ha $\overrightarrow{PQ}+\overrightarrow{QR}=\overrightarrow{PR}$.
	\end{enumerate}
\end{definition}
\begin{definition}{}[Riferimento affine e coordinate affini]
	Un \textbf{riferimento affine}\index{riferimento!affine} $\mathcal{R} = (O,\  e_1,\  e_2,\ \ldots,\  e_n)$ sullo spazio $\aff{V}$ è assegnato fissando un punto $O\in \aff{V}$ detta \textbf{origine}\index{origine affine} ed una base $\basis = (e_1,\  e_2,\ \ldots,\  e_n)$ di $V$. Dunque, per ogni $P\in \aff{V}$ si ha la $n$-upla $(X_1,\  X_2,\ \ldots,\  X_n)$ dette \textit{coordinate affini}\index{coordinate!affini} del punto $P\in\aff{V}$ (uniche per riferimento affine fissato) tale per cui
	\begin{equation*}
		P=\overrightarrow{OV}=x_1e_1+\ldots+x_ne_n
	\end{equation*}
\end{definition}
Per i nostri scopi, parleremo spesso degli spazi affini di dimensione $n$ su $\K$.
\begin{definition}{}[Affinità]
	Un'\textbf{affinità}\seeonlyindex{affinità}{trasformazione!lineare affine} o \textbf{trasformazione lineare affine}\index{trasformazione!lineare affine} di $\aff{\K^n}$ è un'applicazione
	\begin{equation*}
		\funct{}[\phi]{\aff{\K^n}}{\aff{\K^n}}
	\end{equation*}
della forma $\phi\left(x\right)=Ax+b$ con $A\in\GL\left(n,\ \K\right)$ un'applicazione lineare invertibile e $b$ una \textit{traslazione}.
\end{definition}
\begin{definition}{}[Sottospazio affine]
	Un \textbf{sottospazio affine}\index{sottospazio!affine} di $\aff{\K^n}$ è un \textit{traslato} di un sottospazio vettoriale $W\subseteq \K^n$:
	\begin{equation*}
		S=W+x_0=\left\{w+x_0\mid w\in W,\ x_0\in\aff{W}\right\}.
	\end{equation*}
\end{definition}
\begin{remark}{pn}~{}
	\begin{itemize}
		\item $W$ è l'unico traslato di $S$ per l'origine ($x_0=O$) e si dice \textbf{sottospazio direttore}  di $S$\index{sottospazio!direttore}, cioè ne dà appunto la \textit{direzione}. Si definisce $\dim S\coloneqq\dim W$.
		\item Un punto in $\aff{\K^n}$ è un sottospazio affine di dimensione $0$ ($W=\left\{0\right\}$; dopotutto non ha particolarmente senso parlare di direzione del punto).
		\item Una \textbf{retta affine}\index{retta!affine} $r$ in $\aff{\K^n}$ è un sottospazio affine di $\dim 1$: $W=\lin{v}$, cioè $r$ si può individuare assegnando un punto $P\in r$ e un qualsiasi vettore $v$ \textit{parallelo} alla retta $r$.
		\item Un \textbf{piano affine}\index{piano!affine} $\pi$ in $\aff{\K^n}$ è un sottospazio affine di $\dim 2$: $W=\lin{v,\ w}$, cioè $\pi$ si può individuare assegnando un punto $P\in r$ e una coppia di vettori l.i. \textit{paralleli} al piano $\pi$.
		\item Un \textbf{iperpiano affine}\index{iperpiano!affine} è un sottospazio di dimensione $n-1$.
		\item Due sottospazi affini della stessa dimensione si dicono \textbf{paralleli}\index{sottospazio!affine!parallelo} se hanno lo \textit{stesso} sottospazio direttore.
	\end{itemize}
\end{remark}
\begin{example}{n}
Consideriamo $r=W+x_0$ retta affine, che ha dunque $\dim r=\dim W=1$. $W$ è la retta vettoriale in $\K^n$, mentre un qualunque $v\in W\setminus\left\{0\right\}$ è la \textit{direzione} della retta.
\end{example}
Un sottospazio affine $S\subseteq \K^n$ può essere descritto con equazioni cartesiane oppure in forma parametrica.
\begin{itemize}
	\item\textbf{Equazioni cartesiane}. $S$ è visto come l'insieme delle \textit{soluzioni} del sistema lineare
	\begin{equation*}
		Ax=b\qquad\left(\begin{array}{c}
			X_1\\
			\vdots\\
			X_n
		\end{array}\right)\in\aff{\K^{n}}
	\end{equation*}
	con $b$ che descrive la traslazione dovuta a $x_0\in\aff{W}$. In tal caso $W$ è il sottospazio vettoriale delle soluzioni del sistema lineare omogeneo associato $Ax=0$.
	\item \textbf{Forma parametrica}. Supponiamo $\dim S=\dim W=m$. Siano $v_1,\ \ldots,\ v_m\in\K^n$ i vettori di una base di $W$; rispetto ad una base di $\K^n$, e dunque rispetto ad un sistema di riferimento affine con origine $O$, essi sono espressi nelle componenti
	\begin{equation*}
		v_i=\left(V_{i,1},\ \ldots,\ V_{i,n}\right)\in\aff{\K^n}
	\end{equation*}
	Consideriamo $S=W+c$, con il punto $c=\left(C_1,\ \ldots,\ C_n\right)$ rispetto allo stesso sistema affine di prima. I punti $x$ di $S$ in forma parametrica sono dati da
	\begin{equation*}
		x=t_1v_1+\ldots+t_mv_m+c\quad t_1,\ \ldots,\ t_m\in\K,
	\end{equation*}
	da cui otteniamo il sistema $n\times\left(m+1\right)$ seguente
	\begin{equation*}
		\begin{cases}
			\begin{array}{l}
				X_1=t_1V_{1,1}+\ldots+t_mV_{m,1}+C_1\\
				\vdots\\
				X_n=t_1V_{1,n}+\ldots+t_mV_{m,n}+C_n\\			
			\end{array}
		\end{cases}
	\end{equation*}
\end{itemize}
\begin{example}{n}
	La retta $r$ ($\dim W=1$) passante per $c$ con direzione $v$ è descritto parametricamente da
\begin{equation*}
	\begin{cases}
		\begin{array}{l}
			X_1=tV_1+c_1\\
			\vdots\\
			X_n=tV_n+c_n			
		\end{array}
	\end{cases}
\end{equation*}	
\end{example}
Consideriamo ora lo \textit{spazio proiettivo numerico} $\Proj^n=\Proj^n\left(\K\right)=\Proj\left(\K^{n+1}\right)$ e i punti in coordinate omogenee $\left(x_0\colon \ldots\colon x_n\right)$ rispetto ad un dato sistema di riferimento proiettivo. Consideriamo il seguente sottoinsieme di $\Proj^n$:
\begin{equation*}
	U_0\coloneqq \left\{P=\left(x_0\colon\ldots\colon x_n\right)\in \Proj^n\mid x_0\neq 0\right\}
\end{equation*}
La condizione $x_0\neq 0$ è \textit{ben posta}; infatti, se $\lambda\in\K\setminus\left\{0\right\}$, allora $x_0\neq 0\iff\lambda x_0\neq 0$. Consideriamo anche il suo complementare, che è l'iperpiano coordinato rispetto alla prima coordinata omogenea:
\begin{equation*}
	\Proj^n\setminus U_0=H_0=\left\{P=\left(x_0\colon\ldots\colon x_n\right)\in \Proj^n\mid x_0= 0\right\}=\left\{P=\left(0\colon\ldots\colon x_n\right)\in \Proj^n\right\}
\end{equation*}
Sia $P\in U_0$: essendo $a_0\neq 0$ si ha $P=\left(a_0\colon\ldots\colon a_n\right)=\left(1\colon\frac{a_1}{a_0}\ldots\colon \frac{a_n}{a_0}\right)$. In particolare, $\frac{a_1}{a_0},\ \ldots,\ \frac{a_n}{a_0}$ sono univocamente determinate da $P$.
\begin{example}{n}
 Sia $\Proj^2\left(\R\right)=\left\{\text{rette vettoriali in }\R^3\right\}$ con punti di componenti $\left(x_0\colon x_1\colon x_2\right)$. Allora $H_0$ è una retta proiettiva in $\Proj^2\left(\R\right)$ e risulta
 \begin{equation*}
 	\begin{array}{ll}
 		H_0 & =\left\{P=\left(x_0\colon x_1\colon x_2\right)\in \Proj^n\mid x_0=0\right\}=\left\{P=\left(0\colon x_1\colon x_2\right)\in \Proj^n\right\} \\
 		& =\left\{\text{rette vettoriali di }\R^3\text{ contenute nel piano affine }x_0=0\right\}
 	\end{array}
 \end{equation*}
	Infatti, prendiamo $\aff{\R^3}$ e consideriamo il piano $x_0=1$, parallelo al piano $x_0=0$. Se $r\subseteq \R^3$ è una retta vettoriale che \textit{non} appartiene al piano affine $\left\{x_0=0\right\}$ ($r\nsubseteq \left\{x_0=0\right\}$), $r$ interseca il piano $x_0=1$ in un solo punto! In particolare, se $r$ ha direzione $\left(a_0,\ a_1,\ a_2\right)$, il punto nel piano $\left\{x_0=1\right\}$ avrà coordinate $\left(\frac{a_1}{a_0},\ \frac{a_2}{a_0}\right)$.
\end{example}
Possiamo identificare $U_0\subseteq\Proj^n$ con $\aff{\K^n}$. Consideriamo le due funzioni seguenti:
\begin{gather*}\label{spaziaffinitoproiettivi}
	\funct{}[j=j_0]{\aff{\K^n}}{U_0\subseteq \Proj^n}[\left(X_1,\ \ldots,\ X_n\right)][\left(1\colon\ X_1\colon \ldots\colon X_n\right)]\\
	\funct{}[\oldphi]{U_0\subseteq \Proj^n}{\aff{\K^n}}[\left(x_0\colon\ \ldots\colon x_n\right)][\left(\frac{x_1}{x_0},\ \ldots,\ \frac{x_n}{x_0}\right)]
\end{gather*}
\begin{itemize}
	\item $\oldphi$ è ben definita, dato che $x_0\neq 0$ per definizione di $U_0$.
	\item $j$ e $\oldphi$ sono l'una l'inversa dell'altra:
	% https://q.uiver.app/?q=WzAsNixbMCwwLCJcXGFmZntcXGthbXBebn0iXSxbMSwwLCJVXzAiXSxbMiwwLCJcXGFmZntcXGthbXBebn0iXSxbMCwxLCJcXGxlZnQoWF8xLFxcIFxcbGRvdHMsXFwgWF9uXFxyaWdodCkiXSxbMSwxLCJcXGxlZnQoMVxcY29sb24gWF8xXFxjb2xvbiBcXGxkb3RzXFxjb2xvbiBYX25cXHJpZ2h0KSJdLFsyLDEsIlxcbGVmdChYXzEsXFwgXFxsZG90cyxcXCBYX25cXHJpZ2h0KSJdLFswLDEsImoiXSxbMSwyLCJcXG9sZHBoaSJdLFszLDQsIiIsMCx7InN0eWxlIjp7InRhaWwiOnsibmFtZSI6Im1hcHMgdG8ifX19XSxbNCw1LCIiLDAseyJzdHlsZSI6eyJ0YWlsIjp7Im5hbWUiOiJtYXBzIHRvIn19fV1d
	\[\begin{tikzcd}
		{\aff{\K^n}} & {U_0} & {\aff{\K^n}} \\[-25pt]
		{\left(X_1,\ \ldots,\ X_n\right)} & {\left(1\colon X_1\colon \ldots\colon X_n\right)} & {\left(X_1,\ \ldots,\ X_n\right)}
		\arrow["{j}", from=1-1, to=1-2]
		\arrow["{\oldphi}", from=1-2, to=1-3]
		\arrow[from=2-1, to=2-2, maps to]
		\arrow[from=2-2, to=2-3, maps to]
	\end{tikzcd}\]
% https://q.uiver.app/?q=WzAsNixbMCwwLCJVXzAiXSxbMSwwLCJcXGFmZntcXGthbXBebn0iXSxbMiwwLCJVXzAiXSxbMCwxLCJcXGxlZnQoeF8wXFxjb2xvbiBcXGxkb3RzXFxjb2xvbiB4X25cXHJpZ2h0KSJdLFsxLDEsIlxcbGVmdChcXGZyYWN7eF8xfXt4XzB9LFxcIFxcbGRvdHMsXFwgXFxmcmFje3hfbn17eF8wfVxccmlnaHQpIl0sWzIsMSwiXFxsZWZ0KDFcXGNvbG9uXFxmcmFje3hfMX17eF8wfVxcY29sb25cXGxkb3RzXFxjb2xvblxcZnJhY3t4X259e3hfMH1cXHJpZ2h0KT1cXGxlZnQoeF8wXFxjb2xvblxcbGRvdHNcXGNvbG9uIHhfblxccmlnaHQpIl0sWzAsMSwiXFxvbGRwaGkiXSxbMSwyLCJqIl0sWzMsNCwiIiwwLHsic3R5bGUiOnsidGFpbCI6eyJuYW1lIjoibWFwcyB0byJ9fX1dLFs0LDUsIiIsMCx7InN0eWxlIjp7InRhaWwiOnsibmFtZSI6Im1hcHMgdG8ifX19XV0=
\[\begin{tikzcd}
	{U_0} & {\aff{\K^n}} & {U_0} \\[-25pt]
	{\left(x_0\colon \ldots\colon x_n\right)} & {\left(\frac{x_1}{x_0},\ \ldots,\ \frac{x_n}{x_0}\right)} & {\left(1\colon\frac{x_1}{x_0}\colon\ldots\colon\frac{x_n}{x_0}\right)=\left(x_0\colon\ldots\colon x_n\right)}
	\arrow["{\oldphi}", from=1-1, to=1-2]
	\arrow["{j}", from=1-2, to=1-3]
	\arrow[from=2-1, to=2-2, maps to]
	\arrow[from=2-2, to=2-3, maps to]
\end{tikzcd}\]
\end{itemize}
Si ha dunque che $j$ e $\oldphi$ sono \textit{biunivoche}. In questo modo identifichiamo $U_0\subseteq \Proj^n$ con $\aff{\K^n}$, mentre l'iperpiano $H_0$ corrisponde allo spazio proiettivo di dimensione $n-1$; si ha dunque
\begin{equation*}
	\Proj^n=U_0\amalg H_0=\aff{\K^n}\amalg \Proj^{n-1}.
\end{equation*}
La coppia $\left(U_0,\ j\right)$ è detta \textbf{carta affine}\index{carta affine} di $\Proj^n$.\\
In altre parole, $\Proj^n$ si può vedere come un'estensione o \textit{ampliamento} dello spazio affine $\K^n$. Diciamo allora che:
	\begin{itemize}
		\item I punti di $H_0$ sono detti \textbf{punti impropri}\seeonlyindex{punto!improprio}{punto!all'infinito} o \textbf{punti all'infinito}\index{punto!all'infinito}.
		\item $H_0$ è detto \textbf{iperpiano improprio}\seeonlyindex{iperpiano!improprio}{iperpiano!all'infinito} o \textbf{iperpiano all'infinito}\index{iperpiano!all'infinito}.
		\item I punti di $U_0=\aff{\K^n}$ sono detti \textbf{punti propri}\index{punto!proprio}.
	\end{itemize}
\begin{intuitively}{n}
In molti casi, possiamo liberamente parlare di $\aff{\K^n}$ come lo spazio vettoriale $\K^n$ inteso in senso \textit{geometrico} come insieme di punti con un punto qualunque come origine.
\end{intuitively}
\begin{example}{n}
	Consideriamo la retta proiettiva $\Proj^1\left(\K\right)$. L'iperpiano all'infinito è
	  \begin{equation*}
	 		H_0=\left\{P=\left(x_0\colon x_1\right)\in \Proj^1\mid x_0=0\right\}=\left\{\left(0\colon 1\right)\right\},
	 \end{equation*}
mentre invece l'insieme dei punti propri è
 \begin{equation*}
 	U_0=\left\{P=\left(x_0\colon x_1\right)\in \Proj^1\mid x_0\neq 0\right\}.
 \end{equation*}
In particolare, si ha la corrispondenza biunivoca $U_0\stackrel{1\colon1}{\leftrightarrow}\K$ definita come
\begin{equation*}
	\left(x_0\colon x_1\right)=\left(1\colon \frac{x_1}{x_0}\right)\mapsto \frac{x_1}{x_0}\in\K
\end{equation*}
\begin{minipage}{0.56\textwidth}
In altre parole, si può vedere la retta proiettiva come il campo $\K$ con l'aggiunto di un unico punto, l'\textit{infinito} $\infty$.
\begin{equation*}
	\Proj^1=\K\cup\left(0\colon 1\right)=\K\cup\left\{\infty\right\}
\end{equation*}
Se $\K=\R$, essendo $S^1\setminus\left\{1\text{ punto}\right\}\cong \R$, si ha
\begin{equation*}
	\Proj^1\left(\R\right)=\R\cup\left\{\infty\right\}\cong S^1
\end{equation*}
\end{minipage}
\begin{minipage}{0.44\textwidth}
	\includegraphics[trim=0cm 0cm 0cm 0cm,clip,scale=0.75]{images/projlinetocirc.pdf}
\end{minipage}
\vspace{-2mm}
\end{example}
\subsection{Chiusura proiettiva di un sottospazio affine}
\begin{definition}{}[Chiusura proiettiva della retta affine]
	Sia $r\subseteq\aff{\K^n}$ una retta affine. La \textbf{chiusura proiettiva}\index{chiusura!proiettiva!della retta} di $r$ è il sottospazio proiettivo $\overline{r}\subseteq\Proj^n$ generato da $r\subseteq U_0\subseteq\Proj^n$.
\end{definition}
\begin{proposition}{}[Chiusura proiettiva della retta affine]
	$\overline{r}$ è una \textit{retta proiettiva} e si ha
	\begin{equation*}
		\overline{r}=r\cup P_{\infty}
	\end{equation*}
dove $P_{\infty}=\overline{r}\cap H_0$ è detto \textit{punto all'infinito} o \textit{punto improprio} della retta $r$.
\end{proposition}
\begin{proof}{n}
	Sia $v\in\K^n\setminus\left\{0\right\}$ la direzione di $r$ e $w\in r$ un punto della retta. Allora $r$ ha descrizione parametrica in $\aff{\K^n}$
	\begin{equation*}
		\begin{cases}
			\begin{array}{l}
				X_1=tv_1+w_1\\
				\vdots\\
				X_n=tv_n+w_n\\			
			\end{array}
		\end{cases}\quad t\in\K
	\end{equation*}
Consideriamo la retta proiettiva $R\subseteq\Proj^n$ con descrizione parametrica
\begin{equation*}
\begin{cases}
		\begin{array}{l}
			x_0=s\\
			x_1=tv_1+sw_1\\
			\vdots\\
			x_n=tv_n+sw_n\\			
		\end{array}
	\end{cases}\quad \left(s\colon t\right)\in\Proj^1
\end{equation*}
$R$ è la retta proiettiva per i punti
\begin{equation*}
	t=0\ \colon\ \left(1\colon w_1\colon\ldots\colon w_n\right)\quad s=0\ \colon\ \left(0\colon v_1\colon\ldots\colon v_n\right)=P_\infty.
\end{equation*}
Ponendo $s=1$ otteniamo
\begin{equation*}
	\begin{cases}
		\begin{array}{l}
			x_0=1\\
			x_1=tv_1+w_1\\
			\vdots\\
			x_n=tv_n+w_n\\			
		\end{array}
	\end{cases}\quad t\in\K
\end{equation*}
Al variare di $t\in\K$, questi sono tutti e soli i punti di $j\left(r\right)\subseteq U_0\subseteq \Proj^n$. Si ha dunque che $R$ è una retta proiettiva contente $r$:
\begin{equation*}
	\begin{array}{ll}
			R&=r\cup P_{\infty}\\
			P_{\infty}&=R\cap H_0=\left\{\left(0\colon v_1\colon \ldots \colon v_n\right)\right\}
	\end{array}
\end{equation*}
$R$ è necessariamente il più piccolo sottospazio proiettivo contenente $r$, dato che è la retta più un solo punto. Pertanto, $R=\overline{r}$.\qedhere
\end{proof}
\begin{remark}{pn}~{}
	\begin{enumerate}
		\item Il punto improprio di $r$ è $P_{\infty}=\left(0\colon v_1\colon \ldots\colon v_n\right)$ e corrisponde esattamente alla \textit{direzione} $v=\left(v_1,\ \ldots,\ v_n\right)$ di $r$.	Poiché $P_\infty=\left[v\right]$ con $v$ la direzione di $r$, ne segue che l'iperpiano improprio di $\Proj^2\left(\K\right)$ è
	 \begin{equation*}
		\begin{array}{ll}
			H_0 & =\Proj^{n-1}\left(\K\right)=\Proj\left(K^n\right)=\left\{\text{rette vettoriali in }\K^n\right\}=\\
			&=\left\{\text{direzioni delle rette affini in }\aff{\K^n}\right\}
		\end{array}
	\end{equation*}
\item Due rette affini $r_1,\ r_2\subseteq \aff{\K^n}$ hanno lo stesso punto improprio se e solo hanno la \textit{stessa direzione}, cioè se sono \textit{parallele}. Se $r_1\neq r_2$ e $r_1$ e $r_2$ sono parallele, allora $r_1\cap r_2=\emptyset$ in $\aff{\K^n}$, ma $\overline{r_1}\cap \overline{r_2}={P_{\infty}}$ in $\Proj^n$. Ciò ci porta a dire che due rette parallele $r_1$ e $r_2$ si incontrano sempre all'\textit{infinito}!
\item Se $n=2$, cioè operando in $\Proj^2$, due rette distinte $r_1,\ r_2\subseteq \aff{\K^2}$ sono o \textit{incidenti} o \textit{parallele}, ma in $\Proj^2$ si intersecano sempre.
\item Viceversa, sia $l\subseteq \Proj^n$ una retta proiettiva. Abbiamo due casi:
\begin{itemize}
	\item $l\subseteq H_0,\ l\cap U_0= \emptyset$.
	\item $l\nsubseteq H_0\implies l+H_0=\Proj^n$.
\end{itemize}
Infatti, si ha che $l+H_0$ è un sottospazio proiettivo che contiene strettamente $H_0$, dato che $l\nsubseteq H_0$, e usando la formula di Grassmann otteniamo
\begin{equation*}
	\dim\left(l+H_0\right)=\dim l +\dim H_0-\dim \left(l\cap H_0\right)=1+n-1+0=n=\dim \Proj^n,
\end{equation*}
da cui ricaviamo che $l+H_0 = \Proj^n$. Sempre dalla formula di Grassmann
\begin{equation*}
	\dim\left(l\cap H_0\right)=0\implies l\cap H_0=\left\{1 \text{punto}\right\}=\left\{Q\right\},
\end{equation*}
cioè $l\cap U_0=l\setminus\left\{Q\right\}$. In altre parole, $l$ è una retta affine in $\aff{\K^n}$ con un \textit{punto improprio} $Q$ e necessariamente $l$ è la chiusura proiettiva di $l\setminus\left\{Q\right\}$.
\item Sia $n=2$, cioè operiamo in $\Proj^2$. Una retta $r\subseteq \aff{\K^2}$ è descritta da un'equazione lineare
\begin{equation*}
	ax+by+c=0\quad \left(a,\ b\right)\neq\left(0,\ 0\right)
\end{equation*}
con la corrispondenza biunivoca fra le coordinate $\left(x,\ y\right)$ vettoriali e $\left(X_1,\ X_2\right)$ affini. Abbiamo tuttavia anche la corrispondenza con le coordinate omogenee in $\Proj^2$, rispettivamente $\left(x\colon y\colon z\right)$ e $\left(x_0\colon x_1\colon x_2\right)$. Chiamiamo $\left(x\colon y\colon z\right)$ le coordinate omogenee su $\Proj^2$ con
\begin{equation*}
	H_0=\left\{P=\left(x\colon y\colon z\right)\in \Proj^2\mid z=0\right\}=\left\{P=\left(x\colon y\colon0\right)\in \Proj^2\right\}.
\end{equation*}
Allora la chiusura proiettiva $\overline{r}\subseteq \Proj^2$ di $r$ ha in $\Proj^2$ l'equazione lineare omogenea seguente:
\begin{equation*}
		ax+by+cz=0
\end{equation*}
Infatti, per $z=1$ si ottiene l'equazione di $r$, mentre ponendo $z=0$ (cioè il passaggio per $H_0$) troviamo il punto improprio $P_{\infty}$ di $r$:
\begin{equation*}
		\begin{cases}
		\begin{array}{l}
			z=0\\
			ax+by=0\\	
		\end{array}
	\end{cases}\quad P_{\infty}=\left(-b\colon a\colon 0\right)
\end{equation*}
La direzione della retta $ax+by+c=0$ è data dal punto improprio $P_{\infty}$ e corrisponde al vettore $\left(-b\colon a\colon 0\right)$.
\end{enumerate}
\end{remark}
Generalizziamo il concetto di chiusura proiettiva a un generico sottospazio affine.
\begin{definition}{}[Chiusura proiettiva di un sottospazio]
	Dato $S\subseteq \aff{\K^n}$ un sottospazio affine con $S\neq \emptyset$, la \textbf{chiusura proiettiva}\index{chiusura!proiettiva} $\overline{S}\subseteq \Proj^n$ di $S$ è il sottospazio proiettivo generato da $S$. Esso ha dimensione $\dim \overline{S}=\dim S=m$.
\end{definition}
\begin{itemize}
\item\textbf{Equazioni cartesiane}. Se $S$ come sottospazio affine è dato in forma cartesiana dal sistema lineare $h\times\left(n+1\right)$
\begin{gather*}
			Ax+b=0\qquad\left(\begin{array}{c}
		X_1\\
		\vdots\\
		X_n
	\end{array}\right)\in\aff{\K^{n}}\\
	\begin{cases}
	\begin{array}{l}
		a_{1,1}X_1+\ldots+a_{1,n}X_n+b_1=0\\
		\vdots\\
		a_{h,1}X_1+\ldots+a_{h,n}X_n+b_h=0\\			
	\end{array}
\end{cases}
\end{gather*}
allora $\overline{S}$ è descritto dal sistema lineare omogeneo  $h\times\left(n+1\right)$ in $\left(x_0,\ \ldots,\ x_n\right)$ seguente:
\begin{equation*}
	\left(A\mid b\right)x=0\qquad\left(\begin{array}{c}
		x_1\\
		\vdots\\
		x_n\\
		x_0
	\end{array}\right)\in\Proj^n
\end{equation*}
\begin{equation*}
	\label{chiusuraproiettiva}\begin{cases}
		\begin{array}{l}
			a_{1,1}x_1+\ldots+a_{1,n}x_n+b_1x_0=0\\
			\vdots\\
			a_{h,1}x_1+\ldots+a_{h,n}x_n+b_hx_0=0\\			
		\end{array}
	\end{cases}
\end{equation*}
\begin{remark}{n}
	Studiamo le dimensioni di $S$ e $\overline{S}$ usando i sistemi cartesiani appena definiti:
	\begin{equation*}
		\begin{array}{l}
			\dim S=\dim \K^n-\rk\left(A\right)=n-\rk\left(A\right)\\
			\dim \overline{S}=\dim \Proj^n-\rk\left(A\mid b\right)-1=\left(n+1-\rk\left(A\mid b\right)\right)-1=n-\rk\left(A\mid b\right)\\
		\end{array}
	\end{equation*}
	Per Rouché-Capelli vale $\rk A=\rk \left(A\mid b\right)$ in quanto $S\neq \emptyset$. In questo modo abbiamo dimostrato che $\dim \overline{S}=\dim S$.
\end{remark}
	I \textit{punti impropri} del sottospazio affine $S$ sono dati da $\overline{S}\cap H_0$, con $\overline{S}$ la chiusura proiettiva di $S$ e $H_0$ l'iperpiano improprio. Dal sistema \eqref{chiusuraproiettiva} si ha che $\overline{S}\cap H_0$ è dato da
\begin{equation*}
	\begin{cases}
		\begin{array}{l}
			x_0=0\\
			a_{1,1}x_1+\ldots+a_{1,n}x_n=0\\
			\vdots\\
			a_{h,1}x_1+\ldots+a_{h,n}x_n=0\\			
		\end{array}
	\end{cases}
\end{equation*}
Esso corrisponde al sistema lineare omogeneo in $\K^n$ $Ax=0$ associato al sistema lineare $Ax+b=0$ che definisce $S$. In altre parole, $\overline{S}\cap H_0$ corrisponde al \textit{sottospazio vettoriale direttore} $W\subseteq \K^n$ e vale $\overline{S}\cap H_0=\Proj\left(W\right)$ direzione di $S$. La sua dimensione per definizione di direzione è:
\begin{equation*}
	\dim\left(\overline{S}\cap H_0\right)=\dim S-1=\dim\overline{S}-1
\end{equation*}
\item \textbf{Equazioni parametriche}. Se $S$ ($\dim S=m$) è data in \textit{forma parametrica} e il sottospazio direttore $W\subseteq \K^n$ ha una base $\left\{v_1,\ \ldots,\ v_m\right\}$ (tali che $v_i=\left(V_{i,1},\ \ldots,\ V_{i,n}\right)\in\aff{\K^n}$ per un dato sistema di riferimento affine), posto $c\in S$ ricordiamo che l'espressione parametrica di $S$ è
\begin{gather*}
	X=t_1v_1+\ldots+t_mv_m+c\quad t_1,\ \ldots,\ t_m\in\K\\
		\begin{cases}
		\begin{array}{l}
			X_1=t_1V_{1,1}+\ldots+t_mV_{m,1}+C_1\\
			\vdots\\
			X_n=t_1V_{1,n}+\ldots+t_mV_{m,n}+C_n\\			
		\end{array}
	\end{cases}
\end{gather*}
Allora, $\overline{S}$ è il sottospazio generato dagli $m+1$ punti \textit{indipendenti}
\begin{equation*}
	\begin{array}{cc}
		\left(0\colon v_{i,1}\colon\ldots\colon v_{i,n}\right)&i=1,\ \ldots,\ m,\ \text{ e }
		\left(1\colon c_1\colon\ldots\colon c_n\right).
	\end{array}
\end{equation*}
Pertanto, $\overline{S}$ ha descrizione parametrica
\begin{equation*}
			\begin{cases}
		\begin{array}{l}
			x_0=t_0\\
			x_1=t_1v_{1,1}+\ldots+t_mv_{m,1}+t_0c_1\\
			\vdots\\
			x_n=t_1v_{1,n}+\ldots+t_mv_{m,n}+t_0c_n\\			
		\end{array}
	\end{cases}
\end{equation*}
con $\left(t_0\colon\ldots\colon t_m\right)\in\Proj^m$.
\end{itemize}
\subsection{Un esempio di proiettività}
Vediamo un esempio di proiettività di $\Proj^1$.
\begin{example}{n}
	Si consideri $\Proj^1\left(\K\right) = \K \cup \{\infty\}$ con $\infty=(0\colon 1)$. Sia $f$ una proiettività definita come
	\begin{equation*}
		\funct{}[f]{\Proj^1}{\Proj^1}[(x_0\colon x_1)][(ax_0+bx_1\colon cx_0+dx_1)]
	\end{equation*}
	Si ha che $f(0\colon 1)=(b\colon d)$, mentre la sua controimmagine è $f(-b\colon a)=(0\colon 1)$; infatti, siccome le coordinate sono omogenee, basta porre $ax_0+bx_1=0$. Sia $t=\frac{x_1}{x_0}$ la coordinata affine su $\K$, se $x_0\neq 0$ tutti i punti $(x_0\colon x_1)$ si possono scrivere come
	\begin{equation*}
		(x_0\colon x_1)=\left( 1\colon \frac{x_1}{x_0} \right)=(1\colon t),
	\end{equation*}
	il che corrisponde al punto $t\in\K$. Vediamo ora come si comporta l'immagine grazie a queste osservazioni se $ax_0+bx_1\neq 0$:
	\begin{gather*}
		f(x_0\colon x_1)=(ax_0+bx_1\colon cx_0+dx_1)= \left(1\colon \frac{cx_0+dx_1}{ax_0+bx_1} \right)  =  \left( 1 \colon  \frac{ x_0 \left( c+ d \frac{x_1}{x_0} \right) }{ x_0 \left( a+ b\frac{ x_1 }{ x_0 } \right)} \right)=\left( 1\colon \frac{dt+c}{bt+a}\right)
	\end{gather*}
	Dunque la proiettività $f$ corrisponde alla trasformazione
	\begin{equation*}
		\funct{}[F]{\K\cup \{\infty\}}{\K\cup \{\infty\}}\text{ con }F(t)=\begin{cases}
			\frac{dt+c}{bt+a}, & t\in\K, \ t\neq -\frac{a}{b}\\
			\infty, & t=-\frac{a}{b}\\
			\frac{d}{b}, & t=\infty
		\end{cases}
	\end{equation*}
	dove per $t=-\frac{a}{b}$ si ottiene $f(-b\colon a)=(0\colon 1)=\infty$, mentre la prima equazione è detta \textit{trasformazione lineare fratta}, che è definita sulla retta affine tranne dove si annulla il denominatore. Notiamo che $F$ diventa un'affinità
	\begin{equation*}
		\funct{}[F]{\K}{\K}[t][\alpha t+\beta]
	\end{equation*}
	se e solo se il denominatore diventa una costante ponendo $b=0$, ovvero se è della forma $F(t)=\alpha$, il che significa che la proiettività fissa il punto all'infinito, ovvero $f(0\colon 1)=(0\colon 1)$, mentre la parte affine viene mandata in se stessa.	Questo ragionamento si può vedere anche in dimensione superiore.
\end{example}
\subsection{Impratichiamoci! Geometria affine e geometria proiettiva}
\begin{exercise}{n}
	Sia $\K=\R$. Allora, preso $\R^n$ con la topologia Euclidea e $\Proj^n\left(\R\right)$ con la topologia quoziente, mostrare che $U_0$ è un aperto di $\Proj^n\left(\R\right)$ e che $\funct{}[j]{\R^n}{U_0}$ è un omeomorfismo.
\end{exercise}
% aggiungere soluzione
\section{Digressione: spazi proiettivi complessi}
\begin{remember}{n}
	Nel caso di $\R^n$ si è già visto che lo spazio proiettivo reale è un quoziente del tipo $\Proj^n\left(\R\right)=\left(\R^{n+1}\setminus\{0\}\right)/\!\sim$, dunque è dotato in maniera naturale di una topologia. Si può anche vedere come \textit{quoziente della sfera} $S^n$ dove si identificano i punti antipodali grazie alla \textit{suriezione} $\funct{s}[\pi]{S^n}{\Proj^n\left(\R\right)}$; è anche una \textit{varietà topologica compatta} di dimensione $n$. Inoltre $\Proj^1\left(\R\right)\cong S^1$ e abbiamo analizzato il piano proiettivo reale $\Proj^2\left(\R\right)$.
\end{remember}
Anche nel caso complesso per $\Proj^n\left(\C\right)=\left(\C^{n+1}\setminus\{0\}\right)/\!\sim$ si ha in maniera naturale una topologia quoziente data dalla topologia Euclidea su $\C^{n+1}\setminus\{0\} \cong \R^{n+1}\setminus\{0\}$. Vogliamo vedere che è una \textit{varietà topologica compatta} di dimensione $\mathbf{2n}$. Infatti, mentre $\Proj^n\left(\R\right)$ è localmente euclideo di dimensione $n$, si ha che $\Proj^n\left(\C\right)$ localmente si comporta come $\C^n\cong\R^{2n}$, dunque la dimensione topologica è $2n$.
\begin{remember}{n}
	 Ricordiamo la corrispondenza fra numeri complessi, reali e la norma:
	\begin{gather*}
		z_j=x_j+iy_j\implies  z=(z_1,\ \ldots,\ z_{n+1})\in\C^{n+1}\longleftrightarrow (x_1,\ y_1,\ \ldots,\ x_{n+1},\ y_{n+1})\in\R^{2n+2}\\
		\begin{array}{ccc}
			\displaystyle \|z\|^2=\sum_{j=1}^{n+1} \lvert z_j\rvert ^2=\sum_{j=1}^{n+1}\left(\lvert x_j \rvert ^2 +\lvert y_j \rvert ^2\right) & \implies & \displaystyle \ \|\lambda z \| =\sqrt{\sum_{j=1}^{n+1}\lvert \lambda z_j \rvert ^2}=\lvert\lambda\rvert \| z\|,\ \lambda\in\C
		\end{array}
	\end{gather*}
\end{remember}
\begin{theorem}{n}[${\Proj^n\left(\C\right)}$ è una varietà topologica di dimensione $2n$]
\end{theorem}
\begin{proof}{n}~{}
	\begin{itemize}
	\item $\Proj^n\left(\C\right)$ è \textbf{connesso}: è quoziente di $\C^{n+1}\setminus\{0\}$, che è connesso.
	\item $\Proj^n\left(\C\right)$ è \textbf{compatto}: per avere la tesi, si vuole vedere lo spazio come \textit{quoziente di un compatto} o \textit{immagine tramite una funzione continua di un compatto}.
	Ricordiamo la relazione di equivalenza
		\begin{equation*}
			z\sim w\iff \exists\lambda\in\C\setminus\{0\}\colon w=\lambda z.
		\end{equation*}
	Vogliamo dimostrare che un sistema completo di rappresentanti del piano proiettivo sono i \textit{punti della sfera complessa} $S^{2n+1}\subseteq \C^{n+1}=\R^{2n+2}$.
	Se $z\in\C^{n+1}\setminus\{0\}\implies \|z\|\neq 0$ e pertanto $S^{2n+1}\ni\frac{1}{\|z\|}\cdot z\sim z$. Poiché un sistema completo di rappresentanti è la sfera, si ha $\pi(S^{2n+1})=\Proj^n\left(\C\right)$ e dunque $\Proj^n\left(\C\right)$ è compatto.
	\item $\Proj^n\left(\C\right)$ è di Hausdorff: partiamo con una considerazione che segue dal punto precedente. Nel caso reale i punti sulla sfera sono equivalenti solo se \textit{antipodali}; nel caso complesso la questione è differente. Consideriamo $z,\ w\in S^{2n+1}$; per definizione del piano proiettivo
		\begin{equation*}
			z\sim w\iff \exists \lambda\in\C\setminus\{0\} \colon w=\lambda z.
		\end{equation*}
	Siccome $w,\ z\in S^{2n+1}$ hanno norma unitaria $1=\| w\|=\|\lambda z\|= \ \lambda | \|z\|=|\lambda|$. Essendoci \textit{infiniti} numeri di norma $1$ in $\C$, allora ci sono \textit{infiniti} numeri nella stessa classe, e quindi i punti $\lambda z\in S^{2n+1}$ sono tutti equivalenti.	Consideriamo ora $\funct{}[\pi_0=\pi_{\mid S^{2n+1}}]{S^{2n+1}}{\Proj^n\left(\C\right)}$. Per dimostrare che $\Proj^n\left(\C\right)$ è di Hausdorff, è sufficiente dimostrare che $\pi_0$ è un'identificazione \textit{chiusa} per il teorema \ref{quozientehausdorff} (pag. \pageref{quozientehausdorff}), ovvero il piano proiettivo si ottiene come quoziente della sfera.	Sia $C\subset S^{2n+1}$ un chiuso. Allora $\pi_0(C)$ è chiuso in $\Proj^n\left(\C\right)\iff \pi_0^{-1}(\pi_0(C))$ è chiuso in $S^{2n+1}$.
	Per le osservazioni precedenti
	\begin{equation*}
		\pi_0\left(z\right)= \pi_0\left(w\right)\iff \exists\lambda\in\C\text{ con }\lvert\lambda\rvert=1\colon w=\lambda z
	\end{equation*}
	In effetti, la relazione di equivalenza su $S^{2n+1}$ viene da un'azione del gruppo $S^1=\{\lambda\in\C \mid |\lambda |=1\}$ rispetto al prodotto con elemento neutro $1$:
	\begin{equation*}
		\funct{}[F]{S^1\times S^{2n+1}}{S^{2n+1}}[(\lambda, z)][\lambda z]
	\end{equation*}
	\begin{itemize}
		\item $F$ è un'applicazione continua.
		\item $S^1\times S^{2n+1}$ è compatto e Hausdorff.
	\end{itemize}
	$F$ è \textit{chiusa} in quanto funzione continua da un compatto in Hausdorff.	Dato un chiuso $C\subseteq S^{2n+1}$, prendendo la controimmagine dell'immagine agisco sui punti di $C$ con tutti gli elementi di $S^1$, cioè prendo tutte le orbite che intersecano $C$:
	\begin{equation*}
		\pi_0^{-1}(\pi_0(C))=F(\underbrace{S^1\times C}_{\stackrel{\text{chiuso in}}{S^1\times S^{2n+1}}})\subseteq S^{2n+1}\vspace{-2mm}
	\end{equation*}
Segue che $\pi_0^{-1}(\pi_0(C))$ chiuso e $\pi_0(C)$ chiuso in $\Proj^n\left(\C\right)$, cioè $\pi_0$ è applicazione chiusa e $\pi_0$ identificazione per il teorema \ref{condizione sufficiente identificazione} (Manetti 5.4), pag. \pageref{condizione sufficiente identificazione}. Pertanto, $\Proj^n\left(\C\right)$ è anche quoziente di $S^{2n+1}$. Siccome $S^{2n+1}$ è un compatto e di Hausdorff e $\pi_0$ è un'identificazione \textit{chiusa}, allora $\Proj^n\left(\C\right)$ è Hausdorff per il teorema citato in precedenza.
\item $\Proj^n\left(\C\right)$ è \textbf{localmente euclideo di dimensione} $2n$: per dimostrarlo riprendiamo le costruzioni introdotte a pag. \pageref{spaziaffinitoproiettivi}. Consideriamo la famiglia di insiemi
		\begin{equation*}
			U_j\coloneqq\{z_j\neq 0\}=\Proj^n\left(\C\right)\setminus H_j\text{, con }H_j\ j\text{-esimo iperpiano coordinato.}
		\end{equation*}
		Per semplicità sia $j=0$. Si consideri la proiezione al quoziente $\funct{}[\pi]{\C^{n+1}\setminus\{0\}}{\Proj^n\left(\C\right)}$ e la controimmagine $\pi^{-1}(U_0)=\{z\in\C^{n+1}\setminus\{0\}\mid z_0\neq 0\}$, aperto in $\C^{n+1}\setminus\{0\}$ in quanto abbiamo tolto un iperpiano. Segue che $U_0$ è aperto in $\Proj^n\left(\C\right)$ e lo stesso vale per tutti gli $U_j$. Si considerano le seguenti mappe biunivoche, una inversa dell'altra:
		\begin{equation*}
			\funct{}[j]{\C^n}{U_0}[(z_1,\ \ldots,\ z_n)][(1\colon z_1\colon \ldots\colon z_n)]\quad \funct{}[ \oldphi]{U_0}{\C^n}[(z_0\colon\ldots\colon z_n)][\left( \frac{z_1}{z_0},\ \ldots,\ \frac{z_n}{z_0} \right)]
		\end{equation*}
		Mostriamo che $j$ e $\oldphi$ siano omeomorfismi; siccome sono già biunivoche e una inversa dell'altra basta dimostrare che sono entrambe \textit{continue}.\\
			\begin{minipage}[t]{0.51\textwidth}\vspace{1pt}
				Per mostrare che $j$ è continua consideriamo il diagramma a lato, che rappresenta la fattorizzazione di $j$ in $\C^{n+1}\setminus\{0\}$ tramite
					\begin{equation*}
						\widetilde{j}((z_1,\ \ldots,\ z_n))=(1,\ z_1\ldots,\ z_n)
					\end{equation*}
				e la proiezione $\pi$, che in questo caso opera nel seguente modo:
					\begin{equation*}
					\pi((1,\ z_1\ \ldots,\ z_n))=(1\colon z_1\colon\ldots\colon z_n)
				\end{equation*}
				 Siccome $\widetilde{j}$ e $\pi$ sono continue, allora anche la loro composizione $j$ lo è.
			\end{minipage}\hspace{-15pt}
			\begin{minipage}[t]{0.49\textwidth}\vspace{25pt}
				% https://q.uiver.app/?q=WzAsNCxbMSwwLCJcXGNvbXBsZXhzZXRee24rMX1cXHNldG1pbnVzXFx7MFxcfSJdLFswLDIsIlxcY29tcGxleHNldF5uIl0sWzIsMiwiVV8wIl0sWzEsMl0sWzEsMCwiXFx3aWRldGlsZGV7an0iXSxbMSwyLCJqIiwyXSxbMCwyLCJcXHBpIl1d
				\[\begin{tikzcd}
					& {\C^{n+1}\setminus\{0\}} \\
					\\
					{\C^n} & {} & {U_0}
					\arrow["{\widetilde{j}}", from=3-1, to=1-2]
					\arrow["{j}"', from=3-1, to=3-3]
					\arrow["{\pi}", from=1-2, to=3-3]
				\end{tikzcd}\]
			\end{minipage}\\
			\begin{minipage}[t]{0.51\textwidth}\vspace{10pt}
				Per la continuità dell'inversa $\oldphi$ si procede con la fattorizzazione tramite una restrizione dell'inversa di $\pi$ a $\pi^{-1}U_0$
				\begin{equation*}
					\funct{}[p\coloneqq \pi_{\mid_{\pi^{-1}(U_0)}}]{\pi^{-1}(U_0)}{U_0}
				\end{equation*}
				e tramite $\hat{\oldphi}$:
				\begin{equation*}
					\widehat{\oldphi}(z_0,\ \ldots,\  z_n)=\left( \frac{z_1}{z_0},\ \ldots,\ \frac{z_n}{z_0}\right)
				\end{equation*}
			\end{minipage}\hspace{-15pt}
			\begin{minipage}[t]{0.49\textwidth}\hspace{-15pt}\vspace{25pt}
				% https://q.uiver.app/?q=WzAsNCxbMiwwLCJcXHBpXnstMX0oVV8wKSJdLFszLDAsIlxcc3Vic2V0IFxcY29tcGxleHNldF57fW4rMSJdLFswLDIsIlVfMCJdLFszLDIsIlxcY29tcGxleHNldF5uIl0sWzIsMywiXFxvbGRwaGkiLDJdLFswLDIsInAiLDJdLFswLDMsIlxcaGF0e1xcb2xkcGhpfSJdXQ==
				\[\begin{tikzcd}
					&[-35pt]& {\pi^{-1}(U_0)}\subset \C^{n+1} &[-15pt] {} \\
					\\
					{U_0} &&& {\C^n}
					\arrow["{\oldphi}"', from=3-1, to=3-4]
					\arrow["{p}"', from=1-3, to=3-1]
					\arrow["{\widehat{\oldphi}}", from=1-3, to=3-4]
				\end{tikzcd}\]
			\end{minipage}\\
		 Entrambe sono \textit{continue}: la prima è la restrizione di una funzione continua e la seconda, essendo ben definita ($U_0=\{z_0\neq 0\}$), risulta banalmente continua: inoltre, quest'ultima lavora solo con vettori e non con classi di equivalenza! Per dimostrare che $\oldphi$ è continua utilizzeremo le proprietà della topologia quoziente (vedasi \pageref{proprietà identificazione quoziente e mappa continua indotta}), dimostrando che $p$ è un'\textit{identificazione}: è già \textit{continua} e \textit{suriettiva}, dunque basta solo che sia aperta o chiusa per il teorema \ref{condizione sufficiente identificazione}. Osserviamo che $\funct{}[\pi]{\C^{n+1}\setminus\{0\}}{\Proj^n\left(\C\right)}$ è anch'essa un quoziente dato dall'azione del gruppo $G=\C\setminus\{0\}$ rispetto al prodotto per la moltiplicazione su $\C^{n+1}\setminus\{0\}$. In particolare, è un'azione per omeomorfismi. Infatti, fissato  $\lambda\in\C\setminus\{0\}$,
	\begin{equation*}
		\funct{}[\theta_\lambda]{\C^{n+1}\setminus\{0\}}{\C^{n+1}\setminus\{0\}}[z][\lambda z]
	\end{equation*}
	è continua: per la proposizione \ref{proiezione azione gruppo aperta} si ha che $\funct{}[\pi] {\C^{n+1}\setminus\{0\}}{\Proj^n\left(\C\right)}$ è un'applicazione aperta, pertanto anche $p$ che è una sua restrizione è aperta. Dalle considerazioni di cui sopra $p$ è un'identificazione. Ne segue che $U_0$ è un aperto di $\Proj^n\left(\C\right)$ omeomorfo a $\C^n$ e quindi a $\R^{2n}$. Allo stesso modo, $\forall j\in\{0,\ \ldots,\ n\},\ U_j$ è un aperto di $\Proj^n\left(\C\right)$ omeomorfo a $\C^n$ tramite la mappa
	\begin{equation*}
		\funct{}[\oldphi_j]{U_j}{\C^n}[(z_0\colon\ldots\colon z_n)][\left( \frac{z_0}{z_j},\ldots,\frac{z_{j-1}}{z_j}, \frac{z_{j+1}}{z_j},\ldots, \frac{z_n}{z_j} \right)].
	\end{equation*}
	Siccome in \textit{coordinate omogenee} c'è sempre un elemento \textit{non} nullo, allora ogni punto sta in uno degli aperti $U_j$, cioè
	\begin{equation*}
		\Proj^n\left(\C\right)=U_0\cup\ldots\cup U_n
	\end{equation*}
	 $\Proj^n\left(\C\right)$ è \textit{localmente euclideo} di dimensione $2n$, dunque è \textit{varietà topologica compatta}.\qedhere
	\end{itemize}
\end{proof}
\begin{remark}{pn}~{}
	\begin{itemize}
		\item Su un campo $\K$ qualsiasi si ha sempre la mappa $\oldphi_j$, una corrispondenza biunivoca fra i sottoinsiemi $U_j$ (complementari di un iperpiano coordinato) e $\K^n$ (senza aspetto topologico): vale sempre che lo spazio proiettivo è unione di tali $U_j$. In particolare, nel caso reale, tali $U_j$ sono \textit{aperti} ed il ragionamento è \textit{analogo} a quello fatto poc'anzi nel caso complesso.
		\item Non abbiamo dimostrato che è a base numerabile perché, essendo compatto, segue dal teorema \ref{compattoconnessohausdorfflocalmenteuclideo} (pag. \pageref{compattoconnessohausdorfflocalmenteuclideo}), inoltre si potrebbe dimostrare facilmente ‘‘a mano'' mostrando che gli aperti $U_j$ sono a base numerabile, dunque anche la loro unione finita lo è.
	\end{itemize}
\end{remark}

	\subsection{Retta proiettiva complessa}
Cosa succede per la \textit{retta proiettiva complessa}? Essa è una varietà topologica compatta di dimensione $2$ (dunque una \textit{superficie topologica compatta}) che abbiamo già classificato! Scopriamo di che superficie si tratta. Si ha
\begin{equation*}
	\Proj^1\left(\C\right)=U_0\cup\{(0\colon 1)\}\text{ con } U_0\cong\C\cong\R^2
\end{equation*}
In altre parole, la retta proiettiva complessa è un \textit{piano} unito ad un \textit{punto}. Dimostriamo ora che è omeomorfa a $S^2$, detta anche \textbf{sfera di Riemann} \index{sfera!di Riemann}, e analizziamo poi la differenza con il piano proiettivo reale.
\begin{theorem}{n}[${\Proj^1\left(\C\right)\cong S^2}$]
\end{theorem}
\begin{intuitively}{n}
	Per ottenere la sfera si può pensare di \textit{richiudere} il piano $\R^2$ su se stesso, aggiungendo il punto all'infinito $\{(0\colon 1)\}$.
\end{intuitively}
\begin{proof}{n}
	Costruiamo l'omeomorfismo con $S^2\subset\R^3$ attraverso la \textit{proiezione stereografica}. Consideriamo
		%https://q.uiver.app/?q=WzAsNCxbMCwwLCJTXjJcXHNldG1pbnVzXFx7TlxcfSJdLFsyLDAsIlxccmVhbHNldF4yIl0sWzQsMCwiXFxjb21wbGV4c2V0Il0sWzYsMCwiVV8wIl0sWzAsMSwicF9OIl0sWzEsMiwiayJdLFsyLDMsImoiXSxbMCwzLCJGX3t8X3tTXjJcXHNldG1pbnVzXFx7TlxcfX19IiwyLHsiY3VydmUiOjV9XV0=
		\[\begin{tikzcd}
			{S^2\setminus\{N\}} && {\R^2} && {\C} && {U_0}
			\arrow["{p_N}", from=1-1, to=1-3]
			\arrow["{k}", from=1-3, to=1-5]
			\arrow["{j}", from=1-5, to=1-7]
			\arrow["{F_{\mid_{S^2\setminus\{N\}}}}"', from=1-1, to=1-7, curve={height=30pt}]
		\end{tikzcd}\]
	con $\funct{ }[p_N]{S^2\setminus\{N\}}{\R^2}$ la \textit{proiezione stereografica} dal polo nord $N=(0,\ 0,\ 1)$,
	\begin{equation*}
		\funct{}[k]{\R^2}{\C}[\left(x,\ y\right)][x+iy]
	\end{equation*}
	l'identificazione canonica di $\R^2$ con $\C$ e
	\begin{equation*}
		\funct{}[j]{\C}{U_0}[w][(1\colon w)]
	\end{equation*}
	l'immersione di $\C$ come carta affina in $\Proj^[1]\left(\C\right)$. Allora definiamo una funzione $F$ su $S^2\setminus\{N\}$
	\begin{equation*}
		F_{\mid_{S^2\setminus\{N\}}}\coloneqq j\circ k\circ p
	\end{equation*}
	dove poniamo $F(N)\coloneqq (0\colon 1)$ in quanto è l'unico punto rimasto da dover ‘‘mappare'', ricordando che $\Proj^1\left(\C\right)=U_0\cup\{(0\colon 1)\}$.	Siccome $j\ ,\ k,\ p$ sono \textit{tutti} omeomorfismi, allora la loro composizione $F_{\mid_{S^2\setminus\{N\}}}$ è biunivoca e continua su $S^2\setminus\{N\}$. Perché $F$ sia continua in tutti i suoi punti basta verificarla che lo sia in un intorno (aperto) contenente $N$ perché valga il lemma di incollamento. % è l'unico motivo per cui secondo me ha senso questo: si crea F incollando le due proiezioni stereografiche sulle due sfere bucate.
	Considieriamo la proiezione stereografica dal polo sud $S=(0,\ 0,\ -1)$ scegliamo l'aperto $S^2\setminus\{S\}$, intorno di $N$. Dobbiamo mostrare la continuità di $F_{\mid_{S^2\setminus\{S\}}}$. Scriviamo le due proiezioni stereografiche, sia da $N$, sia da $S$; fissato un punto $P_0=(x_0,\ y_0,\ z_0)$ sulla sfera, allora consideriamo le semirette uscenti da $N$ e da $S$ che passano per $P_0$ \footnote{Le equazioni sono scritte in \textit{forma parametrica}, pertanto abbiamo evidenziato il punto per cui passano e la direzione.}:
		\begin{gather*}
			\begin{array}{ccc}
				NP_0\colon \begin{cases}
					x=0+tx_0\\
					y=0+ty_0\\
					z=1+t(z_0-1)
				\end{cases}\quad P_0-N & \quad & SP_0\colon \begin{cases}
					x=0+tx_0\\
					y=0+ty_0\\
					z=-1+t(z_0+1)
				\end{cases}\quad P_0-S
			\end{array}
		\end{gather*}
	Per trovare le immagini delle proiezioni stereografiche intersechiamo le due semirette con il piano $xy$, cioè poniamo $z=0$:
		\begin{gather*}
			\begin{array}{ccc}
				N\ \colon  1+t(z_0-1)=0\implies t=\frac{1}{1-z_0} & \quad & S\ \colon -1+t(z_0+1)=0\implies t=\frac{1}{1+z_0}
			\end{array}
		\end{gather*}
	Notiamo che i denominatori non si annullano in entrambi i casi per la definizione delle proiezioni stereografiche sulla sfera meno $N$ ed $S$ rispettivamente.	Ne segue che l'immagine di $\R^2$ tramite la mappa standard $k$ è
		\begin{gather*}
			\begin{array}{ccc}
				N\ \colon  \left( \frac{x_0}{1-z_0}, \frac{y_0}{1-z_0} \right) \mapsto w=\frac{x_0}{1-z_0} + i\frac{y_0}{1-z_0} \in\C & \quad &  S\ \colon \left( \frac{x_0}{1+z_0}, \frac{y_0}{1+z_0} \right) \mapsto u=\frac{x_0}{1+z_0} + i\frac{y_0}{1+z_0} \in\C
			\end{array}
		\end{gather*}
	Si ha che $w=\frac{1}{\overline{u}}$ e, viceversa, $u=\frac{1}{\overline{w}}$; infatti, quando $P_0\in S^2\setminus\{N,S\}$ allora $w,\ u\in\C\setminus\{0\}$. Verifichiamolo usando le proprietà dei numeri complessi:
		\begin{gather*}
			\frac{1}{\overline{u}} = \frac{1}{ \frac{x_0}{1+z_0} -i\frac{y_0}{1+z_0} } = \frac{1+z_0}{x_0-iy_0} \stackrel{!}{=} \frac{(x_0+iy_0)(1+z_0)}{x_0^2+y_0^2} \stackrel{!!}{=} \frac{(x_0+iy_0)(1+z_0)}{1-z_0^2}=\frac{x_0+iy_0}{1-z_0}=w
		\end{gather*}
	L'uguaglianza (!) vale perché, in quanto $P_0\neq N,\ S$ allora $x_0+iy_0\neq 0$, mentre quella (!!) segue dal fatto che il punto sta sulla sfera, quindi $x_0^2+y_0^2+z_0^2=1\implies x_0^2+y_0^2=1-z_0^2$. Abbiamo dunque ottenuto $F_{\mid_{S^2\setminus\{N,S\}}}$ come una composizione di mappe
		% https://q.uiver.app/?q=WzAsNCxbMCwwLCJTXjJcXHNldG1pbnVzXFx7TixTXFx9Il0sWzIsMCwiXFxyZWFsc2V0XjIiXSxbNCwwLCJcXGNvbXBsZXhzZXQiXSxbNiwwLCJVXzAiXSxbMCwxLCJwX04iXSxbMSwyLCJrIl0sWzIsMywiaiJdLFswLDMsIkZfe3xfe1NeMlxcc2V0bWludXNcXHtOLFNcXH19fSIsMix7ImN1cnZlIjo1fV1d
		\[\begin{tikzcd}
			{S^2\setminus\{N,S\}} && {\R^2} && {\C} && {U_0}
			\arrow["{p_N}", from=1-1, to=1-3]
			\arrow["{k}", from=1-3, to=1-5]
			\arrow["{j}", from=1-5, to=1-7]
			\arrow["{F_{\mid_{S^2\setminus\{N,S\}}}}"', from=1-1, to=1-7, curve={height=30pt}]
		\end{tikzcd}\]
	dove $j(w)=(1\colon w)=\left( 1\colon \frac{1}{\overline{u}} \right)=(\overline{u}\colon 1)$, lavorando in coordinate omogenee. Inoltre, $F$ si estende in modo continuo su $S^2\setminus\{S\}$
		% https://q.uiver.app/?q=WzAsOSxbMCwwLCJTXjJcXHNldG1pbnVzXFx7TixTXFx9Il0sWzIsMCwiXFxyZWFsc2V0XjIiXSxbNCwwLCJcXGNvbXBsZXhzZXQiXSxbOCwwLCJVXzAiXSxbMCwxLCJQIl0sWzQsMSwidSJdLFs2LDAsIlxcY29tcGxleHNldCJdLFs2LDEsIlxcb3ZlcmxpbmV7dX0iXSxbOCwxLCIoXFxvdmVybGluZXt1fVxcY29sb24gMSkiXSxbMCwxLCJwX04iXSxbMSwyLCJrIl0sWzIsNiwiYyJdLFs2LDMsImoiXSxbNCw1LCIiLDAseyJzdHlsZSI6eyJ0YWlsIjp7Im5hbWUiOiJtYXBzIHRvIn19fV0sWzUsNywiIiwwLHsic3R5bGUiOnsidGFpbCI6eyJuYW1lIjoibWFwcyB0byJ9fX1dLFs3LDgsIiIsMCx7InN0eWxlIjp7InRhaWwiOnsibmFtZSI6Im1hcHMgdG8ifX19XV0=
		\[\begin{tikzcd}
			{S^2\setminus\{S\}} && {\R^2} && {\C} && {\C} && {U_0} \\[-25pt]
			{P} &&&& {u} && {\overline{u}} && {(\overline{u}\colon 1)}
			\arrow["{p_S}", from=1-1, to=1-3]
			\arrow["{k}", from=1-3, to=1-5]
			\arrow["{c}", from=1-5, to=1-7]
			\arrow["{j}", from=1-7, to=1-9]
			\arrow[from=2-1, to=2-5, maps to]
			\arrow[from=2-5, to=2-7, maps to]
			\arrow[from=2-7, to=2-9, maps to]
		\end{tikzcd}\]
	con $c$ coniugio dei complessi (che è un omeomorfismo). In questo modo $F_{\mid_{S^2\setminus\{S\}}}$ è composizione di omeomorfismi e quindi continua.	Dunque $F$ è continua, chiusa in quanto funzione da $S^2$ compatto in $\Proj^1\left(\C\right)$ Hausdorff e biunivoca, pertanto $F$ è un \textit{omeomorfismo}.\qedhere
%questo ci dice che f ristretto a s^2-s,n è data da una composizione di mappe: proiezione stereografica da N, identificazione standard con C, mappa j in U_0. Ora al posto di w posso scrivere 1/\overline{u}, che per il quoziente è pari a  grazie alle coordinate omogenee: funziona come un'eliminzione dell'indeterminazione: anche se u=0 non è un problema perché ottengo 0:1. Nella proz ster da S il punto che va nello 0 è il polo nord, e questo ci dice che F si estende in maniera continua su S^n-s mandando P nella proiez ster da S , poi ho la mappa j_1, con tutti omeomorfisi. Dunque F è continua perché composizione di omeomorfismi.
\end{proof}
\begin{remark}{}[$\Proj^1\left(\C\right)\neq \Proj^2\left(\R\right)$]
Notiamo che $\Proj^1\left(\C\right)$ e $\Proj^2\left(\R\right)$ sono entrambe compattificazioni del piano, ma in modo \textit{profondamente diverso}!
\begin{itemize}
	\item $\Proj^1\left(\C\right)=\C\cup\{\infty\}\cong S^2$, è l'unione di un \textbf{piano} con un \textbf{punto all'infinito} ed abbiamo appena dimostrato essere omeomorfo alla \textit{sfera di Riemann} $S^2$.
	\item $\Proj^2\left(\R\right)=\R^2\cup\Proj^1\left(\R\right)\cong\R^2\cup S^1$, è il piano unito alla retta impropria $\Proj^1\left(\R\right)$; topologicamente, esso è l'\textbf{interno del disco} omeomorfo a $\R^2$ unito al bordo $S^1$ con la relazione di equivalenza per i punti antipodali. Per il teorema di classificazione delle superfici compatte (pag. \pageref{classificazionesuperficicompatte}), sappiamo che $S^2\ncong \Proj^2\left(\R\right)$.
\end{itemize}
\end{remark}
\section{Birapporto}
Nei nostri precedenti studi di Topologia abbiamo messo in primo piano lo studio delle \textit{proprietà topologiche}, quegli aspetti di uno spazio topologico che si preservano sotto \textit{omeomorfismi}. Anche nella Geometria Proiettiva risulta di fondamentale importanza la ricerca di \textbf{invarianti} rispetto alle proiettività.\\
Nella geometria Euclidea del piano diverse trasformazioni mantengono relazioni metriche come distanze, angoli e rapporti di distanze, ma passando alla \textit{retta proiettiva} la maggior parte di queste vengono \textit{distorte}. Tuttavia, già nella matematica del tardo periodo greco si trovò che un \textit{rapporto di rapporto di distanze} sul piano si preservava tramite certe trasformazioni.\\
Questo concetto, approfondito e generalizzato (separandolo completamente dalla \textit{distanza Euclidea}) nell'ottica della Geometria proiettiva nel \textit{secolo XIX}, diventò il \textbf{birapporto}, risultando uno degli invarianti proiettivi più usati.
\begin{definition}{}[Birapporto]
	Sia $\Proj^1\left(V\right)$ una retta proiettiva, ovvero $\dim V=2$. Siano $P_1,\ P_2,\ P_3,\ P_4\in\Proj^1\left(V\right)$ dei punti con $P_1,\ P_2,\ P_3$ distinti. Il \textbf{birapporto}\index{birapporto} dei punti $P_1,\ P_2,\ P_3,\ P_4$ (ordinati) è
		\begin{equation*}
			\beta(P_1,\ P_2, P_3, P_4)=\frac{y_1}{y_0}\in\K\cup\{\infty\}\text{ con }\beta=\infty\text{ se }y_0=0
		\end{equation*}
	dove $(y_0\colon y_1)$ sono le coordinate di $P_4$ nel riferimento proiettivo in cui $P_1$ e $P_2$ sono i punti fondamentali e $P_3$ il punto unità, cioè $P_1=(1\colon 0), P_2=(0\colon 1), P_3=(1\colon 1)$.
\end{definition}
\begin{remark}{pn}~{}
	\begin{enumerate}
		\item $y_0,y_1\in\K\implies\beta\in\K\cup\{\infty\}$
		\item $\beta$ è ben definito perché $P_1,\ P_2,\ P_3$ sono distinti e quindi sono in posizione generale\footnote{In una retta proiettiva essere distinti equivale ad essere in posizione generale.}, ovvero determinano in maniera univoca il riferimento proiettivo per il teorema \ref{puntiposizionegeneraleesistenzabase}, pag. \pageref{puntiposizionegeneraleesistenzabase}.
		\item Per ipotesi i primi tre punti sono distinti, mentre non si fanno ipotesi sul quarto punto; vediamo cosa succede in alcuni casi speciali:\label{birapporto2punti}
			\begin{equation*}
				\begin{array}{lllllll}
					P_4=P_1 &\iff& (y_0\colon y_1)=(1\colon 0)&&&\iff& \beta=0\\
					P_4=P_2 &\iff& (y_0\colon y_1)=(0\colon 1)&&& \iff&\beta=\infty\\
					P_4=P_3 &\iff& (y_0\colon y_1)=(1\colon 1)&\iff& y_0=y_1 &\iff&\beta=1
				\end{array}
			\end{equation*}
		Dunque $\beta\in\{0,1,\infty\}$ esattamente quando $P_4$ coincide con uno dei primi 3 punti. Quindi, se $P_1,\ P_2,\ P_3,\ P_4$ sono distinti, allora $\beta\in\K\setminus\{0,1\}$. Viceversa se $a\in\K\setminus\{0,1\}$ e $P_4=(1\colon a)$ allora $\beta=a$, dunque $\beta$ assume tutti i valori possibili in $\K$.
	\end{enumerate}
\end{remark}
Vogliamo ora scoprire come si calcola il birapporto non solo nel sistema di riferimento della definizione, ma in uno qualunque.
\begin{theorem}{}[Birapporto di quattro punti]
	Siano $P_1,\ P_2,\ P_3,\ P_4\in\Proj^1\left(V\right)$ dei punti nella retta proiettiva con $P_1,\ P_2,\ P_3$ distinti. Supponiamo che $P_i=(\lambda_i\colon \mu_i), i=1,\ \ldots,\ 4$. Allora
		\begin{equation*}
			\beta(P_1,\ P_2,\ P_3,\ P_4)=\frac{ \left| \begin{array}{cc}
					\lambda_1 & \lambda_4 \\
					\mu_1 & \mu_4
				\end{array} \right| \cdot \left| \begin{array}{cc}
				\lambda_2 & \lambda_3 \\
				\mu_2 & \mu_3
			\end{array} \right| } { \left| \begin{array}{cc}
			\lambda_1 & \lambda_3 \\
			\mu_1 & \mu_3
		\end{array} \right| \cdot \left| \begin{array}{cc}
		\lambda_2 & \lambda_4 \\
		\mu_2 & \mu_4
	\end{array} \right| }
		\end{equation*}
\end{theorem}
\begin{proof}{n}
	Poiché $P_1\neq P_2$, si ha che $(\lambda_1,\mu_1),(\lambda_2,\mu_2)$ sono una base di $\K^2$. Siano ora
	\begin{equation}\label{birapportoquattropunti1}
		a,\ b\in\K \colon (\lambda_3,\mu_3)=a(\lambda_1,\mu_1)+b(\lambda_2,\mu_2)=(a\lambda_1,a\mu_1)+(b\lambda_2,b\mu_2)
	\end{equation}
	In sostanza, facciamo una \textit{combinazione lineare} e portiamo dentro gli scalari. Per costruzione con questi vettori ottengo la base che dà il riferimento proiettivo con $P_1,\ P_2$ punti fondamentali e $P_3$ punto unità (riscalando in modo tale che sia la somma degli altri due vettori). Per ottenere $P_4$, scrivo il corrispettivo vettore come una combinazione lineare dei vettori della base:
	\begin{equation}\label{birapportoquattropunti2}
		\textcolor{redill}{\circled{\ast}}\quad	\exists c,d\in\K \colon (\lambda_4,\mu_4)=c(a\lambda_1,a\mu_1)+d(b\lambda_2,b\mu_2)
	\end{equation}
	Segue che $P_4$ ha coordinate $(c\colon d)$ nel nuovo riferimento proiettivo e quindi $\beta=\frac{d}{c}$. Per non appesantire la scrittura useremo come la seguente notazione per i determinanti:
	\begin{equation*}
		\delta_{ij}=\left| \begin{array}{cc}
			\lambda_i & \lambda_j \\
			\mu_i & \mu_j
		\end{array} \right|
	\end{equation*}
	Notiamo che la prima combinazione lineare \eqref{birapportoquattropunti} dà il seguente sistema lineare di 2 equazioni in 2 incognite $a,\ b$:
	\begin{equation*}
		\begin{cases}
			\lambda_1 a+\lambda_2 b=\lambda_3\\
			\mu_1 a+\mu_2 b=\mu_3
		\end{cases}
	\end{equation*}
	Risolvendo il sistema con il metodo di Cramer\footnote{Nelle ‘‘Note aggiuntive'', a pag. \pageref{Cramerrimembriancor}, si possono trovare alcuni brevi cenni al metodo di Cramer.} si ottiene:
	\begin{equation*}
		a=\frac{\left| \begin{array}{cc}
				\lambda_3 & \lambda_2 \\
				\mu_3 & \mu_2
			\end{array} \right| }{ \left| \begin{array}{cc}
				\lambda_1 & \lambda_2 \\
				\mu_1 & \mu_2
			\end{array} \right| } = \frac{\delta_{32}}{\delta_{12}}
		\qquad
		b=\frac{\left| \begin{array}{cc}
				\lambda_1 & \lambda_3 \\
				\mu_1 & \mu_3
			\end{array} \right| }{ \left| \begin{array}{cc}
				\lambda_1 & \lambda_2 \\
				\mu_1 & \mu_2
			\end{array} \right| } = \frac{\delta_{13}}{\delta_{12}}
	\end{equation*}
	La seconda combinazione lineare \eqref{birapportoquattropunti2} invece dà un sistema lineare in $c,\ d$, dove sostituisco i valori di $a,\ b$ trovati:
	\begin{equation*}
	\begin{cases}
		(a\lambda_1)c + (b\lambda_2)d=\lambda_4\\
		(a\mu_1)c + (b\mu_2)d=\mu_4
	\end{cases}	\Rightarrow \begin{cases}
		\frac{\delta_{32}}{\delta_{12}}\lambda_1 c + \frac{\delta_{13}}{\delta_{12}}\lambda_2 d = \lambda_4\\
		\frac{\delta_{32}}{\delta_{12}}\mu_1 c + \frac{\delta_{13}}{\delta_{12}}\mu_2 d = \mu_4
	\end{cases}
	\Rightarrow \begin{cases}
		(\delta_{32}\lambda_2)c + (\delta_{13}\lambda_2)d=\delta_{12}\lambda_4\\
		(\delta_{32}\mu_2)c + (\delta_{13}\mu_2)d=\delta_{12}\mu_4	
	\end{cases}
	\end{equation*}
	Applicando ancora Cramer, poiché il determinante è lineare in ogni colonna, possiamo estrarre i $\delta_{ij}$ dalle colonne e riscriverli \textit{riordinando gli indici}: scambiamo l'ordine delle colonne cambiamo il segno, ma facendolo sia al numeratore, sia al denominatore, i segni si semplificano e lo stesso vale per $d$:
		\begin{equation*}
			\begin{array}{l}
			c=\frac{ \left| \begin{array}{cc}
					\delta_{12}\lambda_4 & \delta_{13}\lambda_2 \\
					\delta_{12}\mu_4 & \delta_{13}\mu_2
			\end{array} \right| }{ \left| \begin{array}{cc}
				\delta_{32}\lambda_1 & \delta_{13}\lambda_2 \\
				\delta_{32}\mu_1 & \delta_{13}\mu_2
			\end{array} \right|  } = \frac{ \cancel{ \delta_{12} } \cancel{\delta_{13}} \delta_{42} }{ \delta_{32} \cancel{\delta_{13}}\cancel{\delta_{12}} } = \frac{-\delta_{24} }{-\delta_{23} }=\frac{\delta_{24}}{\delta_{23}} \\
			\\
			d=\frac{ \left| \begin{array}{cc}
				\delta_{32}\lambda_1 & \delta_{12}\lambda_4 \\
				\delta_{32}\mu_1 & \delta_{12}\mu_4
			\end{array} \right| }{ \delta_{32} \delta_{13} \delta_{12} } = \frac{ \cancel{\delta_{32}} \cancel{\delta_{12}} \delta_{14} }{\cancel{\delta_{32}} \delta_{13} \cancel{\delta_{12}} } = \frac{\delta_{14}}{\delta_{13}}\\
			\implies \beta=\frac{d}{c}=\frac{\delta_{14}}{\delta_{13}}\cdot \frac{\delta_{23}}{\delta_{24}}\qedhere
		\end{array}
		\end{equation*}					
\end{proof}
\begin{remark}{n}
	Il birapporto può anche essere definito tramite questa formula, che è \textit{ben definita}: supponendo di moltiplicare i punti per uno scalare, la stessa costante appare al numeratore e al denominatore, dunque si semplifica, pertanto il birapporto così definito \textit{non dipende} dalla scelta delle coordinate omogenee. Inoltre, al denominatore il primo determinante è sempre diverso da $0$ perché \textit{i punti sono distinti}, mentre il secondo determinante al denominatore è \textit{nullo} se e solo $P_2=P_4$, ottenendo la definizione originale.	
\end{remark}

\begin{tipsandtricks}{n}
Se tutti e 4 i punti sono diversi da $(0\colon 1)$, ovvero se $\forall i,\ \lambda_i\neq 0\ P_i=(1\colon z_i),\ i=1,\ \ldots,\ 4$ allora
\begin{equation*}
	\beta(P_1,\ P_2,\ P_3,\ P_4)=\frac{ (z_4-z_1)(z_3-z_2) }{ (z_4-z_2)(z_3-z_1) }.
\end{equation*}
Infatti, $P_i=(\lambda_i\colon\mu_i)=(1\colon z_i)$, cioè  $z_i=\frac{\mu_i}{\lambda_i}$; per la linearità delle colonne si ottiene
	\begin{gather*}
		\left| \begin{array}{cc}
			\lambda_1 & \lambda_4 \\
			\mu_1 & \mu_4
		\end{array} \right| = \lambda_1\lambda_4 \left| \begin{array}{cc}
		1 & 1 \\
		\frac{\mu_1}{\lambda_1} & \frac{\mu_4}{\lambda_4}
		\end{array} \right|= \lambda_1\lambda_4 \left| \begin{array}{cc}
			1 & 1 \\
			z_1 & z_4
		\end{array} \right| = \lambda_1\lambda_2(z_4-z_1)
	\end{gather*}
Procedendo allo stesso modo con gli altri determinanti e applicando la definizione equivalente di \textit{birapporto}, si ottiene il risultato di sopra in quando i $\lambda$ si semplificano.
\end{tipsandtricks}
%LEZ 32
\subsection{Birapporto e trasformazioni proiettive}
\begin{remember}{n}
	Date due rette proiettive $\Proj^1\left(V\right)$ e $\Proj^1\left(V'\right)$ ($\dim \Proj^1\left(V\right)=1=\dim \Proj^1\left(V'\right)$), e $P_1,\ P_2,\ P_3 \in \Proj^1\left(V\right)$ distinti e $Q_1,\ Q_2,\ Q_3 \in \Proj^1\left(V'\right)$ distinti, esiste sempre ed è unica la trasformazione proiettiva $\funct{}[f]{\Proj^1\left(V\right)}{\Proj^1\left(V'\right)}$ tale che $f(P_i)=Q_i, \ i=1,2,3$ grazie al il teorema \ref{puntigeneralitrasformazioni}, pag. \pageref{puntigeneralitrasformazioni}.
\end{remember}
Ci interessa ora provare un risultato più generale: l'esistenza di una proiettività con $4$ punti \textit{non tutti in posizione generale}.
\begin{theorem}{}[Esistenza di una proiettività tra rette proiettive con 4 punti di cui 3 distinti]
	Siano $\Proj^1\left(V\right)$ e $\Proj^1\left(V'\right)$ due rette proiettive, con:
	\begin{itemize}
		\item $P_1,\ P_2,\ P_3,\ P_4\in\Proj^1\left(V\right)$ (di cui i primi 3 distinti).
		\item $Q_1,\ Q_2,\ Q_3,\ Q_4\in\Proj^1\left(V'\right)$ (di cui i primi 3 distinti).
	\end{itemize}
	Allora esiste una trasformazione proiettiva $\funct{}[f]{\Proj^1\left(V\right)}{\Proj^1\left(V'\right)}$ tale che
	\begin{equation*}
		f(P_i)=Q_i,\ \forall i=1,\ \ldots,\ 4  \iff \beta(P_1,\ P_2,\ P_3,\ P_4)=\beta(Q_1,\ Q_2,\ Q_3,\ Q_4).
	\end{equation*}
\end{theorem}
\begin{remark}{n}
	Consideriamo i primi 3 punti e scegliamone dei rappresentanti per cui:
	\begin{equation*}
		v_1,\ v_2,\ v_3\in V \colon P_i=[v_i],\ i=1,\ 2,\ 3\text{ e }v_3=v_1+v_2
	\end{equation*}
	In altri termini, $\{v_1,v_2\}$ è base di $V$ che dà il riferimento proiettivo di $\Proj^1\left(V\right)$ in cui $P_1=(1\colon 0),\ P_2=(0\colon 1),\ P_3=(1\colon 1)$. Sia $P_4=[v_4]$ con $v_4=av_1+bv_2$. Allora $P_4=(a\colon b)$ e $\beta(P_i)=\frac{b}{a}$. Allo stesso modo per l'altra quaterna, siano:
	\begin{equation*}
		v'_1,\ v'_2,\ v'_3\in V' \colon Q_i=[v'_i],\ i=1,\ 2,\ 3\text{ con }v'_3=v'_1+v'_2
	\end{equation*}
	Siccome $P_1,\ P_2,\ P_3$ e $Q_1,\ Q_2,\ Q_3$ sono in posizione generale, allora \textit{esiste ed è unica} una trasformazione proiettiva $\funct{}[f]{\Proj^1\left(V\right)}{\Proj^1\left(V'\right)}$ tale che $f(P_i)=Q_i$. Inoltre, $f=\widetilde{\phi}$ con $\funct{}[\phi]{V}{V'}$ applicazione lineare tale che $\phi(v_1)=v'_1,\ \phi(v_2)=v'_2$, cioè porta una base di $V$ in una base di $V'$; segue che $\phi(v_4)=\phi(av_1+bv_2)=av'_1+bv'_2$ e pertanto $f(P_4)=[\phi(v_4)]=(a\colon b)$ nel riferimento in $\Proj^1\left(V'\right)$.
\end{remark}
\begin{proof}{n}~{}\\
	$\rightimplies$Siccome $f$ è unica e $f(P_4)=Q_4 \implies Q_4=(a\colon b)$ nel riferimento in cui $Q_1=(1\colon 0),\ Q_2=(0\colon 1),\ Q_3=(1\colon 1) \implies \beta(Q_1,\ Q_2,\ Q_3,\ Q_4)=\frac{b}{a}=\beta(P_1,\ P_2,\ P_3,\ P_4)$.\\
	$\leftimplies$Se $\beta(Q_1,\ Q_2,\ Q_3,\ Q_4)=\frac{b}{a}$, allora distinguiamo i casi in cui il birapporto è in $\K$ o \textit{infinito}:
	\begin{itemize}
		\item $\begin{array}{lllll}
			\frac{b}{a}\in\K & \implies & Q_4=\left(1\colon \frac{b}{a} \right)=(a\colon b)=f(P_4) &  &
		\end{array}$
		\item $\begin{array}{lllll}
			\frac{b}{a}=\infty & \implies & a=0 & \implies & Q_4=(0\colon 1)=f(P_4)\end{array}$\qedhere
	\end{itemize}
\end{proof}
\begin{corollary}{}[Birapporto è invariante proiettivo]
	Siano $\Proj^1\left(\K\right)$ e siano dati i punti $P_1,\ P_2,\ P_3,\ P_4\in\Proj^1\left(\K\right)$ di cui i primi 3 distinti, e altri quattro punti $Q_1,\ Q_2,\ Q_3,\ Q_4\in\Proj^1\left(\K\right)$ di cui i primi 3 distinti. Allora esiste una proiettività
	\begin{equation*}
		\funct{}[f]{\Proj^1\left(\K\right)}{\Proj^1\left(\K\right)}
	\end{equation*}
	tale che
	\begin{equation*}
		f(P_i)=Q_i,\ \forall i=1,\ \ldots,\ 4  \iff \beta(P_i)=\beta(Q_i)
	\end{equation*}
cioè il birapporto delle quaterne è \textit{invariante per proiettività}.
\end{corollary}

\begin{remark}{n}
	Sia $\mathcal{S}=\{\text{quaterne ordinate di punti distinti in } \Proj^1\left(\K\right)\}$. Prese due quaterne $\{P_1,\ P_2,\ P_3,\ P_4\}, \{Q_1,\ Q_2,\ Q_3,\ Q_4\} \in\mathcal{S}$, esse sono \textit{proiettivamente equivalenti} se $\exists f$ proiettività tale che $f(P_i)=Q_i
	,\ \forall i=1,\ 2,\ 3,\ 4$ e quindi, in base al teorema precedente, è vero se le due quaterne hanno lo stesso birapporto.Notiamo	che quella appena data è una \textit{relazione di equivalenza} %(verifica per esercizio)
	, tale per cui le classi di equivalenza \textit{proiettiva} di $4$ punti \textit{distinti e ordinati} in $\Proj^1\left(\K\right)$ sono in corrispondenza biunivoca con il campo $\K$ escluso $0$ e $1$ visto che sono 4 punti distinti\footnote{Si veda l'analisi dei casi del birapporto con 2 punti uguali a pag. \pageref{birapporto2punti}}, vale a dire
		\begin{equation*}
			\frac{\mathcal{S}}{\sim} \substack{\longleftrightarrow}{\beta}\in\K\setminus\{0,1\}
		\end{equation*}
	Si ha dunque che ad ogni quaterna di punti distinti associamo il suo \textit{birapporto} (l'applicazione è \textit{suriettiva}) e per ogni elemento nel campo troviamo una \textit{quaterna di punti distinti} con tale birapporto (quozientando, l'applicazione è iniettiva)
\end{remark}
%"il birapporto misura l'equivalenza proiettiva su punti di una retta proiettiva"

\begin{warning}{n}
	In dimensione maggiore generalmente il birapporto \textit{non è definito}, a meno che i $4$ punti non siano \textit{allineati} su una retta proiettiva $r$.
\end{warning}


\begin{example}{n}
	Nel piano proiettivo $\Proj^2\left(\K\right)$ consideriamo due quaterne di punti distinti $\{P_1,\ P_2,\ P_3,\ P_4\}$ e $\{Q_1,\ Q_2,\ Q_3,\ Q_4\}$. Scelta una quaterna, le disposizioni possibili sono le seguenti:
	\begin{itemize}
		\item In \textbf{posizione generale}, cioè a 3 a 3 non allineati.
\begin{center}
			\includegraphics[trim=0cm 0cm 0cm 0cm,clip,scale=0.50]{images/fourpoints1.pdf}
\end{center}
		\item 3 punti allineati su una retta.
\begin{center}
			\includegraphics[trim=0cm 0cm 0cm 0cm,clip,scale=0.50]{images/fourpoints2.pdf}
\end{center}
		\item 4 punti allineati su una retta.
\begin{center}
			\includegraphics[trim=0cm 0cm 0cm 0cm,clip,scale=0.50]{images/fourpoints3.pdf}
\end{center}
	\end{itemize}
	Se i punti $P_i$ sono proiettivamente equivalenti ai punti $Q_i$, allora tali posizioni \textit{devono essere mantenute}: le proiettività mandano \textit{rette in rette}, \textit{posizioni generali in posizioni generali}. Per contronominale, se si verificano casi diversi per le due quaterne possiamo affermare che \textit{non} sono proiettivamente equivalenti.
	\begin{enumerate}
		\item Sia $P_i$, sia $Q_i$ sono in \textit{posizione generale}. Siccome abbiamo 4 punti e $4=\dim\Proj^2\left(\K\right)+2$, allora esiste ed è unica la proiettività $f$ di $\Proj^2\left(\K\right)$ tale che $f(P_1)=Q_i, i=1,\ \ldots,\ 4$.
		\item $P_1,\ P_2,\ P_3$ \textbf{allineati ma non} $P_4$, \textbf{e lo stesso per i} $Q_i$; in altre parole, $P_1,\ P_2,\ P_3\in r$ retta proiettiva e $Q_1,\ Q_2,\ Q_3\in s$ retta proiettiva.
		\begin{center}
		\includegraphics[trim=0cm 0cm 0cm 0cm,clip,scale=0.50]{images/fourpointseq1.pdf}
		\end{center}
		Vogliamo dimostrare che anche in questo caso le quaterne sono proiettivamente equivalenti, estendendo la proiettività fra i tre punti  al quarto.	Scegliamo dei rappresentanti $P_i=[v_i],\ i=1,\ 2,\ 3,\ 4, v_i\in\K^3$ tale che $v_3=v_1+v_2$, lecito in quanto i punti sono allineati. Si ha che $\{v_1,\ v_2,\ v_4\}$ è una base di $\K^3$: $P_4\notin r$ significa che $v_4$ non è linearmente dipendente da $v_1,\ v_2$. Allo stesso modo sia $Q_i=[w_i],\ i=1,\ 2,\ 3,\ 4, w_i\in\K^3$ tale che $w_3=w_1+w_2$ con $\{w_1,\ w_2,\ w_4\}$ base di $\K^3$. Definiamo $\funct{}[\phi]{\K^3}{\K^3}$ lineare tale che
		\begin{equation*}
			\phi(v_1)=w_1,\ \phi(v_2)=w_2,\ \phi(v_4)=w_4.
		\end{equation*}
		Allora $\phi(v_3)=\phi(v_1+v_2)=w_1+w_2=w_3$ e dunque $\funct{}[f=\widetilde{\phi}]{\Proj^2\left(\K\right)}{\Proj^2\left(\K\right)}$ è una proiettività che manda $P_i$ in $Q_i,\ \forall i=1,\ 2,\ 3,\ 4$.
		\item \textbf{Tutti i punti sono allineati}.
		\begin{center}
		\includegraphics[trim=0cm 0cm 0cm 0cm,clip,scale=0.50]{images/fourpointseq2.pdf}
		\end{center}
		Essendo allineati, allora è definito il loro birapporto. Se le due quaterne sono proiettivamente equivalenti, sia $\funct{}[f]{\Proj^2\left(\K\right)}{\Proj^2\left(\K\right)}$ la proiettività: essa porta una quaterna nell'altra e necessariamente una retta nell'altra, ovvero $f(r)=s$; in altre parole, la restrizione alle due rette $\funct{}[f_{\mid_r}]{r}{s}$ porta $P_i$ in $Q_i,\ \forall i$ e quindi $\beta(P_i)=\beta(Q_i)$. Viceversa, se $\beta(P_i)=\beta(Q_i)$ allora esiste una trasformazione proiettiva $\funct{}[g]{r}{s}$che manda $P_i$ in $Q_i,\ \forall i=1,\ 2,\ 3,\ 4$. Si ha che $g$ si estende in maniera non unica ad una proiettività di $\Proj^2\left(\K\right)$. Infatti, dati:
		\begin{itemize}
			\item $r=\Proj\left(U\right),\ U\subset \K^3$.
			\item $s=\Proj\left(W\right),\ W\subset\K^3$.
		\end{itemize}
		con $U$ e $W$ piani vettoriali, si ha che $g=\widetilde{\psi}$ con $\funct{}[\psi]{U}{W}$ isomorfismo lineare. Vogliamo estenderla ad un \textit{automorfismo lineare} $\funct{}[\phi]{\K^3}{\K^3}$ scegliendo basi con due vettori nel piano ed uno esterno, ovvero $u_1,\ u_2\in U$ base di $U$ e $u_3\notin U$. Dunque $\{\psi(u_1),\psi(u_2)\}$ è una base di $W$ e $w_3\notin W$, pertanto $\{u_1,u_2,u_3\}$ è base di $\K^3$ e $\{\psi(u_1),\psi(u_2),w_3\}$ un'altra base. Ponendo $\funct{}[\phi] {\K^3}{\K^3}$ tale che
		\begin{itemize}
			\item $\phi(u_1)\coloneqq \psi(u_1)$;
			\item $\phi(u_2)\coloneqq \psi(u_2)$;
			\item $\phi(u_3)\coloneqq w_3$;
		\end{itemize}
		si ha $f=\widetilde{\phi}$. Dunque, i punti sono \textit{proiettivamente equivalenti} se e solo se hanno lo \textit{stesso birapporto}.
	\end{enumerate}
\end{example}
\subsection{Eserciziamoci! Birapporto}
\begin{exercise}{n}
	Verificare che se $P_1,\ P_2,\ P_3,\ P_4$ sono tutti diversi da $(1\colon 0)$, cioè $P_i=(w_i\colon 1)$,\ per ogni $i=1,\ \ldots,\ 4$, allora vale
	\begin{equation*}
		\beta=\frac{ (w_2-w_1)(w_3-w_2) }{ (w_4-w_2)(w_3-w_1) }
	\end{equation*}
\end{exercise}
	% aggiungere soluzione
		\section{Piano proiettivo duale}
Una \textbf{retta} in $\Proj^2\left(\K\right)$ ha equazione
\begin{equation*}
	r\ \colon a_0x_0+a_1x_1+a_2x_2=0
\end{equation*}
Essa è un'\textbf{equazione lineare omogenea} in coordinate omogenee, dunque è determinata dai coefficienti $a_0,\ a_1,\ a_2$ dell'equazione con la proprietà che devono essere \textit{non tutti nulli}; inoltre, fissati i coefficienti, l'equazione è determinata \textit{a meno di costante moltiplicativa non nulla}. Dunque si può associare a $r$ un \textbf{punto} del piano proiettivo dato dai coefficienti delle coordinate omogenee
\begin{equation*}
	(a_0\colon a_1\colon a_2)\in\Proj^2\left(\K\right)
\end{equation*}
Si ha una \textit{corrispondenza biunivoca} fra le rette in $\Proj^2\left(\K\right)$ e lo spazio proiettivo in sè:
	% https://q.uiver.app/?q=WzAsNCxbMCwwLCJcXHtcXHRleHR7cmV0dGUgaW4gfSBcXHByb2pbMl17XFxrYW1wfVxcfSJdLFszLDAsIlxccHJvalsyXXtcXGthbXB9Il0sWzAsMSwiclxcY29sb24gYV8weF8wK2FfMXhfMSthXzJ4XzI9MCJdLFszLDEsIihhXzBcXGNvbG9uIGFfMVxcY29sb24gYV8yKSJdLFswLDEsIiIsMCx7InN0eWxlIjp7InRhaWwiOnsibmFtZSI6ImFycm93aGVhZCJ9fX1dLFsyLDMsIiIsMCx7InN0eWxlIjp7InRhaWwiOnsibmFtZSI6ImFycm93aGVhZCJ9fX1dXQ==
	\[\begin{tikzcd}
		{\{\text{rette in } \Proj^2\left(\K\right)\}} &&& {\Proj^2\left(\K\right)} \\[-25pt]
		{r\colon a_0x_0+a_1x_1+a_2x_2=0} &&& {(a_0\colon a_1\colon a_2)}
		\arrow[from=1-1, to=1-4, tail reversed]
		\arrow[from=2-1, to=2-4, tail reversed]
	\end{tikzcd}\]
% Notiamo che ciò funziona bene con $\Proj^2\left(\K\right)$, infatti le coordinate omogenee si comportano proprio come i coefficienti dell'equazione.
\begin{example}{n}
In $\Proj^2\left(\R\right)$:
		\begin{equation*}
			l_i\colon x_0+x_1+2x_2=0 \longleftrightarrow (1\colon 1\colon 2)\in\Proj^2\left(\R\right)
		\end{equation*}
\end{example}
\begin{definition}{}[Piano proiettivo duale]
	Inteso $\Proj^2\left(\K\right)$ come lo spazio che \textit{parametrizza le rette} in $\Proj^2\left(\K\right)$, lo chiamiamo \textbf{piano proiettivo duale} \index{piano!proiettivo!duale} e lo denotiamo con $\Proj^2\left(\K\right)^{\ast}$.
\end{definition}
In prima istanza, questo significa semplicemente che si interpreta un punto del \textit{piano duale} come un punto \textit{associato} ad una \textit{retta} di $\Proj^2\left(\K\right)$.
%collgamento co lo spazio vettoriale duale forma lineare, prima costruzione come corrispondenza
\subsection{Fascio di rette}
\begin{definition}{}[Fascio di rette]
	Un \textbf{fascio di rette}\index{fascio!di rette} $\mathcal{F}$ in $\Proj^2\left(\K\right)$ è l'insieme delle rette di equazione
		\begin{equation*}
			\mathcal{F}\colon \lambda l_1+\mu l_2=0, \ (\lambda\colon\mu)\in\Proj^1\left(\K\right),
		\end{equation*}
	dove $l_1,\ l_2$ sono due rette fissate e distinte.
\end{definition}
\begin{remark}{n}
	Possiamo pensare al fascio di rette come una \textbf{collezione di rette}, le cui equazioni si ottengono come \textit{combinazione lineare} delle due rette del fascio con $\lambda,\ \mu$ come \textbf{parametri}.
\end{remark}
\begin{example}{n}
	Consideriamo le rette $l_1\colon x_0+x_1+2x_2=0$ e $l_2\colon 3x_0-2x_1+4x_2=0$. Il fascio di rette determinato da $l_1,\ l_2$ è
		\begin{gather*}
			(\lambda+3\mu)x_0+(\lambda-2\mu)x_1+2(\lambda+2\mu)x_2=0	
		\end{gather*}
	\begin{itemize}
		\item $(\lambda\colon\mu)=(1\colon 0) \rightarrow l_1$.
		\item $(\lambda\colon\mu)=(0\colon 1) \rightarrow l_2$.
		\item $(\lambda\colon\mu)=(1\colon 1) \colon 4x_0-x_1+6x_2=0$.	
	\end{itemize}
\end{example}

\begin{remark}{n}
	Abbiamo detto che ad ogni retta corrisponde un punto del piano proiettivo duale. Il fascio $\mathcal{F}$ corrisponde, sul piano duale $\Proj^2\left(\K\right)^{\ast}$, alla \textbf{retta} passante per i punti corrispondenti a $l_1$ e $l_2$
		\begin{gather*}
			l_1\colon a_0x_0+a_1x_1+a_2x_2=0 \longrightarrow (a_0\colon a_1\colon a_2)\\
			l_2\colon b_0x_0+b_1x_1+b_2x_2=0 \longrightarrow (b_0\colon b_1\colon b_2)
		\end{gather*}
		ossia è descritto da
		\begin{equation*}
			\begin{array}{c}
				\mathcal{F}\colon (\lambda a_0+\mu b_0)x_0+ (\lambda a_1+\mu b_1)x_1 +(\lambda a_2+\mu b_2)x_2=0\\
				\downarrow\\
				(\lambda a_0+\mu b_0 \colon \lambda a_1+\mu b_1 \colon \lambda a_2+\mu b_2)
			\end{array}
		\end{equation*}
	In altri termini, $(\lambda a_0+\mu b_0 \colon \lambda a_1+\mu b_1 \colon \lambda a_2+\mu b_2)$ rappresenta la retta per i due punti duali \textit{in forma parametrica}.
\end{remark}
\begin{example}{n}
	Prendiamo
		\begin{gather*}
			\begin{array}{l}
				l_1\colon x_0+x_1+2x_2=0 \longleftrightarrow (1\colon 1\colon 2)=Q_1\\
				l_2\colon 3x_0-2x_1+4x_2=0 \longleftrightarrow (3\colon -2\colon 4)=Q_2\\
			\end{array}
			\mathcal{F} \longleftrightarrow (\lambda+3\mu \colon \lambda-2\mu \colon 2(\lambda+2\mu))
		\end{gather*}
	Il fascio rappresenta la retta $\overline{Q_1Q_2}$ in forma \textit{parametrica}.
\end{example}
\begin{remark}{n}
	Siccome due rette distinte nel piano si intersecano \textit{in un punto solo}, consideriamo $P\coloneqq l_1\cap l_2$. Allora:
		\begin{itemize}
			\item Ogni retta del fascio passa per $P$ perché è lì che la combinazione lineare \textit{si annulla}.
			\item $P$ è l'unico punto comune a tutte le rette del fascio $\mathcal{F}$.
			\item Viceversa, ogni retta per $P$ appartiene al fascio $\mathcal{F}$.
		\end{itemize}
	Ciò significa che $\mathcal{F}$ è la \textit{famiglia delle rette} per il punto fissato $P$, che è detto \textbf{punto base del fascio}\index{punto!base di un fascio}.
	%P PUZZA DI SOTTOSPAZIO VETT CON VETT NULLO P
\end{remark}
\begin{example}{n}
	Consideriamo
		\begin{align*}
			\begin{cases}
				l_1\colon x_0+x_1+2x_2=0\\
				l_2\colon 3x_0-2x_1+4x_2=0
			\end{cases} &\implies \begin{cases}
				x_0=-x_1-2x_2\\
				-3x_1-6x_2-2x_1+4x_2=0\implies -5x_1-2x_2=0
			\end{cases}\\
		&\implies \begin{cases}
			x_0=4x_1\\
			x_2=-\frac{5}{2}x_1
		\end{cases}
			\end{align*}
		Otteniamo, a meno di multipli, il punto $P=(8\colon 2\colon -5)=l_1\cap l_2$. Tale fascio $\mathcal{F}$ corrisponde alla retta in $\left(\Proj^2\right) ^{\ast}$ per i punti $Q_1=(1\colon 1 \colon 2)$ e $Q_2=(3\colon -2\colon 4)$ che, scritta in forma \textit{parametrica}, corrisponde a $(\lambda+3\mu \colon \lambda-2\mu \colon 2(\lambda+2\mu))$. Cerchiamo ora l'equazione \textit{cartesiana} della retta $\overline{Q_1 Q_2}$ nelle coordinate $(a_0\colon a_1\colon a_2)$:
	\begin{gather*}
	\left| \begin{array}{ccc}
		a_0 & a_1 & a_2\\
		1 & 1 & 2 \\
		3 & -2 & 4 \end{array} \right| = 8a_0 + 2a_1-5a_2=0
	\end{gather*}
	Notiamo che i coefficienti della retta ottenuta sono esattamente le \textbf{coordinate omogenee} di $P$, intersezione delle due rette!
\end{example}
Più in generale, fissato un punto base $P\in\Proj^2\left(\K\right)$, l'insieme delle rette per $P$ in $\Proj^2\left(\K\right)$
\begin{equation*}
	\mathcal{F}_P=\{\text{rette per }P\text{ in }\Proj^2\left(\K\right)\}
\end{equation*}
è un fascio di rette corrispondente a una \textbf{retta} nel piano proiettivo duale $\left(\Proj^2\left(\K\right)\right)^{\ast}$. Se le coordinate del punto sono $P=(c_0\colon c_1\colon c_2)$, la retta corrispondente nel piano proiettivo duale $\left(\Proj^2\left(\K\right)\right)^{\ast}$ ha equazione cartesiana
\begin{equation*}
	c_0a_0+c_1a_1+c_2a_2=0.
\end{equation*}
Infatti, data una retta $r$ qualsiasi di equazione $a_0x_0+a_1x_1+a_2x_2=0$, il punto $P$ appartiene a $r$, cioè $P\in r$, se e solo se vale l'equazione precedente $c_0a_0+c_1a_1+c_2a_2=0$. Per scrivere il fascio $\mathcal{F}$ in forma \textit{parametrica} scelgo due rette distinte passanti per $P$.
%    sostituisco le coordinate di P e trovo quell'eq:   condizione perchè appartenga alla retta   o al fascio, collezione delle rette per p     fisso p e retta che varia (duale )   retta r fissata e p che varia. Ma l'eq è la stessa. Di fatto il fascio si scrive in forma parametrica scegliendo due rette specifiche che passano per p
\begin{remark}{}[Interpretazione affine del fascio di rette proiettive]
	Se interpretiamo $\aff{\K^2}=U_0\subset\Proj^2\left(\K\right)$ e consideriamo il fascio di rette proiettive $\mathcal{F}$ con punto base $P$ in $\Proj^2\left(\K\right)$, abbiamo due possibilità: $P$ è punto base \textit{proprio} o \textit{all'infinito}.
	\begin{itemize}
	\item Se $P$ è un \textbf{punto proprio}, allora $P\in\aff{\K^2}$ e $\mathcal{F}$ corrisponde alle \textit{rette affini} in $\K^2$ per il punto $P$ (passando dalla chiusura proiettiva della retta proiettiva a quella affine).
	\item Se $P$ invece è un \textit{punto improprio}, esso corrisponde ad una \textit{direzione} di rette nel piano affine e $\mathcal{F}$ corrisponde a tutte le rette affini che hanno questa direzione fissata, ovvero è un \textbf{fascio di rette parallele}.
	\end{itemize}
	Il caso proiettivo è interessante perché la distinzione fra questi due tipi di fasci è data solo dal fatto se il punto $P$ è proprio o improprio.  	
\end{remark}
\subsection{Spazi vettoriali duali e spazi proiettivi duali}
Sappiamo che $\Proj^2\left(\K\right)$ è il proiettivizzato di $\K^3$, ovvero $\Proj^2=\Proj^2\left(\K^3\right)$. Inoltre, in \textit{Geometria Uno}, abbiamo definito gli \textbf{spazi vettoriali duali} come\footnote{A differenza della notazione vista in \textit{Geometria Uno}, qui consideriamo gli indici delle coordinate da $0$ a $2$.}:
\begin{equation*}
	\left(\K^3\right)^{\ast}=\Set{\text{forme lineari }\alpha\text{ su }\K^3}=\Set{\funct{}[\alpha]{\K^3}{\K} | \alpha(x_0,\ x_1,\ x_2)= ax_0+a_1x_1+a_2x_2}
\end{equation*}
Quando consideriamo la retta $r\colon a_0x_0+a_1x_1+a_2x_2=0$, $r$ è il proiettivizzato del \textbf{nucleo} di $\alpha$, il quale è un \textit{piano vettoriale} $\ker\alpha\subset\K^3$ la cui equazione è appunto $a_0x_0+a_1x_1+a_2x_2=0$. In altre parole, si ha una corrispondenza fra i \textit{punti della retta proiettiva duale} e le \textit{classi proiettive} delle forme lineari $[\alpha]$:
\begin{equation*}
	\begin{array}{ccc}
		\left( \Proj^2\right)^{\ast}&=&\Proj^2\left(\K^3\right)^{\ast}\\
		r & \longleftrightarrow & [\alpha]\\
	\end{array}
\end{equation*}
Infatti, $\alpha$ è una forma lineare \textit{non nulla} determinata \textit{a meno di multipli}. Si ha che $\{x_0,\ x_1,\ x_2\}$ è una base di $(\K^3)^{\ast}$ che induce le coordinate proiettive $(a_0\colon a_1\colon a_2)$ su $\left( \Proj^2\right)^{\ast}$. Pertanto, tale \textit{interpretazione astratta} diventa operativa fissando la \textit{base duale} delle forme lineari: scrivo $\alpha$ come combinazione lineare della base, e i coefficienti saranno le  data dalle \textit{coordinate proiettive associate}.
Generalizziamo ulteriormente ad uno spazio vettoriale qualunque.
\begin{definition}{}[Spazio proiettivo duale]
	Dato uno spazio vettoriale $V$, il suo \textit{spazio proiettivo associato} $\Proj\left(V\right)$ ed il suo \textit{spazio vettoriale duale} $V^{\ast}=\Set{\text{forme lineari }\funct{}[\alpha]{V}{\K}}$, si definisce lo \textbf{spazio proiettivo duale}\index{spazio!proiettivo!duale} di $\Proj^n\left(V\right)$ come
	\begin{equation*}
		\Proj\left(V\right)^{\ast}\coloneqq\Proj\left(V^{\ast}\right).
	\end{equation*}
	Poiché $\dim V^{\ast} = \dim V$, allora $\Proj\left(V\right)^{\ast} =\Proj\left(V\right)$.
\end{definition}
In particolare, si ha la corrispondenza biunivoca
% https://q.uiver.app/?q=WzAsNCxbMCwwLCJcXHByb2p7Vn1ee1xcYXN0fSJdLFszLDAsIlxce1xcdGV4dHtpcGVycGlhbmkgZGkgfSBcXHByb2p7Vn0gXFx9Il0sWzAsMSwiW1xcYWxwaGFdIl0sWzMsMSwiXFxwcm9qe1xca2VyXFxhbHBoYX1cXHN1YnNldFxcbWF0aGJie1B9KFYpIl0sWzAsMSwiIiwwLHsic3R5bGUiOnsidGFpbCI6eyJuYW1lIjoiYXJyb3doZWFkIn19fV0sWzIsMywiIiwwLHsic3R5bGUiOnsidGFpbCI6eyJuYW1lIjoiYXJyb3doZWFkIn19fV1d
\[\begin{tikzcd}
	{\Proj\left(V\right)^{\ast}} &&& {\{\text{iperpiani di } \Proj\left(V\right) \}} \\[-3mm]
	{[\alpha]} &&& {\mathbb{P}\left(\ker \alpha\right)\subset\Proj\left(V\right)}
	\arrow[from=1-1, to=1-4, tail reversed]
	\arrow[from=2-1, to=2-4, tail reversed]
\end{tikzcd}\]
In coordinate, al piano di equazione $a_0x_0+\ldots +a_nx_n=0$ associamo il punto $(a_0\colon\ldots\colon a_n)$, nello stesso modo in cui ad un vettore associo i coefficienti della scrittura secondo tale base.
\subsection{Impratichiamoci! Piano proiettivo duale}
\begin{exercise}{n}
	In $\Proj^2\left(\R\right)$ scrivere in forma parametrica il fascio delle rette per il punto base $P$ di coordinate $(1\colon -1 \colon 4)$.
\end{exercise}
\begin{solution}{n}
	Scegliamo due rette distinte, a nostro piacere, che passano per il punto $P$; ad esempio, $l_1\colon x_0+x_1=0$ e $l_2\colon 4x_0-x_2=0$. Il fascio sarà
	\begin{gather*}
		\begin{array}{ccc}
			\mathcal{F}\colon  &\lambda(x_0+x_1)+\mu(4x_0-x_2)=0,\ & (\lambda\colon\mu)\in\Proj^1\\
			 &(\lambda+4\mu)x_0+\lambda x_1-\mu x_2=0,\ & (\lambda\colon\mu)\in\Proj^1
		\end{array}
	\end{gather*}
	Se facciamo variare $\lambda$ e $\mu$ otteniamo tutte le rette di $\Proj^2\left(\R\right)$ che passano per $P$.
\end{solution}
