% SVN info for this file
\svnidlong
{$HeadURL$}
{$LastChangedDate$}
{$LastChangedRevision$}
{$LastChangedBy$}

\chapter{Forma canonica di Jordan}
\labelChapter{jordan}

\begin{introduction}
	‘‘BEEP BOOP INSERIRE CITAZIONE QUA BEEP BOOP.''
	\begin{flushright}
		\textsc{NON UN ROBOT,} UN UMANO IN CARNE ED OSSA BEEP BOOP.
	\end{flushright}
\end{introduction}
% inserire citazione
% inserire introduzione carina
\section{Diagonalizzazione simultanea}
\begin{remember}
	Sia $V$ spazio vettoriale, su campo $\kamp$, di dimensione finita. Consideriamo gli \textbf{endomorfismi} di $V$ o, equivalentemente, le matrici $n\times n$ a elementi in $\kamp$ (con $\dim V=n$).
	\begin{itemize}
		\item $A$ determina un endomorfismo di $\kamp^n$ dato da $v\mapsto A\mathbf{v}$.
		\item Matrici associate allo stesso endomorfismo rispetto a basi diverse sono \textbf{simili}\index{matrice!simile}, cioè:
		\begin{equation}
			\exists P\in\gl\left(n\, \kamp\right)\ \colon B=P^{-1}AP
		\end{equation}
		\item Le seguenti affermazioni sono equivalenti:
		\begin{itemize}
			\item $A$ è \textbf{diagonalizzabile}\index{diagonalizzazione}.
			\item $A$ è simile ad una matrice diagonale $A=PDP^{-1}$ con $P$ matrice con \textbf{autovettori} sulle colonne.
			\item $\kamp^n$ ammette una base di \textbf{autovettori}\index{autovettore} di $A$.
			\item $\kamp^n=V_{\lambda_1}\oplus\ldots\oplus V_{\lambda_r}$ con $V_{\lambda_i}$ \textbf{autospazio}\index{autospazio} relativo ad $A$
		\end{itemize}
	\end{itemize}
\vspace{-3mm}
\end{remember}
\begin{define}\textsc{Diagonalizzazione simultanea}.\\
	Siano $A,\ B\in\kamp^{n, m}$ due matrici \textit{diagonalizzabili}. Diciamo che $A$ e $B$ sono \textbf{simultaneamente diagonalizzabili}\index{diagonalizzazione!simultanea} se esiste una base di $\kamp^n$ composta di autovettori sia di $A$ che di $B$.\\
	Equivalentemente, $A$ e $B$ sono \textbf{simultaneamente diagonalizzabili} se esiste una matrice invertibile $P$ tale che $P^{-1}AP$ e $P^{-1}BP$ sono \textit{entrambe diagonali}.
\end{define}
\begin{example}
	Non tutte le matrici diagonalizzabili lo sono simultaneamente. Prendiamo $\realset^2$; si consideri:
	\begin{itemize}
		\item $A$ diagonalizzabile con $2$ autovalori diversi, i cui autospazi sono le rette $y=x$ e $y=-x$.
		\item $B$ diagonalizzabile con $2$ autovalori diversi, i cui autospazi sono le rette $y=0$ e $y=2x$.
	\end{itemize}
Non esiste alcun autovettore comune, dunque $A$ e $B$ \textit{non} sono simultaneamente diagonalizzabili.
\end{example}
Dalla sola definizione non è semplice capire quali matrici sono a tutti gli effetti simultaneamente diagonalizzabili. Tuttavia, il seguente teorema ci permetterà di trovare una condizione necessaria e sufficiente per la diagonalizzazione simultanea.
\begin{theorema}\label{teoremasimdiag}
	Siano $A,\ B\in\kamp^{n, n}$. Allora $A$ e $B$ sono simultaneamente diagonalizzabili se e solo se $A$ e $ B$ sono diagonalizzabili e $A,\ B$ commutano, cioè $AB=BA$.
\end{theorema}
Per dimostrare il teorema, abbiamo tuttavia bisogno del seguente lemma:
\begin{lemming}
	Siano $A,\ B\in\kamp^{n, n}$ tale che $AB=BA$ e sia $W$ un autospazio di $B$. Allora, presa l'azione di $\gl\left(n,\ \kamp\right)$ su $\kamp^{n, n}$, si ha che $A\ldotp W\subseteq W$.
\end{lemming}
\begin{demonstration}
	Sia $\lambda$ l'autovalore di $B$ relativo all'autospazio $W$. Per definizione di autospazio:
	\begin{equation*}
		W=\left\{\mathbf{v}\in V\mid B\ldotp \mathbf{v}=\lambda \mathbf{v}\right\}
	\end{equation*}
Sia $\mathbf{w}\in W$. Vogliamo mostrare che $A\ldotp \mathbf{w}\in W$.
\begin{equation*}
	B\ldotp\left(A\ldotp\mathbf{w}\right)=\left(BA\ldotp\mathbf{w}\right)=\left(AB\ldotp\mathbf{w}\right)=A\ldotp\left(B\ldotp\mathbf{w}\right)=A\ldotp\left(\lambda\mathbf{w}\right)=\lambda\left(A\ldotp\mathbf{w}\right)
\end{equation*}
$A\ldotp \mathbf{w}$ è autovettore rispetto a $\lambda$, pertanto $A\ldotp \mathbf{w}\in W,\ \forall \mathbf{w}$ e dunque segue la tesi.
\end{demonstration}
\begin{demonstration} \textsc{Dimostrazione del teorema} \ref{teoremasimdiag}.\\
	$\impliesdx$ Per ipotesi, $\exists P\in \gl\left(n,\ \kamp\right)$ tale che $D_1=P^{-1}AP$ e $D_2=P^{-1}BP$ sono diagonali; in particolare, in quanto matrici diagonali, esse commutano: $D_1D_2=D_2D_1$. Allora $A=PD_1P^{-1}$ e $B=PD_2P^{-1}$.
	\begin{equation*}
		AB=\left(PD_1P^{-1}\right)\left(PD_2P^{-1}\right)=PD_1D_2P^{-1}=PD_2D_1P^{-1}=\left(PD_2P^{-1}\right)\left(PD_1P^{-1}\right)=BA
	\end{equation*}
$\impliessx$ Procediamo con una \textit{dimostrazione costruttiva}. Sappiamo che:
\begin{itemize}
	\item $A$ diagonalizzabile $\implies \exists \mathbf{v}_1,\ \ldots,\ \mathbf{v}_n$ base di $V$ composta da \textit{autovettori} di $A$.
	\item $B$ diagonalizzabile $\implies V=W_1\oplus\ldots\oplus W_r$ con $W_j$ \textit{autospazi} di $B$
\end{itemize}
Consideriamo $\mathbf{v}_1\in V$. Esso si scrive in modo unico:
\begin{equation*}
	\textcolor{red}{\circled{{\ast}}}\quad \mathbf{v}_1=\mathbf{w}_{1,1}+\ldots+\mathbf{w}_{1,r}\text{ con }\mathbf{w}_{1,j}\in W_j,\ \forall j=1,\ \ldots,\ r
\end{equation*}
$\mathbf{v}_1$ è autovettore di $A$ relativo all'autovalore di $\lambda_1$, dunque $A\mathbf{v}_1=\lambda_1\mathbf{v}$. \textit{Moltiplichiamo} $\mathbf{v}_1$ per $A$:
\begin{equation*}
	\begin{array}{ccc}
		A\ldotp\mathbf{v}_1&=&A\ldotp\mathbf{w}_{1,1}+\ldots+A\ldotp\mathbf{w}_{1,r}\\
		\shortparallel&\\
		\lambda_1\mathbf{v}_1&=&\lambda_1\mathbf{w}_{1,1}+\ldots+\lambda_1\mathbf{w}_{1,r}
	\end{array}
\end{equation*}
Dal lemma appena dimostrato, da $A\ldotp W_j\subseteq W_j,\ \forall j$ segue che $A\ldotp \mathbf{w}_j\in W_j$. Per la chiusura di $W_j$ rispetto al prodotto per uno scalare, abbiamo anche $\lambda \mathbf{w}_j\in W_j,\ \forall j$. Siccome in una somma diretta la decomposizione è unica, deduciamo che:
\begin{equation*}
	A\ldotp \mathbf{w}_{1,1}=\lambda_1 \mathbf{w}_{1,1},\ \ldots,\ A\ldotp \mathbf{w}_{1,r}=\lambda_1 \mathbf{w}_{1,r}
\end{equation*}
In altre parole, $\forall j,\ \mathbf{w}_{1,j}$ è $\mathbf{0}$ oppure un autovettore di $A$ e, per ipotesi, anche di $B$. Procediamo allo stesso modo tutti i vettori della base $\mathbf{v}_1, \ldots,\ \mathbf{v}_n$: si ha che $\mathbf{v}_i=\mathbf{w}_{i,1}+\ldots+\mathbf{w}_{i,r}$ e $\forall i, j,\ $ $\mathbf{w}_{i,j}$ è $\mathbf{0}$ oppure un autovettore comune di $A$ e $B$.\\
Otteniamo un insieme $\left\{\mathbf{w}_{i, j}\right\}$ di autovettori comuni di $A$ e $B$. Per costruzione, lo \textit{span lineare} dei $\left\{\mathbf{w}_{i, j}\right\}$ contiene $\left\{\mathbf{v}_1,\ \ldots,\ \mathbf{v}_n\right\}$ e pertanto è necessariamente pari a $V$!\\
In altre parole, $\left\{\mathbf{w}_{i,\ j}\right\}$ è un sistema di generatori di $V$ e possiamo estrarre da esso una base di $V$ costituita di autovettori comuni ad $A$ e a $B$.
\end{demonstration}
\section{Polinomi e matrici}
\begin{remember}
	Dato un campo $\kamp$, indichiamo con $\kamp\left[t\right]$ l'anello dei polinomi a coefficienti in $\kamp$ nella variabile $t$; un suo elemento $f\left(t\right)\in\kamp\left[t\right]$ è della forma:
	\begin{equation}
		f\left(t\right)=b_nt^n+b_{n-1}t^{n-1}+\ldots+b_1t+b_0\text{ con }b_i\in\kamp
	\end{equation}
\vspace{-6mm}
\end{remember}
Finora abbiamo sempre \textit{valutato} i polinomi in valori del campo $\kamp$. Possiamo invece valutarli in una \textit{matrice} in $\kamp^{n, n}$? Dopotutto, la \textit{somma di matrici} e la \textit{moltiplicazione per uno scalare} sono operazioni \textit{interne} a $\kamp^{n, n}$ e pertanto potremmo pensare che sia lecito.\\ Tuttavia, presi i polinomi $p$ così come sono, non sarebbe \textit{ben definita} $p\left(A\right)$ a causa del \textbf{termine noto}; infatti, non possiamo sommare uno \textit{scalare ad una matrice}! Per ovviare a questo problema, quando valutiamo un polinomio in una matrice $A$ ‘‘\textit{correggiamo}'' il termine noto con la \textbf{matrice identità} $I$:
\begin{equation}
	f\left(A\right)\coloneqq b_nA^n+b_{n-1}A^{n-1}+\ldots+b_1A+b_0I\text{ con }b_i\in\kamp
\end{equation}
In questo modo, $f\left(A\right)\in\kamp^{n, n}$.
\begin{example}
Preso $f\left(t\right)=t^2-3$, il polinomio valutato nella matrice $A$ è $f\left(A\right)=A^2-3I$.
\end{example}
\begin{observe}
	Dati $f,\ g\in\kamp\left[t\right]$ e $A\in\kamp^{n,\ n}$, si ha:
	\begin{itemize}
		\item $\left(f+g\right)\left(A\right)=f\left(A\right)+g\left(A\right)$.
		\item $\left(fg\right)\left(A\right)=f\left(A\right)g\left(A\right)$
	\end{itemize}
\vspace{-3mm}
\end{observe}
\begin{demonstration}
	Prendiamo i polinomi $f\left(t\right)=b_nt^n+b_{n-1}t^{n-1}+\ldots+b_1t+b_0$ e $g\left(t\right)=c_nt^n+c_{n-1}t^{n-1}+\ldots+c_1t+c_0$ e valutiamoli entrambi in $A$: $f\left(A\right)=b_nA^n+b_{n-1}A^{n-1}+\ldots+b_1A+b_0I$ e $g\left(A\right)=c_nA^n+c_{n-1}A^{n-1}+\ldots+c_1A+c_0I$.
	\begin{enumerate}[label=\Roman*]
		\item La somma è ovvia.
		\item Il prodotto è garantito dalla commutatività delle potenze di matrici.
	\end{enumerate}
\vspace{-3mm}
\end{demonstration}
\subsection{Ideale di una matrice}
\begin{define}\textsc{Ideale di una matrice.}\\
	Data $A\in\kamp^{n,\ n}$, definiamo l'\textbf{ideale della matrice}\index{ideale!di una matrice}:
	\begin{equation}
		I_A\coloneqq\left\{f\in\kamp\left[t\right]\mid f\left(A\right)=O\right\}
	\end{equation}
\vspace{-6mm}
\end{define}
\begin{observe}~{}
	\begin{itemize}
		\item $O\in I_A$.
		\item $I_A\neq\left\{O\right\}$; infatti, se consideriamo le seguenti $n^2+1$ matrici in $\kamp^{n,\ n}$:
		\begin{equation*}
			I,\ A,\ A^2,\ A^3,\ \ldots,\ A^{n^2}
		\end{equation*}
	Per il lemma di Steinitz queste matrici sono necessariamente \textit{linearmente dipendenti}, dato che superano in numero $\dim \kamp^{n,\ n}=n^2$, cioè esistono i coefficienti $a_0,\ \ldots,\ a_{n^2}\in\kamp$ \textit{non} tutti nulli tali che:
	\begin{equation*}
		a_0I+a_1A+a_2A^2+\ldots+a_{n^2}A^{n^2}=O
	\end{equation*}
	Allora $p\left(t\right)=a_{n^2}t^{n^2}+\ldots+a_2t^2+a_1t+a_0$ è un polinomio \textit{non} nullo in $I_A$.
	\item $I_A$ soddisfa giustamente la definizione di ideale di $\kamp\left[t\right]$:
	\begin{itemize}
		\item \textsc{$I_A$ è un sottogruppo di $\left(\kamp^{n, n},\ +\right)$}.
		\begin{equation*}
			\left(f+g\right)\left(A\right)=f\left(A\right)+g\left(A\right)=0\implies f+g\in I_A
		\end{equation*}
		\item \textsc{Assorbimento}: se $h\in\kamp\left[t\right]$ si ha :
		\begin{equation*}
			\left(fh\right)\left(A\right)=\underbrace{f\left(A\right)}_{=O}h\left(A\right)=O\implies fh\in I_A
		\end{equation*}
	\end{itemize}
	\end{itemize}
\vspace{-3mm}
\end{observe}
\subsection{Polinomio minimo}
\begin{proposition}
	L'anello $\kamp\left[t\right]$ è ad \textit{ideali principali}: se $I\subseteq \kamp\left[t\right]$ è un ideale, $\exists p$ tale che $I=\left(p\right)$. Il generatore $p$ è \textit{unico} a meno di moltiplicazione per scalari \textit{non} nulli, se prendiamo $P$ \textbf{monico} allora è unico.
\end{proposition}
\begin{demonstration}
Sia $p\in I$ un polinomio \textit{non} nullo di grado \textit{minimo} tra i polinomi in $I_A$.\\
Se $p$ è \textit{costante}, allora $I=\kamp\left[t\right]=\left(1\right)=\left(p\right)$.\\
Supponiamo allora $p$ \textit{non} costante. Vogliamo mostrare che $p$ genera $I$.
Prendiamo $f\in I$ e dividiamolo per $p$:
\begin{equation*}
	\underbrace{f\left(t\right)}_{\in I}=\underbrace{p\left(t\right)q\left(t\right)}_{\in I\text{ per assorbimento}}+r\left(t\right)
\end{equation*}
Con $r\left(t\right)$ polinomio con $\deg r < \deg p$. Notiamo che anche $r\left(t\right)\in I$ per essere vera l'equazione di sopra; in particolare, per la minimalità del grado di $p$ non può esserci un polinomio in $I$ di grado minore di $p$, dunque $r\equiv 0$. Allora $p\mid f$ e dunque ogni polinomio in $I$ è generato da $p$: $I\equiv\left(p\right)$.\\
Se $I=\left(p\right)=\left(\widetilde{p}\right)$, allora $p,\ \widetilde{p}\in I$ e dunque $p\mid \widetilde{p}$, $\widetilde{p}\mid p$, cioè $p=\lambda\widetilde{p}$ con $\lambda\in\kamp\setminus\left\{0\right\}$. Se $p$ è \textit{monico}, l'unico coefficiente $\lambda$ per cui si ha $p=\lambda\widetilde{p}$ è $1$, e dunque $p=\widetilde{p}$, cioè $p$ è unico.
\end{demonstration}
\begin{define}\textsc{Polinomio minimo.}\\
	Sia $A\in\kamp^{n, n}$ e sia $I_A$ ideale dei polinomi che si annullano in $A$. Il \textbf{polinomio minimo}\index{polinomio!minimo} $m_A\left(t\right)$ di $A$ è il  \textit{generatore monico} di $I_A$, ovvero è il polinomio monico \textit{non} nullo di grado minimo tra i polinomi in $I_A$.
\end{define}
\begin{example} Cerchiamo il polinomio minimo della seguente matrice.
	\begin{equation*}
		A=\left(\begin{array}{cc}
			0 & 1 \\
			1 & 0
		\end{array}\right)\quad A^2=\left(\begin{array}{cc}
		0 & 1 \\
		1 & 0
	\end{array}\right)\left(\begin{array}{cc}
	0 & 1 \\
	1 & 0
\end{array}\right)=\left(\begin{array}{cc}
1 & 0 \\
0 & 1
\end{array}\right)=I
	\end{equation*}
Notiamo che $A^2-I=O$, dunque $p\left(t\right)=t^2-1\in I_A$. Poiché $m_A\left(t\right)\mid p\left(t\right)$, esso può essere solo $t-1$, $t+1$, $t^2-1$. Escludiamo sempre il caso $m_A\left(t\right)=1$, in quanto allora si avrebbe $I_A=\kamp\left[t\right]$ e ciò non è mai vero per questo anello (i polinomi di grado $0$ non si annullano in generale sulle matrici!).\\
Se fosse $m_A\left(t\right)=t-1$, allora $m_A\left(A\right)=A-I\neq O$ e dunque $t-1\notin I_A$. In modo analogo $m_A\left(t\right)=t+1\implies A+I\neq 0\implies t-1\notin I_A$. L'unica possibilità è allora $m_A\left(t\right)=t^2-1$.
\end{example}
\begin{observe}
	Se $A$ e $B$ sono simili, allora $I_A=I_B\subseteq \kamp\left[t\right]$ e quindi $m_A\left(t\right)=m_B\left(t\right)$.
\end{observe}
\begin{demonstration}
	Sia $M\in\gl\left(n,\ \kamp\right)$ la matrice che rende $A$ simile a $B$: $B=M^{-1}AM$. Le potenze di matrici simili sono simili anch'esse:
	\begin{equation*}
		\begin{array}{l}
			B=M^{-1}AM\\
			B^2=M^{-1}A^2M\\
			\ldots\\
			B^k=M^{-1}A^kM
		\end{array}
	\end{equation*}
Se $p\left(t\right)=c_dt^d+\ldots+c_0$, allora:
\begin{equation*}
	\begin{array}{ll}
		M^{-1}p\left(A\right)M&=M^{-1}\left(c_dA^d+\ldots+c_0I\right)M=c_d\left(M^{-1}A^dM\right)+\ldots+c_1\left(M^{-1}AM\right)+c_0I=\\
		&=c_dB^d+\ldots+c_1B+c_0I=p\left(B\right)
	\end{array}
\end{equation*}
Ovvero $M^{-1}p\left(A\right)M=\left(B\right)$. Pertanto, $p\left(A\right)=O$ se e solo se $p\left(B\right)=O$, cioè se $I_A=I_B$.
\end{demonstration}
\section{Teorema di Cayley-Hamilton}
\begin{remember}
	Ricordiamo alcune definizioni e proprietà utili legate al \textbf{determinante}:
	\begin{itemize}
		\item Il \textbf{complemento algebrico}\index{completamento algebrico} $\left(i,\ j\right)$ di una matrice quadrata $M$ è:
		\begin{equation}
			M_{i,j}=\left(-1\right)^{i+j}\det\left(\begin{array}{c}
				\text{matrice ottenuta da }M \text{ cancellando}\\
				\text{la riga }i\text{ e la colonna }j
			\end{array}\right)
		\end{equation}
	\item La \textbf{matrice aggiunta}\index{matrice!aggiunta} $\mathrm{adj}\left(M\right)$ di una matrice quadrata $M$ è la matrice $\left(M_{i,j}\right)$ che, al posto $\left(i,\ j\right)$ ha il complemento algebrico $M_{j,\ i}$.\footnote{Attenzione all'ordine degli indici!}
	\item La \textbf{regola di Laplace}\index{regola!di Laplace} afferma che:
	\begin{equation}
		\mathrm{adj}\left(M\right)M=M\mathrm{adj}\left(M\right)=\det\left(M\right)I
	\end{equation}
	Inoltre, se $\det\left(M\right)\neq 0$, allora:
	\begin{equation}
		M^{-1}=\frac{1}{\det M}\mathrm{adj}\left(M\right)
	\end{equation}
\item Il \textbf{polinomio caratteristico}\index{polinomio!caratteristico} di $A$ è:
\begin{equation}
	C_A\left(t\right)=\det\left(tI-A\right)=\left(-1\right)^n\det\left(A-tI\right)
\end{equation}
In particolare, il polinomio caratteristico è un polinomio \textit{monico}.
	\end{itemize}
\vspace{-3mm}
\end{remember}
\begin{observe}\label{polinomicoeffmatrisci}
Una matrice i cui elementi sono polinomi in $\kamp\left[t\right]$ può essere scritta in modo unico come polinomio in $t$ con coefficienti delle matrici in $\kamp^{n,\ n}$.
\end{observe}
\begin{example}~{}
	\begin{equation*}
		\left(\begin{array}{cc}
			2t^2 & 3t+1 \\
			t^2-4t & 0
		\end{array}\right)=\left(\begin{array}{cc}
		2 & 0 \\
		1 & 0
	\end{array}\right)t^2+\left(\begin{array}{cc}
	0 & 3 \\
	-4 & 0
\end{array}\right)t+\left(\begin{array}{cc}
0 & 1 \\
0 & 0
\end{array}\right)
	\end{equation*}
\end{example}
\begin{theorema}\textsc{Teorema di Cayley-Hamilton}.\index{teorema!di Cayley-Hamilton}\\
Sia $A\in \mathbb{K}^{n,\ n}$. Allora $C_A\left(t\right)\in I_A$, cioè $C_A\left(A\right)=0$. In altre parole, $m_A\left(t\right)\mid C_A\left(t\right)$.\\
In particolare $\deg m_A\left(t\right)\leq n$.
\end{theorema}
\begin{demonstration}
	Sia $M\coloneqq tI-A$. Allora:
	\begin{equation*}
		C_A\left(t\right)=\det M=t^n+b_{n-1}t^{n-1}+\ldots+b_1t+b_0
	\end{equation*}
Consideriamo l'\textit{aggiunta} di $M$, $\mathrm{adj}\left(M\right)=\mathrm{adj}\left(tI-A\right)$.
Poiché gli elementi di $tI-A$ sono polinomi in $\kamp\left[t\right]$ di grado minore o uguale di $1$ e gli elementi di $\mathrm{adj}\left(M\right)$ sono polinomi in $\kamp\left[t\right]$ di grado minore o uguale di $n-1$, $\mathrm{adj}\left(M\right)$ si scrive come polinomio di grado minore e uguale a $n-1$ con coefficienti in $\kamp^{n, n}$ (grazie all'osservazione di pag. \pageref{polinomicoeffmatrisci}).
\begin{equation*}
	\mathrm{adj}\left(M\right)=C_{n-1}t^{n-1}+C_{n-2}t^{n-2}+\ldots+C_1t+C_0,\ C_i\in\kamp^{n,\ n}
\end{equation*}
Usando la regola di Laplace:
\begin{equation*}
	\begin{array}{ccc}
		M\mathrm{adj}\left(M\right)=\det\left(M\right)I&&\\
		\Downarrow&&\\
		\left(tI-A\right)\mathrm{adj}\left(M\right)=C_A\left(t\right)I&=&\left(t^n+b_{n-1}t^{n-1}+\ldots+b_1t+b_0\right)I\\
		\shortparallel &=& \ovaled{It^n+b_{n-1}It^{n-1}+\ldots+b_1It+b_0I}\\
		\left(tI-A\right)\left(C_{n-1}t^{n-1}+C_{n-2}t^{n-2}+\ldots+C_1t+C_0\right)&&\\
		={\tikz[baseline=(char.base)]\node[anchor=south west, draw,rectangle, rounded corners, inner sep=2pt, minimum size=7mm,
		text height=6mm](char){$\begin{array}{l}
				{\small C_{n-1}t^n+C_{n-2}t^{n-1}+\ldots+C_1t^2+C_0t+}\\
				{\small-AC_{n-1}t^{n-1}-AC_{n-2}t^{n-2}+\ldots-AC_1t+AC_0}
			\end{array}$} ;}&&
	\end{array}
\end{equation*}
Uguagliamo i due termini evidenziati, sommando le matrici coefficienti termine a termine. Si ha il sistema:
\begin{equation*}
	\begin{cases}
		\begin{array}{ll}
			C_{n-1}=I&\colon t^n\\
			C_{n-2}-AC_{n-1}=b_{n-1}I&\colon t^{n-1}\\
			C_{n-3}-AC_{n-2}=b_{n-2}I&\colon t^{n-2}\\
			\ldots&\\
			C_{0}-AC_{1}=b_{1}I&\colon t\\
			-AC_0=b_0I&\colon 1\\
		\end{array}
	\end{cases}
\end{equation*}
Sostituiamo a cascata le equazioni dalla seconda in giù:
\begin{equation*}
	\begin{cases}
		\begin{array}{ll}
			C_{n-2}=A+b_{n-1}I\\
			C_{n-3}=A^2+b_{n-1}A+b_{n-2}I\\
			\ldots\\
			C_0=A^{n-1}+b_{n-1}A^{n-2}+\ldots+b_1I
		\end{array}
	\end{cases}
\end{equation*}
Sostituiamo $C_0$ nell'ultima:
\begin{equation*}
	\underbrace{A^n+b_{n-1}A^{n-1}+\ldots+b_1A+b_0I}_{C_A\left(A\right)}=O
\end{equation*}
Abbiamo dunque ottenuto la tesi.
\end{demonstration}
\begin{observe}
	Si ha che $m_A\left(t\right)\mid C_A\left(t\right)\implies C_A\left(t\right)=m_A\left(t\right)q\left(t\right)$, con $q\left(t\right)\in\kamp\left[t\right]$. In altre parole, le \textit{radici} del \textit{polinomio minimo} sono \textit{autovalori}.
\end{observe}
Il seguente teorema afferma un legame ancora più forte tra polinomio minimo e autovalori di una matrice.
\begin{theorema}
	Sia $A\in \mathbb{K}^{n,\ n}$ e $m_A\left(t\right)$ il suo polinomio minimo. Allora, preso $\lambda\in\mathbb{K}$:
	\begin{equation}
		m_A\left(\lambda\right)=0\iff \lambda\text{ è un autovalore di }A
	\end{equation}
\vspace{-6mm}
\end{theorema}
\begin{demonstration}~{}\\
	$\impliesdx$ Segue dal teorema di Cayley-Hamilton perché $m_A\left(\lambda\right)=0\implies C_A\left(\lambda\right)=0\implies \lambda$ autovalore.\\
	$\impliessx$ Sia $\lambda$ un autovalore di $A$ con autovettore associato $\underline{v}$. Si ha:
	\begin{gather*}
		A\underline{v}=\lambda \underline{v}\\
		A^2\underline{v}=A\left(A\underline{v}\right)=A\left(\lambda \underline{v}\right)=\lambda A\underline{v}=\lambda^2 \underline{v}\\
	\end{gather*}
Allo stesso modo si arriva a $A^k\underline{v}=\lambda^n\underline{v}$. Preso un generico polinomio $p\left(t\right)\in\mathbb{K}\left[t\right]$, esso si può esprimere come:
\begin{equation*}
	p=\sum_{i=0}^{d}c_i t_i\quad c_i\in\mathbb{K}
\end{equation*}
Allora $\displaystyle p\left(A\right)=\sum_{i=0}^{d}c_i A^i$ e dunque:
\begin{align*}
	p\left(A\right)\underline{v}&=\left(\sum_{i=0}^{d}c_i A^i\right)\underline{v}=\sum_{i=0}^{d}c_i\left( A^i\underline{v}\right)=\sum_{i=0}^{d}c_i\left( \lambda^i\underline{v}\right)=\underbrace{\left(\sum_{i=0}^{d}c_i \lambda^i\right)}_{\in\mathbb{K}}\underline{v}=p\left(\lambda\right)\underline{v}
\end{align*}
Consideriamo ora un polinomio $p\in I_A$. Per sua definizione $p\left(A\right)=0$; in particolare, da quanto scritto sopra:
\begin{equation*}
	O\underline{v}=p\left(\lambda\right)\underline{v}
\end{equation*}
Ed essendo $v$ un autovettore, $v\neq 0$; dall'equazione sopra necessariamente segue $p\left(\lambda\right)=0$. In particolare, essendo $p\in I_A$ generato dal polinomio minimo $m_A$ (cioè $p\left(t\right)=m_A\left(t\right)q\left(t\right)$ con $q\left(t\right)\neq 0$), segue che $m_A\left(\lambda\right)=0$.
\end{demonstration}
\section{Forma canonica di Jordan}
D'ora in poi, se non altresì specificato, considereremo $\mathbb{K}=\complexset$, cioè tratteremo di matrici $A\in \complexset^{n,\ n}$ e endomorfismi fra spazi vettoriali complessi.
\begin{observe}
Poichè $\complexset$ è algebricamente chiuso, ogni polinomio $p\in\complexset\left[t\right]$ si fattorizza completamente come prodotto di fattori lineari:
\begin{equation}
	C_A\left(t\right)=\left(t-\lambda_1\right)^{m_1}\ldots\left(t-\lambda_r\right)^{m_r}\text{ con } m_i \text{ molteplicità algebrica di } \lambda_i
\end{equation}
Nel caso del polinomio minimo, si ha:
Poichè $\complexset$ è algebricamente chiuso, ogni polinomio $p\in\complexset\left[t\right]$ si fattorizza completamente come prodotto di fattori lineari:
\begin{equation}
	m_A\left(t\right)=\left(t-\lambda_1\right)^{h_1}\ldots\left(t-\lambda_r\right)^{h_r}\text{ con } 1\leq h_i\leq m_i \forall i=1,\ldots,\ r
\end{equation}
\end{observe}
Sia $A\in \complexset^{n,\ n}$ una matrice associata a un endomorfismo $\funz{f}{V}{V}$. Se $f$ è diagonalizzabile, esiste una base in cui la matrice di $f$ è diagonale. Anche quando tuttavia la matrice non è diagonalizzabile, vogliamo cercare una base in cui la matrice di $f$ è \textit{particolarmente semplice}.
\begin{define}
	Un \textbf{blocco di Jordan}\index{blocco di Jordan} $J=J_k\left(\lambda\right)$, di autovalore $\lambda\in\complexset$ e dimensione $K$, è una matrice quadrata $k\times k$ con sulla diagonale solo l'autovalore e sopra ogni elemento della diagonale $1$:
	\begin{equation}%\setlength\arraycolsep{0.5mm}
		    J=J_k\left(\lambda\right) = \left(
		\begin{array}{ccccc}
\lambda	& 1 		&  0		& \ldots 	& 0 \\
0		& \lambda 	& \ddots	& 			& \vdots\\
\vdots	&  			& \ddots	& 1 		& 0\\
\vdots	& 			&   		& \lambda 	& 1\\
0		&  \dots  	&  \dots 	&  0 		& \lambda
		\end{array}
		\right)
	\end{equation}
\end{define}
\begin{observe}~{}
	\begin{itemize}
		\item $J$ è determinato da $\lambda$ e $k$.
		\item Il polinomio caratteristico di $J$ è $C_J\left(t\right)=\left(t-\lambda\right)^k$, cioè $\lambda$ è l'unico autovalore di $J$ con molteplicità algebrica $k$.
	\end{itemize}	
\end{observe}
\begin{observe}
Definiamo il blocco di Jordan di dimensione $k$ con autovalore zero, necessario per calcolare l'autospazio $V_\lambda$:
		\begin{equation}\setlength\arraycolsep{0.5mm}
			N=J-\lambda I= \left(
				\begin{array}{ccccc}
				0	& 1 		&  0		& \ldots 	& 0 \\
				0		& 0 	& \ddots	& 			& \vdots\\
				\vdots	&  			& \ddots	& 1 		& 0\\
				\vdots	& 			&   		& 0 	& 1\\
				0		&  \dots  	&  \dots 	&  0 		& 0
			\end{array}
			\right)
		\end{equation}
Si ha che $\rk N=k-1\implies \dim V_{\lambda}=\dim \ker N=k-\rk N= 1$, cioè $J$ \textit{non} è \textit{mai} diagonalizzabile se $k>1$, dato che $1=\dim V_{\lambda}\leq m_\lambda = k$.\\
Se la base $\basis$ dello spazio $V$ (in cui stiamo operando con l'endomorfismo associato a $J$) è $\left\{\underline{e}_1,\ \ldots,\ \underline{e}_k\right\}$, notiamo che $\underline{e}_1$ è l'unico autovettore di $N$ e $V_\lambda=\mathcal{L}\left(\underline{e}_1\right)$. Si vede che $J$ agisce in modo particolare sui vettori di $\basis$:
\begin{equation*}
\begin{cases}
J\underline{e}_1=\lambda \underline{e}_1\\
J\underline{e}_2=\underline{e}_1+\lambda \underline{e}_2\\
\ldots\\
J\underline{e}_k=\underline{e}_{k-1}+\lambda \underline{e}_k
\end{cases}
\end{equation*}
Anche $N$ agisce in modo altrettanto particolare sui vettori di $\basis$:
\begin{equation*}
	\begin{cases}
		N\underline{e}_1=\underline{0}\\
		N\underline{e}_2=\underline{e}_1\\
		\ldots\\
		N\underline{e}_k=\underline{e}_{k-1}
	\end{cases}
\end{equation*}
Cioè, cominciando da $\underline{e}_k$ e applicando $N$ ripetutamente otteniamo gli altri vettori della base.
\begin{center}
	\begin{tikzcd}
		\underline{e}_{1} & \underline{e}_{2} \arrow[l, "N", bend left] & \dots \arrow[l, "N", bend left] & \underline{e}_{k-1} \arrow[l, "N", bend left] & \underline{e}_k \arrow[l, "N", bend left]
	\end{tikzcd}
\end{center}
Ad esempio, con $N^2$ si ha:
\begin{equation*}
	\begin{cases}
		N^2\underline{e}_1=\underline{0}\\
		N^2\underline{e}_2=N\left(N\underline{e}_2\right)=N\underline{e}_1=\underline{0}\\
		\ldots\\
		N^2\underline{e}_k=N\left(N\underline{e}_k\right)=N\underline{e}_{k-1}=\underline{0}
	\end{cases}
\end{equation*}
Infatti, se guardiamo la matrice $N^2$, si ha:
		\begin{equation*}
	N^2=\left(J-\lambda I\right)^2= \left(
	\begin{array}{ccccc}
		0		& 0 		&  1		& \ldots 	& 0 \\
		\vdots	& \ddots 	& 0			& 1			& \vdots\\
				&  			& \ddots	& 1 		& 0\\
		\vdots	& 			&   		& 0 		& 1\\
		0		&  \dots  	&  \dots 	& \dots 		& 0
	\end{array}
	\right)
\end{equation*}
Si ha dunque, ad ogni potenza successiva di $N$, lo ‘‘spostamento'' della diagonale di $1$ verso destra. In particolare:
		\begin{equation*}
	N^{k-1}=\left(J-\lambda I\right)^{k-1}= \left(
	\begin{array}{ccccc}
		0		& \dots 	&  	\dots		& 0 	& 1 \\
		\vdots	& \ddots 	& 			& 		& 0\\
				&  			& 			&  		& \vdots \\
		\vdots	& 			&   		& \ddots 		& \vdots\\
		0		&  \dots  	&  \dots 	& \dots & 0
	\end{array}
	\right)
\end{equation*}
E in questo caso si ha la relazione con i vettori della base:
\begin{equation*}
	\begin{cases}
		N^{k-1}\underline{e}_i=\underline{0}\ \forall i=1,\ldots,\ k-1\\
		\ldots\\
		N^{k-1}\underline{e}_k=\underline{e}_1
	\end{cases}
\end{equation*}
Studiando l'immagine dell'applicazione associata ad $N$, essendo la base dell'immagine i vettori colonna l.i., si ha $\Im N^{k-1}=\mathcal{L}\left(e_1\right)$.\\
Come già affermato dunque, è $\underline{e}_k$ a determinare l'\textit{intera} base di $V$ tramite la moltiplicazione per $N$.\\
Come ultima osservazione fondamentale, notiamo inoltre che $N^k=O$, cioè $N$ è una matrice \textbf{nilpotente}\index{matrice!nilpotente} di ordine $k$.
\end{observe}
\begin{define}
	Una matrice quadrata si dice in \textbf{forma di Jordan}\index{forma di Jordan} se ha solo blocchi di Jordan lungo la diagonale, mentre altrove è nulla.
\end{define}
\begin{example}
La seguente matrice $9\times 9$  è in forma di Jordan, con blocchi $J_3\left(2\right)$, $J_2\left(i\right)$, $J_3\left(i\right)$ e $J_1\left(-4\right)$:
	\begin{equation*}
		A=
		\tikz[baseline]{%
			\node[matrix of math nodes,matrix anchor=west,left delimiter=(,right delimiter=),ampersand replacement=\&] (M) {%
				2 	\& 1 	\& 0	\& \color{gray}{0}	\&	\color{gray}{0} 	\&	\color{gray}{0} 	\&	\color{gray}{0} 	\&	\color{gray}{0} 	\&	\color{gray}{0} 	\\
				0	\& 2 	\& 1 	\& \color{gray}{0}	\&	\color{gray}{0}	\&	 \color{gray}{0}	\&	\color{gray}{0} 	\&	\color{gray}{0} 	\&	\color{gray}{0} 	\\
				0	\& 0 	\& 2 	\& \color{gray}{0}	\& \color{gray}{0}	\&	\color{gray}{0}	\&	\color{gray}{0} 	\&	\color{gray}{0} 	\&	\color{gray}{0} 	\\
				\color{gray}{0}	\& \color{gray}{0} 	\& \color{gray}{0}  	\& i	\& 1	\&	\color{gray}{0}	\&	\color{gray}{0} 	\&	\color{gray}{0} 	\&	\color{gray}{0} 	\\
				\color{gray}{0}	\& \color{gray}{0} 	\& \color{gray}{0} 	\& 0	\& i	\&	\color{gray}{0}	\&	\color{gray}{0}	\&	\color{gray}{0}	\&	\color{gray}{0}	\\
				\color{gray}{0}	\&  \color{gray}{0}	\& \color{gray}{0} 	\& \color{gray}{0}	\& \color{gray}{0}	\&	i	\&	1	\&	0	\&	\color{gray}{0}	\\
				\color{gray}{0}	\& \color{gray}{0} 	\& \color{gray}{0}  	\& 	\color{gray}{0}	\& \color{gray}{0}	\&	0	\&	i	\&	1 	\&	\color{gray}{0}	\\
				\color{gray}{0}	\&  \color{gray}{0}	\&  \color{gray}{0} 	\& \color{gray}{0}		\& \color{gray}{0}	\&	0	\&	0	\&	i 	\&	\color{gray}{0}	\\
				\color{gray}{0}	\& \color{gray}{0} 	\&  \color{gray}{0} 	\& \color{gray}{0}		\& \color{gray}{0}	\&	\color{gray}{0}	\&	\color{gray}{0}	\&	\color{gray}{0} 	\&	-4	\\				
			};
			\draw (M-1-1.north west) rectangle (M-3-3.south east);
			\draw (M-3-3.south east) rectangle (M-5-5.south east);
			\draw (M-5-5.south east) rectangle (M-8-8.south east);
			\draw (M-8-8.south east) rectangle (M-9-9.south east);
		}
	\end{equation*}
\end{example}

\begin{observe}
	Una matrice \textit{diagonale} è in forma di Jordan, con un unico blocco di ordine $1$ (cioè senza alcun $1$ nell'elemento sopra).
\end{observe}
\begin{observe}
	Se $A$ è in forma di Jordan, sulla diagonale compaiono tutti gli autovalori con la loro \textit{molteplicità}. Dunque, se $\lambda$ è un autovalore, la somma delle \textit{dimensioni} dei blocchi relativi a $\lambda$ è uguale alla \textit{molteplicità algebrica} $m_\lambda$ di $\lambda$.
\end{observe}
\begin{theorema}
	Sia $V$ uno spazio vettoriale complesso di $\dim n$ e $f$ un endomorfismo di $V$. Allora esiste una base di $V$ in cui la matrice di $f$ è in forma di Jordan. Inoltre, la forma di Jordan è unica a meno dell'ordine dei blocchi.\\
	\textit{In termini matriciali}, ogni $A\in\complexset^{n,\ n}$ è simile ad una matrice in forma di Jordan, unica a meno dell'ordine dei blocchi.
\end{theorema}
Per agevolare la dimostrazione, faremo uso di un lemma importante. Tuttavia, prima di dimostrarlo, facciamo alcune osservazioni per comprendere meglio ciò di cui parliamo.