% SVN info for this file
\svnidlong
{$HeadURL$}
{$LastChangedDate$}
{$LastChangedRevision$}
{$LastChangedBy$}

\chapter{Connessione e compattezza}
\labelChapter{Connessocompatto}

\begin{introduction}
‘‘\emph{Lisa}: Allora, dov'è mio padre?\\
\emph{Professor Frink}: Beh, sarebbe ovvio anche per l'individuo più scriteriato, laureato e con specializzazione in Topologia iperbolica, che Homer Simpson è piombato... nella terza dimensione. [...] \emph{\textbf{[disegna sulla lavagna]}} Ecco un comunissimo quadrato—\\
\emph{Commissario Winchester}: Ehi ehi, rallenta, capuepopolo!\\
\emph{Professor Frink}: —ma supponiamo di estendere il quadrato oltre le due dimensioni del nostro universo, lungo l'ipotetica asse z, qui. \emph{\textbf{[tutti sussultano]}} Così si ottiene un oggetto tridimensionale noto come “cubo”, o meglio “Frinkaedro”, in onore dello scopritore!
\begin{flushright}
	\textsc{I Simpson,} La paura fa novanta VI.
\end{flushright}
\end{introduction}
\lettrine[findent=1pt, nindent=0pt]{L}{e} due proprietà che danno il nome a questo capitolo sono estremamente importanti, in quanto sono due dei principali \textit{invarianti} studiati in topologia. Entrambe rappresentano una \textit{generalizzazione} di alcuni aspetti affrontati più o meno esplicitamente durante lo studio dell'Analisi:
\begin{itemize}
	\item Ci sono sottoinsiemi del piano i cui punti possono essere \textit{connessi} da una linea arzigogolata, una spezzata o un segmento, mente
	\item Ci sono sottoinsiemi \textit{limitati} le cui successioni di punti \textit{convergono} nel sottoinsieme.
\end{itemize}
Vedremo che la \textbf{connessione} e la \textbf{compattezza} sono definite in modo abbastanza basilare, seppur non necessariamente siano intuitive a primo acchito. Tuttavia, proprio in virtù di questa semplicità, sono applicabili in tanti contesti diversi; in particolare, la compattezza come la definiremo ci permetterà di prendere informazioni note \textit{localmente} ed estenderle in modo che valgano globalmente in tutto lo spazio.
\section{Connessione}
\begin{define}[Spazio connesso e spazio sconnesso.]~{}\\
Uno spazio topologico $X$ si dice \textbf{connesso}\index{spazio!connesso} se gli unici sottoinsiemi aperti e chiusi sono $\emptyset,\ X$.\\
Uno spazio non \textit{connesso} si dice \textbf{sconnesso} oppure \textbf{non connesso}.
\end{define}
\begin{lemming}[Condizioni equivalenti della sconnessione; Manetti, 4.2.]~{}\label{sconnesso}\\
Sono condizioni equivalenti:
\begin{enumerate}
	\item $X$ è \textit{sconnesso}.
	\item $X=A\cup B$ con $A,\ B$ aperti, non vuoti, disgiunti.
	\item $X=A\cup B$ con $A,\ B$ chiusi, non vuoti, disgiunti.
\end{enumerate}
\vspace{-3mm}
\end{lemming}
\begin{demonstration}~{}\\
$2\iff3)$ Sono equivalenti: se $A$ è aperto e disgiunto da $B$ tale che $X=A\cup B$ significa che $B=\mathcal{C}A=X\setminus A$ e dunque chiuso; analogamente per $B$ aperto si ha che $A$ è chiuso: allora $A,\ B$ chiusi e aperti propri.\\
$1\implies2)$ Esiste $\emptyset\subsetneqq A \subsetneqq X$ con $A$ aperto e chiuso. Allora basta porre $B=\mathcal{C}A=X\setminus A$: essendo il complementare di $A$ è aperto e chiuso, sono disgiunti e tali per cui $B\neq X,\ B\neq \emptyset$. $A$ e $B$ soddisfano la tesi.\\
$2\implies1)$ $A$ aperto, $B$ aperto $\implies A$ chiuso perché $A=\mathcal{C}X=X\setminus B$. Inoltre $A$ non vuoto, $B$ non vuoto $\implies A\neq X$. Dunque $A$ è aperto, chiuso e $A\neq \emptyset,\ X$ e pertanto soddisfa la tesi: esiste un sottoinsieme aperto e chiuso che non il vuoto o l'insieme stesso.
\end{demonstration}
\begin{tips}
	Il lemma \ref{sconnesso} (Manetti, 4.2) ci dice che è sufficiente trovare \textit{solo due} aperti (o chiusi) che soddisfano la condizione di cui sopra per affermare la sconnessione. Viceversa, per dimostrare la connessione, dobbiamo dimostrare che \textit{per ogni coppia} di aperti (o chiusi) non vuoti, la cui unione è $X$, essi non siano disgiunti.
\end{tips}
\begin{examples} Esempi di spazi topologici \textit{sconnessi} in topologia Euclidea:
	\begin{itemize}
		\item $X=\realset\setminus\left\{0\right\}=\left(-\infty,\ 0\right)\cup \left(0,\ +\infty\right)$.
		\item $X=\left[0,\ 1\right]\cup \left(2,\ 3\right)$.
	\end{itemize}
\vspace{-3mm}
\end{examples}
\begin{lemming}[Connesso è disgiunto o sottoinsieme di un aperto/chiuso; Manetti, 4.4.]~{}\label{connessodisgiuntoosottoinsieme}\\
Sia $X$ spazio topologico e $A\subseteq X$ con $A$ aperto e chiuso. Sia $Y\subseteq X,\ Y$ \textit{connesso}. Allora $Y\cap A=\emptyset$ (cioè $Y\subseteq Y\setminus A$) oppure $Y\subseteq A$.
\end{lemming}
\begin{demonstration}
Consideriamo $Y\cap A$: esso è intersezione di due aperti e chiusi per ipotesi ($Y$ è aperto e chiuso perché \textit{connesso}), cioè è aperto e chiuso. Essendo $Y$ \textit{connesso}, un suo sottoinsieme aperto e chiuso o è l'insieme vuoto oppure è l'insieme stesso, cioè $Y\cap A=\emptyset$ (cioè $Y\subseteq Y\setminus A$) oppure $Y\cap A=Y$ (cioè $Y\subseteq A)$.
\end{demonstration}
\begin{theorema}[Connessione di ${\left[0,\ 1\right]}$; Manetti, 4.6.]~{}\\
Con la topologia Euclidea, $X=\left[0,\ 1\right]$ è \textit{connesso}.
\end{theorema}
\begin{demonstration}
Supponiamo che $X=\left[0,\ 1\right]=C\cup D$ con $C,\ D$ entrambi chiusi.\\
Dobbiamo dimostrare che $C,\ D$ \textit{non} sono disgiunti, ovvero $C\cap D\neq 0$. Supponiamo sia $0\in C$ e poniamo $d=\inf D$. Essendo $D$ un chiuso, $d\in \overline{D}=D$.
\begin{itemize}
	\item Se $d=0$, $d\in C\cap D\neq \emptyset$.
	\item Se $d>0$ allora $\left[0,\ d\right)\subseteq C$ perché \textit{non sta} in $D$. Il passaggio alla chiusura mantiene l'inclusione, dunque $\left[0,\ d\right]\subseteq \overline{C}=C$. Segue che $d\in C$ e dunque $C\cap D\neq \emptyset$.
\end{itemize}
\vspace{-3mm}
\end{demonstration}
\begin{theorema}[Immagine continua di un connesso è un connesso; Manetti, 4.7.]~{}\\
	L'immagine continua di un \textit{connesso} è un \textit{connesso}:
	\begin{equation}
		\funz{f}{X}{Y}\text{ continua},\ X\text{ connesso}\implies f\left(X\right)\text{ connesso}
	\end{equation}
\vspace{-6mm}
\end{theorema}
\begin{demonstration}
	Sia $Z\subseteq f\left(X\right)$, $Z$ aperto, chiuso in $f\left(X\right)$ non vuoto. Per dimostrare che $f\left(X\right)$ sia connesso ci è sufficiente dimostrare che $Z=f\left(X\right)$: in questo modo gli unici aperti e chiusi sono i sottoinsiemi impropri:
	\begin{itemize}
		\item $Z$ aperto: $\exists A$ aperto in $Y\ \colon Z=A\cap f\left(X\right)$.
		\item $Z$ chiuso: $\exists C$ chiuso in $Y\ \colon Z=C\cap f\left(X\right)$.
	\end{itemize}
Allora:
	\begin{itemize}
	\item $f^{-1}\left(Z\right)=f^{-1}\left(A\right)\cap f^{-1}\left(f\left(X\right)\right)=f^{-1}\left(A\right)\implies f^{-1}\left(Z\right)$ è uguale alla controimmagine continua di un aperto in $Y$, cioè è uguale ad un aperto di $X$.
	\item $f^{-1}\left(Z\right)=f^{-1}\left(C\right)\cap f^{-1}\left(f\left(X\right)\right)=f^{-1}\left(C\right)\implies f^{-1}\left(Z\right)$ è uguale alla controimmagine continua di un chiuso in $Y$, cioè è uguale ad un chiuso di $X$
	\end{itemize}
Segue che $f^{-1}\left(Z\right)$ è aperto e chiuso in $X$. Notiamo inoltre che, essendo $Z\neq \emptyset$, allora $f^{-1}\left(Z\right)\neq \emptyset$: essendo $X$ \textit{connesso} per ipotesi, necessariamente $f^{-1}\left(Z\right)=X$.
\end{demonstration}
\begin{observe}
Dal teorema precedente segue che essere \textit{connesso} è una proprietà topologica! Infatti, se vale per una qualunque funzione continua $\funz{f}{X}{Y}$, allora varrà anche per omeomorfismi tra $X$ e $Y$; in particolare, si avrà per suriettività che $f\left(X\right)=Y$ connesso.
\end{observe}
\subsection{Connessione per archi}
\begin{define}[Arco.]~{}\\
Un \textbf{arco}\seeonlyindex{arco}{cammino} o \textbf{cammino}\index{cammino} $\alpha$ da un punto $x$ a un punto $y$ in uno spazio topologico $X$ è una funzione continua che parametrizza un \textit{percorso} finito fra gli estremi $x$ e $y$:
\begin{equation}
\funz{\alpha}{\left[0,\ 1\right]}{X} \text{ continua}\ \colon \alpha\left(0\right)=x,\ \alpha\left(1\right)=y
\end{equation}
\vspace{-6mm}
\end{define}
\begin{define}[Connessione per archi.]~{}\\
Uno spazio topologico $X$ si dice \textbf{connesso per archi} o \textbf{c.p.a.}\index{spazio!connesso per archi}\seeonlyindex{spazio!c.p.a.}{spazio!connesso per archi} o \textit{path-connected} se per ogni coppia di punti in $X$ esiste un arco che li collega:
\begin{equation}
\forall x,\ y\in X\ \exists \funz{\alpha}{\left[0,\ 1\right]}{X} \text{ continua}\ \colon \alpha\left(0\right)=x,\ \alpha\left(1\right)=y
\end{equation}
\vspace{-6mm}
\end{define}
\begin{theorema}[$X$ c.p.a. implica $X$ connesso; Manetti, 4.7.]
\end{theorema}
\begin{demonstration}
	Sia $X=A\cup B$, con $A,\ B$ aperti non vuoti. Vogliamo dimostrare che $A\cap B\neq \emptyset$. Essendo non vuoti, prendiamo $a\in A,\ b\in B$. In quanto $X$ è \textbf{c.p.a.}, esiste il cammino (continuo) $\funz{\alpha}{\left[0,\ 1\right]}{X}$ tale che $\alpha\left(a\right)=a,\ \alpha\left(1\right)=b$.\\
	Studiamo la controimmagine di $\alpha$:
	\begin{gather*}
		\alpha^{-1}\left(X\right)=\alpha^{-1}\left(A\cup B\right)=\left[0,\ 1\right]\\
		\left[0,\ 1\right]=\alpha^{-1}\left(A\cup B\right)=\alpha^{-1}\left(A\right)\cup \alpha^{-1}\left(B\right)
	\end{gather*}
$\alpha^{-1}\left(A\right),\ \alpha^{-1}\left(B\right)$ sono entrambi aperti e non vuoti in quanto controimmagini (continue) di aperti non vuoti ($0\in \alpha^{-1}\left(A\right),\ 1\in \alpha^{-1}\left(B\right)$).\\
Poiché $\left[0,\ 1\right]$ è connesso, allora le controimmagini trovate \textit{non} sono disgiunte. Segue allora che $\exists t\in \alpha^{-1}\left(A\right)\cap \alpha^{-1}\left(B\right)$ e quindi:
\begin{equation*}
\alpha\left(t\right)\in\alpha\left(\alpha^{-1}\left(A\right)\cap \alpha^{-1}\left(B\right)\right)\subseteq\alpha\left(\alpha^{-1}\left(A\right)\right)\cap\alpha\left(\alpha^{-1}\left(B\right)\right)=A\cap B
\end{equation*}
\end{demonstration}
\vspace{-3mm}
\begin{theorema}[Immagine continua di uno spazio c.p.a. è un c.p.a.]~{}\\
	L'immagine continua di uno spazio \textbf{c.p.a.} è \textbf{c.p.a.}:
	\begin{equation}
		\funz{f}{X}{Y}\text{ continua},\ X\textbf{ c.p.a.}\implies f\left(X\right)\textbf{ c.p.a.}
	\end{equation}
	\vspace{-6mm}
\end{theorema}
\begin{demonstration}
	Considerati $y,\ z\in f\left(X\right)$, vogliamo trovare un cammino tra i due punti. Poiché $y,\ z\in f\left(X\right)$, consideriamo $a,\ b\in X$ tali che $y=f(a)$ e $z=f(b)$. Poichè $a,\ b\in X$ \textbf{c.p.a.}, esiste un cammino $\alpha$ tale per cui $\alpha\left(0\right)=a$ e $\alpha\left(1\right)=b$; componiamolo ora con la funzione $f$.
	\[
	\begin{tikzcd}
		{f\circ \alpha\ \colon \left[0,\ 1\right]} \arrow[r, "\alpha"] & X \arrow[r, "f"] & Y
	\end{tikzcd}
	\]
	La composizione $f\circ \alpha$ è continua perché composizione di funzioni continue, inoltre:
	\begin{gather*}
		\left(f\circ \alpha\right)\left(0\right)=f\left(\alpha\left(0\right)\right)=f\left(a\right)=y\\
		\left(f\circ \alpha\right)\left(1\right)=f\left(\alpha\left(1\right)\right)=f\left(b\right)=z
	\end{gather*}
	Dunque $f\circ \alpha$ è un cammino fra due punti $y,\ z$ arbitrari in $f\left(X\right)$ e pertanto $f\left(X\right)$ è \textbf{c.p.a.}
\end{demonstration}
\begin{observe}
	Essere \textbf{c.p.a.} è invariante topologico.
\end{observe}
\begin{define}[Giunzione di cammini.]~{}\\
	Dati due cammini in uno spazio $X$:
	\begin{gather*}
		\funz{\alpha}{\left[0,\ 1\right]}{X}\quad \alpha\left(0\right)=x,\ \alpha\left(1\right)=y\\
		\funz{\beta}{\left[0,\ 1\right]}{X}\quad \beta\left(0\right)=y,\ \beta\left(1\right)=z
	\end{gather*}
	Allora possiamo creare un cammino $\alpha \ast \beta$ con la \textbf{giunzione di cammini}\index{cammino!giunzione di cammini}:
	\begin{equation}
		\left(\alpha\ast\beta\right)\left(t\right)=\begin{cases}
			\alpha\left(2t\right)\quad\text{se }0\leq t\leq \frac{1}{2}\\
			\beta\left(2t-1\right)\quad\text{se }\frac{1}{2}\leq t\leq 1\\	
		\end{cases}
	\end{equation}
%\vspace{-6mm}
\end{define}
\begin{lemming}[Unione di c.p.a. non disgiunta è c.p.a.]~{}\label{unionecpa}\\
	Sia $A,\ B$ \textbf{c.p.a}, $A\cap B\neq \emptyset\implies A\cup B$ \textbf{c.p.a.}
\end{lemming}
\begin{demonstration}
	Se $x, y\in A$ oppure $x,\ y\in B$ esiste per ipotesi un arco che li collega. Dobbiamo allora trovare un arco in $A\cup B$ da $x$ a $y$ $\forall x\in A, y\in B$. Preso $z\in A\cap B$, per ipotesi esistono due cammini ad esso:
	\begin{gather*}
		\funz{\alpha}{\left[0,\ 1\right]}{A}\quad \alpha\left(0\right)=x,\ \alpha\left(1\right)=z\\
		\funz{\beta}{\left[0,\ 1\right]}{B}\quad \beta\left(0\right)=z,\ \beta\left(1\right)=z	
	\end{gather*}
	Usando la \textit{giunzione di cammini}, si ha:
	\begin{equation}
		\left(\alpha\ast\beta\right)\left(t\right)=\begin{cases}
			\alpha\left(2t\right)\quad\text{se }0\leq t\leq \frac{1}{2}\\
			\beta\left(2t-1\right)\quad\text{se }\frac{1}{2}\leq t\leq 1\\	
		\end{cases}
	\end{equation}
	Il cammino $\funz{\alpha\ast\beta}{\left[0,\ 1\right]}{A\cup B}$ è quello richiesto.
\end{demonstration}
\begin{observes}~{}\label{giunzionecpa}
	\begin{itemize}
		\item Usando la giunzione di cammini, si ha che:
		\begin{equation*}
			X\text{ è \textbf{c.p.a.}}\iff \exists z\in X\ \colon \forall x\in X\quad
			\exists \funz{\alpha}{\left[0,\ 1\right]}{X}\ \colon \alpha\left(0\right)=z,\ \alpha\left(1\right)=x
		\end{equation*}
		In altre parole, uno spazio è \textbf{c.p.a.} se e solo se esiste un punto per cui ogni altro punto è collegato tramite un arco.
		\item Per ogni arco $\alpha$ esiste l'arco inverso, percorso al contrario: $\overline{\alpha}\left(t\right)=\alpha\left(1-t\right)$.
	\end{itemize}
\vspace{-3mm}
\end{observes}
\begin{define}[Segmento.]~{}\label{segmento}\\
In $\realset^n$, un \textbf{segmento}\index{segmento} $\overline{PQ}$ è la combinazione lineare tra i punti $P$ e $Q$, parametrizzato come:
	\begin{equation}
		\overline{PQ}=\left\{P+tQ\mid t\in\left[0,\ 1\right]\right\}
	\end{equation}
\vspace{-6mm}
\end{define}
\begin{define}[Sottoinsieme convesso.]~{}\\
	Un sottoinsieme $Y\subseteq\realset^n$ è \textbf{convesso}\index{sottoinsieme!convesso} se per ogni coppia di punti esiste un segmento che li collega contenuto interamente in $Y$.
	\begin{equation}
		\forall P,\ Q\in Y\quad \overline{PQ}\subseteq Y
	\end{equation}
\vspace{-6mm}
\end{define}
\begin{define}[Sottoinsieme stellato.]~{}\\
	Un sottoinsieme $Y\subseteq\realset^n$ è \textbf{stellato}\index{sottoinsieme!stellato} per $P$ se esiste un $P\in Y$ tale che per ogni altro punto esiste un segmento che li collega contenuto interamente in $Y$.
	\begin{equation}
		\exists P \in Y\ \colon \forall Q\in Y\quad \overline{PQ}\subseteq Y
	\end{equation}
\vspace{-6mm}
\end{define}
\begin{examples}~{}
\begin{itemize}
	\item Gli intervalli aperti e semiaperti sono \textbf{c.p.a}, dunque sono \textit{connessi}: l'arco $\alpha$ è banalmente il segmento pari all'intervallo aperto.
	\item Preso $X\subseteq\realset^n$ \textit{convesso}, qualunque segmento è anche per costruzione un arco: $X$ è anche \textbf{c.p.a} e dunque \textit{connesso}.
	\item $X=\realset^{2}\setminus\left\{0\right\}$ \textit{non} è \textit{convesso} (per $\left(0,\ 1\right)$ e $\left(0,\ -1\right)$ non si hanno segmenti interni ad $X$) ma è \textbf{c.p.a.} (basta prendere un cammino che ‘‘giri attorno'' all'origine) e dunque è \textit{connesso}.
	\item Preso $X\subseteq\realset^n$ \textit{stellato} per $P\in X$, qualunque segmento con $P$ è anche per costruzione un arco: $X$ è anche \textbf{c.p.a} per l'osservazione \ref{giunzionecpa} e dunque {connesso}.
	\item Ogni insieme \textit{convesso} è anche \textit{stellato} per $P$, basta fissare un qualunque punto come nostro $P$. In generale, un insieme è convesso se e solo se è stellato per ogni suo punto.
\end{itemize}
\vspace{-3mm}
\end{examples}
\subsection{Connessione nella topologia euclidea}
Vediamo ora che conseguenze hanno questi teoremi in $\realset$ con la topologia Euclidea.
\begin{theorema}[Condizioni equivalenti della connessione su $\realset$.]~{}\\
	Sia $I\subseteq \realset$. Le seguenti affermazioni sono equivalenti:
		\begin{enumerate}
	\item $I$ è un intervallo, ovvero $I$ è \textit{convesso}.
	\item $I$ è \textbf{c.p.a.}
	\item $I$ è connesso.
		\end{enumerate}
	\vspace{-3mm}
\end{theorema}
\begin{demonstration}~{}\\
	$1) \implies 2)$ Siccome $I$ è convesso $\implies$ $I$ stellato $\implies$ $I$ \textbf{c.p.a.} $\implies$ $I$ connesso. \\
	$2) \implies 3)$ Vale in generale che \textbf{c.p.a.} $\implies$ connesso.\\
	$3) \implies 1)$ Per \textit{contronominale} mostriamo che $I$ non intervallo $\implies$ $I$ sconnesso. $I$ non intervallo significa che
		\begin{gather*}
			\exists\ a<b<c,\ a,\ c\in I,\ b\notin I \\
			b\notin I \implies I= \left(\underbrace{ I\cap \left(-\infty ,b\right)}_{\in a}\ \right) \cup \left( \underbrace{I\cap \left(b ,+\infty\right)}_{\in c}\ \right)
		\end{gather*}
	ovvero $I$ è unione di aperti, non vuoti e disgiunti $\implies$ $I$ sconnesso.	
\end{demonstration}
\begin{observe}~{}\label{teorema esistenza zeri funzioni continue, s^n cpa}
		\begin{itemize}
	\item Come conseguenza immediata di questo teorema si ha il \textbf{teorema di esistenza degli zeri}\index{teorema! di esistenza degli zeri} per funzioni continue da $\realset$ in $\realset$; per tali funzioni vale che l'immagine continua di un intervallo $\left[a,\ b\right]$ è un intervallo $\left[f\left(a\right),\ f\left(b\right)\right]$, pertanto se gli estremi dell'intervallo sono tali per cui $f\left(a\right)<0<f\left(b\right)$ allora la funzione ammette uno zero, in quanto $0\in\left[f\left(a\right),\ f\left(b\right)\right]$.
	\item Per $n\geq 1$ la \textbf{sfera}\index{sfera} $\displaystyle S^n \coloneqq \left\{ \left(x_1,\ldots,x_{n+1}\right) \mid \sum_{i=1}^{n+1}x_i^2=1 \right\}$ è \textbf{c.p.a.}; infatti, $\forall x,y\in S^n$ si trova sempre un arco dato dall'intersezione di $S^n$ e del piano $H$ passante per il centro della sfera, $x$ e $y$.
		\end{itemize}
	\vspace{-3mm}
\end{observe}
Mostriamo un risultato per funzioni continue da $S^n$ in $\realset$.
\begin{theorema}[Funzioni continue da continue da $S^n$ in $\realset$.]~{}\label{non iniettività S^n in realset}\\
Sia $\funz f {S^n} \realset$ una funzione continua. Allora $\exists x\in S^n \ \colon f(x)=f(-x)$. In particolare $f$ non è iniettiva.
\end{theorema}
\begin{demonstration}
	Costruiamo una funzione $g(x)=f(x)-f(-x)$: essa è continua perché somma di funzioni continue. Siccome $S^n$ è connesso allora $g\left( S^n\right)\subseteq\realset$ è connesso $\implies$ per il teorema precedente $g\left(S^n\right)$ è un intervallo.\\
	Si considerino un punto $y\in S^n$ arbitrario e le sue immagini $g(y)$ e $g(-y)$: esse appartengono all'intervallo dell'immagine $g\left( S^n\right)$, quindi se ne può considerare il loro punto medio:
		\begin{gather*}
			\frac{1}{2}\left[ g(y) + g(-y)\right]=\frac{1}{2} \left[ f(y) -f(-y) + f(-y) -f(y) \right]= 0\\
			\implies 0\in g\left(S^n\right)\exists x\in S^n \colon g(x)=0, \text{ ovvero } f(x)=f(-x)
		\end{gather*}
	\vspace{-3mm}
\end{demonstration}
Come conseguenza di questo teorema si ha che un aperto di $\realset$ non sarà mai omeomorfo ad un aperto di $\realset^n$.
\begin{theorema}[Aperti di $\realset$ non omeomorfi ad aperti di $\realset^n$.]~{}\\
Sia $I\subseteq\realset$ e $U\subseteq\realset^n$, con $n\geq 2$. Se $I,\ U$ sono aperti allora $I$ non è omeomorfo a $U$.	
\end{theorema}	
\begin{demonstration}
	Si consideri un omeomorfismo $\funz g U I$. Siccome $U\subseteq\realset^n$ aperto allora esiste una palla aperta di raggio $\epsilon$ contenuta in $U$, se ne considera il bordo $S^n\subseteq U$. Si considera dunque la restrizione $\funz {g_{|_{S^n}}} {S^n} I$, che per il teorema precedente non è iniettiva. Dunque $g$ non è un omeomorfismo.	
\end{demonstration}

\begin{observe}
	Il teorema appena visto è un caso particolare del \textbf{teorema dell'invarianza della dimensione}\index{teorema!dell'invarianza della dimensione}: ‘‘Siano $U\subseteq\realset^n, V\subseteq\realset^m$ aperti. Se $U\cong V \implies n=m$. Equivalentemente $n\neq m\implies U\ncong V$''. A pag. \pageref{invarianzadimensionen=2} si può trovare la dimostrazione di un altro caso particolare, quello per gli aperti di $\realset^2$.
\end{observe}
% LEZ 08
\subsection{Unioni e prodotti di spazi connessi}
\begin{theorema}[Unione arbitraria di sottospazi connessi è un connesso.]~{}\label{unione sottospazi connessi}\\
Siano $\left\{ X_i \right\}_{i\in I}$ una famiglia di sottoinsiemi di uno spazio topologico $X$. Se ogni $X_i$ è connesso e $\displaystyle\bigcap_{i\in I}X_i\neq\emptyset$ allora $\displaystyle\bigcup_{i\in I}X_i$ è connesso.	
\end{theorema}
\begin{demonstration}
	Sia $\displaystyle Z\subseteq Y\coloneqq \bigcup_{i\in I}X_i$ un aperto e chiuso non vuoto. Vogliamo dimostrare che $Z=Y$, cosicché $Y$ risulti connesso. Basta mostrare l'inclusione $Y\subseteq Z$.\\
	Se considera l'intersezione di $Z$ e di un connesso fra gli $\left\{X_i\right\}$, essa sarà banale per il lemma \ref{connessodisgiuntoosottoinsieme}:
	\begin{gather*}
		X_i \cap Z = \begin{cases}
			\emptyset & \\
			X_i	&		
		\end{cases}
	\end{gather*}
	Essendo $Z\neq\emptyset$ e contenuto nell'unione $\displaystyle \bigcup_{i\in I}X_i$, $\exists i_0$ tale per cui $X_{i_0}\cap Z\neq \emptyset$, da cui segue che $X_{i_0}\cap Z=X_{i_0} \implies X_{i_0}\subseteq Z $.\\
	Siccome $\displaystyle \bigcap_{i\in I}X_i\neq\emptyset$, esisterà sicuramente $x\in\bigcap_{i\in I}X_i\subseteq X_{i_0}\subseteq Z \implies x\in Z$. Poiché $x\in X_i, \forall i$ per definizione di intersezione, segue che $\forall i\in I, \ X_i\cap Z\neq\emptyset$. Quindi, per $\forall i,\ X_i\subseteq Z \implies Y\subseteq Z \implies Y=Z$, quindi $Y$ è connesso perché l'unico aperto e chiuso non vuoto è se stesso.
\end{demonstration}
\begin{theorema}[Prodotto di connessi è connesso.]~{}\label{prodotto connessi}\\
$X, Y$ sono spazi topologici connessi $\iff X\times Y$ è connesso.	
\end{theorema}
\begin{demonstration}~{}\\
	$\impliessx$ Segue dalla continuità e suriettività delle proiezioni e dal fatto che l'immagine continua di un connesso è connessa:
		\begin{gather*}
			\funz p {X\times Y} X \text{continua e suriettiva } \implies p(X)=X \text{ connesso}\\
			\funz q {X\times Y} Y \text{continua e suriettiva } \implies q(Y)=Y \text{ connesso}	
		\end{gather*}	
	$\impliesdx$ Notiamo che, essendo $X\times\{y\}\cong X$ con $X$ connesso, al variare di $y\in Y$ abbiamo delle fibre tutte connesse che spazzano lo spazio prodotto $X\times Y$
	\begin{equation*}
		X\times Y=\bigcup_{y\in Y}X\times\{y\}
	\end{equation*}
	Tuttavia, le $X\times\{y\}$ sono tutte \textit{disgiunte} le une dalle altre e non è sufficiente per affermare la connessione di $X\times Y$!\\
	L'idea è quindi quella di creare una sorta di ‘‘guida'' \textit{connessa} la cui intersezione con le fibre $X\times\{y\}$ non sia vuota; posto $X_y=X\times\{y\}$, prendiamo $x_0\in X$ e definiamo $Y_{x_0}=\{x_0\}\times Y\cong Y$ connesso. Poichè $Z_y=X_y\cup Y_{x_0}$ è connesso $\forall y$ perché unione di connessi la cui intersezione è $\{\left(x_0,\ y\right)\}$, dalle osservazioni precedenti segue che:
		\begin{equation*}
			X\times Y=\bigcup_{y\in Y}Z_y=\bigcup_{y\in Y}\left(X_y \cup Y_{x_0}\right),\ \bigcap_{y\in Y}\left(Z_y\right)=Y_{x_0}\neq\emptyset	
		\end{equation*}
	Dunque $X\times Y$ è unione di connessi la cui intersezione non è vuota, quindi per il teorema precedente è connesso.
\end{demonstration}
\begin{attention}
	L'intersezione di connessi non è necessariamente un connesso! Prendiamo su $\realset^2$ con topologia Euclidea la circonferenza unitaria $S^1=\{\left(x,\ y\right)\in \realset^2\mid x^2+y^2=1\}$ e il segmento sull'asse $x$ dato da $[-1,\ 1]\times\{0\}$: essi sono due \textbf{c.p.a.} (e quindi connessi), ma la loro intersezione è $\{\left(-1,\ 0\right),\ \left(1,\ 0\right)\}$ che è chiaramente sconnessa.\\
	Tuttavia, ciò è vero se lo spazio è $\realset$ con la topologia Euclidea: poiché i connessi sono gli intervalli, la loro intersezione rimane comunque un intervallo (o al più un punto) che è connesso.
\end{attention}
\subsection{Spazi connessi non c.p.a.}
Approfondiamo ora la differenza fra essere spazio connesso o \textbf{c.p.a.}, mostrando esempi di un tipo ma non dell'altro. Prima, però, dimostreremo un teorema sulla caratterizzazione di un \textit{insieme denso} che ci tornerà utile.
\begin{theorema}[Caratterizzazione di un insieme denso.]~{}\\
	Sia $X$ uno spazio topologico e $A\subseteq X$ un suo sottoinsieme, allora:
		\begin{gather*}
			A \text{ è denso }\iff \forall U\subseteq X \text{ aperto e }U\neq\emptyset, \ U\cap A\neq\emptyset	
		\end{gather*}
	\vspace{-6mm}
\end{theorema}
\begin{demonstration}~{}\\
	$\impliesdx$ Se $A$ è denso allora $\overline{A}=X$. Supponiamo che $\exists V$ aperto $\colon V\cap A=\emptyset$. Siccome $V$ è aperto allora $X\setminus V$ è chiuso, inoltre $V\cap A=\emptyset$, quindi $A\subseteq X\setminus V$. Essendo $A$ contenuto in un chiuso allora lo sarà anche la sua chiusura, siccome è il più piccolo chiuso che lo contiene:
		\begin{equation*}
			A\subseteq X\setminus V \implies \overline{A}=X\subseteq \overline{X\setminus V}=X\setminus V \implies V=\emptyset
		\end{equation*}
	Ne segue che l'unico aperto che non interseca $A$ è l'insieme vuoto. \\
	$\impliessx$ Consideriamo un chiuso $K\supseteq A$. Siccome è chiuso allora il suo complementare $X\setminus K$ è aperto. Per ipotesi dunque si ha che $V\cap A\neq \emptyset$ oppure $V=\emptyset$, passando al complementare si ottiene che:
	\vspace{-1mm}
		\begin{equation*}
			\begin{array}{llllllll}
				A\subseteq K &\implies& X\setminus K \subseteq X\setminus A &\implies& V\subseteq X\setminus A &\implies& V\cap A=\emptyset&\\
				& \implies& V=\emptyset& \implies& K=X& \implies & \overline{A}=X&
			\end{array}
		\vspace{-2mm}
		\end{equation*}
	L'ultima implicazione è dovuta al fatto che ogni chiuso che contiene $A$ si è dimostrato essere solo $X$ per cui esso sarà la sua chiusura.
\end{demonstration}

\begin{theorema}[Chiusura e connessione.]~{}\label{chiusuraconnessa}\\
Sia $X$ uno spazio topologico e $Y\subseteq X$ connesso, allora:
		\begin{equation}
			\forall W \colon Y\subseteq W \subseteq \overline{Y} \implies W \text{ connesso}
		\end{equation}
	In particolare la chiusura di un connesso è connessa.
\end{theorema}
\begin{demonstration}
	Consideriamo un sottoinsieme $Z\subseteq W$ aperto, chiuso e non vuoto; dimostriamo che $Z$ deve coincidere con $W$, cioè $W$ ammette come aperto e chiuso non vuoto solo se stesso.
		\begin{gather*}
			Z\subseteq W \text{ aperto } \implies \exists A\subseteq X \text{ aperto } \colon Z=W\cap A \\
			Z\subseteq W \text{ chiuso } \implies \exists C\subseteq X \text{ chiuso } \colon Z=W\cap C
		\end{gather*}
	Studiamo adesso l'intersezione di $Z$ con $Y$:
		\begin{gather*}
			Z\cap Y=A\cap W\cap Y \stackrel{!}{=} A\cap Y \text{ aperto in Y}\\
			Z\cap Y=C\cap W\cap Y \stackrel{!}{=} C\cap Y \text{ chiuso in Y}
		\end{gather*}
	Il passaggio indicato con (!) è dovuto al fatto che $Y\subseteq W$. Poichè $Z\cap Y$ è aperto e chiuso in $Y$, se $Z\cap Y$ è non vuoto seguirà che $Y\subseteq Z$ in quanto $Y$ è connesso. Per provare che tale intersezione non è vuota utilizziamo il teorema precedente:
		\begin{gather*}
			Y \text{ denso in W, infatti } \mathcal{cl}_W(Y)=\mathcal{cl}_X(Y)\cap W=\overline{Y}\cap W=W\\
			Z \text{ aperto in } W\text{ e } Y \text{ denso in } W \implies Z\cap Y \neq \emptyset
		\end{gather*}
	Come già detto, si ha che $Z\cap Y=Y \implies Y\subseteq Z$. Poichè $Y$ è denso in $W$ e $Z$ è chiuso in $W$, il quale contiene $Y$, si ha:
		\begin{equation*}
			W=\mathcal{cl}_W(Y)\subseteq \overline{Z}=Z \implies W\subseteq Z \implies W=Z \implies W \text{ connesso}
		\end{equation*}
\vspace{-3mm}
\end{demonstration}
Vediamo ora degli esempi di spazi connessi ma non \textbf{c.p.a.}
\begin{example}\textsc{Seno del topologo.}\\
\begin{minipage}{0.47\textwidth}
Sia $Y\subseteq \realset^2$ con la topologia euclidea e $Y=\left\{ \left( x,\ \sin\frac{1}{x} \right) \mid x>0 \right\}$, detto anche \textbf{seno del topologo}\index{seno del topologo}. Esso è \textbf{c.p.a.} perché per connettere due punti basta percorrere la curva stessa del grafico. Quindi $Y$ è connesso, dunque per il teorema \ref{chiusuraconnessa} (pag. \pageref{chiusuraconnessa}) $\overline{Y}$ è connesso.\\
Tuttavia $\overline{Y}$ non è \textbf{c.p.a.} in quanto $\overline{Y}=Y\cup \left\{ (0,y) \mid -1\leq y \leq 1 \right\}$ ed i punti sull'asse delle $y$ e sulla curva $Y$ non si possono connettere tramite un arco continuo.
\end{minipage}
\hspace{-7mm}
\begin{minipage}{0.52\textwidth}
	\includegraphics[trim=0cm 0.5cm 0.5cm 1.25cm,clip,scale=0.50]{images/topologistsine.pdf}
\end{minipage}
\end{example}
\vspace{6mm}
\begin{example}\textsc{La pulce ed il pettine.}\\
	Si consideri il ‘‘pettine'' come il seguente sottospazio di $\realset ^2$ con la topologia euclidea:
		\begin{equation*}
			Y= \left\{ (x,\ 0) \mid 0\leq x\leq 1 \right\} \cup \bigcup_{\stackrel{r\in\rationalset}{ 0\leq r\geq 1}} \left\{ (r,\ y) \mid 0\leq y \leq 1 \right\}
		\end{equation*}
\begin{minipage}{0.62\textwidth}
Presi due punti su $Y$ si possono collegare fra loro scendendo alla base del pettine $\left[0,\ 1\right]$ e risalendo sui ‘‘denti'' di ascissa razionale. Quindi $Y$ è \textbf{c.p.a.}, allora $Y$ è connesso e $\overline{Y}=\left[0,\ 1\right]\times \left[0,\ 1\right]$.\\
Si consideri ora la ‘‘pulce'', ovvero un punto $P$ di ascissa irrazionale ed ordinata 1, ad esempio $P=\left(\frac{\sqrt{2}}{2},\ 1\right)$. Sia $Z=Y\cup P$; per il teorema precedente segue che $Z$ è connesso, infatti:
\begin{gather*}
	Y\subseteq Z \subseteq \overline{Y}=\left[0,\ 1\right]\times \left[0,\ 1\right]	
\end{gather*}
	\end{minipage}
	\begin{minipage}{0.37\textwidth}
		\includegraphics[trim=1.1cm 0.5cm 0.5cm 1.25cm,clip,scale=0.50]{images/comb.pdf}
	\end{minipage}\\
Tuttavia $Z$ non è \textbf{c.p.a.}: preso un cammino $\funz \alpha {\left[0,\ 1\right]} {Z\subseteq \realset^2}$ tale che $\alpha(t)= \left( x(t), y(t)\right)$ con $\alpha (0)=(0,0)$ e $\alpha(1)=P$, per continuità $y(t)\neq 0 \implies x(t)\in\rationalset$, ma \textit{non} è vero per $P$ che ha ascissa irrazionale, dunque non esiste un cammino continuo che colleghi l'origine e $P$. Ne consegue che $Z$ non è \textbf{c.p.a.}
\end{example}
\subsection{Componenti connesse}
L'intuizione geometrica che ci ha portati alla definizione di connessione è stata ‘‘di quanti pezzi è fatto uno spazio?''. Se uno spazio è connesso è fatto di un solo ‘‘pezzo'', cerchiamo ora di definire cosa sono i ‘‘pezzi'' e come sono fatti.
\begin{define}[Componente connessa.]~{}\\
	Sia $X$ uno spazio topologico e $C\subseteq X$. Si dice che $C$ è una \textbf{componente connessa}\index{componente!connessa} se:
		\begin{itemize}
			\item $C$ è \textit{connesso}.
			\item $C$ è \textbf{massimale}\index{massimale}, ovvero $C\subseteq A$, $A$ connesso $\implies C=A$.
		\end{itemize}
	Scelto $x\in X$ si può definire la \textbf{componente connessa di un punto}\index{componente!connessa di un punto}, ovvero:
	\begin{equation}
		\displaystyle C(x)=\bigcup \{C \mid C\text{ connesso},\ x\in C\}
	\end{equation}
\vspace{-6mm}
\end{define}
La componente connessa di \textit{un punto} è effettivamente una componente connessa: infatti, è connessa perché unione di connessi con intersezione \textit{non vuota} ($x$ stesso) e se $C(x)\subseteq A \implies x\in A \implies A\subseteq C(x) \implies A=C(x)$.\\
Vediamo ora qualche proprietà delle componenti connesse, in particolare che sono chiuse e formano una partizione.
\begin{theorema}[Componenti connesse sono chiuse e formano una partizione.]~{}\\
	Sia $X$ uno spazio topologico, allora:
		\begin{enumerate}
			\item le componenti connesse sono chiuse.
			\item le componenti connesse formano una partizione di $X$.
		\end{enumerate}
	\vspace{-3mm}
\end{theorema}
\begin{demonstration}
	~{}
	\begin{enumerate}[label=\Roman*]
		\item Sia $C$ una componente connessa. Per ogni insieme vale che $C\subseteq\overline{C}$, ma $C$ è connesso, quindi $\overline{C}$ è connesso. Siccome $C$ è massimale allora $C=\overline{C}$, ovvero è chiuso.
		\item Per dimostrare che le componenti connesse formano una partizione di $X$ dobbiamo mostrare che $X$ è unione disgiunta delle componenti connesse. Prima di tutto notiamo come l'unione delle componenti connesse $C\left(x\right)$ al variare dei punti $x\in X$ coprono lo spazio:
			\begin{equation*}
				\forall x\in X,\ x\in C(x) \implies X=\bigcup_{x\in X}C(x)
			\end{equation*}
		Mostriamo ora che sono disgiunte. Prendiamo due componenti connesse $C$ e $D$ e supponiamo per assurdo che non siano disgiunte; grazie alla proprietà di massimalità delle componenti connesse segue che:
		\begin{gather*}
			C\cap D\neq\emptyset \implies C\cup D \text{ connesso } \implies C=C\cup D=D
		\end{gather*}
	\end{enumerate}
\vspace{-6mm}
\end{demonstration}
\begin{example}
	Sia $\rationalset\subseteq\realset$ con la topologia Euclidea. La componenti connesse di $\rationalset$ sono i punti, quindi i punti sono chiusi in $\rationalset$, il che è una riconferma dato che sappiamo che $\rationalset$ è \textbf{Hausdorff}. Tuttavia non possono essere aperti altrimenti avremmo la topologia discreta!\\
	Inoltre siccome $\rationalset$ ha più di una componente connessa significa che non è connesso! Invece $\realset$ è connesso grazie all'assioma di completezza.
\end{example}
\begin{observe}
	Dati due spazi omeomorfi si ha che hanno lo stesso numero di componenti connesse in quanto l'immagine continua di connessi è connessa. Quindi il \textbf{numero di componenti connesse} ci fornisce un criterio per determinare quando due spazi non sono omeomorfi!
\end{observe}


		\section{Compattezza}
\begin{define}[Ricoprimento aperto e sottoricoprimento.]~{}\\
	Sia $X$ uno spazio topologico. Un \textbf{ricoprimento aperto}\index{ricoprimento}\index{ricoprimento!aperto} di $X$ è una famiglia $\mathcal{A}=\{A_i \}_{i\in I}$ di aperti di $X$ tali che $\displaystyle X=\bigcup_{i\in I} A_i$. \\
	Un \textbf{sottoricoprimento}\index{sottoricoprimento} $\mathcal{B}$ di un ricoprimento aperto $\mathcal{A}$ è una famiglia di aperti di $\mathcal{A}$ la cui unione è ancora tutto $X$.
\end{define}		

\begin{examples}\textsc{Esempi di ricoprimenti aperti.}
	\begin{itemize}
		\item $\displaystyle \realset=(-\infty ,\ 2)\cup(0,\ +\infty)$ è un ricoprimento aperto
		\item $\displaystyle \realset =\bigcup_{n\in\naturalset}(-n,\ n)$ è un ricoprimento aperto
		\item $\displaystyle \realset=\bigcup_{p \text{ primo}}(-p,p)$ è un ricoprimento aperto
	\end{itemize}
\end{examples}

\begin{define}[Spazio compatto.]~{}\\
	Uno spazio topologico $X$ si dice \textbf{compatto}\index{spazio!compatto} se dato un qualsiasi ricoprimento aperto $\mathcal{A}$ si può sempre estrarre un sottoricoprimento \textit{finito} $\mathcal{B}$.
\end{define}
L'importanza della definizione risiede nel fatto che non si chiede che esista un ricoprimento $\mathcal{A}$ finito (basterebbe banalmente $X$ stesso che è aperto) bensì che da $\mathcal{A}$ si possa sempre estrarre un \textit{numero finito di aperti} che ricopra ancora $X$.

\begin{examples}\textsc{Esempi di spazi non compatti.}
	\begin{itemize}
		\item $\realset$ con la topologia euclidea: se si considera il ricoprimento aperto $\displaystyle \realset=(-\infty , 2)\cup(0,+\infty)$, esso non ammette sottoricoprimento finito.
		\item Gli intervalli aperti o semiaperti della forma $[a,\ b)$ hanno come ricoprimento aperto $\mathcal{A}=\left\{ \left[ a, \ b-\frac{1}{n}\right) \right\}_{n\in \naturalset}$, che non ammette un sottoricoprimento finito.
	\end{itemize}
\end{examples}

\begin{theorema}[Immagine continua di un compatto è un compatto]\label{immagine compatto}
	Dati $X,Y$ spazi topologici, $\funz f X Y$ continua, allora
		\begin{gather*}
			X \text{ compatto } \implies f(X) \text{ compatto }
		\end{gather*}
	\vspace{-6mm}
\end{theorema}
\begin{demonstration}
	Sia  $\mathcal{A}=\{A_i\}$ ricoprimento aperto di $f(X)$. Per definizione di ricoprimento allora:
	\begin{equation*}
		f(X)\subseteq \bigcup_{i\in I}A_i
	\end{equation*}
	Si considerino ora le controimmagini degli aperti $A_i$ tramite $f$, aperte in quanto $f$ è continua.
	\begin{equation*}
	X\subseteq f^{-1}\left(\bigcup_{i\in I}A_i\right)=\bigcup_{i\in I}f^{-1}\left(A_i\right)
	\end{equation*}	
	Da ciò si evince che $\mathcal{A}'=\left\{f^{-1}(A_i)\right\}$ è un ricoprimento aperto di $X$; essendo $X$ compatto, si può estrarre un sottoricoprimento finito di $X$. Riapplicando la funzione $f$ troviamo un sottoricoprimento finito del ricoprimento originale:
	\begin{gather*}
		X=f^{-1}(A_1) \cup	\ldots \cup f^{-1}(A_n) \implies f(X)\subseteq A_1\cup \ldots \cup A_n \implies f(X) \text{ compatto}
	\end{gather*}
	\vspace{-6mm}
\end{demonstration}
Da questo teorema segue che essere compatti è una \textbf{proprietà topologica}.
% LEZ 09
\begin{theorema}[${[0,\ 1]}$ è un compatto.]~{}\\
	L'intervallo $[0,\ 1]\subseteq\realset$ con la topologia euclidea è compatto.
\end{theorema}
\begin{demonstration}
	Sia $\mathcal{A}=\{ A_i\}_{i\in I}$ un ricoprimento aperto di $[0,\ 1]$ con $A_i$ aperti in $\realset$.
	Sia $X=\{ t\in\realset \mid [0,t] \text{ è coperto da un numero finito di } A_i \}$. Questo insieme \textit{non} è vuoto; infatti, per $t=0$:
		\begin{gather*}
			[0,\ t]=[0,\ 0]=\{0\}\implies\exists A_0\in\mathcal{A}\colon \{0\}\subseteq A_{0} \implies 0\in X\implies X\neq\emptyset
		\end{gather*}
	Siccome $X$ non è vuoto, per la completezza dei reali ne posso considerare l'estremo superiore $b=\sup X$. Ci sono due casi, $b>1$  e $b\leq 1$: dimostriamo che il primo è possibile mentre il secondo è assurdo per definizione di estremo superiore:
		\begin{itemize}
			\item $b>1$: $\exists t\in X \colon 1<t<b \implies [0,\ 1]\subseteq [0,\ t] \subseteq A_1\cup \ldots \cup A_n$
			\item $b\leq 1$: $b\in [0,\ 1] \implies \exists A_0\in\mathcal{A} \colon b\in A_0$ con $A_0$ aperto; per definizione della topologia Euclidea esisterà una palla aperta centrata $b$ contenuta in $A_0$:
			\begin{equation*}
				\exists \delta >0 \colon B_\delta (b)=(b-\delta, \ b+\delta )\subseteq A_0
			\end{equation*}
			Sia $0<h<\delta$. Consideriamo $b+h$ e l'intervallo $[0,\ b+1]$:
				\begin{gather*}
					[0,\ b+h]=[0,\ t]\cup [t,\ b+h]\subseteq \underbrace{A_1\cup\ldots\cup A_n}_{t\in X}\cup \underbrace{A_0}_{B_\delta (b)\subseteq A_0}
				\end{gather*}	
			Quindi $b+h$ è coperto da un numero finito di aperti, pertanto $b+h\in X$, il che è assurdo perché $b=\sup X$.
		\end{itemize}
	 Segue che solo il caso $b>1$ è lecito, da cui si ha che $[0,\ 1]$ è coperto da un numero finito di aperti del ricoprimento iniziale e quindi è compatto.
\end{demonstration}
Notiamo che questo teorema implica che un intervallo $[a,\ b]\subseteq\realset$ è compatto, essendo omeomorfo a $[0,\ 1]$.\\
Vediamo ora un esempio di spazio compatto che non abbia la topologia euclidea.
\begin{example}
	Uno spazio $X$ con la \textbf{topologia cofinita} $CF$ è compatto.\\
	Ricordiamo che gli aperti nella topologia $CF$ sono i sottoinsiemi il cui complementare è finito, quindi gli aperti sono pari a $X$ privato di un numero finito di punti.\\
	Preso un ricoprimento aperto $\mathcal{A}=\{A_i\}$, scegliamo un aperto $A_0=X\setminus\{x_1,\ \ldots,\ x_n\}$. Per ricavare un sottoricoprimento finito di $\mathcal{A}$ è sufficiente considerare, per ogni punto $x_i$ che \textit{non} appartiene all'aperto $A_0$, un aperto del ricoprimento che lo contenga. In questo modo $X=A_0\cup A_1\cup A_n$ e quindi $X$ è compatto
\end{example}
\begin{observe}
Notiamo che se $X$ è \textbf{finito} allora $X$ è compatto per \textit{qualsiasi} topologia: poiché la sua cardinalità è finita, la sarà anche quella del suo \textit{insieme delle parti}, da cui scelgo gli aperti della topologia; un qualunque ricoprimento sarà necessariamente finito. I casi interessanti di spazi compatti sono quelli il cui insieme di sostegno \textit{non è finito}.\\
Inoltre, se $X$ ha la topologia \textbf{discreta} vale anche il \textit{viceversa}:
	\begin{gather*}
		X \text{ top. discreta } \implies \left( X \text{ compatto } \iff X \text{ finito}\right)
	\end{gather*}
$\impliesdx$ Consideriamo il ricoprimento aperto $\mathcal{A}=\left\{ A_x\right\}_{x\in X}$, con $A_x\coloneqq \{x\}$ aperti in quanto $X$ ha la topologia discreta. Siccome $X$ è compatto allora esiste un sottoricoprimento finito, ovvero un numero finito di aperti di $\mathcal{A}$ che ricopre $X$, ossia:
\begin{equation*}
	X=\{x_1\}\cup\{x_2\}\cup\ldots\{x_n\}\implies X={x_1,\ x_2,\ \ldots,\ x_n}\implies X \text{ finito.}
\end{equation*}
\vspace{-6mm}
\end{observe}
\subsection{Relazioni fra compattezza e altre proprietà topologiche}
\begin{theorema}[Chiuso in un compatto è compatto; Manetti, 4.41.1.]~{}\label{chiuso in compatto}\\
Un chiuso in un compatto è un compatto, ovvero se $X$ è uno spazio topologico compatto, $C\subseteq X$ chiuso allora $C$ è compatto.
\end{theorema}
\begin{demonstration}
	Sia $\mathcal{A}=\{A_i \}_{i\in I}$ un ricoprimento di $X$, sia $C\subseteq X$ chiuso, allora $A\coloneqq X\setminus C$ è aperto in $X$.\\
	Sia $\mathcal{A}'=\{A_i,\ A\}$ ricoprimento aperto di $X$. Siccome $X$ è compatto esiste un suo sottoricoprimento finito
		\begin{gather*}
			X=A_1\cup\ldots\cup A_n\cup A \implies C=X\setminus A=A_1\cup\ldots\cup A_n
		\end{gather*}
	ovvero $C$ è compatto.
\end{demonstration}

\begin{lemming}[Unione finita di compatti è un compatto; Manetti, 4.41.2.]
\end{lemming}
\begin{demonstration}
Preso un ricoprimento aperto $\mathcal{A}$ di $K_1\cup\ldots\cup K_n$, estraiamo un sottoricoprimento finito $\widetilde{\mathcal{A}}_i$ per ogni $K_i$ compatto: l'unione $\widetilde{\mathcal{A}}_1\cup\ldots\cup \widetilde{\mathcal{A}}_n$ è un sottoricoprimento finito di $\mathcal{A}$ che copre $K_1\cup\ldots\cup K_n$.
\end{demonstration}
% Vediamo ora una relazione importante fra due proprietà topologiche: compattezza e \textbf{Hausdorff}.
\begin{theorema}[Compatto in un Hausdorff è chiuso; Manetti, 4.48.]~{}\label{compatto in hausdorff chiuso}\\
Se $X$ è di Hausdorff e $K\subseteq X$ è compatto, allora $K$ è chiuso.
\end{theorema}
\begin{demonstration}
	Per dimostrare che $K$ è chiuso mostriamo che il suo complementare è aperto, in particolare mostrando che sia intorno di ogni suo punto.
	Allora, fissato un $x_0$ arbitrario nel complementare, dovrà esistere un aperto contenente $x_0$ (che può coincidere o meno con $X\setminus K$) disgiunto da $K$.
		\begin{equation*}
				K \text{ chiuso } \iff X\setminus K \text{ aperto } \iff \exists A\subseteq X\setminus K \text{ con } A \text{ aperto } \colon x_0\in A \iff A\cap K=\emptyset
		\end{equation*}
	Per ipotesi $X$ è di \textbf{Hausdorff}; fissato $x_0$ in $X\setminus K$, per ogni $y\in K$ chiaramente $x\neq y$, dunque esisteranno due intorni (aperti) $U_y\in I(x_0)$ e $V_y\in I(y)$ (dipendenti da $y$) tali che $U_y\cap V_y=\emptyset$. Consideriamo allora, al variare di $y$, l'unione di tutti gli intorni (aperti) $V_y$:
	\begin{equation*}
			V=\bigcup_{y\in K}V_y \implies K\subseteq V
	\end{equation*}
	Gli aperti $\{V_y\}$ formano un ricoprimento di $K$, dunque, essendo $K$ compatto, estraiamo un sottoricoprimento $V_{y_1}\cup\ldots\cup V_{y_n}$.\\
	Consideriamo allora gli intorni aperti di $x_0$ definiti in precedenza e selezioniamo quelli dati da $y_1,\ \ldots,\ y_n$. Sia:
	\begin{equation*}
		U=U_{y_1}\cap\ldots\cap U_{y_n} \in I(x_0)
	\end{equation*}
	Gli aperti $V$ e $U$ per costruzione \textit{non} si intersecano ($V\cap U=\emptyset$), in particolare $V\cap K=\emptyset$ perché $K\subseteq V$. Allora, per le osservazioni iniziali $U\subseteq X\setminus K$, cioè $X\setminus K$ è intorno di $x_0$. Per l'arbitrarietà di $x_0$, segue che $X\setminus K$ è aperto e quindi $K$ chiuso.
\end{demonstration}
\begin{theorema}[Compatto in $\realset$ se e solo chiuso e limitato; Manetti, 4.42.]~{}\label{compatto chiuso e limitato R}\\
Un sottospazio $K\subseteq \realset$ è compatto $\iff K$ chiuso e limitato.
\end{theorema}
\begin{demonstration}~{}\\
	$\impliesdx$ Siccome $\realset$ è di \textbf{Hausdorff} e $K$ è compatto allora per il teorema precedente $K$ è chiuso.\\
	Per vedere che è limitato consideriamo un ricoprimento aperto $\mathcal{A}=\left\{ (-n,\ n)\cap K\right\}_{n\in\naturalset}$ di $K$. Siccome è compatto allora esiste un sottoricoprimento finito, ovvero:
	\begin{gather*}
		K\subseteq (-n_1,\ n_1)\cup\ldots\cup(-n_m,\ n_m) \implies K\subseteq (-M,\ M), \ M\coloneqq \max n_i
	\end{gather*}
	quindi $K$ è limitato. \\
	$\impliessx $ $K$ è limitato, quindi $K\subseteq [-n, \ n]$ che è compatto; poiché $K$ è chiuso per ipotesi e contenuto in un compatto, per il teorema \ref{chiuso in compatto} è anch'esso compatto.
\end{demonstration}
\begin{attention}
	Il teorema precedente \textit{non} afferma che gli unici compatti di $\realset$ sono gli intervalli chiusi e limitati! Anche una loro \textit{unione finita} (anche disgiunta) potrebbe esserlo.
\end{attention}
\begin{theorema}[Funzione su compatto in $\realset$ ammette massimo/minimo; Manetti, 4.43.]~{}\label{weierstrass}\\
Sia $\funz f X \realset$ con $X$ compatto e $\realset$ con la topologia euclidea. Se $f$ è continua allora ammette massimo e minimo.
\end{theorema}
\begin{demonstration}
	$f$ continua e $X$ compatto $\implies f(X)$ compatto, e per il teorema precedente ciò equivale al fatto che $f(X)$ è chiuso e limitato.
	\begin{equation*}
		\left.
		\begin{array}{lcl}
			f\left(x\right)\text{ limitata}&\implies& \sup \left\{f\left(x\right)\right\}<+\infty\\
			f\left(x\right)\text{ chiusa}&\implies& \sup \left\{f\left(x\right)\right\}=\max \left\{f\left(x\right)\right\}\\
		\end{array}
		\right\}
		\implies f\left(x\right)\text{ ammette massimo.}
	\end{equation*}
	Analoga è la dimostrazione per il minimo.
\end{demonstration}

\begin{attention}
	Per poter parlare di massimo e minimo di una funzione c'è bisogno di un \textit{ordinamento} sul codominio; il dominio $X$ potrebbe anche non averne uno!
\end{attention}
Vogliamo ora vedere come si comporta la compattezza rispetto al prodotto, prima però va dimostrato un lemma che ci tornerà utile nella dimostrazione del teorema.
\begin{lemming}[Tube lemma.]~{\label{tube lemma}}\\
Siano $X,Y$ spazi topologici con $Y$ compatto, $x_0\in X$, $A\subseteq X\times Y\colon A$ aperto e $\{x_0\}\times Y\subseteq A$. Allora $\exists U\subseteq X$ con $x_0\in U$, aperto tale che $\{x_0\}\times Y \subseteq U\times Y \subseteq A$
\end{lemming}
\begin{demonstration}
	L'aperto $A$ si può esprimere come unione di aperti della base della topologia prodotto:
	\begin{equation*}
		A\text{ aperto in }X\times Y\implies A=\bigcup_{i\in I}\left( U_i\times V_i \right)
	\end{equation*}
	Con $U_i\subseteq X$, $V_i\subseteq Y$. $\mathcal{U}=\{U_i\times V_i\}$ è un ricoprimento di $\{x_0\}\times Y$; poiché $\{x_0\}\times Y$ è compatto in quanto omeomorfo a $Y$, esiste un suo sottoricoprimento finito da $\mathcal{U}$:
	\begin{equation*}
		\{x_0\}\times Y\subseteq (U_1\times V_1)\cup\ldots\cup (U_n\times V_n)
	\end{equation*}
	%Se necessario si eliminano gli aperti che sono disgiunti da $\{x_0\}\times Y$
	Poniamo $U=U_1\cap\ldots\cap U_n$:
	\begin{equation*}
		\{x_0\}\times Y\subseteq U\times Y \subseteq (U_1\times V_1)	\cup\ldots\cup (U_n\times V_n) \subseteq A
	\end{equation*}
	$U$ è l'aperto che soddisfa la tesi.
\end{demonstration}
\begin{theorema}[Prodotto di compatti è compatto; Manetti, 4.49.2.]~{}\label{prodotto compatti}\\
$X,Y$ compatti $\iff X\times Y$ è compatto.
\end{theorema}
\begin{demonstration}~{}\\
	$\impliessx $ Le proiezioni $p$ e $q$ sono funzioni continue e suriettive. Poiché $X\times Y$ compatto e $p\left(X\times Y\right)=X$ e $q\left(X\times Y\right)=Y$, allora $X,\ Y$ sono compatti. \\
	$\impliesdx $ Sia $\mathcal{A}=\{A_i\}$ un ricoprimento aperto di $X\times Y$. Per ipotesi $Y$ è compatto, dunque per ogni $x$ si ha $\{x\}\times Y$ compatto in quanto $\{x\}\times Y\cong Y$.
	Estraiamo un sottoricoprimento finito da $\mathcal{A}$ che copra $\{x\}\times Y$:
	\begin{equation*}
		\{x\}\times Y\subseteq A_{x,1}\cup\ldots\cup A_{x,n}=A_x
	\end{equation*}
	Gli $A_{x_i}$ dipendono dal punto $x$ scelto. Poiché $A_x$ è aperto, Per il \textit{Tube lemma} allora:
		\begin{gather*}
			\exists U_x\subseteq X \text{ aperto }\colon \{ x\}\times Y \subseteq U_x\times Y \subseteq A_x=A_{x,1}\cup\ldots\cup A_{x,n}
		\end{gather*}
	Al variare di $x\in X$ si ha un ricoprimento $\mathcal{U}=\left\{U_x\right\}_{x\in X}$ di $X$ compatto, dunque possiamo estrarne un sottoricoprimento finito $X=U_{x_1}\cup\ldots\cup U_{x_m}$. Usando la proiezione sulla componente $X$:
		\begin{equation*}
			\begin{array}{ll}
				X\times Y &= p^{-1}(X)=p^{-1}\left( U_{x_1}\cup\ldots\cup U_{x_m}\right)= (U_{x_1}\times Y)\cup\ldots\cup (U_{x_m}\times Y)\\ &\subseteq A_{x_1}\cup\ldots\cup A_{x_m} 
				\subseteq \left( A_{x_1 , 1}\cup\ldots\cup A_{x_1, n_1}\right) \cup\ldots\cup \left( A_{x_m, 1}\cup\ldots\cup A_{x_m, n_m} \right)
			\end{array}
		\end{equation*}
	Poichè $X\times Y$ è coperto da un unione finita di un unione finita di aperti del ricoprimento $\mathcal{A}$, segue che è compatto.
\end{demonstration}
Noto che il prodotto di compatti è compatto, possiamo generalizzare la caratterizzazione dei compatti in $\realset$ (teorema \ref{compatto chiuso e limitato R} al caso dello spazio $\realset ^n$.
\begin{theorema}[Compatto in $\realset^n$ se e solo chiuso e limitato; Manetti, 4.42.]~{}\label{compatto chiuso e limitato R^n}\\
$K\subseteq \realset^n$ compatto $\iff K$ chiuso e limitato.
\end{theorema}
\begin{demonstration}~{}\\
	$\impliesdx$ $K$ è compatto in $\realset^n$ che è un Hausdorff, quindi $K$ è chiuso per il teorema \ref{compatto in hausdorff chiuso}. Per dimostrare che è limitato consideriamo un ricoprimento di palle aperte centrate nell'origine e utilizziamo l'ipotesi di $K$ compatto:
		\begin{gather*}
			K\subseteq \bigcup_{n\in\naturalset} B_n(\mathbf{0}) \implies K\subseteq B_{n_1}(\mathbf{0})\cup\ldots\cup B_{n_m}(\mathbf{0}) \subseteq B_M (\mathbf{0})
		\end{gather*}
	Con $M=\max n_i$.\\
	$\impliessx$ $K$ è limitato, quindi $K\subseteq [-a,\ a]^n$ che è compatto perché prodotto di compatti, ma $K$ è anche chiuso, quindi per il teorema \ref{chiuso in compatto} è compatto.
\end{demonstration}
\begin{digression}
In realtà vale un teorema più generale, che si dimostrerà poi nel corso di \textit{Istituzioni di Analisi}.\\
\textsc{Teorema:} Sia $X$ uno spazio metrico completo, allora $K\subseteq X$ compatto $\iff K$ chiuso e \textbf{totalmente limitato}\index{spazio!totalmente limitato}, cioè $\forall\epsilon >0, \ K$ è contenuto in un'unione finita di palle di raggio $\epsilon$.\\
In $\realset^n$ vale limitato $\iff$ totalmente limitato, ma in generale no; ad esempio, consideriamo lo spazio metrico delle funzioni continue su $[0,\ 1]$ con distanza dell'estremo superiore:
	\begin{gather*}
		\mathcal{C}\left( [0, \ 1] \right) \coloneqq \left\{ \funz f {[0,\ 1]} \realset \mid f \text{ continua } \right\},  \ \ \text{con } \mvf{d}{f}{g} =\sup_{x\in [0, \ 1]} |f(x)-g(x)|
	\end{gather*}
La palla di centro l'origine $0$ e raggio $1$:
\begin{equation*}
	B_1(\mathbf{0})=\left\{ \funz f {[0,\ 1]} \realset \mid f \text{ continua }, -1\geq f(x) \leq 1 \ \right\}
\end{equation*}
È chiusa e limitata, tuttavia in $\mathcal{C}\left( [0, \ 1] \right)$ non è compatta.
\end{digression}
\begin{theorema}[Funzione continua da compatto ad Hausdorff è chiusa; Manetti, 4.52.]~{}\label{da compatto in T_2 è chiuso}
Se $\funz f X Y$ continua con $X$ è compatto e $Y$ di \textbf{Hausdorff}, allora $f$ è chiusa.
\end{theorema}
\begin{demonstration}
	Per mostrare che $f$ è chiusa consideriamo $C\subseteq X$ chiuso e mostriamo che $f(C)$ è chiuso usando i teoremi \ref{chiuso in compatto}, \ref{immagine compatto} e \ref{compatto in hausdorff chiuso}:
		\begin{equation*}
				C\subseteq X \text{ chiuso in compatto} \implies C \text{ compatto} \implies f(C) \text{ compatto in }\mathbf{T2} \implies f(C) \text{ chiuso}
		\end{equation*}
\end{demonstration}
In generale vale il \textbf{teorema di Kuratowsi-Mròwka}: $Y$ è compatto se e solo se per qualsiasi spazio topologico $X$ la proiezione $\funz {p_X} {X\times Y} X$ è chiusa; noi ne dimostreremo una versione più debole.
% LEZ 10
\begin{theorema}[Prodotto con compatto implica proiezione chiusa; Manetti, 4.49.1.]~{}\\
Siano $X,\ Y$ spazi topologici con $Y$ compatto, allora la proiezione $\funz p {X\times Y} X$ è chiusa.
\end{theorema}
\begin{demonstration}
	Preso un $C\subseteq X\times Y$ chiuso vogliamo mostrare che $p(C)\subseteq X$ è chiuso. Per far ciò, mostriamo che il suo complementare $X\setminus p(C)$ è aperto in quanto intorno di ogni suo punto.\\
	Chiaramente, se $p(C)=X$ allora è già chiuso; se invece $p(C)\neq X$ allora $\exists x_0\in X\setminus p(C)$. Si consideri la fibra di $x_0$ tramite la proiezione $p$:
		\begin{gather*}
			p^{-1}(\{x_0\} )=\{x_0\}\times Y \subseteq (X\times Y)\setminus C
		\end{gather*}
	Con $(X\times Y)\setminus C$ aperto perché complementare in $X\times Y$ del chiuso $C$. Si rientra nelle ipotesi del \textit{Tube lemma}:
		\begin{equation*}
			\exists U\subseteq X \text{ aperto } \colon \{x_0\}\times Y \subseteq U\times Y\subseteq (X\times Y)\setminus C		\end{equation*}
	Poiché la proiezione è continua, si ha:
	\begin{equation*}
		\begin{array}{ll}
			p^{-1}(U)=U\times Y\subseteq (X\times Y)\setminus C &\implies p^{-1}(U)\cap C=\emptyset \\
			& \implies U\cap p(C)=\emptyset \implies x_0\in U\subseteq X\setminus p(C)
		\end{array}	
	\end{equation*}
Segue dunque che $X\setminus p\left(C\right)$ è intorno di $x_0$. Poiché questo è vero $\forall x_0\in X\setminus p\left(C\right)$, $X\setminus p\left(C\right)$ è intorno di ogni suo punto, quindi aperto. Segue pertanto la tesi.
\end{demonstration}