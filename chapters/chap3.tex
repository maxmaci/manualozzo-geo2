% SVN info for this file
\svnidlong
{$HeadURL$}
{$LastChangedDate$}
{$LastChangedRevision$}
{$LastChangedBy$}

\chapter{Gruppi topologici}
\labelChapter{gruppi topologici}

\begin{introduction}
‘‘BEEP BOOP INSERIRE CITAZIONE QUA BEEP BOOP.''
\begin{flushright}
	\textsc{NON UN ROBOT,} UN UMANO IN CARNE ED OSSA BEEP BOOP.
\end{flushright}
\end{introduction}

\section{Gruppi topologici}
Conoscendo le strutture di gruppo e spazio topologico su un insieme vogliamo vedere come possono essere compatibili fra loro.
\begin{define}
	Un insieme $G$ si dice gruppo topologico se
		\begin{itemize}
			\item $G$ è un gruppo
			\item $G$ è uno spazio topologico
			\item Le operazioni sono continue, ovvero le mappe \  $\funztot \mu {G\times G} G {(x,\ y)} {xy}$ \ e \ $\funztot i G G x {x^{-1}}$ sono funzioni continue.
		\end{itemize}
\end{define}
Vediamo ora degli esempi noti di gruppi topologici.
\begin{examples}
	\begin{itemize}
		\item $\left( \realset^n, \ +, \ \mathcal{E_{ucl}} \right), \left( \complexset ^n, \ +, \ \mathcal{E_{ucl}} \right)$
		\item $\left( \realset^*, \ \bullet,\ \mathcal{E_{ucl}} \right), \left( \realset^*, \ \bullet, \ \mathcal{E_{ucl}} \right)$ con la topologia indotta di sottospazio
		\item $\left( M_{n,m}(\realset),\ \bullet, \mathcal{E_{ucl}} \right)$ con la topologia indotta di sottospazio di $\realset^{n,m}$, ad esempio $\intv$
	\end{itemize}	
\end{examples}
\begin{observe}
	I gruppi topologici $\gl (n,\realset )$ e $\gl (n, \realset )$ sono aperti di $M_{n,n}$.\newline
	Infatti considerata la funzione determinante $\funz \det {\realset^{n,n}} \realset$ essa è continua in quanto per calcolare il determinante si opera solo con somme e prodotti. Si ha che $\gl(n, \realset)$ è il complementare dell'insieme delle matrici che hanno determinante nullo, il quale è un chiuso in quanto controimmagine di un chiuso quale $\{0\}$ di una funzione continua. Dunque tale gruppo topologico è aperto, analogamente per $\complexset$.
\end{observe}
Vediamo ora altri sottoinsiemi di $M_{n,n}$:
	\begin{itemize}
		\item $\Sl$ dato da $\{\det A=1\} $ è il \textit{gruppo speciale lineare}
		\item $\Or$ determinato dall'equazione $A^{t}A=I$ è il \textit{gruppo ortogonale}
		\item $\So=\Or\cap\Sl$ è il \textit{gruppo speciale ortogonale}
		\item $\U$ determinato dall'equazione $A^{t*}\overline{A}=I$ è il \textit{gruppo unitario}
		\item $\Su=\U\cap\Sl$ è il \textit{gruppo speciale unitario}
	\end{itemize}

Ci sono delle operazioni sulle matrici che sono continue:
	\begin{itemize}
		\item \textit{moltiplicazione matriciale}: continua perché definita tramite somme e prodotti di elementi delle matrici
		\item \textit{inversa}: è una funzione che ad una matrice $A$ associa $\displaystyle \frac{1}{\det A}$ per prodotti e somme di elementi della matrice, dunque è continua
	\end{itemize}

\begin{observe}
	Per i gruppi topologici in generale vale la moltiplicazione destra e sinistra sono omeomorfismi:
		\begin{gather*}
			\funztot {L_h} G G g {hg} \text{   e   } \funztot {R_n}  G G g {gh} \\
			(L_h)^{-1}=L_{h^{-1}} \text{   e   } (R_h)^{-1}=R_{h^-1}
		\end{gather*}
	Ne segue che un gruppo topologico è \textbf{omogeneo}, ovvero
		\begin{gather*}
			\forall g,h\in G \ \exists \funz \phi G G \text{ omeomorfismo t.c.} \phi(g)=h
		\end{gather*}
	infatti basta porre $\phi=L_{hg^{-1}}$ oppure $\phi=R_{g^{-1}h}$
\end{observe}
Il seguente teorema ci permette di caratterizzare i gruppi topologici Hausdorff grazie alla chiusura dell'elemento neutro. 
%vorei mettere un punto esclamativo, è una cosa figa
\begin{theorema}
	Sia $G$ un gruppo topologico, $e\in G$ il suo elemento neutro, si ha che 
		\begin{gather*}
			G \ T_2 \iff \{e\} \text{ chiuso }
		\end{gather*}
\end{theorema}
\begin{demonstration}~{}\\
	$\impliesdx G \ T_2 \implies G \ T_1 \implies$ tutti i punti sono chiusi, in particolare anche $\{e\}$. \newline
	$\impliessx$ Per dimostrare che $G \ T_2$ si utilizza la caratterizzazione con la diagonale chiusa, sfruttando l'omogeneità dei gruppi topologici:
		\begin{gather*}
			\funztot \phi {G\times G} G {(g,h)} {gh^{-1}} \ \ , \ \ (g,h)=(h,g) \iff g=h\iff \phi\left( (g,h)\right)=gh^{-1}=e \\
			\implies \Delta_G =\phi^{-1}\left(\{e\}\right)
		\end{gather*}
	Per ipotesi $\{e\}$ è chiusa, quindi $\Delta_G$ è chiuso e dunque $G \ \ T_2$.
\end{demonstration}

\begin{observe}
	$\gl(n,\realset)$ è sconnesso. Mostriamo che è unione di due aperti non vuoti disgiunti sfruttando la funzione determinante $\funz \det {\gl(n,\realset)} {\realset^*}$, infatti essendo continua  le controimmagini di aperti saranno aperti:
		\begin{gather*}
			\begin{array}{lcl}
				\det^{-1} \left( (0,+\infty)\right) & = & \gl ^+ (n,\realset) \\ 
				\det^{-1} \left( (-\infty,0)\right) & = & \gl ^- (n,\realset) \\
			\end{array}
		\implies \gl(n,\realset)=\gl^+(n,\realset) \sqcup \gl^-(n,\realset)
		\end{gather*}
\end{observe}
 Dimostriamo un lemma che generalizza il teorema \ref{unione sottospazi connessi}
 e che ci sarà utile nella dimostrazione successiva sulla connessione di alcuni gruppi topologici.
\begin{lemming} \textsc{Manetti 4.18} \\
	Sia $\funz f X Y$ continua. Se $f$ è suriettiva aperta o chiusa, $Y$ è connesso e le fibre sono connesse, ovvero se $\forall y\in Y \ f^{-1}(y)$ è connesso, allora $X$ è connesso.
\end{lemming}
\begin{demonstration}
	Supponiamo che $f$ sia aperta e consideriamo $A_1\neq\emptyset\neq A_2$ aperti (per $f$ chiusa si considerano dei chiusi e si procede in modo analogo) t.c. $X=A_1\cup A_2$. Per dimostrare che $X$ è connesso mostriamo che $A_1\cap A_2\neq\emptyset$:
		\begin{gather*}
			\begin{array}{lcl}
				f \text{ aperta } & \implies & f(A_1), f(A_2) \text{ aperti}\\
				f \text{ suriettiva} & \implies & f(X)=Y \implies f(A_1\cup A_2)=f(A_1)\cup f(A_2)=Y \\
				Y \text{ connesso } & \implies & f(A_1)\cap f(A_2)\neq\emptyset \implies \exists y_0\in f(A_1)\cap f(A_2) \implies \begin{cases} 
					f^{-1}(y_0) \cap A_1\neq \emptyset \\
					f^{-1}(y_0) \cap A_2\neq \emptyset 
				\end{cases}
			\end{array} \\
				\begin{cases}
					\left( f^{-1}(y_0)\cap A_1 \right)\cup \left( f^{-1}(y_0)\cap A_2 \right)=f^{-1}(y_0) \\
					\text{fibre connesse }
				\end{cases}
				 \implies  f^{-1}(y_0)\cap A_1\cap A_2\neq\emptyset \implies A_1\cap A_2\neq\emptyset
		\end{gather*}
\end{demonstration}

\begin{theorema}
	$\forall n\geq 1, \gl^+(n,\realset)$ e $\gl(n,\complexset)$ sono connessi.
\end{theorema}
\begin{demonstration}
	
\end{demonstration}

%\LEZ 11
\begin{observe}
	
\end{observe}

\begin{theorema}
	
\end{theorema}
\begin{demonstration}
	
\end{demonstration}

\begin{observe}
	
\end{observe}