% SVN info for this file
\svnidlong
{$HeadURL$}
{$LastChangedDate$}
{$LastChangedRevision$}
{$LastChangedBy$}

\chapter{Gruppi topologici}
\labelChapter{gruppi topologici}

\begin{introduction}
‘‘Di questi giorni, l'angelo della topologia e il diavolo dell'algebra astratta lottano per l'anima di ogni singola disciplina matematica.''
\begin{flushright}
	\textsc{Hermann Weyl,} esorcista topologico.
\end{flushright}
\end{introduction}
\lettrine[findent=1pt, nindent=0pt]{A}{bbiamo} definito la struttura di \textit{spazio topologico} su un insieme con lo scopo principale di poter definire formalmente la \textit{continuità} di una funzione. Un altro tipo di struttura fondamentale per la Matematica è quella di \textit{gruppo}, un insieme dotato di una \textit{operazione binaria} che soddisfa le condizioni di chiusura, associatività, identità ed invertibilità.\\
In questo capitolo studieremo il \textbf{gruppo topologico}, un oggetto matematico che è dotato contemporaneamente sia di una struttura di \textit{gruppo}, sia di una di \textit{spazio topologico}. In questo modo, potremo eseguire operazioni algebriche e parlare di continuità allo stesso tempo. Vedremo inoltre le relazioni fra alcune proprietà che abbiamo già visto e questo nuovo oggetto di studio.
\section{Gruppi topologici}
Conoscendo le strutture di \textit{gruppo} e \textit{spazio topologico} su un insieme, vogliamo vedere come possono essere \textit{compatibili} fra loro.
\begin{definition}{}[Gruppo topologico]
	Un insieme $G$ si dice \textbf{gruppo topologico}\index{gruppo topologico} se:
		\begin{itemize}
			\item $G$ è un \textit{gruppo};
			\item $G$ è uno \textit{spazio topologico};
			\item l'\textit{operazione} e l'\textit{inverso} sono funzioni \textit{continue}:
			\begin{equation*}
				\funct{}[\mu]{G\times G}{G}[(x,\ y)][x\cdot y]\qquad \funct{}[i]{G}{G}[x][x^{-1}]
			\end{equation*}
		\end{itemize}
\end{definition}
Vediamo ora degli esempi noti di gruppi topologici.
\begin{example}{pn}~{}
	\begin{itemize}
		\item $\left( \R^n, \ +\ , \ \eucl \right), \left( \C ^n, \ +\ , \ \eucl \right)$.
		\item $\left( \R^*, \ \cdot\ ,\ \eucl \right), \left( \C^*, \ \cdot\ , \ \eucl \right)$ con la topologia indotta di sottospazio.
		\item $\left( M_{n,m}(\R),\ \cdot\ , \eucl \right)$ con la topologia indotta di sottospazio di $\R^{n,m}$.
	\end{itemize}
\end{example}
\begin{remark}{}[{$\GL (n,\K )$} è un aperto di {$M_{n,n}\left(\K\right)$}, con {$\K=\R$} o {$\C$}]
	Consideriamo la funzione del \textit{determinante} $\funct{}[\det]{\R^{n,n}}{\R}$: essa è continua in quanto per calcolare il determinante si opera solo con somme e prodotti. Si ha che $\GL(n, \R)$ è il complementare dell'insieme delle matrici che hanno determinante nullo, il quale è un chiuso in quanto controimmagine di $\{0\}$, chiuso in $\R$, tramite una funzione continua.
	\begin{equation*}
		\GL(n, \R)=M_{n,n}(\R)\setminus\mathrm{det}^{-1}\left(\left\{0\right\}\right)
	\end{equation*}
	Dunque tale gruppo topologico è aperto; analogamente vale per il caso con $\C$.
\end{remark}
Vediamo ora altri sottogruppi moltiplicativi di $M_{n,n}$:
	\begin{itemize}
		\item $\SL$, dato da $\{\det A=1\} $, è il \textbf{gruppo speciale lineare}\index{gruppo!speciale lineare}.
		\item $\Or$, determinato dall'equazione $A^{t}A=I$, è il \textbf{gruppo ortogonale}\index{gruppo!ortogonale}.
		\item $\SO=\Or\cap\SL$ è il \textbf{gruppo speciale ortogonale}\index{gruppo!speciale ortogonale}.
		\item $\U$, determinato dall'equazione $A^{t}\overline{A}=I$, è il \textbf{gruppo unitario}\index{gruppo!unitario}.
		\item $\SU=\U\cap\SL$ è il \textbf{gruppo speciale unitario}\index{gruppo!speciale unitario}.
	\end{itemize}
Tutti questi sono gruppi topologici, in quanto le operazioni sulle matrici sono continue:
	\begin{itemize}
		\item \textit{Moltiplicazione matriciale}: è una funzione continua perché definita tramite somme e prodotti di elementi delle matrici.
		\item \textit{Inversione}: è una funzione che ad una matrice $A$ associa la sua inversa $A^{-1}$, ottenuta moltiplicando $\displaystyle \frac{1}{\det A}$ per una matrice ottenuta con particolari prodotti e somme di elementi della matrice $A$; per questo motivo è continua.
	\end{itemize}
\begin{remark}{n}
	Per i gruppi topologici in generale vale la \textit{moltiplicazione destra} e \textit{sinistra}:
		\begin{align*}
			\funct{}[L_h]{G}{G}[g][hg]&\funct{}[R_n]{G}{G}[g][gh] \\
			(L_h)^{-1}=L_{h^{-1}}&(R_h)^{-1}=R_{h^{-1}}
		\end{align*}
	In particolare sono omeomorfismi. Ne segue che un gruppo topologico è \textbf{omogeneo}: per ogni $g,\ h\in G$ esiste un omeomorfismo $\funct{}[\phi]{G}{G}$ tale che $\phi(g)=h$.	Infatti, basta porre $\phi\coloneqq L_{hg^{-1}}$ oppure $\phi\coloneqq R_{g^{-1}h}$.
\end{remark}
Il seguente teorema ci permette di caratterizzare i gruppi topologici di Hausdorff grazie alla chiusura dell'elemento neutro.
\begin{theorem}{}[Caratterizzazione dei gruppi topologici di Hausdorff]
	Sia $G$ un gruppo topologico ed $e\in G$ il suo elemento neutro. Si ha che $G$ di Hausdorff se e solo se $\{e\}$ è chiuso.
\end{theorem}
\begin{proof}{n}~{}\\
	$\rightimplies$ Se $G$ è di Hausdorff è $T_1$, quindi tutti i punti sono chiusi ed in particolare lo è $\{e\}$.\\
	$\leftimplies$ Per dimostrare che $G$ è di Hausdorff si utilizza la caratterizzazione con la diagonale chiusa. Definita la mappa continua
	\begin{equation*}
		\funct{}[\phi]{G\times G}{G}[(g,h)][gh^{-1}],
	\end{equation*}
	si ha che
		\begin{equation*}
			(g,h)\in\Delta_G \iff g=h\iff \phi\left( (g,h)\right)=gh^{-1}=e.
		\end{equation*}
	Ma allora $\Delta_G =\phi^{-1}\left(\{e\}\right)$, ed essendo $\{e\}$ chiuso per ipotesi, $\Delta_G$ è chiusa e dunque $G$ è di Hausdorff.\qedhere
\end{proof}

\begin{remark}{}[$\GL(n,\R)$ è sconnesso]
$\GL(n,\R)$ è unione di due aperti non vuoti disgiunti: notando che le controimmagini di aperti tramite la funzione determinante $\funct{}[\det]{\GL(n,\R)}{\R\setminus\{0\}}$ sono aperti in quanto $\det$ è continua, si ha
\begin{equation*}
		\mathrm{det}^{-1} \left( (0,+\infty)\right)=\GL ^+ (n,\R)\qquad\mathrm{det}^{-1} \left( (-\infty,0)\right)=\GL ^- (n,\R).
\end{equation*}
Di conseguenza, $\GL(n,\R)=\GL^+(n,\R) \amalg \GL^-(n,\R)$.
\end{remark}
 Dimostriamo un lemma che generalizza il teorema \ref{unione sottospazi connessi} e che ci sarà utile nella dimostrazione successiva sulla connessione di alcuni gruppi topologici.
\begin{lemma}{}[Connessione per fibre; Manetti, 4.18]\label{connessione e fibre}
	Sia $\funct{}[f]{X}{Y}$ continua. Se $f$ è suriettiva aperta o chiusa, $Y$ è connesso e per ogni $y\in Y$ le fibre $f^{-1}(y)$ sono connesse, allora $X$ è connesso.
\end{lemma}
\begin{proof}{n}
	Supponiamo che $f$ sia aperta e consideriamo $A_1,A_2\neq\emptyset$ aperti\footnote{Per $f$ chiusa si considerano dei chiusi $A_1,A_2\neq\emptyset$ e si procede in modo analogo.} tale che $X=A_1\cup A_2$. Per dimostrare che $X$ è connesso mostriamo che $A_1\cap A_2\neq\emptyset$.\\
	Poiché $f$ aperta, $f(A_1), f(A_2)$ sono aperti. Inoltre, poiché $f$ suriettiva, 
	\begin{equation*}
		Y=f(X)=f(A_1\cup A_2)=f(A_1)\cup f(A_2)
	\end{equation*}
	$Y$ è uno spazio connesso e ricoperto dai due aperti $f(A_1)$ e $f(A_2)$, dunque la loro intersezione $f(A_1)\cap f(A_2)$ è non vuota; sia $y_0\in f(A_1)\cap f(A_2)$. Ma allora
	\begin{equation*}
		f^{-1}(y_0) \cap A_1\neq \emptyset\qquad f^{-1}(y_0) \cap A_2\neq \emptyset.
	\end{equation*}
	Osserviamo che
	\begin{equation*}
		\left(f^{-1}(y_0)\cap A_1 \right)\cup \left( f^{-1}(y_0)\cap A_2 \right)=f^{-1}(y_0)\cap\left(A_1\cup A_2\right)=f^{-1}(y_0)\cap X=f^{-1}(y_0).
	\end{equation*}
	Le fibre sono connesse: essendo ricoperte da due aperti, si ha
	\begin{equation*}
			\left(f^{-1}(y_0)\cap A_1 \right)\cap \left(f^{-1}(y_0)\cap A_2 \right)=f^{-1}(y_0)\cap A_1\cap A_2\neq\emptyset
	\end{equation*}
	da cui segue che $A_1\cap A_2\neq\emptyset$.
\end{proof}
\begin{theorem}{n}[{$\GL^+(n,\R)$} e {$\GL(n,\C)$} sono connessi]
\end{theorem}
\begin{proof}{n}
	Si procede per induzione su $n$ per $\GL^+(n,\R)$; il caso $\GL(n,\C)$ è analogo.\\
	$n=1)\quad$ Si ha $\GL^+(1,\R)=(0,+\infty)$ e $\GL(1,\C)=\C\setminus\{0\}=\R^2\setminus\{\mathbf{0}\}$, che sappiamo essere connessi.\\
	$n>1)\quad$ Supponiamo ora che $\GL^+(n-1,\R)$ sia connesso. Per dimostrare che $\GL(n,\R)$ è connesso, cerchiamo una funzione continua e suriettiva da $\GL^+(n,\R)$ ad un connesso che soddisfi le ipotesi del lemma precedente. A tal scopo, consideriamo la funzione che mappa una matrice $n\times n$ alla sua prima colonna:
	\begin{equation*}
		\funct{}[p]{\R^{n,n}}{\R^n}[A][p(A)].
	\end{equation*}
	Siccome $\R^{n,n}=\R^n\times\R^{n,n-1}$ allora $p$ è una \textit{proiezione}, dunque per il punto 2. della proposizione \ref{topprodotto} $p$ è aperta. La restrizione di $p$ a $\GL^+(n,\R)$
	\begin{equation*}
		\funct{}[p]{\GL^+(n,\R)}{\R^n\setminus\{\mathbf{0}\}}
	\end{equation*}
	è una funzione continua, suriettiva e aperta perché restrizione di una funzione che ha tutte e tre queste proprietà; inoltre, $\R^n\setminus\{\mathbf{0}\}$ è connesso per $n>1$.\\
	Rimane soltanto da mostrare che le fibre sono tutte connesse: per far ciò è sufficiente mostrare che siano tutte omeomorfe ad una particolare fibra connessa. Consideriamo
		\begin{equation*}
			y_0=\begin{pmatrix}
					1 \\ 0 \\ \vdots \\ 0
				\end{pmatrix}\in \R^n\setminus\{\mathbf{0}\} \implies p^{-1}(y_0)=
				\left(\begin{array}{c|c}
				1 & \begin{array}{ccc}
					\ast & \cdots & \ast
				\end{array}\\
				\hline
				\begin{array}{c}
					0\\
					\vdots\\
					0
				\end{array} & A
				\end{array}\right)
%				
%				\begin{pmatrix}
%						1      & \ast & \cdots & \ast\\
%						0      &   &       &  \\
%						\vdots &   & A     &  \\
%						0      &   &       &
%					\end{pmatrix}
		\end{equation*}
	con $(\ast,\dots, \ast)\in\R^{n-1}$ arbitrario in quanto non influisce nel calcolo del determinante e con $A\in\GL^+(n-1,\R)$. Segue che $p^{-1}(y_0)=\R^{n-1}\times\GL^+(n-1,\R)$, dunque $p^{-1}(y_0)$ è una fibra connessa visto che le due componenti lo sono per ipotesi. Sia $y\in\R^n\setminus\{\mathbf{0}\}$ e sia $A\in\GL^+(n,\R)$ tale per cui $p(A)=y$, cioè $y$ è la prima colonna di $A$.
	Vogliamo mostrare che
	 \begin{equation*}
	 	p^{-1}(y)=Ap^{-1}(y_0)=L_A\left( p^{-1}(y_0)\right),
	 \end{equation*}
	 dove $L_A$ è l'omeomorfismo della moltiplicazione sinistra
	 \begin{equation*}
	 	\funct{}[L_A]{\GL^+(n,\R)}{\GL^+(n,\R)}[B][AB],
	 \end{equation*}
	 in modo tale da avere tutte le fibre omeomorfe a $p^{-1}(y_0)$.
	 \begin{itemize}
	 	\item[$\leftinclude$]  Presa $B\in p^{-1}(y_0)$, essa deve essere della forma
	 \begin{equation*}
	 	B=
	 	\left(\begin{array}{c|c}
	 		1 & \begin{array}{ccc}
	 			\ast & \cdots & \ast
	 		\end{array}\\
	 		\hline
	 		\begin{array}{c}
	 			0\\
	 			\vdots\\
	 			0
	 		\end{array} & C
	 	\end{array}\right).
	 \end{equation*}
	 Osservando che
	 \begin{equation*}
	 	p(AB)=Ap(B)=A\begin{psmallmatrix}
	 		1 \\ 0 \\ \vdots \\ 0
	 	\end{psmallmatrix}= p(A)=y,
	 \end{equation*}
	 si ha $AB\in p^{-1}\left(y\right)$.\\
	 \item[$\rightinclude$] Presa $C\in p^{-1}(y)$, essa deve essere della forma
	 \begin{equation*}
	 	C=
	 	\left(\begin{array}{c|c}
	 		y & \begin{array}{ccc}
	 			\ast & \cdots & \ast\\
	 			\vdots & \ddots & \vdots\\
	 			\ast & \cdots & \ast
	 		\end{array}
 		\end{array}\right).
	 \end{equation*}
	 Poniamo allora $B=A^{-1}C$:
	 \begin{equation*}
	 	p(B)=p(A^{-1}C)=A^{-1}p(C)=A^{-1}y=A^{-1}p(A)=p(A^{-1}A)=p(I)=y_0.
	 \end{equation*}
	 Poiché $B\in p^{-1}(y_0)$, allora $C\in Ap^{-1}(y_0)$.
	 \end{itemize}
Siccome tutte le fibre sono tutte omeomorfe ad una fibra connessa, allora sono tutte connesse e valgono le ipotesi del lemma precedente, dunque $\GL^+(n,\R)$ è connesso.\qedhere
\end{proof}
\begin{corollary}{n}[$\SL{(n,\R)}$ e $\SL{(n,\C)}$ sono connessi]
\end{corollary}
\begin{proof}{n}
	Siccome $\GL^+(n,\R)$ e $\GL(n,\C)$ sono connessi, basta considerare la seguente funzione:
		\begin{gather*}
			\funct{}[f]{\GL^+(n,\R)}{\SL(n,\R)}[A][
				\begin{pmatrix}
					\frac{a_{1,1}}{\det A} & a_{1,2} & \cdots  & a_{1,n} \\
					\vdots                 & \vdots  & \ddots  & \vdots \\
					\frac{a_{n,1}}{\det A} & a_{n,2} & \cdots  & a_{n,n}
				\end{pmatrix}
			]			
		\end{gather*}
	Siccome $f$ è continua e suriettiva e $\GL^+(n,\R)$ è connesso, $f(\GL^+)=\SL$ è connesso.\qedhere
\end{proof}

\begin{corollary}{n}[$\Or$ \textit{non} è connesso.]
\end{corollary}
\begin{proof}{n}
	Siccome $\Or$ è sottogruppo di $\GL$ e la connessione è una proprietà topologica allora $\Or$ non è connesso. In particolare, si può dividere in base a $\det =+1$ e $\det =-1$.\qedhere
\end{proof}

\begin{theorem}{n}[$\SO{(n)}, \U{(n)}$ e $\SU{(n)}$ sono compatti e connessi]
\end{theorem}
\begin{proof}{n}
	Per dimostrare che sono compatti, essendo sottospazi di $\GL(n,\R)\subseteq\R^{n,n}\cong \R^{n^2}$ per il \textit{teorema di Heine-Borel}\ref{compatto chiuso e limitato R^n} basta dimostrare che sono chiusi e limitati. Poiché questi sottospazi sono definiti come luoghi di zeri di polinomi in $a_{ij}$, allora sono chiusi:
		\begin{align*}
			\SO(n)&\coloneqq\Set{A\in\GL(n,\R) | A^{t}A=I, \det A=1}\\
			\U(n)&\coloneqq\Set{A\in\GL(n,\R) | A^{t}\overline{A}=I}\\
			\SU(n)&\coloneqq\Set{A\in\GL(n,\R) | A^{t}\overline{A}=I, \det A=1}
		\end{align*}
	Siccome $\SU(n)\subseteq \U(n)\subseteq \SO(n)$ basta dimostrare che $\SO(n)$ è limitato, usando la norma Euclidea\footnote{La norma Euclidea di matrici $\R^{n,n}$ corrisponde a visualizzare la matrice $A\in\R^{n,n}$ come un vettore in $\R^{n^2}$ e usare la norma Euclidea ben nota degli spazi vettoriali reali.} su $\R^{n,n}$. Se $A\in\SO(n)$, la condizione $A^{t}A=I$ implica che 
		\begin{gather*}
			 \sum_{i=1}^n a_{ij}^2=1, \forall j=1,\dots,n \implies \norm{A}=\sum_{i,j=1}^n a_{ij}^2=n.
		\end{gather*}
	Ma allora $\SO(n)\subseteq S_{\sqrt{n}}\subseteq\R^{n,n}\cong \R^{n^2}$, dove $S^{n^2-1}_{\sqrt{n}}$ è la sfera di raggio $\sqrt{n}$ in $\R^{n,n}\cong \R^{n^2}$: dunque, $\SO(n)$ è limitato. Ne segue che anche $\U(n)$ e $\SU(n)$ lo sono, dunque sono tutti chiusi e limitati in $\R^{n,n}$ e quindi sono compatti.\\
	Per dimostrare che sono \textit{connessi} si procede analogamente al teorema precedente: consideriamo la proiezione sulla prima colonna
	\begin{equation*}
		\funct{}[p]{\SO(n)}{S^{n-1}\subseteq \R^n}[A][p(A)],
	\end{equation*}
 	che è una funzione continua, suriettiva e chiusa in quanto è funzione da un compatto a valori in un Hausdorff, e le cui sue fibre sono connesse, dato che
	\begin{equation*}
		p^{-1}\left(\begin{pmatrix} 1 \\ 0 \\ \vdots \\ 0 \end{pmatrix} \right) = \left(\begin{array}{c|c}
			1 & \begin{array}{ccc}
				0 & \cdots & 0
			\end{array}\\
			\hline
			\begin{array}{c}
				0\\
				\vdots\\
				0
			\end{array} & A
		\end{array}\right)
	\end{equation*}
con $A\in \SO(n-1)$. Segue che per il lemma \ref{connessione e fibre} $\SO(n)$ è connesso.\qedhere
\end{proof}
\begin{remark}{n}
	$\GL$ e $\SL$ \textit{non} sono compatti perché non sono limitati; inoltre, $\GL$ è aperto e non chiuso.
\end{remark}
