% SVN info for this file
\svnidlong
{$HeadURL$}
{$LastChangedDate$}
{$LastChangedRevision$}
{$LastChangedBy$}

\chapter{Gruppi topologici}
\labelChapter{gruppi topologici}

\begin{introduction}
‘‘Di questi giorni, l'angelo della topologia e il diavolo dell'algebra astratta lottano per l'anima di ogni singola disciplina matematica.''
\begin{flushright}
	\textsc{Hermann Weyl,} esorcista topologico.
\end{flushright}
\end{introduction}
\noindent Nel \autoref{chap:spazitopologici} abbiamo definito la struttura di \textit{spazio topologico} su un insieme con lo scopo principale di poter definire formalmente la \textit{continuità} di una funzione. Un'altro tipo di struttura fondamentale per la Matematica è quella di \textit{gruppo}, un insieme dotato di una \textit{operazione binaria} che soddisfa le condizioni di chiusura, associatività, identità ed invertibilità.\\
In questo capitolo studieremo il \textbf{gruppo topologico}, un oggetto matematico che è dotato contemporaneamente sia di una struttura di \textit{gruppo}, sia di una di \textit{spazio topologico}. In questo modo, potremo eseguire operazioni algebriche e parlare di continuità allo stesso tempo. Vedremo inoltre le relazioni fra alcune proprietà che abbiamo già visto e questo nuovo oggetto di studio.
\section{Gruppi topologici}
Conoscendo le strutture di \textit{gruppo} e \textit{spazio topologico} su un insieme, vogliamo vedere come possono essere \textit{compatibili} fra loro.
\begin{define}[Gruppo topologico.]~{}\\
	Un insieme $G$ si dice \textbf{gruppo topologico}\index{gruppo topologico} se:
		\begin{itemize}
			\item $G$ è un \textit{gruppo}.
			\item $G$ è uno \textit{spazio topologico}.
			\item L'\textit{operazione} e l'\textit{inverso} sono funzioni \textit{continue}:
			\begin{equation}
				\funztot \mu {G\times G} G {(x,\ y)} {x\cdot y}\qquad \funztot i G G x {x^{-1}}
			\end{equation}
		\end{itemize}
	\vspace{-3mm}
\end{define}
Vediamo ora degli esempi noti di gruppi topologici.
\begin{examples}~{}
	\begin{itemize}
		\item $\left( \realset^n, \ +, \ \mathcal{E_{ucl}} \right), \left( \complexset ^n, \ +, \ \mathcal{E_{ucl}} \right)$.
		\item $\left( \realset^*, \ \cdot,\ \mathcal{E_{ucl}} \right), \left( \realset^*, \ \cdot, \ \mathcal{E_{ucl}} \right)$ con la topologia indotta di sottospazio.
		\item $\left( M_{n,m}(\realset),\ \cdot, \mathcal{E_{ucl}} \right)$ con la topologia indotta di sottospazio di $\realset^{n,m}$, ad es. $\intv$.
	\end{itemize}
\vspace{-3mm}
\end{examples}
\begin{observe}
	I gruppi topologici $\gl (n,\realset )$ e $\gl (n, \complexset )$ sono aperti di $M_{n,n}$.\newline
	Infatti, considerata la funzione del \textit{determinante} $\funz \det {\realset^{n,n}} \realset$ essa è continua in quanto per calcolare il determinante si opera solo con somme e prodotti.\\
	Si ha che $\gl(n, \realset)$ è il complementare dell'insieme delle matrici che hanno determinante nullo, il quale è un chiuso in quanto controimmagine di un chiuso quale $\{0\}$ di una funzione continua. Dunque tale gruppo topologico è aperto, e analogamente vale per il caso con $\complexset$.
\end{observe}
Vediamo ora altri sottoinsiemi di $M_{n,n}$:
	\begin{itemize}
		\item $\Sl$, dato da $\{\det A=1\} $, è il \textit{gruppo speciale lineare}\index{gruppo!gruppo speciale lineare}.
		\item $\Or$, determinato dall'equazione $A^{t}A=I$, è il \textit{gruppo ortogonale}\index{gruppo!gruppo ortogonale}.
		\item $\So=\Or\cap\Sl$ è il \textit{gruppo speciale ortogonale}\index{gruppo!gruppo speciale ortogonale}.
		\item $\U$, determinato dall'equazione $A^{t*}\overline{A}=I$, è il \textit{gruppo unitario}\index{gruppo!gruppo unitario}.
		\item $\Su=\U\cap\Sl$ è il \textit{gruppo speciale unitario}\index{gruppo!gruppo speciale unitario}.
	\end{itemize}
Ci sono delle operazioni sulle matrici che sono continue:
	\begin{itemize}
		\item \textit{Moltiplicazione matriciale}: continua perché definita tramite somme e prodotti di elementi delle matrici
		\item \textit{Inversa}: è una funzione che ad una matrice $A$ associa $\displaystyle \frac{1}{\det A}$ per prodotti e somme di elementi della matrice, in particolare è continua.
	\end{itemize}
\begin{observe}
	Per i gruppi topologici in generale vale la \textit{moltiplicazione destra} e \textit{sinistra}:
		\begin{gather*}
			\funztot {L_h} G G g {hg} \text{   e   } \funztot {R_n}  G G g {gh} \\
			(L_h)^{-1}=L_{h^{-1}} \text{   e   } (R_h)^{-1}=R_{h^{-1}}
		\end{gather*}
	In particolare sono omeomorfismi. Ne segue che un gruppo topologico è \textbf{omogeneo}, ovvero:
		\begin{gather*}
			\forall g,h\in G \ \exists \funz \phi G G \text{ omeomorfismo}\ \colon \phi(g)=h
		\end{gather*}
	Infatti, basta porre $\phi=L_{hg^{-1}}$ oppure $\phi=R_{g^{-1}h}$.
\end{observe}
Il seguente teorema ci permette di caratterizzare i gruppi topologici di \textbf{Hausdorff} grazie alla chiusura dell'elemento neutro.
\begin{theorema}[G un gruppo topologico di Hausdorff se e solo se il neutro è chiuso.]~{}\\
	Sia $G$ un gruppo topologico, $e\in G$ il suo elemento neutro, si ha che:
		\begin{gather*}
			G \ \text{di \textbf{Hausdorff}} \iff \{e\} \text{ chiuso }
		\end{gather*}
	\vspace{-6mm}
\end{theorema}
\begin{demonstration}~{}\\
	$\impliesdx G$ di \textbf{Hausdorff} $\implies G \ T_1 \implies$ tutti i punti sono chiusi, in particolare anche $\{e\}$. \newline
	$\impliessx$ Per dimostrare che $G$ è di \textbf{Hausdorff} si utilizza la caratterizzazione con la diagonale chiusa, sfruttando l'omogeneità dei gruppi topologici:
		\begin{gather*}
			\funztot \phi {G\times G} G {(g,h)} {gh^{-1}} \ \ , \ \ (g,h)=(h,g) \iff g=h\iff \phi\left( (g,h)\right)=gh^{-1}=e \\
			\implies \Delta_G =\phi^{-1}\left(\{e\}\right)
		\end{gather*}
	Per ipotesi $\{e\}$ è chiusa, quindi $\Delta_G$ è chiuso e dunque $G$ è di \textbf{Hausdorff}.
\end{demonstration}

\begin{observe}
	$\gl(n,\realset)$ è sconnesso. Mostriamo che è unione di due aperti non vuoti disgiunti sfruttando la funzione determinante $\funz \det {\gl(n,\realset)} {\realset^*}$, infatti essendo continua  le controimmagini di aperti saranno aperti:
		\begin{gather*}
			\begin{array}{lcl}
				\det^{-1} \left( (0,+\infty)\right) & = & \gl ^+ (n,\realset) \\
				\det^{-1} \left( (-\infty,0)\right) & = & \gl ^- (n,\realset) \\
			\end{array}
		\implies \gl(n,\realset)=\gl^+(n,\realset) \amalg \gl^-(n,\realset)
		\end{gather*}
	\vspace{-3mm}
\end{observe}
 Dimostriamo un lemma che generalizza il teorema \ref{unione sottospazi connessi} e che ci sarà utile nella dimostrazione successiva sulla connessione di alcuni gruppi topologici.
\begin{lemming}[$Y$ connesso con fibre connesse tramite $f$ suriettiva aperta/chiusa implica $X$ connesso; Manetti, 4.18.]~{}\\
	Sia $\funz f X Y$ continua. Se $f$ è suriettiva aperta o chiusa, $Y$ è connesso e le fibre sono connesse, ovvero se $\forall y\in Y \ f^{-1}(y)$ è connesso, allora $X$ è connesso.
\end{lemming}
\begin{demonstration}
	Supponiamo che $f$ sia aperta e consideriamo $A_1\neq\emptyset\neq A_2$ aperti (per $f$ chiusa si considerano dei chiusi e si procede in modo analogo) t.c. $X=A_1\cup A_2$. Per dimostrare che $X$ è connesso mostriamo che $A_1\cap A_2\neq\emptyset$:
		\begin{gather*}
			\begin{array}{lcl}
				f \text{ aperta } & \implies & f(A_1), f(A_2) \text{ aperti}\\
				f \text{ suriettiva} & \implies & f(X)=Y \implies f(A_1\cup A_2)=f(A_1)\cup f(A_2)=Y \\
				Y \text{ connesso } & \implies & f(A_1)\cap f(A_2)\neq\emptyset \implies \exists y_0\in f(A_1)\cap f(A_2) \implies \begin{cases}
					f^{-1}(y_0) \cap A_1\neq \emptyset \\
					f^{-1}(y_0) \cap A_2\neq \emptyset
				\end{cases}
			\end{array} \\
				\begin{cases}
					\left( f^{-1}(y_0)\cap A_1 \right)\cup \left( f^{-1}(y_0)\cap A_2 \right)=f^{-1}(y_0) \\
					\text{fibre connesse }
				\end{cases}
				 \implies  f^{-1}(y_0)\cap A_1\cap A_2\neq\emptyset\implies\\
				 \implies A_1\cap A_2\neq\emptyset
		\end{gather*}
\end{demonstration}
\begin{theorema}[$\protect\forall n\geq 1, {\gl^+(n,\realset)}$ e ${\gl(n,\complexset)}$ sono connessi.]
\end{theorema}
\begin{demonstration}
	Si procede per induzione su $n$ per $\gl^+(n,\realset)$, il caso $\gl(n,\complexset)$ è analogo.\newline
	$n=1 \ )$ \ \ $\begin{cases}
			\gl^+(1,\realset)=(0,+\infty) \\
			\gl(n,\complexset)=\complexset\setminus\{0\}=\realset^2\setminus{0}
		\end{cases}$ connessi.\newline
	$n>1 \ )$ \ \ Supponiamo che $\gl^+(n-1,\realset)$ sia connesso e dimostriamo che lo è anche $\gl(n,\realset)$. Cerchiamo dunque una funzione continua da $\gl^+(n,\realset)$  a $\gl(n-1,\realset)$ che soddisfi le ipotesi del lemma precedente. Pertanto si considera la funzione prima colonna $\displaystyle \funztot p {\realset^{n,n}} {\realset^n} A {p(A)}$ che mappa la prima colonna di $A$. Siccome $\realset^{n,n}=\realset^n\times\realset^{n,n-1}$ allora $p$ è una proiezione, dunque per il punto 2 della proposizione \ref{topprodotto}	è aperta. Restringiamo ora $p$ a $\gl^+(n,\realset)$ nel modo seguente $\funz p {\gl^+(n,\realset)} {\realset^n\setminus\{\mathbf{0}\}}$ che è una funzione continua, suriettiva e aperta, inoltre $\realset^n\setminus\{\mathbf{0}\}$ è connesso per $n>1$. Calcoliamo ora le fibre e mostriamo che sono tutte omeomorfe fra loro e connesse, quindi prima ne troviamo una, mostriamo che è connessa poi dimostriamo che sono tutte omeomorfe. A questo scopo consider
		\begin{gather*}
			y_0=\begin{pmatrix}
					1 \\ 0 \\ \vdots \\ 0
				\end{pmatrix}\in \realset^n\setminus\{\mathbf{0}\} \implies p^{-1}	(y_0)=\begin{pmatrix}
						1      & * & \cdots & *\\
						0      &   &       &  \\
						\vdots &   & A     &  \\
						0      &   &       &
					\end{pmatrix}
		\end{gather*}
	con $(*,\dots, *)\in\realset^{n-1}$ arbitrario in quanto non influisce nel calcolo del determinante e con $A\in\gl^+(n-1,\realset)$, da cui segue che $p^{-1}(y_0)=\realset^{n-1}\times\gl^+(n-1,\realset)$, dunque $p^{-1}(y_0)$ è conness visto che i due fattori lo sono per ipotesi.\newline
	Mostriamo ora che tutte le fibre sono omeomorfe a $p^{-1}(y_0)$. Sia $y\in\realset^n\setminus\{\mathbf{0}\}$ e sia $A\in\gl^+(n,\realset)$ tale che $p(A)=y$, ovvero $y$ è la prima colonna di $A$. In generale vale la relazione $p(AB)=Ap(B)$ e la moltiplicazione sinistra $\displaystyle \funztot {L_A} {\gl^+(n,\realset)} {\gl^+(n,\realset)} B {AB}$ è un omeomorfismo. Dimostriamo che vale $p^{-1}(y)=L_A\left( p^{-1}(y_0)\right)=Ap^{-1}(y_0)$, in modo tale da avere tutte le fibre omeomorfe fra loro:
	\begin{equation*}
		\begin{array}{ll}
			\includesx & \displaystyle \quad B\in p^{-1}(y_0) \implies B=\begin{pmatrix}
				1	   & \cdots  & \cdot  \\
				0 	   & \cdots  & \cdot   \\
				\vdots & \ddots  & \vdots   \\
				0      & \cdots  & \cdot
			\end{pmatrix} \implies	p(AB)=Ap(B)=A\begin{psmallmatrix}
				1 \\ 0 \\ \vdots \\ 0
			\end{psmallmatrix}= p(A)=y\\
			\includedx & \displaystyle \quad C\in p^{-1}(y) \implies C=\begin{pmatrix}
				y & * & \cdots & *
			\end{pmatrix}
		\end{array}
	\end{equation*}
Poniamo allora $B=A^{-1}C$:
			\begin{equation*}
				\begin{array}{ll}
					p(B)&=p(A^{-1}C)=A^{-1}p(C)=A^{-1}y=A^{-1}p(A)=\\
					&=p(A^{-1}A)=p(I)=y_0
				\end{array}
			\end{equation*}
	Poiché $C=AB$, allora $B\in p^{-1}(y_0)$.\\
	Quindi, siccome tutte le fibre sono tutte omeomorfe ad una fibra connessa, allora sono tutte connesse e valgono le ipotesi del lemma precedente, per cui $\gl^+(n,\realset)$ è connesso.
\end{demonstration}
%WELP LA FORMATTAZIONE DEL GATHER* è ORRIBILE MA NON SO COME MIGLIORARLA

%\LEZ 11
\begin{corollary}[$\Sl{(n,\realset)}$ e $\Sl{(n,\complexset)}$ sono connessi.]
\end{corollary}
\begin{demonstration}
	Siccome $\gl^+(n,\realset)$ e $\gl(n,\complexset)$ sono connessi, basta considerare la seguente funzione:
		\begin{gather*}
			\funztot f {\gl^+(n,\realset)} {\Sl(n,\realset)} A {
				\begin{pmatrix}
					\frac{a_{1,1}}{\det A} & a_{1,2} & \cdots  & a_{1,n} \\
					\vdots                 & \vdots  & \ddots  & \vdots \\
					\frac{a_{n,1}}{\det A} & a_{n,2} & \cdots  & a_{n,n}
				\end{pmatrix}
			}			
		\end{gather*}
	Siccome $f$ è continua e suriettiva e $\gl^+(n,\realset)$ è connesso allora $f(\gl^+)=\Sl$ è connesso.
\end{demonstration}

\begin{corollary}[$\Or$ \textit{non} è connesso.]
\end{corollary}
\begin{demonstration}
	Siccome $\Or$ è sottogruppo di $\gl$ e la connessione è una proprietà topologica allora $\Or$ non è connesso. In particolare si può dividere in base a $\det =+1$ e $\det =-1$.
\end{demonstration}

\begin{theorema}[$\So{(n)}, \U{(n)}$ e $\Su{(n)}$ sono compatti e connessi.]
\end{theorema}
\begin{demonstration}
	Per dimostrare che sono \textit{compatti} essendo sottospazi di $\realset^{n\times n}$ per il teorema \ref{compatto chiuso e limitato R^n} basta dimostrare che sono chiusi e limitati. In particolare essendo definiti tramite equazioni che sono luoghi di zeri di polinomi in $a_{ij}$ allora sono chiusi:
		\begin{gather*}
			\So(n):\begin{cases}
					A^{t}A=I\\
					\det A=1
				\end{cases}, \ \
			U(n): A^{t}\overline{A}=I, \ \
			\Su(n): \begin{cases}
				A^{t}\overline{A}=I\\
				\det A=1
			\end{cases}
		\end{gather*}
	Siccome $\Su(n)\subseteq \U(n)\subseteq \So(n)$ basta dimostrare che $\So(n)$ è limitato:
		\begin{gather*}
			A\in\So(n) \implies \sum_{i=1}^n a_{ij}^2=1, \forall i=1,\dots,n \implies \sum_{i=1}^n a_{ij}^2=n \implies \So(n)\subseteq S_{\sqrt{n}}\subseteq\realset^{n\times n}
		\end{gather*}
	dove $S_{\sqrt{n}}$ è la sfera di raggio $\sqrt{n}$, dunque $\So(n)$ è limitato. Ne segue che anche $\U(n)$ e $\Su(n)$ lo sono, dunque sono tutti chiusi e limitati in $\realset^{n\times n}$, quindi compatti. \newline
	Per dimostrare che sono \textit{connessi} si procede analogamente al teorema precedente sfruttando il lemma che lo precede. Si consideri $\funz p {\So(n)} {S^{n-1}\subseteq \realset^n}$ funzione prima colonna, essa è continua, suriettiva e chiusa in quanto è un compatto in un \textbf{Hausdorff}, e le sue fibra sono connesse:
	\begin{equation*}
		p^{-1}\left( \begin{psmallmatrix} 1 \\ 0 \\ \vdots \\ 0 \end{psmallmatrix} \right) = \begin{psmallmatrix}
			1      & 0 & \cdots & 0\\
			0      &   &       &  \\
			\vdots &   & A     &  \\
			0      &   &       &
		\end{psmallmatrix}
	\end{equation*}
Con $A\in \So(n-1)$, dunque per il lemma precedente $\So(n)$ è connesso.
\end{demonstration}

\begin{observe}
	$\gl$ e $\Sl$ \textit{non} sono compatti perché non sono limitati, inoltre $\gl$ è aperto e non chiuso.
\end{observe}
