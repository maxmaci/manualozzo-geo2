% SVN info for this file
\svnidlong
{$HeadURL$}
{$LastChangedDate$}
{$LastChangedRevision$}
{$LastChangedBy$}

\chapter{Note aggiuntive}
\labelChapter{footnotes}

\begin{introduction}
‘‘BEEP BOOP INSERIRE CITAZIONE QUA BEEP BOOP.''
\begin{flushright}
	\textsc{NON UN ROBOT,} UN UMANO IN CARNE ED OSSA BEEP BOOP.
\end{flushright}
\end{introduction}

\noindent Riportiamo alcune note e dimostrazioni aggiuntive che possono risultare utili al lettore.

\section{Capitolo 6: successioni}
La seguente dimostrazione sulla non prima-numerabilità del quoziente $\nicefrac{\realset}{\integerset}$ è adattata da Brian M. Scott \cite{scott:nonum} su Mathematics Stack Exchange.
\begin{demonstration}\label{dimostrazionenonnumerabilità}
Si consideri la contrazione di $\integerset$ in $\realset$ ad un punto, cioè il quoziente $\nicefrac{\realset}{\integerset}$ e si definisca la classe di equivalenza degli interi come $[0]$.\\
Sia $\left\{U_n:n\in\naturalset\right\}$ una famiglia di intorni aperti di $[0]$; cerchiamo un intorno aperto di $[0]$ che non ne contiene nessuno come sottoinsieme, mostrano in tal modo che non formano un sistema fondamentale di intorni di $[0]$ e pertanto che $\nicefrac{\realset}{\integerset}$ non è primo-numerabile per $[0]$.\\
Sia $\pi$ la mappa quoziente. Per ogni $n\in\naturalset$ e $k\in\integerset$ esiste un $\epsilon_{n,k}\in(0,1)$ tale che: 
\begin{equation*}
U_n\supseteq \pi\left[\inter_{k\in\integerset}(k-\epsilon_{n,k},k+\epsilon_{n,k})\right]\
\end{equation*}
Per $k\in\integerset$ sia $\delta_k=\frac12\epsilon_{k,k}$, e sia:
\begin{equation*}
V=\pi\left[\bigcup_{k\in\integerset}(k-\delta_k,k+\delta_k)\right]
\end{equation*}
Chiaramente $V$ è un intorno aperto di $[0]$, e vogliamo dimostrare che $U_n\nsubseteq V$ per ogni $n\in\naturalset$. Per mostrare ciò, fissiamo $n\in\naturalset$; si ha $\delta_n<\epsilon_{n,n}$, quindi possiamo sceglie un numero reale $x\in(n+\delta_n,n+\epsilon_{n,n})$. Ma allora $\pi(x)\in U_n\setminus V$, e dunque $U_n\nsubseteq V$.
\end{demonstration}
\section{Capitolo 11: forma canonica di Jordan}
La seguente dimostrazione sul determinante di una matrice a blocchi e del suo polinomio caratteristico si basa su integrazioni proprie da Jean Marie \cite{jeanmarie:blockdet} e da Ben Grossmann \cite{bengrossman:blockdet} su Mathematics Stack Exchange. \label{dimostrazionedeterminantematriceblocchi}
\begin{demonstration}
	Data una matrice quadrata $\left(
	\begin{array}{c|c}
		\mathbf{A} & \mathbf{B}\\
		\hline
		\mathbf{C} & \mathbf{D}
	\end{array}
	\right)$
	definita dai blocchi $\mathbf{A}$ di dimensione $n\times n$, $\mathbf{B}$ di dimensione $n\times m$, $\mathbf{C}$ di dimensione $m\times n$ e $\mathbf{D}$ di dimensione $m\times m$. Supponendo $\mathbf{A}$ blocco invertibile, si può scomporre la matrice nel seguente modo:
	\begin{equation}
		\left(
		\begin{array}{c|c}
			\mathbf{A} & \mathbf{B}\\
			\hline
			\mathbf{C} & \mathbf{D}
		\end{array}
		\right)=\left(
		\begin{array}{c|c}
			\mathbf{I_n} & \mathbf{0}\\
			\hline
			\mathbf{CA^{-1}} & \mathbf{I_m}
		\end{array}
		\right)\left(
		\begin{array}{c|c}
			\mathbf{A} & \mathbf{0}\\
			\hline
			\mathbf{0} & \mathbf{D-CA^{-1}B}
		\end{array}
		\right)
		\left(
		\begin{array}{c|c}
			\mathbf{I_n} & \mathbf{A^{-1}B}\\
			\hline
			\mathbf{0} & \mathbf{I_m}
		\end{array}
		\right)
	\end{equation}
	Calcoliamo il determinante di $\left(
	\begin{array}{c|c}
		\mathbf{A} & \mathbf{B}\\
		\hline
		\mathbf{C} & \mathbf{D}
	\end{array}
	\right)$, notando che $\left(
	\begin{array}{c|c}
		\mathbf{I_n} & \mathbf{0}\\
		\hline
		\mathbf{CA^{-1}} & \mathbf{I_m}
	\end{array}
	\right)$ e $\left(
	\begin{array}{c|c}
		\mathbf{I_n} & \mathbf{A^{-1}B}\\
		\hline
		\mathbf{0} & \mathbf{I_m}
	\end{array}
	\right)$ sono triangolari con diagonale di $1$.
	\begin{equation*}
		\begin{array}{ll}
			\det\left(\begin{array}{c|c}
				\mathbf{A} & \mathbf{B}\\
				\hline
				\mathbf{C} & \mathbf{D}
			\end{array}\right)&=\det\left(\left(
		\begin{array}{c|c}
		\mathbf{I_n} & \mathbf{0}\\
		\hline
		\mathbf{CA^{-1}} & \mathbf{I_m}
	\end{array}
\right)\left(
\begin{array}{c|c}
\mathbf{A} & \mathbf{0}\\
\hline
\mathbf{0} & \mathbf{D-CA^{-1}B}
\end{array}
\right)
\left(
\begin{array}{c|c}
\mathbf{I_n} & \mathbf{A^{-1}B}\\
\hline
\mathbf{0} & \mathbf{I_m}
\end{array}
\right)\right)=\\&=
			\det\left(
			\begin{array}{c|c}
				\mathbf{I_n} & \mathbf{0}\\
				\hline
				\mathbf{CA^{-1}} & \mathbf{I_m}
			\end{array}
			\right)\det\left(
			\begin{array}{c|c}
				\mathbf{A} & \mathbf{0}\\
				\hline
				\mathbf{0} & \mathbf{D-CA^{-1}B}
			\end{array}
			\right)
			\det\left(
			\begin{array}{c|c}
				\mathbf{I_n} & \mathbf{A^{-1}B}\\
				\hline
				\mathbf{0} & \mathbf{I_m}
			\end{array}
			\right)=\\&=
\det\left(
\begin{array}{c|c}
	\mathbf{A} & \mathbf{0}\\
	\hline
	\mathbf{0} & \mathbf{D-CA^{-1}B}
\end{array}
\right)
		\end{array}
	\end{equation*}
Ci serve calcolare il determinante di una \textit{matrice diagonale a blocchi}. Presa allora:
\begin{equation*}
	\left(
	\begin{array}{c|c}
		\mathbf{P} & \mathbf{0}\\
		\hline
		\mathbf{0} & \mathbf{Q}
	\end{array}
	\right)
\end{equation*}
Possiamo riscriverla come:
\begin{equation*}
	\left(
	\begin{array}{c|c}
		\mathbf{P} & \mathbf{0}\\
		\hline
		\mathbf{0} & \mathbf{I_m}
	\end{array}
	\right)\left(
	\begin{array}{c|c}
		\mathbf{I_n} & \mathbf{0}\\
		\hline
		\mathbf{0} & \mathbf{Q}
	\end{array}
	\right)
\end{equation*}
Grazie alle formule di Laplace, possiamo calcolare il determinante delle due matrici sfruttando le matrici identità presenti. Ad esempio, sviluppando rispetto le righe o le colonne sulla seconda:
\begin{equation*}
\det\left(
\begin{array}{c|c}
	\mathbf{I_n} & \mathbf{0}\\
	\hline
	\mathbf{0} & \mathbf{Q}
\end{array}
\right)=1\cdot \det\left(
\begin{array}{c|c}
	\mathbf{I_{n-1}} & \mathbf{0}\\
	\hline
	\mathbf{0} & \mathbf{Q}
\end{array}
\right)=1\cdot 1\cdot \det\left(
\begin{array}{c|c}
	\mathbf{I_{n-2}} & \mathbf{0}\\
	\hline
	\mathbf{0} & \mathbf{Q}
\end{array}
\right)=\ldots = 1^n\cdot \det \mathbf{Q} =\det \mathbf{Q}
\end{equation*}
Il risultato è analogo per la prima. Dunque, concludendo:
\begin{gather}
\det\left(
\begin{array}{c|c}
	\mathbf{A} & \mathbf{B}\\
	\hline
	\mathbf{C} & \mathbf{D}
\end{array}
\right)=\det\left(\mathbf{A}\right)\det\left(\mathbf{D-CA^{-1}B}\right)\\
\det\left(
\begin{array}{c|c}
	\mathbf{A} & \mathbf{0}\\
	\hline
	\mathbf{0} & \mathbf{D}
\end{array}
\right)=\det\left(\mathbf{A}\right)\det\left(\mathbf{D}\right)
\end{gather}
Come ultima conseguenza, se vogliamo studiare il polinomio caratteristico $C_A\left(t\right)$ di una matrice $A$ a blocchi diagonali $\mathbf{B}$ e $\mathbf{C}$, abbiamo che:
\begin{equation}
C_A\left(t\right)=\det\left(
\begin{array}{c|c}
	\mathbf{B}-t I & \mathbf{0}\\
	\hline
	\mathbf{0} & \mathbf{C}-t I
\end{array}
\right)=\det\left(\mathbf{B}-t I\right)\det\left(\mathbf{C}-t I\right)=C_B\left(t\right)C_C\left(t\right)
\end{equation}
\end{demonstration}