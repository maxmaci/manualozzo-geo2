% SVN info for this file
\svnidlong
{$HeadURL$}
{$LastChangedDate$}
{$LastChangedRevision$}
{$LastChangedBy$}

\chapter{Note aggiuntive}
\labelAppendix{footnotes}
\addtocontents{define}{\noindent\textls{\textsc{\textcolor{reddo}{Appendice A:}
\nowtitle}}
}{}
\addtocontents{theorema}{\noindent\textls{\textsc{\textcolor{reddo}{Appendice A:}
			\nowtitle}}
}{}
\begin{introduction}
‘‘Le note a piè di pagina sono le superfici ingannatrici che permettono ai paragrafi tentacolari di aderire alla realtà più ampia della biblioteca.''
\begin{flushright}
	\textsc{Nicholson Baker,} bibliotecario di Cthulhu.
\end{flushright}
\end{introduction}

\noindent Riportiamo alcune note, precisazioni e dimostrazioni complementari agli argomenti dei capitoli principali che possono risultare utili al lettore
\section{Capitolo 1: spazi topologici}
\subsection{Alcune proprietà della continuità}
Le seguenti dimostrazioni sulle continuità di funzioni sono da \cite{munkres:2000topology}.
\begin{theorema}[Inclusione è funzione continua; Munkres, 18.2b.]~{}\\
	Se $Z\subseteq X$ è sottospazio, l'inclusione $\incl{i}{Z}{X}$ è una funzione continua.
\end{theorema}
\begin{demonstration}
	Se $A$ è un aperto in $X$, allora $i^{-1}\left(A\right)=A\cap Z$ è un aperto in $Z$ per definizione della topologia di sottospazio.
\end{demonstration}
\begin{theorema}[Restrizione di una funzione continua è continua; Munkres, 18.2d.]~{}\\
	Se $\funz{f}{X}{Y}$ è una funzione continua, allora $\funz{f_{\mid Z}}{Z}{Y}$ è continua per ogni sottospazio $Z\subseteq X$.
\end{theorema}
\begin{demonstration}
	La funzione $f_{\mid Z}$ è la composizione dell'inclusione $\incl{i}{Z}{X}$ con la funzione $\funz{f}{X}{Y}$. Poiché la composizione di funzioni continue è continua (teorema \ref{compfunzcont}, pag. \pageref{compfunzcont}), segue la tesi.
\end{demonstration}
Le seguente dimostrazioni sulle restrizioni di omeomorfismi è un'elaborazione personale sulle basi dei teoremi precedenti.
\begin{corollary}[Restrizione di un omeomorfismo è omeomorfismo.]~{}\\
Se $\funz{f}{X}{Y}$ è un omeomorfismo, allora $\funz{f_{\mid Z}}{Z}{f\left(Z\right)}$ è omeomorfismo per ogni sottospazio $Z\subseteq X$. In particolare, $\funz{f_{X\setminus Z}}{X\setminus Z}{Y\setminus f\left(Z\right)}$.
\end{corollary}
\begin{demonstration}
	La restrizione di una funzione è sempre una biezione e, per il teorema precedente, è anche continua. Ciò vale sia per $f$, sia per l'inversa $f^{-1}$ e segue dunque la prima tesi. La seconda parte del corollario vale perché, se $f$ è biettiva, allora:
	\begin{equation*}
		f\left(X\setminus Z\right)=f\left(X\right)\setminus f\left(Z\right)=Y\setminus f\left(Z\right)
	\end{equation*}
\vspace{-3mm}
\end{demonstration}
\begin{attention}
	Non vale il viceversa del teorema precedente: \textit{non è vero} che se $A,\ B\subseteq X$ sono omeomorfi, allora $X\setminus A$ e $X\setminus B$ sono omeomorfi!\\
	Preso ad esempio $A=S^2\setminus\left\{(0,\ 0,\ 1),\ (0,\ 0,\ -1)\right\}$ e $B=S^1\times \realset$. Si ha che $A$ e $B$ sono \textit{omeomorfi} (per proiezione dall'origine), ma $\realset^3\setminus A$ è connesso per archi, mentre $\realset^3\setminus B$ non è \textit{neppure connesso}.
\end{attention} 
\section{Capitolo 6: assiomi di numerabilità e successioni}
\subsection{Non prima numerabilità del quoziente}
La seguente dimostrazione sulla non prima numerabilità del quoziente $\nicefrac{\realset}{\integerset}$ è adattata da \cite{scott:nonum} su Mathematics Stack Exchange.
\begin{demonstration}\label{dimostrazionenonnumerabilità}
Si consideri la contrazione di $\integerset$ in $\realset$ ad un punto, cioè il quoziente $\nicefrac{\realset}{\integerset}$ e si definisca la classe di equivalenza degli interi come $[0]$.\\
Sia $\left\{U_n:n\in\naturalset\right\}$ una famiglia di intorni aperti di $[0]$; cerchiamo un intorno aperto di $[0]$ che non ne contiene nessuno come sottoinsieme, mostrano in tal modo che non formano un sistema fondamentale di intorni di $[0]$ e pertanto che $\nicefrac{\realset}{\integerset}$ non è primo numerabile per $[0]$.\\
Sia $\pi$ la mappa quoziente. Per ogni $n\in\naturalset$ e $k\in\integerset$ esiste un $\epsilon_{n,k}\in(0,1)$ tale che: 
\begin{equation*}
U_n\supseteq \pi\left[\inter_{k\in\integerset}(k-\epsilon_{n,k},k+\epsilon_{n,k})\right]\
\end{equation*}
Per $k\in\integerset$ sia $\delta_k=\frac12\epsilon_{k,k}$, e sia:
\begin{equation*}
V=\pi\left[\bigcup_{k\in\integerset}(k-\delta_k,k+\delta_k)\right]
\end{equation*}
Chiaramente $V$ è un intorno aperto di $[0]$, e vogliamo dimostrare che $U_n\nsubseteq V$ per ogni $n\in\naturalset$. Per mostrare ciò, fissiamo $n\in\naturalset$; si ha $\delta_n<\epsilon_{n,n}$, quindi possiamo sceglie un numero reale $x\in(n+\delta_n,n+\epsilon_{n,n})$. Ma allora $\pi(x)\in U_n\setminus V$, e dunque $U_n\nsubseteq V$.
\end{demonstration}
\section{Capitolo 8: gruppo fondamentale}
\subsection{Funzioni iniettive e sottogruppi}\label{isomorfismosottogruppo}
La seguente dimostrazione è adattata da \cite{HagenVonEitzen:injectivesubgroup} su Mathematics Stack Exchange.
\begin{lemming}[Funzione iniettiva fra gruppi implica isomorfismo con sottogruppo.]~{}\\
Dati due gruppi $G$ e $H$, se $\funz{f}{G}{H}$ è omomorfismo \textit{iniettivo} allora $G$ è \textit{isomorfo} al sottogruppo $K=f\left(G\right)$ di $H$.
\end{lemming}
\begin{demonstration}
	Banalmente, preso $K=f\left(G\right)$ abbiamo ristretto l'omeomorfismo iniettivo alla sua immagine, rendendolo suriettivo e dunque biettivo.\\
	Per verificare che $K$ è sottogruppo usiamo ora il criterio seguente: un sottoinsieme $K$ di $H$ è un sottogruppo se non è vuoto e $a,\ b\in K\implies ab^{-1}\in K$. Poiché $G$ è non vuoto, $K=f\left(G\right)$ non è vuoto. Presi allora $a,\ b\in K$, troviamo $x,\ y\in G$ tali che $f\left(x\right)=a,\ f\left(y\right)=b$. Allora:
	\begin{equation*}
		ab^{-1}=f\left(x\right)f\left(y\right)^{-1}f\left(xy^{-1}\right)\in K
	\end{equation*}
\vspace{-6mm}
\end{demonstration}
\subsection{Gruppi liberi}\label{gruppolibero}
\begin{define}[Parola.]~{}\\
	Dato un gruppo $G$, una \textbf{parola}\index{parola} è un qualunque prodotto di elementi del gruppo e dei loro inversi.
\end{define}
Se $a,\ b,\ c$ sono elementi del gruppo $G$, alcune possibili parole sono $ab$, $abca^{-1}$, $ccb^{-1}ab$.  Due parole sono considerate \textit{distinte} se non possiamo ricondurci dall'una all'altra attraverso gli assiomi di gruppi: ad esempio, $ab=acc^{-1}a$, ma $a\neq b^{-1}$. Una parola può essere semplificata in due modi differenti:
\begin{itemize}
	\item Togliendo l'elemento neutro $e$ o una coppia di elementi adiacenti $aa^{-1}$ o $a^{-1}a$.
	\item Sostituendo ad una coppia $ab$ il loro prodotto $d$ in $G$ oppure ad una serie di $k$ termini $a\ldots a$ la potenza $a^k$.
\end{itemize}
Una parola che non può essere semplificata ulteriormente è detta \textbf{ridotta}\index{parola!ridotta}.
\begin{define}[Gruppo libero.]~{}\\
Il \textbf{gruppo libero}\index{gruppo!libero} $F_S$ su un insieme $S$ è il gruppo i cui elementi sono tutte le parole ridotte date dagli elementi di $S$; l'operazione è la \textbf{concatenazione}\index{concatenazione} di parole e l'elemento neutro è la \textit{parola vuota} (la parola senza alcun elemento di $S$).\\
Gli elementi di $S$ sono detti \textbf{generatori}\index{generatori} e il numero di generatori è il \textbf{rango} del gruppo libero.\\
Un gruppo $G$ è detto \textbf{libero} se isomorfo al gruppo libero $F_S$ generato dal sottoinsieme $S\subseteq G$.
\end{define}
\begin{example}
	$\left(\integerset,\ +\right)$ è un gruppo libero di rango 1, generato da, ad esempio, $S=\{1\}$. Esso è un gruppo libero \textit{abeliano}.
\end{example}
\begin{observes}~{}
	\begin{itemize}
		\item Ogni gruppo finito di rango $\geq 2$ è \textit{non} abeliano.
		\item Ogni gruppo non triviale \textit{finito} non può essere libero, in quanto gli elementi dell'insieme $S$ generante $F_S$ hanno ordine infinito. 
	\end{itemize}
\vspace{-3mm}
\end{observes}
\begin{define}[Prodotto libero.]~{}\\
	Dati due gruppi $G$ e $H$, il \textbf{prodotto libero}\index{prodotto!libero} $G\ast H$ è il gruppo i cui elementi sono le parole ridotte date da generatori $g_i\in G$ e in $h_j\in H$, ad esempio $g_1h_1\ldots g_k h_k$.
\end{define}
\begin{example}~{}
		\begin{itemize}
		\item Se $G=\left<a\right>$ e $H=\left<b\right>$ sono gruppi ciclici infiniti, ogni elemento di $G\ast H$ è dato da prodotti alternati di potenze di $a$ e potenze di $b$, cioè $G\ast H\cong F_S$ con $S=\{x,\ y\}$. Alcune parole sono, ad esempio, $a^3b^5a^{-1}b$ o $b^{-2}ab^{-3}$.
		\item Preso $\left(\integerset,\ +\right)$, $\integerset\ast\integerset$ è un gruppo libero di rango 2 come l'esempio precedente: infatti, ogni gruppo ciclico infinito è isomorfo a $\integerset$, dunque ci riconduciamo all'esempio precedente. Possiamo quindi scrivere $\integerset\ast\integerset=\left<a\right>\ast\left<b\right>$, indicando $\{a\}$ come il generatore $\{\textcolor{redill}{1}\}$ del primo $\integerset$ e $\{b\}$ come il generatore $\{\textcolor{blueill}{1}\}$ del secondo $\integerset$.\\
		Alcune parole sono, ad esempio, $a^3b^5a^{-1}b=(\textcolor{redill}{3})(\textcolor{blueill}{5})(\textcolor{redill}{-1})(\textcolor{blueill}{1})$ o $b^{-2}ab^{-3}=(\textcolor{blueill}{-2})(\textcolor{redill}{1})(\textcolor{blueill}{-3})$.
	\end{itemize}
\vspace{-3mm}
\end{example}
\section{Somma wedge}
\begin{define}[Somma wedge.]~{}\\
	Se $\left(X,\ x_0\right)$ e $\left(X,\ y_0\right)$ sono due spazi topologici di cui prendiamo i punti $x_0\in X$ e $y_0\in Y$. La somma wedge di $X$ e $Y$ è lo spazio quoziente dell'unione disgiunta di $X$ e $Y$ in cui identifichiamo solo i due punti, cioè poniamo $x_0\sim y_0$:
	\begin{equation}
		X\vee Y=\frac{X\amalg Y}{\sim}
	\end{equation}
Nel caso di una famiglia di spazi $\left\{\left(X_i,\ x_i\right)\right\}_{i\in I}$, la relazione $\sim$ è tale per cui $\left\{x_i\right\}_{i\in I}$ sono tutti identificati fra di loro; la somma wedge della famiglia è definita come:
\begin{equation*}
	\bigvee_{i\in I} X_i = \frac{\coprod_{i\in I}X_i}{\sim}
\end{equation*}
\end{define}
\begin{example}
	Il \textit{bouquet} di $n$ circonferenze è una somma wedge di $n$ circonferenze.
\end{example}
In questa sede non approfondiamo ulteriormente, ma è interessante sapere come con certi spazi topologici che si comportano ‘‘bene'' il \textit{teorema di Van Kampen} ci dà condizioni per cui il gruppo fondamentale di $X\vee Y$ è il gruppo libero dei gruppi fondamentali di $X$ e $Y$:
\begin{equation}
	\gruf{X\vee Y}{\ }=\gruf{X}{\ }\ast\gruf{Y}{\ }
\end{equation}
\begin{example}
	Il \textit{bouquet} di $2$ circonferenze $S^1\vee S^1$ ha gruppo fondamentale $\gruf{S^1\vee S^1}{\ }=\gruf{S^1}{\ }\ast\gruf{S^1}{\ }$  $\gruf{S^1\vee S^1}{\ }=\integerset\ast\integerset$.
\end{example}
\section{Capitolo 10: approfondimenti di Algebra Lineare}
\subsection{Determinante di una matrice a blocchi}
La seguente dimostrazione sul determinante di una matrice a blocchi e del suo polinomio caratteristico si basa su integrazioni proprie da \cite{jeanmarie:blockdet} e da \cite{bengrossman:blockdet} su Mathematics Stack Exchange. \label{dimostrazionedeterminantematriceblocchi}
\begin{demonstration}
	Data una matrice quadrata $\left(
	\begin{array}{c|c}
		\mathbf{A} & \mathbf{B}\\
		\hline
		\mathbf{C} & \mathbf{D}
	\end{array}
	\right)$
	definita dai blocchi $\mathbf{A}$ di dimensione $n\times n$, $\mathbf{B}$ di dimensione $n\times m$, $\mathbf{C}$ di dimensione $m\times n$ e $\mathbf{D}$ di dimensione $m\times m$. Supponendo $\mathbf{A}$ blocco invertibile, si può scomporre la matrice nel seguente modo:
	\begin{equation}
		\left(
		\begin{array}{c|c}
			\mathbf{A} & \mathbf{B}\\
			\hline
			\mathbf{C} & \mathbf{D}
		\end{array}
		\right)=\left(
		\begin{array}{c|c}
			\mathbf{I_n} & \mathbf{0}\\
			\hline
			\mathbf{CA^{-1}} & \mathbf{I_m}
		\end{array}
		\right)\left(
		\begin{array}{c|c}
			\mathbf{A} & \mathbf{0}\\
			\hline
			\mathbf{0} & \mathbf{D-CA^{-1}B}
		\end{array}
		\right)
		\left(
		\begin{array}{c|c}
			\mathbf{I_n} & \mathbf{A^{-1}B}\\
			\hline
			\mathbf{0} & \mathbf{I_m}
		\end{array}
		\right)
	\end{equation}
	Calcoliamo il determinante di $\left(
	\begin{array}{c|c}
		\mathbf{A} & \mathbf{B}\\
		\hline
		\mathbf{C} & \mathbf{D}
	\end{array}
	\right)$, notando che $\left(
	\begin{array}{c|c}
		\mathbf{I_n} & \mathbf{0}\\
		\hline
		\mathbf{CA^{-1}} & \mathbf{I_m}
	\end{array}
	\right)$ e $\left(
	\begin{array}{c|c}
		\mathbf{I_n} & \mathbf{A^{-1}B}\\
		\hline
		\mathbf{0} & \mathbf{I_m}
	\end{array}
	\right)$ sono triangolari con diagonale di $1$.
	\begin{equation*}
		\begin{array}{ll}
			\det\left(\begin{array}{c|c}
				\mathbf{A} & \mathbf{B}\\
				\hline
				\mathbf{C} & \mathbf{D}
			\end{array}\right)&=\det\left(\left(
		\begin{array}{c|c}
		\mathbf{I_n} & \mathbf{0}\\
		\hline
		\mathbf{CA^{-1}} & \mathbf{I_m}
	\end{array}
\right)\left(
\begin{array}{c|c}
\mathbf{A} & \mathbf{0}\\
\hline
\mathbf{0} & \mathbf{D-CA^{-1}B}
\end{array}
\right)
\left(
\begin{array}{c|c}
\mathbf{I_n} & \mathbf{A^{-1}B}\\
\hline
\mathbf{0} & \mathbf{I_m}
\end{array}
\right)\right)=\\&=
			\det\left(
			\begin{array}{c|c}
				\mathbf{I_n} & \mathbf{0}\\
				\hline
				\mathbf{CA^{-1}} & \mathbf{I_m}
			\end{array}
			\right)\det\left(
			\begin{array}{c|c}
				\mathbf{A} & \mathbf{0}\\
				\hline
				\mathbf{0} & \mathbf{D-CA^{-1}B}
			\end{array}
			\right)
			\det\left(
			\begin{array}{c|c}
				\mathbf{I_n} & \mathbf{A^{-1}B}\\
				\hline
				\mathbf{0} & \mathbf{I_m}
			\end{array}
			\right)=\\&=
\det\left(
\begin{array}{c|c}
	\mathbf{A} & \mathbf{0}\\
	\hline
	\mathbf{0} & \mathbf{D-CA^{-1}B}
\end{array}
\right)
		\end{array}
	\end{equation*}
Ci serve calcolare il determinante di una \textit{matrice diagonale a blocchi}. Presa allora:
\begin{equation*}
	\left(
	\begin{array}{c|c}
		\mathbf{P} & \mathbf{0}\\
		\hline
		\mathbf{0} & \mathbf{Q}
	\end{array}
	\right)
\end{equation*}
Possiamo riscriverla come:
\begin{equation*}
	\left(
	\begin{array}{c|c}
		\mathbf{P} & \mathbf{0}\\
		\hline
		\mathbf{0} & \mathbf{I_m}
	\end{array}
	\right)\left(
	\begin{array}{c|c}
		\mathbf{I_n} & \mathbf{0}\\
		\hline
		\mathbf{0} & \mathbf{Q}
	\end{array}
	\right)
\end{equation*}
Grazie alle formule di Laplace, possiamo calcolare il determinante delle due matrici sfruttando le matrici identità presenti. Ad esempio, sviluppando rispetto le righe o le colonne sulla seconda:
\begin{equation*}
\det\left(
\begin{array}{c|c}
	\mathbf{I_n} & \mathbf{0}\\
	\hline
	\mathbf{0} & \mathbf{Q}
\end{array}
\right)=1\cdot \det\left(
\begin{array}{c|c}
	\mathbf{I_{n-1}} & \mathbf{0}\\
	\hline
	\mathbf{0} & \mathbf{Q}
\end{array}
\right)=1\cdot 1\cdot \det\left(
\begin{array}{c|c}
	\mathbf{I_{n-2}} & \mathbf{0}\\
	\hline
	\mathbf{0} & \mathbf{Q}
\end{array}
\right)=\ldots = 1^n\cdot \det \mathbf{Q} =\det \mathbf{Q}
\end{equation*}
Il risultato è analogo per la prima. Dunque, concludendo:
\begin{gather}
\det\left(
\begin{array}{c|c}
	\mathbf{A} & \mathbf{B}\\
	\hline
	\mathbf{C} & \mathbf{D}
\end{array}
\right)=\det\left(\mathbf{A}\right)\det\left(\mathbf{D-CA^{-1}B}\right)\\
\det\left(
\begin{array}{c|c}
	\mathbf{A} & \mathbf{0}\\
	\hline
	\mathbf{0} & \mathbf{D}
\end{array}
\right)=\det\left(\mathbf{A}\right)\det\left(\mathbf{D}\right)
\end{gather}
Come ultima conseguenza, se vogliamo studiare il polinomio caratteristico $C_A\left(t\right)$ di una matrice $A$ a blocchi diagonali $\mathbf{B}$ e $\mathbf{C}$, abbiamo che:
\begin{equation}
C_A\left(t\right)=\det\left(
\begin{array}{c|c}
	\mathbf{B}-t I & \mathbf{0}\\
	\hline
	\mathbf{0} & \mathbf{C}-t I
\end{array}
\right)=\det\left(\mathbf{B}-t I\right)\det\left(\mathbf{C}-t I\right)=C_B\left(t\right)C_C\left(t\right)
\end{equation}
\end{demonstration}
\subsection{Convergenza uniforme}\label{convergenzauniforme}
Tutti i ragionamenti qui presenti si applicano anche alle successioni e serie di potenze.
\begin{define}[Convergenza uniforme.]~{}\\
	Dato un insieme $E$ e un successione $\left(f_n\right)_{n\in\naturalset}$ con $\funz{f_n}{E}{X}$ con $X$ metrico, si dice che la successione è uniformemente convergente \index{convergenza!uniforme} su $E$ con limite $\funz{f}{E}{X}$ se:
	\begin{equation}
		\forall\epsilon>0\ \exists N\in\naturalset\ \colon \forall n\geq N,\ x\in E \mvf{d}{f_n\left(x\right)}{f\left(x\right)}<\epsilon
	\end{equation}
\vspace{-6mm}
\end{define}
\begin{theorema}[Criterio di Weierstrass o M-test.]~{}\\
	Sia $\left(f_n\right)_{n\in\naturalset}$ una successione di funzioni \textit{reali} o \textit{complesse} definite su un insieme $A$ e che esista una successione di numeri \textit{non negativi} $\left(M_n\right)_{n\in\naturalset}$ che soddisfino la seguente relazione:
	\begin{equation}
		\forall n\geq 1,\ x\in A\ \colon \lvert f_n\left(x\right)\rvert \leq M_n,\ \sum_{n=1}^{\infty}M_n<\infty
	\end{equation}
Allora la serie:
\begin{equation}
	\sum_{n=1}^{\infty}f_n
\end{equation}
Converge \textit{assolutamente} e \textit{uniformemente} su $A$
\end{theorema}
Si usa spesso l'\textit{M-test} assieme al \textbf{teorema del limite uniforme}.
\begin{theorema}[Teorema del limite uniforme.]~{}\\
Sia $\left(f_n\right)_{n\in\naturalset}$ una successione di funzioni \textit{reali} o \textit{complesse} continue sullo spazio topologico $A$ nel quale sono definite; se la successione converge uniformemente su $A$ allora il limite converge ad una funzione continua. In particolare, lo stesso si ha nel caso di una serie.
\end{theorema}
\section{Capitolo 11: geometria proiettiva}
\subsection{Regola di Cramer}\label{Cramerrimembriancor}
\begin{theorema}[Regola di Cramer.]~{}\\
	Si consideri un sistema $A\mathbf{x}=\mathbf{b}$ di $n$ equazioni lineari in $n$ incognite, con $\det A \neq 0$. Il sistema ha un unica soluzione $\mathbf{x}$, le cui componenti sono:
	\begin{equation}
		x_i=\frac{\det A_i}{\det A}\quad i=1,\ \ldots,\ n
	\end{equation}
	Con $A_i$ la matrice ottenuta sostituendo la $i$-esima colonna di $A$ col vettore $\mathbf{b}$.
\end{theorema}
Ad esempio, dato il sistema lineare in 2 equazioni e 2 incognite (in cui $a_{11}a_{22}-a_{12}a_{21}\neq 0$):
\begin{equation*}
	\begin{cases}
		a_{11}x_1+a_{12}x_2=b_1\\
		a_{21}x_1+a_{22}x_2=b_2
	\end{cases}\iff
\left(\begin{array}{cc}
	a_{11} & a_{12} \\
	a_{21} & a_{22}
\end{array}\right)\left(\begin{array}{c}
x_1 \\
x_2
\end{array}\right)=\left(\begin{array}{c}
\textcolor{redill}{b_1} \\
\textcolor{redill}{b_2}
\end{array}\right)
\end{equation*}
Possiamo trovare $x_1$ e $x_2$ con la regola di Cramer:
\begin{equation*}
	x_1=\frac{\left|\begin{array}{cc}
			\textcolor{redill}{b_1} & a_{12} \\
			\textcolor{redill}{b_2} & a_{22}
		\end{array}\right|}{\left|\begin{array}{cc}
		a_{11} & a_{12} \\
		a_{21} & a_{22}
	\end{array}\right|}\quad x_2=\frac{\left|\begin{array}{cc}
	 a_{11} & \textcolor{redill}{b_1} \\
	 a_{21} & \textcolor{redill}{b_2}
\end{array}\right|}{\left|\begin{array}{cc}
a_{11} & a_{12} \\
a_{21} & a_{22}
\end{array}\right|}
\end{equation*}