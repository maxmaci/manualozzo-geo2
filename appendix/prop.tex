% SVN info for this file
\svnidlong
{$HeadURL$}
{$LastChangedDate$}
{$LastChangedRevision$}
{$LastChangedBy$}

\chapter{Proprietà varie ed eventuali}
\labelAppendix{proprietà}

\begin{introduction}
‘‘BEEP BOOP INSERIRE CITAZIONE QUA BEEP BOOP.''
\begin{flushright}
	\textsc{NON UN ROBOT,} UN UMANO IN CARNE ED OSSA BEEP BOOP.
\end{flushright}
\end{introduction}

\noindent Riportiamo alcune proprietà utili per il lettore.
\section{Immagine e controimmagine}
Data una funzione $\funz{f}{X}{Y}$, per ogni sottoinsieme $A\subseteq X$ e $B\subseteq Y$ valgono le seguenti proprietà:
\begin{center}
	\begin{tabular}{l|l}
	\multicolumn{1}{c|}{\textbf{Immagine}} 	& \multicolumn{1}{c}{\textbf{Controimmagine}}\\ \hline
	$f(X)\subseteq Y$				& $f^{-1}(Y) = X$ 	\\ 
	$f(f^{-1}(Y)) = f(X)$			& $f^{-1}(f(X))= X$ \\
	$f(f^{-1}(B)) \subseteq B$ \footnotemark	& $f^{-1}(f(A)) \supseteq A$ \footnotemark	\\
	$f(f^{-1}(B)) = B \cap f(X)$	& $(f \mid_A)^{-1}(B) = A \cap f^{-1}(B)$											\\
	$f(f^{-1}(f(A))) = f(A)$		& $f^{-1}(f(f^{-1}(B))) = f^{-1}(B)$											\\
	$f(A) = \emptyset \iff A = \emptyset$	&  $f^{-1}(B) = \emptyset \iff B \subseteq Y \setminus f(X)$					\\
	$f(A) \supseteq B \iff \exists C \subseteq A \colon f(C) = B$ & $f^{-1}(B) \supseteq A \iff f(A) \subseteq B$\\
	$f(A) \supseteq f(X \setminus A) \iff f(A) = f(X)$ & $f^{-1}(B) \supseteq f^{-1}(Y \setminus B) \iff f^{-1}(B) = X$\\
	$f(X \setminus A) \supseteq f(X) \setminus f(A)$ & $f^{-1}(Y \setminus B) = X \setminus f^{-1}(B)$ \\
	$f(A \cup f^{-1}(B)) \subseteq f(A) \cup B$ & $f^{-1}(f(A) \cup B) \supseteq A \cup f^{-1}(B)$ \\
	$f(A \cap f^{-1}(B)) = f(A) \cap B$ & $f^{-1}(f(A) \cap B) \supseteq A \cap f^{-1}(B)$ \\
	\multicolumn{2}{c}{$f(A) \cap B = \emptyset \iff A \cap f^{-1}(B) = \emptyset$}
\end{tabular}
\footnotetext{Uguale se $B\subseteq f(X)$, cioè se $f$ \textit{suriettiva}.}
\footnotetext{Uguale se $f$ \textit{iniettiva}.}	
\end{center}

Date le funzioni $\funz{f}{X}{Y}$ e $\funz{f}{Y}{Z}$, valgono le seguenti proprietà:
\begin{itemize}
	\item $(g \circ f)(A) = g(f(A))$
	\item $(g \circ f)^{-1}(C) = f^{-1}(g^{-1}(C))$
\end{itemize}
Data una funzione $\funz{f}{X}{Y}$ e dati i sottoinsiemi $A_1,\ A_2\subseteq X$ e $B_1,\ B_2\subseteq Y$ valgono le seguenti proprietà:
\begin{center}
	\begin{tabular}{l|l}
		\multicolumn{1}{c|}{\textbf{Immagine}} 	& \multicolumn{1}{c}{\textbf{Controimmagine}}\\ \hline
		$A_1 \subseteq A_2 \implies f(A_1) \subseteq f(A_2)$	& $B_1 \subseteq B_2 \implies f^{-1}(B_1) \subseteq f^{-1}(B_2)$ 	\\ 
		$f(A_1 \cup A_2) = f(A_1) \cup f(A_2)$ \footnotemark	& $f^{-1}(B_1 \cup B_2) = f^{-1}(B_1) \cup f^{-1}(B_2)$ \\
		$f(A_1 \cap A_2) \subseteq f(A_1) \cap f(A_2)$ \footnotemark	& $f^{-1}(B_1 \cap B_2) = f^{-1}(B_1) \cap f^{-1}(B_2)$	\\
		$f(A_1 \setminus A_2) \supseteq f(A_1) \setminus f(A_2)$ & $f^{-1}(B_1 \setminus B_2) = f^{-1}(B_1) \setminus f^{-1}(B_2)$
	\end{tabular}
\footnotetext{\label{note1}Uguale se $f$ \textit{iniettiva}.}
\footnotetext{Si veda \ref{note1}.}
\end{center}
Data una funzione $\funz{f}{X}{Y}$ e date le famiglie di sottoinsiemi $\{A_i\}\subseteq\setpart{X}$ e $\{B_i\}\subseteq\setpart{Y}$ (con $I$ un insieme di indici anche \textit{infinito} o \textit{non numerabile}) valgono le seguenti proprietà:
\begin{center}
	\begin{tabular}{l|l}
		\multicolumn{1}{c|}{\textbf{Immagine}} 	& \multicolumn{1}{c}{\textbf{Controimmagine}}\\ \hline
		$\displaystyle f\left(\bigcup_{i\in I}A_i\right) = \bigcup_{i\in I} f(A_i)$	& $\displaystyle f^{-1}\left(\bigcup_{i\in I}B_i\right) = \bigcup_{i\in I} f^{-1}(B_i)$ 	\\ 
		$\displaystyle f\left(\bigcap_{i\in I}A_i\right) \subseteq \bigcap_{i\in I} f(A_i)$\footnotemark	& $\displaystyle f^{-1}\left(\bigcap_{i\in I}B_i\right) = \bigcap_{i\in I} f^{-1}(B_i)$
	\end{tabular}
\footnotetext{Si veda \ref{note1}.}
\end{center}
\section{Invarianti topologici più comuni}
\begin{multicols}{2}
	\textls{\textsc{\textcolor{reddo}{Cardinalità}}}
	\begin{itemize}
		\item \textbf{Cardinalità} $\vert X\vert$ dello spazio $X$.
		\item \textbf{Cardinalità} $\vert \topo\vert$ della topologia.
	\end{itemize}
\textls{\textsc{\textcolor{reddo}{Assiomi di separazione}}}
\begin{itemize}
	\item \textbf{T1}: per ogni coppia di punti distinti $x$ e $y$ c'è un intorno aperto contenente $x$ ma non $y$; alternativamente, tutti i punti sono chiusi.
	\item \textbf{T2} o \textbf{Hausdorff}: per ogni coppia di punti distinti $x$ e $y$ esistono due intorni disgiunti. Spazi di \textbf{Hausdorff} sono sempre \textbf{T1}.
	\item \textbf{Discreta}: ogni sottoinsieme è aperto.
\end{itemize}
\textls{\textsc{\textcolor{reddo}{Assiomi di numerabilità}}}
\begin{itemize}
	\item \textbf{Separabile}: lo spazio contiene un sottoinsieme denso numerabile.
	\item \textbf{Primo numerabile}: ogni punto ha un sistema fondamentale di intorni numerabile.
	\item \textbf{Secondo numerabile}: la topologia ha una base di cardinalità numerabile. Spazi secondo numerabili sono sempre separabili e primi numerabili.
\end{itemize}
\textls{\textsc{\textcolor{reddo}{Connessione}}}
\begin{itemize}
	\item \textbf{Connesso}: gli unici aperti e chiusi sono il vuoto e lo spazio stesso; alternativamente, lo spazio \textit{non} è l'unione di una coppia di aperti/chiusi non vuoti e disgiunti.
	\item \textbf{Connesso per archi}: ogni punto ha un sistema fondamentale di intorni numerabile.
	\item \textbf{Connesso per archi}: per ogni coppia di punti $x,\ y$ c'è un cammino $\funz{\alpha}{\left[0,\ 1\right]}{X}$ (con $\alpha\left(0\right)=x$ e $\alpha\left(1\right)=y$). Spazi \textbf{c.p.a.} sono sempre connessi.
	\item \textbf{Semplicemente connesso}: spazio \textbf{c.p.a.} il cui gruppo fondamentale è banale.
	\item \textbf{Contraibile}: spazio con tipo di omotopia di un punto; alternativamente, l'identità $Id_X$ è omotopica ad una funzione costante. Ogni spazio contraibile è semplicemente connesso.
\end{itemize}
\textls{\textsc{\textcolor{reddo}{Compattezza}}}
\begin{itemize}
	\item \textbf{Compatto}: ogni ricoprimento aperto ammette un sottoricoprimento finito.
	\item \textbf{Compatto per successioni}: ogni successione ha una sottosuccessione convergente.
\end{itemize}
\textls{\textsc{\textcolor{reddo}{Metrica}}}
\begin{itemize}
	\item \textbf{Metrico}: la topologia è indotta da una distanza; alternativamente, è omeomorfo ad uno spazio metrico.
\end{itemize}
\end{multicols}
\section{Invarianti omotopici più comuni}
Tutti gli invarianti omotopici sono anche invarianti per omeomorfismi.
\begin{multicols}{2}
\begin{itemize}
	\item \textbf{Connesso per archi}.
	\item \textbf{Semplicemente connesso}.
	\item Se gli spazi sono \textbf{c.p.a.}, i \textbf{gruppi fondamentali} sono \textit{isomorfi}. Se non sono \textit{c.p.a.}, $\gruf{X}{x_0}$ è omeomorfo a $\pi{Y}{y_0}$ con $\funz{f}{\left(X,\ x_0\right)}{\left(Y,\ f\left(x_0\right)\right)}$ equivalenza omotopica.
\end{itemize}
\end{multicols}
\section{Invarianti proiettivi}
\begin{multicols}{2}
\begin{itemize}
	\item \textbf{Birapporto di quattro punti}.
	\item \textbf{Grado di una conica}.
	\item \textbf{Irriducibilità di una conica}.
	\item \textbf{Conica degenere}.
\end{itemize}	
\end{multicols}