% SVN info for this file
\svnidlong
{$HeadURL$}
{$LastChangedDate$}
{$LastChangedRevision$}
{$LastChangedBy$}

\chapter{Elenchi}
\labelChapter{elenchi}
\begin{multicols}{2}
	\listofdefines
	\listoftheoremas
\end{multicols}
\begin{comment}

\chapter{Elenchi ...}
\labelChapter{elenchi}

\begin{introduction}
‘‘BEEP BOOP INSERIRE CITAZIONE QUA BEEP BOOP.''
\begin{flushright}
	\textsc{NON UN ROBOT,} UN UMANO IN CARNE ED OSSA BEEP BOOP.
\end{flushright}
\end{introduction}

\noindent Elenchiamo, separati per capitoli, i principali teoremi e proposizioni.
\section{Capitolo 1: spazi topologici}
\textsc{\textbf{Teorema 1.1.0.}} \textsc{Teorema delle basi. (Manetti, 3.7)}\\
	Sia $X$ un insieme e $\basis\subseteq\setpart{X}$ una famiglia di sottoinsiemi di $X$. $\basis$ è la base di un'\textit{unica} topologia \textit{se e solo se}:
	\begin{enumerate}
		\item \textit{L'insieme} $X$ \textit{deve essere scritto come unione di elementi della famiglia}: $\displaystyle X=\union_{B\in \basis}B$.
		\item \textit{Per ogni punto dell'intersezione di elementi della famiglia deve esserci un'altro elemento di essa che contiene il punto ed è sottoinsieme dell'intersezione}:
		\begin{equation}
			\forall A, B\in\basis\ \forall x\in A\cap B\ \exists C\in \basis\ \colon x\in C\subseteq A\cap B
		\end{equation}
	\end{enumerate}
\textsc{\textbf{Teorema 1.2.0.}} \textsc{Composizione di funzioni continue. (Manetti, 3.26)}\\
	La \textit{composizione} di funzioni continue è continua.
	\begin{equation}
		\funz{f}{Y}{Z},\ \funz{g}{X}{Y}\text{ continue}\implies \funz{f\circ g}{X}{Z}\text{ continua}
	\end{equation}
	\vspace{-6mm}\\
\textsc{\textbf{Teorema 1.2.1.}} \textsc{Continuità per punti. (Manetti, 3.28)}\\
	Siano $X$, $Y$ spazi topologici e $\funz{f}{X}{Y}$ funzione. $f$ è continua per aperti $\iff$ $f$ è continua in $x\ \forall x\in X$.\\
\textsc{\textbf{Teorema 1.2.1.}} \textsc{Continuità per punti. (Manetti, 3.28)}\\
Siano $X$, $Y$ spazi topologici e $\funz{f}{X}{Y}$ funzione. $f$ è continua per aperti $\iff$ $f$ è continua in $x\ \forall x\in X$.\\
\textsc{\textbf{Teorema 1.6.0.}} \textsc{Base della topologia prodotto.} \textsc{(Manetti, 3.61)}\\
	\begin{enumerate}
		\item Una \textit{base} della topologia $\mathcal{P}$ è data dagli insiemi della forma $U\times V$ dove $U\subseteq P$ aperto, $V\subseteq Q$ aperto.
		\item $p,\ q$ sono aperte; inoltre $\forall \left(x,\ y\right)\in P\times Q$ le restrizioni:
		\begin{gather}
			\funztot{p_{\mid}}{P\times \left\{y\right\}}{P}{\left(x,\ y\right)}{x}\\
			\funztot{q_{\mid}}{\left\{x\right\}\times Q}{Q}{\left(x,\ y\right)}{y}
		\end{gather}
		Sono \textit{omeomorfismi}.\\
		\item Data $\funz{f}{X}{P\times Q}$ con $X$ spazio topologico, si ha che:
		\begin{equation}
			f\text{ continua}\iff f_1=p\circ f,\ f_2=q\circ f\text{ continue}
		\end{equation}
	\end{enumerate}
\textsc{\textbf{Teorema 1.6.0.}} \textsc{Base della topologia prodotto.} \textsc{(Manetti, 3.61)}
\begin{enumerate}
	\item Una \textit{base} della topologia $\mathcal{P}$ è data dagli insiemi della forma $U\times V$ dove $U\subseteq P$ aperto, $V\subseteq Q$ aperto.
	\item $p,\ q$ sono aperte; inoltre $\forall \left(x,\ y\right)\in P\times Q$ le restrizioni:
	\begin{gather}
		\funztot{p_{\mid}}{P\times \left\{y\right\}}{P}{\left(x,\ y\right)}{x}\\
		\funztot{q_{\mid}}{\left\{x\right\}\times Q}{Q}{\left(x,\ y\right)}{y}
	\end{gather}
	Sono \textit{omeomorfismi}.\\
	\item Data $\funz{f}{X}{P\times Q}$ con $X$ spazio topologico, si ha che:
	\begin{equation}
		f\text{ continua}\iff f_1=p\circ f,\ f_2=q\circ f\text{ continue}
	\end{equation}
\end{enumerate}
\textsc{\textbf{Teorema 1.7.0.}} \textsc{Diagonale e Hausdorff.}  \textsc{(Manetti, 3.69)}\\
	Sia $X$ spazio topologico. La \textbf{diagonale}\index{diagonale} $\Delta\subseteq X\times X$ è l'insieme delle coppie che hanno uguali componenti:
	\begin{equation}
		\Delta=\left\{\left(x,\ x\right)\mid x\in X\right\}
	\end{equation}
	Si ha:
	\begin{equation}
		X \text{ di \textbf{Hausdorff}} \iff \Delta \textsc{ chiuso in }X\times X
	\end{equation}
	\vspace{-6mm}\\
\textsc{\textbf{Teorema 1.8.0.}} \textsc{Prodotto di Hausdorff.}\\
	$X,\ Y$ di \textbf{Hausdorff}$\iff X\times Y$ di \textbf{Hausdorff}.
\section{Capitolo 2: connessione e compattezza}
\section{Capitolo 3: gruppi topologici}
\section{Capitolo 4: topologia quoziente}
\section{Capitolo 5: azioni di gruppo}
\section{Capitolo 6: assiomi di numerabilità e successioni}
\section{Capitolo 7: omotopia}
\section{Capitolo 8: gruppo fondamentale}
\section{Capitolo 9: varietà topologiche}
\section{Capitolo 10: approfondimenti di Algebra Lineare}
\section{Capitolo 11: geometria proiettiva}
\section{Capitolo 12: coniche proiettive}
	contenuto...
\end{comment}